% response-marshall.tex
% ----------------------------------------------------------------------
% response-header.tex
% ----------------------------------------------------------------------

% Base class and packages
\documentclass[11pt]{article}

% Included in online comment header
\usepackage[pdftex]{graphicx}
\usepackage[pdftex]{color}
\usepackage{amssymb}
%\usepackage{times}

% Additional packages
\usepackage[T1]{fontenc}
\usepackage{geometry}
\usepackage[hidelinks]{hyperref}
\usepackage{natbib}

% Graphic path of main manuscript
\graphicspath{{../figures/}}

% Replacements for Copernicus commands
\newcommand{\unit}[1]{\ensuremath{\mathrm{#1}}}
\newcommand{\chem}[1]{\ensuremath{\mathrm{#1}}}
\newcommand{\urlprefix}[0]{}

% Default font and spacing
\renewcommand\familydefault{\sfdefault}
\setlength{\parskip}{1.2ex}
\setlength{\parindent}{0em}
\linespread{1.5}

% color defined in comment template
\definecolor{journalname}{rgb}{0.34,0.59,0.82}

% todo command which should not be used in final version
\definecolor{todored}{rgb}{0.439,0.157,0.145}  % from Advances in Geoscience
\newcommand{\todo}[1]{\textcolor{todored}{TODO: #1}}

\linespread{1.0}

\begin{document}
\textbf{Authors' response to the Editor}
\bigskip

Dear Hilmar Gudmundsson,

We greatly apologize for delays accumulated during the peer review of our
manuscript. To some extent, the delay has been caused by our wish to address
the reviews as well as possible. In addition to a change of the first author's
working environment, setting up and running the sensitivity study asked by
Alexander Jarosch on a different supercomputer, using a new allocation that was
meant for a different project and became effective only in October, took us
considerable time.

We believe that we have addressed all points raised by the reviews and hereby
submit our revised manuscript to The Cryosphere. Please find hereafter a
summary of changes made to the manuscript since its first publication in The
Cryosphere Discussion, and a marked-up file showing all changes made.

With best regards,

Julien Seguinot on behalf of all authors.

\bigskip

Changes not related to the reviews:

\begin{itemize}

    \item We have updated all simulations to a more recent and stable version
    of PISM (0.7.2).

    \item We have rescaled palaeo-temperature proxy records with a higher
    precision than in the discussion version.

    \item We have clarified that given ice volumes concern only sea-level
    relevant ice and areas only grounded ice.

\end{itemize}

\bigskip

Changes following the review by S.~J.~Marshall:

\begin{itemize}

    \item \textbf{Sect.~1 (Introduction):}
    We have added references regarding geological evidence for late-glacial
    readvances.

    \item \textbf{Sect.~1 (Introduction):}
    We have added a short discussion of more regional palaeo-temperature proxy
    records which we deemed less adequate to force the model in order to
    justify the choice of less regional proxy records used in our study.

    \item \textbf{Sect.~2.2 (Ice rheology):}
    We have clarified the advantage and the main limitation of the hybrid SIA
    and SSA modelling approach.

    \item \textbf{Sect.~5.1.2 (Ice configuration during MIS 2):}
    In the light of modelled Last Glacial Maximum ice sheet extent, we have
    discussed further the limitations of forcing the model with fixed
    present-day temperature and precipitations patterns.

    \item \textbf{Sect.~6 (Conclusions):}
    We highlighted that although the GRIP and EPICA forcing yield best results
    in our study, more regional palaeo-climate reconstructions are needed.

    \item \textbf{Various places:}
    We have applied grammar corrections.

\end{itemize}

\bigskip

Changes following the review by A.~H.~Jarosch:

\begin{itemize}

    \item \textbf{Sect.~2.1 (Model overview):}
    We have clarified that ETOPO1 data were bilinearly interpolated to the
    model grids.

    \item \textbf{Sect.~2.1 (Model overview):}
    We have explained the new subdivisions of Sect.~2 (Model setup).

    \item \textbf{Sect.~2.2 (Ice rheology):}
    In this new subsection, we have added much detail on the model physics and
    on our choice of parameter for the sensitivity study to ice rheological
    parameters.

    \item \textbf{Sect.~2.2 (Ice rheology):}
    We removed the formulation ``SSA as a sliding law for the SIA'' which could
    induce in error.

    \item \textbf{Sect.~2.2 (Ice rheology):}
    We have made clear that temperatures are computed in the bedrock to a depth
    of 3\,km below the ice sheet base where geothermal heat flux is applied.

    \item \textbf{Sect.~2.3 (Basal sliding):}
    In this new subsection, we have added much detail on the model physics and
    on our choice of parameter for the sensitivity study to basal sliding
    parameters.

    \item \textbf{Sect.~2.6 (Climate forcing):}
    We have clarified that NARR temperature and precipitation fields were
    bilinearly interpolated to the model grids and admitted their limitations
    in reproducing steep climatic gradients characteristic of the study area.

    \item \textbf{Sect.~4 (Sensitivity to ice flow parameters):}
    In this new section, we present results of the new sensitivity study to ice
    rheological and basal sliding parameters

    \item \textbf{Sect.~5.1.2 (Ice configuration during MIS 2):}
    We have recognized the limitations of the NARR with simulating orographic
    processes in some areas of steep topography and acknowledged that using
    downscaling methods could indeed reduce some of the inconsistence between
    model results and geological reconstructions.

    \item \textbf{Sect.~5.1.2 (Ice configuration during MIS 2):}
    We have balanced the final statement concerning modelled ice volume maxima
    using the uncertainty estimates obtained in the new Sect.~4 (Sensitivity to
    ice flow parameters).

    \item \textbf{Sect.~5.1.3 (Ice configuration during MIS 4):}
    We have balanced the final statement concerning modelled ice volume maxima
    using the uncertainty estimates obtained in the new Sect.~4 (Sensitivity to
    ice flow parameters).

    \item \textbf{Sect.~5.2.2 (Major ice-dispersal centres):}
    In the light of modelled accumulation areas, we have shown that present-day
    ice volumes contained in the ETOPO1 basal topography data likely does not
    affect the conclusions of our study.

    \item \textbf{Sect.~5.2.2 (Major ice-dispersal centres):}
    We have balanced the final statement concerning modelled ice volume minima
    using the uncertainty estimates obtained in the new Sect.~4 (Sensitivity to
    ice flow parameters).

    \item \textbf{Table~1:}
    The default parameter table was extended to include more parameter values
    and references to their source.

    \item \textbf{Table~3:}
    This new table contains ice rheological and basal sliding parameter values
    used in the sensitivity study.

    \item \textbf{Table~5:}
    This new table contains modelled ice volume and area extremes corresponding
    to the results of the sensitivity study (similarly to Table~4).

    \item \textbf{Fig.~2:}
    This new figure illustrates the different parametrisations of ice rheology
    used in the sensitivity study.

    \item \textbf{Fig.~3:}
    This new figure illustrates the different parametrisations of basal sliding
    used in the sensitivity study.

    \item \textbf{Fig.~7:}
    This new figure contains modelled sea-level relevant ice volume time series
    resulting from the sensitivity study (similarly to Fig.~5).

    \item \textbf{Various places:}
    We have applied grammar corrections.

\end{itemize}

\bigskip

Changes following the review by A.~Stumpf:

\begin{itemize}

    \item \textbf{Abstract:}
    We have reworked the first two sentences to clarify that the Cordilleran
    ice sheet is less understood than its counterparts not by lack of previous
    studies but due to its complexity.

    \item \textbf{Fig.~1:}
    We have enlarged the location map to include full geographic names instead
    of abbreviations.

    \item \textbf{Figs.~4, 6, 8--12, 14--16:}
    We have added rivers, lakes and graticules on all model output maps.

    \item \textbf{Sect.~5.2.3 (Erosional imprint on the landscape):}
    We have added a reference to Stumpf et~al. (2000).

    \item \textbf{Sect.~5.3.3 (Deglacial flow directions):}
    We have added references to Stumpf et~al. (2000) and to Ferbey et~al. (2013).

\end{itemize}

\end{document}
