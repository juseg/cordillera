% response-marshall.tex
% ----------------------------------------------------------------------
% response-header.tex
% ----------------------------------------------------------------------

% Base class and packages
\documentclass[11pt]{article}

% Included in online comment header
\usepackage[pdftex]{graphicx}
\usepackage[pdftex]{color}
\usepackage{amssymb}
%\usepackage{times}

% Additional packages
\usepackage[T1]{fontenc}
\usepackage{geometry}
\usepackage[hidelinks]{hyperref}
\usepackage{natbib}

% Graphic path of main manuscript
\graphicspath{{../figures/}}

% Replacements for Copernicus commands
\newcommand{\unit}[1]{\ensuremath{\mathrm{#1}}}
\newcommand{\chem}[1]{\ensuremath{\mathrm{#1}}}
\newcommand{\urlprefix}[0]{}

% Default font and spacing
\renewcommand\familydefault{\sfdefault}
\setlength{\parskip}{1.2ex}
\setlength{\parindent}{0em}
\linespread{1.5}

% color defined in comment template
\definecolor{journalname}{rgb}{0.34,0.59,0.82}

% todo command which should not be used in final version
\definecolor{todored}{rgb}{0.439,0.157,0.145}  % from Advances in Geoscience
\newcommand{\todo}[1]{\textcolor{todored}{TODO: #1}}


\begin{document}
\textbf{Authors' response to S.~J.~Marshall}
\bigskip

% ----------------------------------------------------------------------
% Interactive comment text begins
% ----------------------------------------------------------------------

\newcommand{\sechead}[1]{\bigskip\noindent\textbf{#1}}
\newcommand{\referee}[1]{\bigskip\textcolor{journalname}{\textit{#1}}}
\newcommand{\msquote}[1]{\begin{quote}\textit{#1}\end{quote}}
\newcommand{\doi}[1]{doi:\allowbreak\href{http://dx.doi.org/#1}{#1}}

To S.~J.~Marshall,

We thank you very much for this positive review of our manuscript.

% ----------------------------------------------------------------------

\sechead{1 \quad Summary comments}

\referee{%
    Seguinot and colleagues provide the first detailed glaciological modelling
    of that I am aware of for the Cordilleran Ice Sheet in western North
    America, making this a novel and long overdue contribution. The authors
    have not only made new advances with this contribution, they have done so
    in an impressive leap forward. This is an excellent and carefully-presented
    study which is likely to rejuvenate interest and debate in Cordilleran Ice
    Sheet reconstructions. The balance between numerical modelling and glacial
    geological/geomorphological considerations is unusually strong, and the
    authors can be commended for this emphasis. This adds tremendous value to
    the results and increases confidence in the modelling, and I also
    appreciate that the authors point out areas where the numerical model is
    not in accord with the geological record.}

\referee{%
    The manuscript is well-written and beautifully illustrated, and I have very
    few substantive comments. The choices made by the authors are logical and
    well-explained, and they reach several well-substantiated conclusions: a
    two-phase Cordilleran glaciation, a reasonably robust estimate of CIS
    volume at LGM, the general model of CIS growth through multiple alpine
    icefields, and the importance of the Skeena Mountain inception centre. One
    can always quibble with specific aspects of the model design and climate
    scenarios, but the authors have explored a reasonable span of `solution
    space' and these aspects of the Cordilleran ice sheet history appear to be
    robust features of the simulations.}

\referee{%
    The modelling strategy and results presented here stand to be widely cited,
    and I expect that it will serve as a springboard for additional studies
    from others in the international community. I recommend this manuscript for
    publication in The Cryosphere without reservations.}

% ----------------------------------------------------------------------

\sechead{2 \quad Specific comments}

\referee{%
    The Cordilleran Ice Sheet is difficult to model due to its complex
    topography and multiple inception centres (and possibly multiple
    domes/divides), strong regional climatic gradients, which require
    relatively high-resolution climate input fields, and a dearth of
    paleoclimate proxies for western North America to inform spatial and
    temporal variations in climate conditions during the glacial period. The
    authors confront these challenges well, with a adequate ice sheet model
    resolution (5 to 10\,km) and ice physics, carefully calibrated `control'
    climatology (published in Seguinot et~al., 2014), and a good
    exploration of different paleoclimate time series histories in this
    contribution.}

Thank you for summarizing so well the most challenging aspects of our study.

\referee{%
    Nevertheless, it is not clear that ice-core based paleoclimate proxies from
    Greenland or Antarctica are appropriate for western North America. This may
    be particularly true of Greenland proxies, where the amplitude of D-O
    (millennial) climate variations is exceptionally strong and is likely to be
    regional. Because these remote ice core records are `scaled' based on only
    one constraint, producing a CIS maximum configuration that resembles the
    geological record, it is difficult to assess the pre-LGM simulations or the
    details of the modelled ice divide structure, ice thickness, etc. The
    robustness of the conclusion that Greenland and Antarctic ice core records
    are good proxies of glacial climate variability in western North America is
    therefore not so clear, but it is admittedly hard to do better at this
    time. I do wonder if there is any hope from more regional climate proxies
    such as the Logan ice cores or the off-shore Vancouver Island sediment
    records that are cited from Cosma et al. This is worth a short discussion.}

Thank you for raising this point. Before explaining our choice of
palaeo-climate proxy records, we would like to clarify that we do not consider
the GRIP and EPICA ice core records, our preferred climate drivers, as
satisfactory proxies of climate variability in western North America. Instead,
we think that more regional proxy records are needed in order to better
understand the dynamics of the Cordilleran ice sheet through the last glacial
cycle, and have updated the conclusions to highlight this fact in the
manuscript.

To choose palaeo-climate proxy records to force the model, we first focused on
records that are recognized as proxies for temperature. These include oxygen
isotope (\chem{\delta^{18}O}) records from the Greenland and Antarctic ice
sheets. Besides ice core records, we use alkenone unsaturation index
(\chem{U^{K'}_{37}}) series from oceanic sediment cores, a known proxy for
sea-surface temperatures \citep{Prahl.Wakeham.1987, Prahl.etal.1988,
Muller.etal.1998}. However, we do not use \chem{\delta^{18}O} records from benthic
foraminifera, which have been evaluated worldwide \citep{Lisiecki.Raymo.2005},
including for core~MD02-2496 near Vancouver Island \citep{Cosma.etal.2008}, but
are commonly interpreted as proxies for global ice volume on land rather than
for temperatures \citep{Shackleton.1967}. Second, we chose to use only records
that span over the last 120\,ka in order to avoid spinning-up the model using a
different record than the one being tested. The Mount Logan ice core
\chem{\delta^{18}O} record covers only the last 30\,Ka and has been interpreted
as a proxy for source region rather than for palaeo-temperature
\citep{Fisher.etal.2004, Fisher.etal.2008} and thus does not fit our criteria.

Concerning sediment cores offshore Vancouver Island, sea-surface temperatures
have been reconstructed for core~JT96-09 from alkenone unsaturation indices
over the last 16 ka \citep{Kienast.McKay.2001}, and for the nearby
core~MD02-2496 from the Mg/Ca ratio in planktonic foraminifera (\emph{N.
pachyderma} and \emph{G. bulloides}) between 12 and
21\,\chem{^{14}C}\,cal\,ka \citep{Taylor.etal.2014}. Very recently, this
planktonic foraminifera record has been extended to cover the period from 10 to
ca.~50\,\chem{^{14}C}\,cal\,ka \citep{Taylor.etal.2015}, and thus it
could perhaps be used to model the growth and decay of the Cordilleran ice
sheet during
the Marine Oxygen Isotope Stage (MIS) 2, but such a simulation could not be
directly compared with other runs without spinning-up the model using a
different proxy record.

A shortened version of the preceding explanation has been added to the climate
forcing section of the manuscript.

\referee{%
    Similarly it is difficult to know the errors and uncertainties associated
    with the assumption of fixed modern-day spatial patterns for temperature
    and precipitation. I suspect that the sensitivity of this assumption far
    exceeds that associated with the different paleoclimate proxies. Such that,
    for example, one could readily imagine different assumptions, such as a
    maritime effect that gives reduced glacial cooling near the coast vs. in
    the interior, that is a stronger effect than the difference between
    different paleo-climate proxies with respect to the timing of LGM, ice
    divide structure, etc. But this is a very reasonable start, what the
    authors have done -- there is always going to be more parameter space to
    explore in future studies. As above, I would perhaps just suggest a small
    discussion of the authors' opinion on this question, the uncertainty or
    possible influence of this assumption of modern-day climate patterns.}

Indeed, we chose to keep modern-day spatial patterns for temperature and
precipitation constant in order to limit the number of degrees of freedom in
our study.

Regarding temperature, it is reasonable to think that the changes were greater
inland than near the coast. However, our simulations already produce an
excess of ice inland. Including such a temperature continentality gradient
in the model while keeping the precipitation pattern constant would thus cause
an even greater mismatch between the model results and the geologically
reconstructed Last Glacial Maximum (LGM) ice margins. We have introduced a
short comment of this effect in Sect.~5.1.2 (previously~4.1.2, Ice
configuration during MIS~2).

However, the mismatch we observe between the modelled and reconstructed LGM ice
margins let us think that the assumption of fixed modern-day precipitation
patterns is more critical than the assumption of fixed modern-day temperature
patterns. During phases of ice sheet growth, the presence of mountain ice caps
on the Coast Mountains likely resulted in a decrease of precipitation inland
(Sect.~5.1.2). Furthermore, the adiabatic warming associated with moisture
depletion in the interior (Sect.~5.1.2) may have counterbalanced the potential
continentality gradient discussed above.

\referee{%
    Several minor points and grammatical corrections are included in the
    attached text. Nothing that will require much thought -- this is a really
    impressive piece of research, overall, and I am hard-pressed to find any
    criticism of it. It is one of the easiest reviews I have ever done.
    Congratulations to the authors and thanks for this fine work.}

We have corrected grammatical mistakes given in the attachment. Thank you for
spotting them. And thank you again very much for supporting our manuscript with
such enthusiasm!

% ----------------------------------------------------------------------

\begin{thebibliography}{69}

\bibitem[{Cosma et~al.(2008)Cosma, Hendy, and Chang}]{Cosma.etal.2008}
Cosma, T., Hendy, I., and Chang, A.: Chronological constraints on Cordilleran
  Ice Sheet glaciomarine sedimentation from core MD02-2496 off Vancouver Island
  (western Canada), Quaternary Sci. Rev., 27, 941--955,
  \doi{10.1016/j.quascirev.2008.01.013}, 2008.

\bibitem[{Fisher et~al.(2008)Fisher, Osterberg, Dyke, Dahl-Jensen, Demuth,
  Zdanowicz, Bourgeois, Koerner, Mayewski, Wake, Kreutz, Steig, Zheng, Yalcin,
  Goto-Azuma, Luckman, and Rupper}]{Fisher.etal.2008}
Fisher, D., Osterberg, E., Dyke, A., Dahl-Jensen, D., Demuth, M., Zdanowicz,
  C., Bourgeois, J., Koerner, R.~M., Mayewski, P., Wake, C., Kreutz, K., Steig,
  E., Zheng, J., Yalcin, K., Goto-Azuma, K., Luckman, B., and Rupper, S.: The
  Mt Logan Holocene--late Wisconsinan isotope record: tropical Pacific--Yukon
  connections, The Holocene, 18, 667--677, \doi{10.1177/0959683608092236},
  2008.

\bibitem[{Fisher et~al.(2004)Fisher, Wake, Kreutz, Yalcin, Steig, Mayewski,
  Anderson, Zheng, Rupper, Zdanowicz, Demuth, Waszkiewicz, Dahl-Jensen,
  Goto-Azuma, Bourgeois, Koerner, Sekerka, Osterberg, Abbott, Finney, and
  Burns}]{Fisher.etal.2004}
Fisher, D.~A., Wake, C., Kreutz, K., Yalcin, K., Steig, E., Mayewski, P.,
  Anderson, L., Zheng, J., Rupper, S., Zdanowicz, C., Demuth, M., Waszkiewicz,
  M., Dahl-Jensen, D., Goto-Azuma, K., Bourgeois, J.~B., Koerner, R.~M.,
  Sekerka, J., Osterberg, E., Abbott, M.~B., Finney, B.~P., and Burns, S.~J.:
  Stable Isotope Records from Mount Logan, Eclipse Ice Cores and Nearby
  Jellybean Lake. Water Cycle of the North Pacific Over 2000 Years and Over
  Five Vertical Kilometres: Sudden Shifts and Tropical Connections, G\'{e}ogr.
  phys. Quatern., 58, 337, \doi{10.7202/013147ar}, 2004.

\bibitem[{Kienast and McKay(2001)}]{Kienast.McKay.2001}
Kienast, S.~S. and McKay, J.~L.: Sea surface temperatures in the subarctic
  northeast Pacific reflect millennial-scale climate oscillations during the
  last 16 kyrs, Geophys. Res. Lett., 28, 1563--1566,
  \doi{10.1029/2000GL012543}, 2001.

\bibitem[{Lisiecki and Raymo(2005)}]{Lisiecki.Raymo.2005}
Lisiecki, L.~E. and Raymo, M.~E.: A Pliocene-Pleistocene stack of 57 globally
  distributed benthic $\delta^{18}$O records, Paleoceanography, 20, PA1003,
  \doi{10.1029/2004pa001071}, 2005.

\bibitem[{M\"{u}ller et~al.(1998)M\"{u}ller, Kirst, Ruhland, von Storch, and
  Rosell-Mel{\'{e}}}]{Muller.etal.1998}
M\"{u}ller, P.~J., Kirst, G., Ruhland, G., von Storch, I., and
  Rosell-Mel{\'{e}}, A.: Calibration of the alkenone paleotemperature index
  $U^{K'}_{37}$ based on core-tops from the eastern South Atlantic and the
  global ocean (60$^\circ$N-60$^\circ$S), Geochim. Cosmochim. Ac., 62,
  1757--1772, \doi{10.1016/s0016-7037(98)00097-0}, 1998.

\bibitem[{Prahl and Wakeham(1987)}]{Prahl.Wakeham.1987}
Prahl, F.~G. and Wakeham, S.~G.: Calibration of unsaturation patterns in
  long-chain ketone compositions for palaeotemperature assessment, Nature, 330,
  367--369, \doi{10.1038/330367a0}, 1987.

\bibitem[{Prahl et~al.(1988)Prahl, Muehlhausen, and Zahnle}]{Prahl.etal.1988}
Prahl, F.~G., Muehlhausen, L.~A., and Zahnle, D.~L.: Further evaluation of
  long-chain alkenones as indicators of paleoceanographic conditions, Geochim.
  Cosmochim. Ac., 52, 2303--2310, \doi{10.1016/0016-7037(88)90132-9}, 1988.

\bibitem[{Seguinot et~al.(2014)Seguinot, Khroulev, Rogozhina, Stroeven, and
  Zhang}]{Seguinot.etal.2014}
Seguinot, J., Khroulev, C., Rogozhina, I., Stroeven, A.~P., and Zhang, Q.: The
  effect of climate forcing on numerical simulations of the {C}ordilleran ice
  sheet at the {L}ast {G}lacial {M}aximum, The Cryosphere, 8, 1087--1103,
  \doi{10.5194/tc-8-1087-2014}, 2014.

\bibitem[{Shackleton(1967)}]{Shackleton.1967}
Shackleton, N.: Oxygen Isotope Analyses and Pleistocene Temperatures
  Re-assessed, Nature, 215, 15--17, \doi{10.1038/215015a0}, 1967.

\bibitem[{Taylor et~al.(2014)Taylor, Hendy, and Pak}]{Taylor.etal.2014}
Taylor, M., Hendy, I., and Pak, D.: Deglacial ocean warming and marine margin
  retreat of the Cordilleran Ice Sheet in the North Pacific Ocean, Earth
  Planet. Sc. Lett., 403, 89--98, \doi{10.1016/j.epsl.2014.06.026}, 2014.

\bibitem[{Taylor et~al.(2015)Taylor, Hendy, and Pak}]{Taylor.etal.2015}
Taylor, M.~A., Hendy, I.~L., and Pak, D.~K.: The California Current System as a
  transmitter of millennial scale climate change on the northeastern Pacific
  margin from 10 to 50 ka, Paleoceanography, 30, 1168--1182,
  \doi{10.1002/2014pa002738}, 2015.

\end{thebibliography}

% ----------------------------------------------------------------------
% Interactive comment text ends
% ----------------------------------------------------------------------

\end{document}
