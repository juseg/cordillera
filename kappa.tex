% kappa.tex
% ======================================================================

% Base class and packages
\documentclass{article}
\usepackage{amsmath}
\usepackage{bm}
\usepackage{lineno}
\usepackage{natbib}
\usepackage[onehalfspacing]{setspace}
\usepackage{physics}
\usepackage[colorlinks, citecolor=blue]{hyperref}
\usepackage[sort]{cleveref}  % needs to be loaded after hyperref
%\usepackage{graphicx}

% Copernicus-style internal references
\crefname{chapter}{Chap.}{Chaps.}
\crefname{equation}{Eq.}{Eqs.}
\crefname{figure}{Fig.}{Figs.}
\crefname{section}{Sect.}{Sects.}
\crefname{table}{Table}{Tables}

% My commands
%\newcommand{\note}[1]{\textbf{[NOTE: #1]}}
\newcommand{\todo}[1]{\emph{[\textbf{Todo:} #1]}}
%\newcommand{\aref}[0]{\textbf{[ref.]}}

% Vectors and tensors
\newcommand{\vect}[1]{\va*{#1}} % bold arrow vectors
\newcommand{\tens}[1]{\vb*{#1}} % bold tensors

% Differential operators
\renewcommand{\div}[1]{\mathrm{div}\,#1}            % divergence
\renewcommand{\grad}[1]{\vect{\mathrm{grad}}\,#1}   % gradient
\newcommand{\tdiv}[1]{\vect{\mathrm{div}}\,#1}      % tensor divergence
\newcommand{\tgrad}[1]{\tens{\mathrm{grad}}\,#1}    % tensor gradient
\newcommand{\matdv}[1]{\pdv{#1}{t}+\vect{v}\cdot\grad{}\,#1}  % material dv.

% Common notations
\newcommand{\doteps}[0]{\dot{\epsilon}} % epsilon dot
\newcommand{\IDT}[0]{\tens{\delta}}     % Identity tensor
\newcommand{\CST}[0]{\tens{\sigma}}     % Cauchy stress tensor
\newcommand{\DST}[0]{\tens{\tau}}       % deviatoric stress tensor
\newcommand{\SRT}[0]{\tens{\doteps}}    % strain-rate tensor
\newcommand{\vv}[0]{\vect{v}}           % velocity vector
\newcommand{\vsia}[0]{\vv_{\mathrm{SIA}}}   % SIA velocity
\newcommand{\vssa}[0]{\vv_{\mathrm{SSA}}}   % SSA velocity

% Common units
\newcommand{\unit}[1]{\ensuremath{\mathrm{#1}}}
\newcommand{\degree}[0]{\ensuremath{^{\circ}}}
\newcommand{\degC}[0]{\unit{{\degree}C}}

% Document properties
\title{Numerical simulation of the Cordilleran ice sheet \\
       through the last glacial cycle}
\author{Julien Seguinot}

% ======================================================================
\begin{document}
% ======================================================================

\maketitle
\linenumbers

% ======================================================================
\section{Introduction}
% ======================================================================
% Ice as viscous matter
% Glaciers and ice sheets
% The last glacial cycle

% Fig. crevasses
% Fig. glacier flowing
% Fig. recent climate evolution
% Fig. present and palaeo-ice sheets

% ======================================================================
\section{Field area description}
% ======================================================================
% Geographic setting
% Climatic setting
% Palaeo-glaciology

% Fig. map of North America
% Fig. map of the Cordillera?
% Fig. photos of the Cordillera
% Fig. photos of fieldwork
% Fig. Johan lineation map?

% ======================================================================
\section{Numerical ice sheet model}
% ======================================================================

% ----------------------------------------------------------------------
\subsection{Overview}
% ----------------------------------------------------------------------

The simulations presented in this thesis use Parallel Ice Sheet Model (PISM),
an open-source, finite-difference, shallow ice-sheet model
\citep{PISM-authors.2014}. The model reads bedrock topography, sea level,
geothermal heat flux and climate forcing as inputs, and computes the thermal
and dynamic state of the ice sheet, and the associated thermal and
deformational response of an idealised lithosphere. The atmospheric and oceanic
envelopes are not part of the model. However, PISM includes boundary models
that provide simple parametrisations of their effect on the ice model at
supposedly well-defined interfaces (Fig.~??).

% Fig. PISM = ice + bedrock

This section does not attempt to provide an extensive description of PISM,
which can be found in the on-line documentation \citep{PISM-authors.2014}.
Instead, it aims to highlight modelling choices made in the context of this
thesis, and give an overview of the physics involved in these choices.
Notations used and corresponding parameter choices are summarized in Table~??.

% ----------------------------------------------------------------------
\subsection{Field equations}
% ----------------------------------------------------------------------

The thermodynamic core of the ice sheet model relies on four elementary field
equations. Firstly, ice flow is assumed \emph{incompressible}. Hence, the ice
density, $\rho$, is constant, and the
conservation of mass implies a conservation of volume, which can be expressed
in terms of the ice velocity vector, $\vv$, by
\begin{equation}
    \label{eqn:incompressibility}
    \div{\vv} = 0 \,.
\end{equation}

Secondly, the \emph{balance of stresses} is expressed by the Stokes equation,
thus assuming creeping flow. In other words, ice is considered as a slow-moving
fluid whose deformation is entirely controlled by internal friction, here
expressed by the Cauchy stress tensor, $\CST$, and gravity, $\vect{g}$:
\begin{equation}
    \label{eqn:stressbalance}
    \tdiv{\CST} + \rho\,\vect{g} = \vect{0} \,.
\end{equation}

Thirdly, after defining the strain-rate tensor
${\SRT = \frac{1}{2}(\tgrad{\vv} + \tgrad{\vv}^{\mathrm{T}})}$,
and the deviatoric stress tensor, ${\DST = \CST - p\,\IDT}$,
where $p=\frac{1}{3}\tr(\CST)$ is the hydrostatic pressure and
$\tens{\delta}$ is the identity tensor, a \emph{constitutive law} for ice based
on laboratory experiments is used to relate these two quantities. It reads:
\begin{equation}
    \label{eqn:glenslaw}
    \SRT = A\,\tau_e^{n-1}\,\DST \,.
\end{equation}
The equivalent stress, $\tau_e$, is defined by
${\tau_e}^2 = -\mathrm{II}_{\DST} = \frac{1}{2} \tr(\DST^2)$,
where $\mathrm{II}_{\DST}$ is the second invariant of the stress tensor.
The ice softness coefficient, $A$, depends on the deviation of temperature $T$
from the pressure-melting point $T_m$ through
\begin{equation}
    A = A_0\,e^\frac{-Q}{R(T-T_m)} \,.
\end{equation}

Letting ${\doteps_e}^2 = \frac{1}{2} \tr(\SRT^2)$ and $B=A^{-1/n}$, the
constitutive law (\ref{eqn:glenslaw}) can be rewritten using a sometimes more
convenient, inverted formulation,
\begin{equation}
    \label{eqn:invglenslaw}
    \DST = B\,\doteps_e^{1/n-1}\,\SRT \,.
\end{equation}
By analogy to Newtonian flow, an apparent viscosity, $\nu$, can then be defined
by $\DST = \nu\,\SRT$, thus yielding

\begin{equation}
    \label{eqn:viscosity}
    \nu = \frac{B}{2}\,\doteps_e^{1/n-1} \,.
\end{equation}

Finally, the evolution of temperature within the ice is governed by the
\emph{conservation of energy}. In PISM, the conservation of energy uses an
enthalpy formulation in order to account for thermodynamic effects associated
with internal phase changes in the glacier. This enthalpy ($H$) formulation
reads
\begin{equation}
    \label{eqn:enthalpy}
    \rho \left(\matdv{H}\right)
        = -\div{\vect{q}} + \frac{\nu\doteps_e^2}{4} \,,
\end{equation}
where $\vect{q}$ represents the heat flow and
${\frac{\nu\dot{\epsilon_e}^2}{4} = \tr(\DST\SRT)}$ is a
source term corresponding to strain heat release. In the case where ice
temperature, $T$, is below the pressure-melting point, the enthalpy can be
expressed as $H=c\,T$, while the heat flow is given by Fourier's law,
$\vect{q} = k\,\grad{T}$. Consequently, the enthalpy equation
(\ref{eqn:enthalpy}) can be rewritten using a more familiar, cold-ice,
temperature formulation,
\begin{equation}
    \label{eqn:temperature}
    \matdv{T} = \frac{k}{\rho c} \Delta T
                + \frac{\nu\doteps_e^2}{4\rho c} \,,
\end{equation}
where $c$ denotes the specific heat capacity of the ice and $k$ is its thermal
conductivity.

\Cref{eqn:incompressibility,eqn:stressbalance,eqn:glenslaw,eqn:enthalpy}
constitute the thermodynamic basis
of our model. However, their resolution in full form implies a computational
demand too high for multi-millennial, continental-scale applications, such as
the numerical modelling of the Cordilleran ice sheet through the last glacial
cycle, thus requiring shallow approximations.

% ----------------------------------------------------------------------
\subsection{Shallow approximations}
% ----------------------------------------------------------------------

PISM, the ice sheet model that we use, employ two well-documented
approximations of ice flow equations: the shallow ice approximation and the
the shallow shelf approximation. Although both approximations can be derived
from a rigorous scaling approach, we don't expand these lengthy derivations
here, assuming instead the simplifications and focusing on their effect.

% Fig. SIA + SSA

Distinguishing the hydrostatic and deviatoric components of the stress tensor,
the balance of stresses (\ref{eqn:stressbalance}) can be expanded into
\begin{equation}
    \tdiv{\DST} - \grad{p} + \rho\,\vect{g} = \vect{0} \,.
\end{equation}

Projecting along the three Cartesian coordinate axis, this reads
\begin{subequations}
\label{eqn:fullstokes}
\begin{align}
    \pdv{\tau_{xx}}{x} + \pdv{\tau_{xy}}{y} + \pdv{\tau_{xz}}{z}
        &= \pdv{p}{x} \,, \\
    \pdv{\tau_{yx}}{x} + \pdv{\tau_{yy}}{y} + \pdv{\tau_{yz}}{z}
        &= \pdv{p}{y} \,, \\
    \pdv{\tau_{zx}}{x} + \pdv{\tau_{zy}}{y} + \pdv{\tau_{zz}}{z}
        &= \pdv{p}{z} - \rho g \,.
\end{align}
\end{subequations}

The shallow ice approximation (SIA) assumes that basal friction is high enough
for the vertical shear stresses $\{\tau_{xz}, \tau_{yz}\}$ to predominate over
all other components of the deviatoric stress tensor. In this framework, the
projected stress-balance can be simplified to
\begin{subequations}
\label{eqn:sia}
\begin{align}
    \label{eqn:siax}
    \pdv{\tau_{xz}}{z} &= \pdv{p}{x} \,, \\
    \label{eqn:siay}
    \pdv{\tau_{yz}}{z} &= \pdv{p}{y} \,, \\
    \label{eqn:siaz}
    0 &= \pdv{p}{z} - \rho g \,.
\end{align}
\end{subequations}

Neglecting atmospheric pressure, \Cref{eqn:siaz} yield
\begin{equation}
    p = \rho g (s-z) \,,
\end{equation}
which can be re-introduced into \cref{eqn:siax,eqn:siay}, thus yielding the
expression of the horizontal shear stresses, sometimes referred to as driving
stresses,
\begin{subequations}
\begin{align}
    \tau_{xz} &= -\rho g (s-z) \pdv{s}{x} \,, \\
    \tau_{yz} &= -\rho g (s-z) \pdv{s}{x} \,.
\end{align}
\end{subequations}

Using the constitutive law (\ref{eqn:glenslaw}), vertical derivatives of
horizontal velocities can be related to the horizontal shear stresses:
\begin{subequations}
\begin{align}
    \pdv{v_x}{z} &= 2A (\tau_{xz}^2 + \tau_{yz}^2)^{n-1} \tau_{xz} \,, \\
    \pdv{v_y}{z} &= 2A (\tau_{xz}^2 + \tau_{yz}^2)^{n-1} \tau_{yz} \,.
\end{align}
\end{subequations}

Thus, within the framework of the SIA, horizontal velocities $\vv_{SIA}$ can be
directly expressed as a function of the slope gradient,
\begin{subequations}
\begin{align}
    \pdv{v_x}{z} &= -2A (\rho g)^n (s-z)^n
                    \left(\pdv{s}{x}+\pdv{s}{y}\right)^{n-1} \pdv{s}{x} \,, \\
    \pdv{v_y}{z} &= -2A (\rho g)^n (s-z)^n
                    \left(\pdv{s}{x}+\pdv{s}{y}\right)^{n-1} \pdv{s}{y} \,,
\end{align}
\end{subequations}
which yields, in vectorial notation,
\begin{equation}
    \label{eqn:vsia}
    \pdv{\vsia}{z} = -2A\,(\rho g)^n\,(s-z)^n\,|\grad{s}|^{n-1}\,\grad{s} \,.
\end{equation}

The shallow shelf approximation (SSA), on the opposite of the SIA, assumes that
basal friction is low enough for ice to deform predominantly by horizontal
expansion within the entire ice column. In this context, the stress-balance can
be simplified to
\begin{subequations}
\label{eqn:ssa}
\begin{align}
    \label{eqn:ssax}
    \pdv{\tau_{xx}}{x} + \pdv{\tau_{xy}}{y} + \pdv{\tau_{xz}}{z}
        &= \pdv{p}{x} \,, \\
    \label{eqn:ssay}
    \pdv{\tau_{yx}}{x} + \pdv{\tau_{yy}}{y} + \pdv{\tau_{yz}}{z}
        &= \pdv{p}{y} \,, \\
    \label{eqn:ssaz}
    \pdv{\tau_{zz}}{z} &= \pdv{p}{z} - \rho g \,,
\end{align}
\end{subequations}
and horizontal velocities can be assumed independent of depth, so that
\begin{align}
    \pdv{v_x}{z} &= 0 \,, \\
    \pdv{v_y}{z} &= 0 \,.
\end{align}

Once again neglecting atmospheric pressure, and remembering that $\tr(\DST)=0$,
\Cref{eqn:ssaz} yields
\begin{align}
    p &= \rho g (s-z) - \tau_{zz} \,, \\
      &= \rho g (s-z) + \tau_{xx} + \tau_{yy} \,,
\end{align}
which can be re-introduced into \cref{eqn:ssax,eqn:ssay}:
\begin{subequations}
\begin{align}
    2\pdv{\tau_{xx}}{x} - \pdv{\tau_{yy}}{x} + \pdv{\tau_{xy}}{y}
        + \pdv{\tau_{xz}}{z} &= \rho g \pdv{s}{x} \,, \\
    \pdv{\tau_{yx}}{x} + 2\pdv{\tau_{yy}}{y} - \pdv{\tau_{xx}}{y}
        + \pdv{\tau_{yz}}{z} &= \rho g \pdv{s}{y} \,.
\end{align}
\end{subequations}

Using the inverse formulation of the constitutive law \eqref{eqn:invglenslaw},
components of the deviatoric stress tensor can be replaced by corresponding
velocity components, yielding
\begin{subequations}
\begin{align}
    \pdv{x} \left[2\bar{\nu}
                  \left(2\pdv{v_x}{x} + \pdv{v_y}{y}\right) \right]
        + \pdv{y} \left[\bar{\nu}
                        \left(\pdv{v_x}{y} + \pdv{v_y}{x}\right) \right]
        + \pdv{\tau_{xz}}{z} &= \rho g \pdv{s}{x} \,, \\
    \pdv{x} \left[\bar{\nu}
                  \left(\pdv{v_x}{y} + \pdv{v_y}{x}\right) \right]
        + \pdv{y} \left[2\bar{\nu}
                        \left(\pdv{v_x}{x} + 2\pdv{v_y}{y}\right) \right]
         + \pdv{\tau_{yz}}{z} &= \rho g \pdv{s}{y} \,,
\end{align}
\end{subequations}
where $\bar{\nu}$ is the depth-integrated apparent viscosity defined by
\begin{equation}
    \bar{\nu} = \frac{1}{H}\int_b^s\nu \,.
\end{equation}

A final integration over the entire ice column yields
\begin{subequations}
\begin{align}
    \label{eqn:vssa}
    \pdv{x} \left[2\bar{\nu}h
                  \left(2\pdv{v_x}{x} + \pdv{v_y}{y}\right) \right]
        + \pdv{y} \left[\bar{\nu}h
                        \left(\pdv{v_x}{y} + \pdv{v_y}{x}\right) \right]
        + \tau_{bx} &= \rho gh \pdv{s}{x} \,, \\
    \pdv{x} \left[\bar{\nu}H
                  \left(\pdv{v_x}{y} + \pdv{v_y}{x}\right) \right]
        + \pdv{y} \left[2\bar{\nu}h
                        \left(\pdv{v_x}{x} + 2\pdv{v_y}{y}\right) \right]
        + \tau_{by} &= \rho gh \pdv{s}{y} \,,
\end{align}
\end{subequations}
where $\tau_{bx}$ and $\tau_{by}$ correspond to $x$ and $y$-components of the
basal traction, and will be determined by a sliding law.

In PISM, the SIA and SSA velocities are eventually added,
\begin{equation}
    \label{eqn:siassa}
    \vv = \vsia + \vssa \,.
\end{equation}
Because $\vsia$ is small where $\vssa$ is large and
reciprocally, we can expect the model to perform well in regions where
conditions for applicability of either the SIA or the SSA are met. However,
\cref{eqn:siassa} is a heuristic, and its validity in the zone of
transition between shear flow and longitudinal flow has not yet been
mathematically proven.

% ----------------------------------------------------------------------
\subsection{Basal sliding}
% ----------------------------------------------------------------------

Although the SIA is valid only under non-sliding conditions, the SSA requires
a sliding law as basal boundary condition, which in fact constitutes the main
control on SSA velocities and in turn locations of ice streams.

The simulations presented in this thesis use a pseudo-plastic sliding law,
\begin{equation}
    \label{eqn:pseudoplastic}
    \vect{\tau}_b = -\tau_c \frac{\vv_b}{{v_{th}}^q\,|\vv_b|^{1-q}} \,,
\end{equation}
where $\vect{\tau}_b$ is the basal traction force, $\vv_b$ is the basal
velocity, and $v_{th}$ is an arbitrary velocity threshold. In the case of
$q=0$, \cref{eqn:pseudoplastic} corresponds to a purely plastic law,
\begin{equation}
    \label{eqn:plastic}
    \vect{\tau}_b = -\tau_c \frac{\vv_b}{|\vv_b|} \,.
\end{equation}

However a non-zero exponent is chosen here in order to improve convergence. The
yield stress $\tau_c$ is related to till properties by the Mohr-Coulomb
criterion,
\begin{equation}
   \tau_c = c_0 + N\,\tan{\phi} \,.
\end{equation}

The till cohesion $c_0$ is assumed to be zero. The effective pressure on the
till, $N$, is determined by the modelled amount of water at the bed. However,
two different parametrisations are used within this thesis. In Paper~I, using
PISM~0.5, the effective pressure is linearly related to water content by a
simple parametrisation,
\begin{equation}
    N = \rho gH (1 - \alpha \frac{W}{W_{max}}) \,.
\end{equation}

In Paper~IV, using PISM~0.6, the effective pressure is determined by an
empirical parametrisation based on laboratory experiments on Antarctic till,
\begin{equation}
    N = \delta \rho gH \, 10^{(e_0/C_c) (1 - (W/W_{max}))} \,.
\end{equation}

Drainage of water at the base of the ice sheet, although implemented in PISM,
is not included in simulations presented here for the sake of simplicity.
In turn, the amount of water at the bed, $W$, corresponds to the local
accumulation of basal melt-water. In varies from zero to $W_{max}$, a threshold
above which further melt exits the system.

Finally, the till friction angle $\phi$ is taken a~function of modern bed
elevation, with lowest values occurring at low elevations, thus accounting for
a weakening of till associated with the presence of marine sediments:
\begin{equation}
    \phi(x,y) =
    \begin{cases}
        \phi_0 & \text{if}\ b \le b_0 \,, \\
        \phi_0 + (\phi_1-\phi_0) \frac{b - b_0}{b_1-b_0}
                & \text{if}\ b_0 < b < b_1 \,, \\
        \phi_1 & \text{if}\ b_1 \le b \,,
    \end{cases}
\end{equation}
where $b_0=0$\,m is modern sea-level and $b_1=200$\,m is a rough average of
highest shoreline elevations in the model domain. The values of $\phi_1$ and
$\phi_2$ are chosen differently in Paper~I (10 and 30$^\circ$) and Paper~IV
(15 and 45$^\circ$), but these choices are roughly equivalent in the case of a
saturated till ($W=W_{max}$).

% ----------------------------------------------------------------------
\subsection{Bedrock response}
% ----------------------------------------------------------------------

The modelled ice sheet interacts with the underlying bedrock in two ways.
Firstly, bedrock temperatures are computed to a given depth by appending the
ice enthalpy model (\ref{eqn:enthalpy}) with an underlying bedrock thermal
model. The only process accounted for is heat conduction, hence this model
consists of a simple diffusion equation,
\begin{equation}
    \pdv{T}{t} = \frac{k_b}{\rho_b c_b} \Delta T \,,
\end{equation}
where $k_b$, $\rho_b$ and $c_b$ are the thermal conductivity, density and
specific heat capacity of the bedrock. Bedrock temperature is conditioned at
depth by a fixed upwards geothermal heat flux $q_G$. The bedrock thermal model
is necessary for modelling the Cordilleran ice sheet on multi-millennial
time-scales, because temperature fluctuations caused by climate change and
isolating effects of the ice sheet penetrate several hundred meters into the
rock, and eventually affect basal ice temperatures.

Secondly, bedrock elevation evolve in response to the ice load. The bedrock
deformation model used in this thesis describes the flexure of an elastic
lithosphere on top of an infinite half-space viscous mantle
\citep{Lingle.Clark.1985}. It can be described by a single differential
equation of the bed elevation, $b$,
\begin{equation}
    2\nu_m\,|\grad|\,\pdv{b}{t} + \rho_l g b + D\,\Delta^2 b = \sigma_{zzb} \,,
\end{equation}
where $\nu_m$ is mantle viscosity, $\rho_l$ is lithosphere density, $D$ is the
lithosphere's flexural rigidity, and $\sigma_{zzb}$ corresponds to the ice load
\citep{Bueler.etal.2007}. Here, $\Delta^2$ designs the biharmonic (Laplacian
square) operator of
linear elastic theory, while $|\grad|$ is a pseudo-differential operator
defined through the Fourier transform by \citet[Eq.~6]{Bueler.etal.2007}. In
the left-hand part of this equation, the first term accounts for mantle
relaxation, the second for point-wise isostasy, and the third for elastic
flexure. Due to high mantle viscosity, $\nu_m$, there exists a time lag between
the ice sheet growth and the isostatic bedrock response.
%$\nu=1\times10^{21}$\,Pa\,s
%$D=5\times10^{24}$\,N\,m
%$\rho_r = 3300$\,kg\,m$^{-3}$

% ----------------------------------------------------------------------
\subsection{Surface mass balance}
% ----------------------------------------------------------------------

\newcommand{\PDD}[0]{\mathrm{PDD}}
\newcommand{\sPDD}[0]{\sigma_{\mathrm{PDD}}}

At the interface between the ice sheet and the atmosphere (Fig.~??),
surfice mass fluxes are computed from monthly mean surface air temperature,
$T_m$, and monthly precipitation, $P_m$, by a temperature-index model
\citep[e.g.,][]{Hock.2003}. Surface accumulation, $a_s$, equals precipitation
when temperature is below ${T_0=0\,\degC}$, and decreases to zero linearly
with temperature between $T_0$ and ${T_1=2\,\degC}$,
\begin{equation}
    a_s =
    \begin{cases}
        0       & \text{if}\ T_m \le T_0 \,, \\
        P_m \frac{T_m-T_0}{T_1-T_0}
                & \text{if}\ T_0 < b < T_1 \,, \\
        P_m     & \text{if}\ T_1 \le T_m \,,
    \end{cases}
\end{equation}

Surface melt, $m_s$, is assumed proportional to the number of positive degree
days (PDD), defined over an arbitrary time interval, $[t_1, t_2]$, as the
integral of positive Celcius temperature $T-T_0$,
\begin{equation}
    \mathrm{PDD} = \int_{0}^{A}\max(T(t)-T_0,0)\,dt \,.
\end{equation}

For multimillenial simulations of palaeo-ice sheet evolution such as the ones
presented in this thesis, daily or hourly temperature data is not available,
forcing the PDD computation to rely on an idealised representation of the
annual temperature cycle $T_{ac}$. Sub-annual temperature variability around
the freezing point, however, significantly affects surface melt on a multi-year
scale \citep{Arnold.Mackay.1964}. It is then typically included in the models
under an assumption of normal temperature distribution, using a standard
deviation parameter $\sPDD$ in the PDD computation \citep{Braithwaite.1984},
PDDs can then be computed using a double-integral formulation
\citep{Reeh.1991},
\begin{equation}
    \PDD = \frac{1}{\sPDD\sqrt{2\pi}}
        \int_{t_1}^{t_2} \mathrm{d}t
        \int_{0}^{\infty} \mathrm{d}T \,
        T \exp\left({-\frac{(T-T_{ac})^2}{2\sPDD^2}}\right) \,,
\end{equation}
or more efficiently using an error function formulation
\citep{Calov.Greve.2005},
\begin{equation}
    \label{eqn:calovgreve}
    \PDD = \int_{t_1}^{t_2} \mathrm{d}t
        \left[\frac{\sPDD}{\sqrt{2\pi}}
                \exp\left({-\frac{T_{ac}^2}{2\sPDD^2}}\right)
              + \frac{T_{ac}}{2} \, \mathrm{erfc}
                \left(-\frac{T_{ac}}{\sqrt{2}\sPDD}\right)\right] \,.
\end{equation}

In the simulations presented here, \cref{eqn:calovgreve} is numerically
approximated using week-long sub-intervals. For each sub-interval, surface melt
is computed from PDDs and the available snow and ice depths using an algorithm
documented in the on-line documentation of PISM \citep{PISM-authors.2014}. This
algorythms employ different melt factors for snow ($F_s$) and for ice ($F_i$),
as derived from mass-balance measurements on contemporary glaciers in the
Coast and Rocky Mountains of British Columbia \citep{Shea.etal.2009}. In
Paper~I, the temperature standard deviation $\sPDD$ is a constant model
parameter. In Paper~IV, it is computed from daily temperature values from the
North American Regional Reanalysis \citep[NARR,][]{Mesinger.etal.2006}, using
a method initially developed in Paper~II, and further improved in Paper~III.
%3.04\,\unit{mm\,{\degree}C^{-1}\,day^{-1}} for snow and
%4.59\,\unit{mm\,{\degree}C^{-1}\,day^{-1}} for ice

% ----------------------------------------------------------------------
\subsection{Atmospheric forcing}
% ----------------------------------------------------------------------

Atmospheric forcing of the model consists of a present-day monthly climatology,
$\{T_{m0}, P_{m0}\}$, modified by a lapse-rate correction, ${\Delta}T_{LR}$,
and a temperature-offset correction, ${\Delta}T_{TS}$,
\begin{subequations}
\begin{align}
    T_m(t, x, y) &= T_{m0}(x, y) + {\Delta}T_{LR}(t)
                                 + {\Delta}T_{TS}(t, x, y) \,, \\
    P_m(t, x, y) &= P_{m0}(x, y) \,.
\end{align}
\end{subequations}

The present-day climatology, $\{T_{m0}, P_{m0}\}$, is computed from
near-surface air temperature and precipitation rate fields extracted from
observational data and atmospheric reanalyses.
The lapse-rate correction, ${\Delta}T_{LR}$, is computed in relation to the
climate input reference bedrock topography, $b_{ref}$,
\begin{align}
    {\Delta}T_{LR}(t, x, y) &= -\gamma [s(t, x, y)-b_{ref}(x, y)] \\
                            &= -\gamma [h(t, x, y)+b(t, x, y)-b_{ref}(x, y)]\,,
\end{align}
thus accounting for the evolution of ice thickness, $h$, on the one hand, and
for differences between the the ice flow model basal topography, $b$, and the
climate input topography, $b_{ref}$, on the other hand. All simulations use an
annual temperature lapse rate of $\gamma = 6\,\unit{\degC\,km^{-1}}$.

Paper~I tests the sensitivity of the model to different input present-day
climatologies, $\{T_{m0}, P_{m0}\}$, while the temperature-offset corrections,
${\Delta}T_{TS}$, uses constant values ranging from -10 to 0\,\degC. Paper~IV,
on the opposite, tests model sensitivity to time-dependent temperature-offset
corrections derived from different palaeo-temperature proxy records.

% ----------------------------------------------------------------------
\subsection{Numerical implementation}
% ----------------------------------------------------------------------




% ======================================================================
\section{Software tools and contributions}
% ======================================================================
% Computational workflow
% Contributions to PISM
% PyPDD
% pismplotlib
% r.in.worldclim
% r.out.pism

% Fig. flow chart
% Fig. pypdd example
% Fig. pismplotlib example?

% ======================================================================
\section{Results summary}
% ======================================================================
% Constant-climate runs
% Effect of daily temperature variability
% Transient-climate runs

% Fig. 3km LGM quiver
% Fig. sigma distribution map
% Fig. sigma effect on Cordillera runs
% Fig. deglaciation streamplot
% Fig. cumulative ice displacement map

% ======================================================================
\section{Future perpectives}
% ======================================================================
% Model improvements
% Fieldwork
% Geomorphological-oriented plots

% References
\newcommand{\urlprefix}[0]{}  % remove default "URL" prefix
\bibliographystyle{copernicus}
\bibliography{refs/references.bib}

% ----------------------------------------------------------------------
\end{document}
% ----------------------------------------------------------------------
