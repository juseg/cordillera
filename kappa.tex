% kappa.tex
% ----------------------------------------------------------------------

% Base class and packages
\documentclass{article}
\usepackage{amsmath}
\usepackage{bm}
\usepackage{lineno}
\usepackage[onehalfspacing]{setspace}
\usepackage{physics}
%\usepackage[colorlinks]{hyperref}
%\usepackage{natbib}
%\usepackage{graphicx}

% My commands
%\newcommand{\note}[1]{\textbf{[NOTE: #1]}}
%\newcommand{\todo}[1]{\emph{[\textbf{Todo:} #1]}}
%\newcommand{\aref}[0]{\textbf{[ref.]}}
\renewcommand{\grad}[1]{\vec{\textup{grad}}\,#1}
\renewcommand{\div}[1]{\textup{div}\,#1}
\newcommand{\tensdiv}[1]{\vec{\textup{div}}\,#1}

% Document properties
\title{Numerical simulation of the Cordilleran ice sheet \\
       through the last glacial cycle}
\author{Julien Seguinot}

% ----------------------------------------------------------------------
\begin{document}
% ----------------------------------------------------------------------

\maketitle
\linenumbers

% ----------------------------------------------------------------------
\section{Introduction}
\label{sec:intro}
% ----------------------------------------------------------------------
% Ice as viscous matter
% Glaciers and ice sheets
% The last glacial cycle

% Fig. crevasses
% Fig. glacier flowing
% Fig. recent climate evolution
% Fig. present and palaeo-ice sheets

% ----------------------------------------------------------------------
\section{Field area description}
% ----------------------------------------------------------------------
% Geographic setting
% Climatic setting
% Palaeo-glaciology

% Fig. map of North America
% Fig. map of the Cordillera?
% Fig. photos of the Cordillera
% Fig. photos of fieldwork
% Fig. Johan lineation map?

% ----------------------------------------------------------------------
\section{Numerical ice flow model}
% ----------------------------------------------------------------------
\subsection{Overview}
\subsection{Field equations}

Incompressibility of flow (conservation of volume)

\begin{equation}
    \div{\vec{v}} = 0
\end{equation}

Balance of stresses (Stokes equation)

\begin{equation}
    \label{eq:stresses}
    \tensdiv{\bm\sigma} + \rho\,\vec{g} = \vec{0}
\end{equation}

Constitutive law for ice (Glen's law)

\begin{equation}
    \bm{\dot\epsilon} = A_0\,e^\frac{-Q}{RT*}\,\tau_e^{n-1}\,\bm{\tau}
\end{equation}

Thermal equation (conservation of energy)

\begin{equation}
    \frac{\partial T}{\partial t}
        + \vec{v} \cdot \vec{\mathrm{grad}}\,T
        = \frac{k}{\rho c} \Delta T
        + \frac{4\mu \dot{\epsilon_e}^2}{\rho c}
\end{equation}

\subsection{Shallow approximations}

Complete stress balance

\begin{equation}
    \left\{\begin{aligned}
        \pdv{\tau_{xx}}{x} + \pdv{\tau_{xy}}{y} + \pdv{\tau_{xz}}{z}
            &= \pdv{p}{x} \\
        \pdv{\tau_{yx}}{x} + \pdv{\tau_{yy}}{y} + \pdv{\tau_{yz}}{z}
            &= \pdv{p}{y} \\
        \pdv{\tau_{zx}}{x} + \pdv{\tau_{zy}}{y} + \pdv{\tau_{zz}}{z}
            &= \pdv{p}{z} - \rho g
    \end{aligned}\right.
\end{equation}

Shallow ice approximation

\begin{equation}
    \left\{\begin{aligned}
        \pdv{\tau_{xz}}{z} &= \pdv{p}{x} \\
        \pdv{\tau_{yz}}{z} &= \pdv{p}{y} \\
        0 &= \pdv{p}{z} - \rho g
    \end{aligned}\right.
\end{equation}


\begin{equation}
    \pdv{\vec{v}}{z} = 2A\,(\rho g)^n\,(s-z)^n\,|\grad{s}|^{n-1}\,\grad{s}
\end{equation}

Shallow shelf approximation

\begin{equation}
    \left\{\begin{aligned}
        \pdv{\tau_{xx}}{x} + \pdv{\tau_{xy}}{y} + \pdv{\tau_{xz}}{z}
            &= \pdv{p}{x} \\
        \pdv{\tau_{yx}}{x} + \pdv{\tau_{yy}}{y} + \pdv{\tau_{yz}}{z}
            &= \pdv{p}{y} \\
        \pdv{\tau_{zz}}{z} &= \pdv{p}{z} - \rho g
    \end{aligned}\right.
\end{equation}


\begin{equation}
    \left\{\begin{aligned}
        \pdv{x} \left[2\bar{\nu}H
                      \left(2\pdv{v_x}{x} + \pdv{v_y}{y}\right) \right]
            + \pdv{y} \left[\bar{\nu}H
                            \left(\pdv{v_x}{y} + \pdv{v_y}{x}\right) \right]
            + \tau_{bx} &= \rho_i gH \pdv{h}{x} \\
        \pdv{x} \left[\bar{\nu}H
                      \left(\pdv{v_x}{y} + \pdv{v_y}{x}\right) \right]
            + \pdv{y} \left[2\bar{\nu}H
                            \left(\pdv{v_x}{x} + 2\pdv{v_y}{y}\right) \right]
            + \tau_{bx} &= \rho_i gH \pdv{h}{x} \\
    \end{aligned}\right.
\end{equation}

\begin{equation}
    \bar{\nu} = \frac{\bar{B}}{2}\,\dot{\epsilon}^{\frac{1-n}{n}}
              = \frac{1}{2 \bar{A}^{\frac{1}{n}}}\,
                \dot{\epsilon}^{\frac{1-n}{n}}
\end{equation}

Basal sliding

\begin{equation}
    \bm{\tau}_b = -\tau_c \frac{\vec{v}_b}
                               {{u_{th}}^q\,|\vec{v}_b|^{1-q}}
\end{equation}

\begin{equation}
   \tau_c = c_{0} + \tan{\phi}\cdot N_{til}
\end{equation}

\newcommand{\phimin}{\phi_{\mathrm{min}}}
\newcommand{\phimax}{\phi_{\mathrm{max}}}
\newcommand{\bmin}{b_{\mathrm{min}}}
\newcommand{\bmax}{b_{\mathrm{max}}}

\begin{equation}
  \phi(x,y) =
  \begin{cases}
    \phimin, & b(x,y) \le \bmin, \\
    \phimin + (b(x,y) - \bmin) \,M, & \bmin < b(x,y) < \bmax, \\
    \phimax, & \bmax \le b(x,y).
  \end{cases}
  \label{eq:phipiecewise}
\end{equation}

\begin{equation}
N_{til} = \delta P_o \, 10^{(e_0/C_c)
          \left(1 - (W_{til}/W_{til}^{max})\right)}
\end{equation}

\subsection{Bedrock response}
\subsection{Surface mass balance}
\subsection{Atmospheric forcing}
\subsection{Numerical implementation}


% Fig. PISM = ice + bedrock

% ----------------------------------------------------------------------
\section{Software tools and contributions}
% ----------------------------------------------------------------------
% Computational workflow
% Contributions to PISM
% PyPDD
% pismplotlib
% r.in.worldclim
% r.out.pism

% Fig. flow chart
% Fig. pypdd example
% Fig. pismplotlib example?

% ----------------------------------------------------------------------
\section{Results summary}
% ----------------------------------------------------------------------
% Constant-climate runs
% Effect of daily temperature variability
% Transient-climate runs

% Fig. 3km LGM quiver
% Fig. sigma distribution map
% Fig. sigma effect on Cordillera runs
% Fig. deglaciation streamplot
% Fig. cumulative ice displacement map

% ----------------------------------------------------------------------
\section{Future perpectives}
% ----------------------------------------------------------------------
% Model improvements
% Fieldwork
% Geomorphological-oriented plots

% ----------------------------------------------------------------------
\end{document}
% ----------------------------------------------------------------------
