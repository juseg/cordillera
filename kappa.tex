% kappa.tex
% ----------------------------------------------------------------------

% Base class and packages
\documentclass{article}
\usepackage{lineno}
\usepackage[onehalfspacing]{setspace}
\usepackage{physics}
%\usepackage[colorlinks]{hyperref}
%\usepackage{natbib}
%\usepackage{graphicx}

% My commands
%\newcommand{\note}[1]{\textbf{[NOTE: #1]}}
%\newcommand{\todo}[1]{\emph{[\textbf{Todo:} #1]}}
%\newcommand{\aref}[0]{\textbf{[ref.]}}

% Document properties
\title{Numerical simulation of the Cordilleran ice sheet \\
       through the last glacial cycle}
\author{Julien Seguinot}

% ----------------------------------------------------------------------
\begin{document}
% ----------------------------------------------------------------------

\maketitle
\linenumbers

% ----------------------------------------------------------------------
\section{Introduction}
\label{sec:intro}
% ----------------------------------------------------------------------
% Ice as viscous matter
% Glaciers and ice sheets
% The last glacial cycle

% Fig. crevasses
% Fig. glacier flowing
% Fig. recent climate evolution
% Fig. present and palaeo-ice sheets

% ----------------------------------------------------------------------
\section{Field area description}
% ----------------------------------------------------------------------
% Geographic setting
% Climatic setting
% Palaeo-glaciology

% Fig. map of North America
% Fig. map of the Cordillera?
% Fig. photos of the Cordillera
% Fig. photos of fieldwork
% Fig. Johan lineation map?

% ----------------------------------------------------------------------
\section{Numerical ice flow model}
% ----------------------------------------------------------------------
% Overview
% Field equations
% Bedrock response
% Surface mass balance
% Atmospheric forcing
% Numerical implementation

% Fig. PISM = ice + bedrock

% ----------------------------------------------------------------------
\section{Software tools and contributions}
% ----------------------------------------------------------------------
% Computational workflow
% Contributions to PISM
% PyPDD
% pismplotlib
% r.in.worldclim
% r.out.pism

% Fig. flow chart
% Fig. pypdd example
% Fig. pismplotlib example?

% ----------------------------------------------------------------------
\section{Results summary}
% ----------------------------------------------------------------------
% Constant-climate runs
% Effect of daily temperature variability
% Transient-climate runs

% Fig. 3km LGM quiver
% Fig. sigma distribution map
% Fig. sigma effect on Cordillera runs
% Fig. deglaciation streamplot
% Fig. cumulative ice displacement map

% ----------------------------------------------------------------------
\section{Future perpectives}
% ----------------------------------------------------------------------
% Model improvements
% Fieldwork
% Geomorphological-oriented plots

% ----------------------------------------------------------------------
\end{document}
% ----------------------------------------------------------------------
