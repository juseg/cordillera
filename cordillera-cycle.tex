% cordillera-climate.tex
% ----------------------------------------------------------------------

% Copernicus manuscript
\documentclass[tc, ms]{copernicus}

% Copernicus final print
%\documentclass[tc]{copernicus}

% Copernicus discussion paper
%\documentclass[tcd, hvmath]{copernicus_discussions}

% Copernicus-like latex2rtf compatible
%% copernicus_rtf.tex
% ------------------

% Base class and packages
\documentclass{article}
\usepackage{color}
\usepackage{geometry}
\usepackage{graphicx}
\usepackage{setspace}
\onehalfspacing

% Replacements for bibtex commands
\newcommand{\citep}[1]{(\textcolor{blue}{#1})}
\newcommand{\citet}[1]{\textcolor{blue}{#1}}

% Replacements for Copernicus commands
\newcommand{\introduction}[0]{\section{Introduction}}
\newcommand{\conclusions}[0]{\section{Conclusions}}
\newcommand{\tophline}[0]{\hline}
\newcommand{\middlehline}[0]{\hline}
\newcommand{\bottomhline}[0]{\hline}
\newcommand{\unit}[1]{\ensuremath{\mathrm{#1}}}
\newcommand{\degree}[0]{\ensuremath{^{\circ}}}

% Ignore other Copernicus commands
\newcommand{\runningtitle}[1]{}
\newcommand{\runningauthor}[1]{}
\newcommand{\received}[1]{}
\newcommand{\correspondence}[1]{}
\newcommand{\pubdiscuss}[1]{}
\newcommand{\revised}[1]{}
\newcommand{\accepted}[1]{}
\newcommand{\published}[1]{}



% Coloured hyperlinks
\usepackage[colorlinks]{hyperref}

% Encoding
\usepackage[T1]{fontenc}

% Figure directory
\graphicspath{{figures/}}

% My commands
\newcommand{\idea}[1]{\textbf{[IDEA: #1]}}
\newcommand{\note}[1]{\textbf{[NOTE: #1]}}
\newcommand{\todo}[1]{\textbf{[TODO: #1]}}
\newcommand{\aref}[0]{\textbf{[ref.]}}
\renewcommand{\citep}[1]{\aref}
\renewcommand{\citet}[1]{\aref}

% ----------------------------------------------------------------------
\begin{document}\hack{\sloppy}
% ----------------------------------------------------------------------

% Title
\title{Numerical simulation of the Cordilleran ice sheet
       through the last glacial cycle}

% Authors
\author[1,2]{J.~Seguinot}
\author[3]{M.~Margold}
\author[2]{I.~Rogozhina}
\author[1]{A.~Stroeven}
\runningauthor{J.~Seguinot et~al.}
\correspondence{J.~Seguinot (julien.seguinot@natgeo.su.se)}

% Running title
\runningtitle{Climate forcing for Cordilleran ice sheet simulations}

% Affiliations
\affil[1]{Department of Physical Geography and Quaternary Geology and the
          Bolin Centre for Climate Research, Stockholm University,
          Stockholm, Sweden}
\affil[2]{Helmholtz Centre Potsdam, GFZ German Research Centre for Geosciences,
          Potsdam, Germany}
\affil[3]{Department of Geography, Durham University, UK}

% For Copernicus
\received{}
\accepted{}
\published{}

% Title
\firstpage{1}
\maketitle

% Abstract
\begin{abstract}

  Despite more than a century of geological observations, the Cordilleran ice
  sheet of North America remains poorly understood in terms of its former
  extent, volume and dynamics. Although geomorphological evidence is abundant,
  its complexity is such that whole ice-sheet reconstructions of advance and
  retreat patterns are lacking. Here we use a numerical ice sheet model
  calibrated against field-based evidence to attempt a quantitative
  reconstruction of the Cordilleran ice sheet history through the last glacial
  cycle. A series of simulations is driven by time-dependent temperature
  offsets from six proxy records located around the globe. Although this
  approach reveals large variations in model response to evolving atmospheric
  forcing, all simulations produce two major glaciation events during MIS~4
  (61.9--55.4\,\unit{kyr}) and MIS~2 (29.5--16.9\,\unit{kyr}). The timing of
  glaciation is
  better reproduced using temperature reconstructions from Greenland and
  Antarctic ice cores than from regional ocean sediment cores. During most of
  the last glacial cycle, the modelled ice cover is discontinuous and
  restricted to high mountain areas. However, widespread precipitation over the
  Skeena mountains favours the persistence of a central ice dome throughout the
  glacial cycle. It acts as a nucleation centre before the last glacial maximum
  and hosts the last remains of Cordilleran ice during the
  early Holocene (10.9--9.5\,\unit{kyr}).

\end{abstract}

\note{Please read through the abstract, introduction and figure captions. All
      other sections are very preliminary at this stage. I need some feedback
      on the (currently embryonic) discussion part.}

% ----------------------------------------------------------------------
\introduction
\label{sec:intro}
% ----------------------------------------------------------------------

During the last glacial cycle, glaciers and ice caps of the North American
Cordillera were more extensive than today. During the last glacial maximum, a
contiguous blanket of ice extended from Denali Peak in the north to the Puget
Sound in the south \aref. This ice mass is known as the former Cordilleran ice
sheet (Fig.~\ref{fig:locmap}).

For more than a century, exploration and geological investigation of the
Cordillera have led to a large collection of evidence of the former ice cover \aref. This
evidence consists of mapped boundaries of the former ice extent \aref, directions of
past sliding velocities \aref and locations of former melt-water streams \aref, as well as
the timing of glacier retreat identified from radiocarbon and cosmogenic dating \aref.

Field-based evidence allowed to reconstruct, with a high degree of confidence,
the maximal extent attained by the Cordilleran ice sheet during the last glacial
cycle \aref. However, former ice thickness and the ice sheet's contribution to the
last glacial maximum sea-level low-stand remain poorly constrained. Moreover,
our understanding of the glaciation history is mainly restricted to the
deglaciation phase and on a regional scale, while little is known about the ice
sheet evolution prior to the last glacial maximum extent \aref. Although
time-evolving, whole ice-sheet reconstructions of glacial advance and retreat
patterns are available for the Laurentide and Eurasian ice sheets \aref, such is not
the case for the Cordilleran ice sheet, where complex arrangements of ice flow
directions emanating from multiple glaciation centres have been identified \aref.

The present study uses a numerical ice sheet model (PISM, \aref) calibrated against
field-based evidence to perform a quantitative reconstruction of the
Cordilleran ice sheet history through the last glacial cycle. Although
numerical modelling has been established as a useful tool to improve our
understanding
of the Cordilleran ice sheet more than twenty years ago \aref, the ubiquitously
mountainous topography of the region has presented a major challenge to its
application, which has only been overcome by recent development of numerical ice
sheet models and underlying scientific computing tools \aref. In addition to the topographic
complexity, strong climatic contrasts characteristic of the North American
Cordillera require the use of a high-resolution temperature and precipitation
forcing as input data to the ice sheet model \aref.

Because past climate conditions are subject to considerable uncertainty, our
palaeoclimate forcing is a simplistic approximation of time-evolved temperature
and precipitation fields derived from a combination of a present-day
atmospheric reanalysis (NARR, \aref), lapse-rate corrections and temperature offset
time series. The latter are
obtained by scaling six different palaeo-temperature reconstructions from
proxy records around the globe, including two \chem{\delta^{18}O} records from
Greenland ice cores \aref, two \chem{\delta^{18}O} records from Antarctic ice cores \aref,
and two alkenone unsaturation index records from Northwest Pacific oceanic
sediment cores \aref. Model output is compared to geomorphological evidence in terms
of timing and extent of glaciation and patterns of deglaciation.


% ----------------------------------------------------------------------
\section{Model setup}
\label{sec:model}
% ----------------------------------------------------------------------

\note{Currently, this section contains parts from the previous paper.}

The simulations presented here were run using Parallel Ice Sheet Model (PISM,
development version~11b0a7f), an open-source, finite-difference, shallow ice
sheet model \aref. The model takes basal topography, sea level, geothermal heat flux
and climate forcing, as input fields and computes the evolution of ice extent
and thickness in time, its thermal and dynamic state, and
the associated lithospheric response. Our modelling domain encompasses the
entire area covered by the Cordilleran ice sheet at the last glacial maximum
(Fig.~\ref{fig:locmap}).

To reconstruct the successive phases of growth and decay of the last Cordilleran
ice sheet, palaeo-climatic conditions of the last glacial cycle are mimicked
by applying time-dependent temperature offsets derived from multiple
palaeo-temperature proxy records. Each simulation starts from assumed ice-free
conditions at -120\,kyr, and runs to present. These computations were
performed on 16 to 128 cores at the Swedish National Supercomputing
Center.

\subsection{Ice thermodynamics}

PISM is a~shallow model, which implies that the balance of stresses is
approximated based on their predominant components.
The Shallow Shelf Approximation (SSA) is used as a ``sliding law'' for the
Shallow Ice Approximation (SIA) \citep{bueler-brown-2009,winkelmann-etal-2011}.
SIA and SSA velocities are computed by finite difference methods on a~10\,km
resolution horizontal grid of 300 by 150 points (the modelling domain). Ice
softness depends on temperature and water content through an enthalpy
formulation \citep{aschwanden-blatter-2009,aschwanden-etal-2012}. Enthalpy is
computed in three dimensions in up to 51 layers within the
ice, and temperature is additionally computed in 31 layers in bedrock to
a~depth of 3\,km. Surface air temperature from the atmospheric forcing
provides the upper boundary condition to the ice enthalpy model, and
a~geothermal heat flux of 70\,\unit{mW\,m^{-2}} provides the lower boundary
condition to the bedrock thermal model. Although this uniform value does not
account for the high spatial geothermal variability in the region, it is on
average representative of available heat flow measurements
\citep{artemieva-mooney-2001,blackwell-richards-2004}.

A~pseudo-plastic sliding law \citep{aschwanden-etal-2013} relates the
bed-parallel shear stress and the sliding velocity. The yield stress is
modelled using the Mohr--Coulomb criterion. The till friction angle $\phi$
varies from 15 to 45{\degree}. It is taken as a~function of modern bed
elevation, with lowest values occurring at low elevation, thus accounting
for a~weakening of till associated with the presence of marine sediments
\citep{martin-etal-2011,aschwanden-etal-2013}. The till effective pressure is
related to the modelled amount of water in the till, using a formula derived
from laboratory experiments with till extracted from an Antarctic ice stream.
Basal topography is derived from the ETOPO1 combined topography
and bathymetry dataset with a~resolution of 1\,arc-min \citep{data:etopo1}.

\note{Do we need more details (equations) on this new sliding law?}

Sea level is lowered as a function of time according to the SPECMAP
reconstruction, and basal topography responds to ice load
following a bedrock deformation model that includes point-wise isostasy,
elastic lithosphere flexure and viscous mantle deformation in a~semi-infinite
half-space \citep{lingle-clark-1985,bueler-etal-2007}. It uses a lithosphere
density of 3300\,\unit{kg\,m^{-3}}, a flexural rigidity of $5 \times
10^{24}$\,\unit{N\,m} and a mantle viscosity of $1 \times
10^{21}$\,\unit{Pa\,s}. Due to the high mantle viscosity, there exists a time
lag between ice sheet growth and isostatic bedrock response.

\subsection{Surface mass-balance}

Ice surface accumulation and ablation are computed from monthly mean
near-surface air temperature, monthly precipitation, and monthly standard
deviation of near-surface temperature by a~temperature-index (positive
degree-day) model \citep{hock-2003}. Ice accumulation is equal to precipitation
when temperature is below 0\,\unit{{\degree}C}, and decreases to zero linearly
with temperature between 0 and 2\,\unit{{\degree}C}. Ice ablation is computed
from the number of positive degree-days, defined as the integral of
temperatures above 0\,\unit{{\degree}C} in one year.

The positive degree-day integral \citep{calov-greve-2005} is numerically
approximated using week-long sub-intervals. It accounts for temperature
variability assuming a~normal distribution along a~central (input) value. The
temperature standard deviation is part of the forcing climatology. It was
computed from daily temperature values from the North American Regional
Reanalysis \citep{data:narr}, excluding variability associated with the
seasonal cycle itself \citet{seguinot-rogozhina-2014}. The
ablation model incorporates degree-day factors of
3.04\,\unit{mm\,{\degree}C^{-1}\,day^{-1}} for snow and
4.59\,\unit{mm\,{\degree}C^{-1}\,day^{-1}} for ice, as derived from
mass-balance measurements on contemporary glaciers from the Coast Mountains and
Rocky Mountains in British Columbia \citep{shea-etal-2009}.

\subsection{Atmospheric forcing}

Atmospheric forcing of the model consists of a present-day monthly climatology
modified by time-dependent temperature offsets and a lapse-rate correction.
\begin{align}
    T_m(t, x, y) &= T_{m0}(x, y) + {\Delta}T_{TS}(t) + {\Delta}T_{LR}(t, x, y) \\
    P_m(t, x, y) &= P_{m0}(x, y)
\end{align}

The present-day climatology $\{T_m, P_m\}$ was computed from the near-surface air temperature and
precipitation rate fields of the North American Regional Reanalysis (NARR) over
the period 1979--2000 (Fig.~\ref{fig:atm}). Modern climate of the North
American Cordillera is characterised by strong geographic variations of
seasonality, timing of the precipitation peak and daily temperature variability.
Our choice of data from the NARR is motivated by the need for an accurate,
high-resolution precipitation forcing, as outlined in a previous sensitivity
study.

Temperature offset time-series ${\Delta}T_{TS}$ are derived from proxy records from
the Greenland Ice Core Project (GRIP), the North Greenland Ice Core Project
(NGRIP), the European Project for Ice Coring in Antarctica (EPICA), the Lake
Vostok ice core, and Ocean Drilling Program (ODP) sites 1012 and 1020, both
located off the Californian shore. Palaeo-temperatures from the GRIP and NGRIP
records were calculated using a quadratic equation \aref:
\begin{equation}
    {\Delta}T_{TS}(t) = -11.88[\chem{\delta^{18}O(t)}-\chem{\delta^{18}O}(0)]
                        -0.1925[\chem{\delta^{18}O(t)}^2-\chem{\delta^{18}O}(0)^2]
\end{equation}
while the temperature reconstructions from the Antarctic and oceanic cores were
provided by their authors \aref. All records were scaled linearly in
order to simulate comparable ice volumes and realistic ice extents at the last
glacial maximum (Table~\ref{tab:records}, Fig.~\ref{fig:timeseries}).

Prior to surface mass balance computation, the model dynamically applies
a~lapse-rate correction to surface air temperature,
\begin{align}
    {\Delta}T_{LR}(t, x, y) &= -\gamma [z_{s}(t, x, y)-z_{ref}] \\
                            &= -\gamma [h(t, x, y)+z_{b}(t, x, y)-z_{ref}]
\end{align}


This correction accounts
for the evolution of ice thickness on the one hand, and differences between the
climate forcing reference topography and the ice flow model basal topography on
the other hand. An annual lapse rate of
$\gamma = 6\,\unit{{\degree}C\,km^{-1}}$
is used in all simulations.


% ----------------------------------------------------------------------
\section{Results}
\label{sec:results}
% ----------------------------------------------------------------------

Despite large differences in the input temperature offset time-series, the
model output presents consistent features that can be observed across the range
of forcing data used. In all simulations, ice cover is relatively restricted
during most of the glacial cycle, while two major glaciation events occur
during MIS~4 (61.9--55.4\,\unit{kyr}) and MIS~2 (29.5--16.9\,\unit{kyr},
Fig.~\ref{fig:timeseries}). A local ice minimum is consistently attained in between these
two events during MIS~3 (52.2--45.6\,\unit{kyr}).

However, the magitude and timing of these events vary significantly
between our simulations. For instance, the use of GRIP and NGRIP forcing
results in highest variability in modelled ice volume. Simulations driven by
the Antarctic and GRIP records reach a last glacial maximum in ice volume at
19.5--16.9\,\unit{kyr}, while simulations driven by oceanic records
and the NGRIP record attain it prior to 20.0\,\unit{kyr}.

Despite the different timing,
snapshots of model output show relatively consistent patterns of glaciation
(Fig.~\ref{fig:snapshots}). In all simulations, a central ice cap persists over
the Skeena mountains between the two glaciation events. However, the size of
this ice cap, as well as the magnitude of the MIS~4, depends sensitively on the temperature
offset time-series used to drive the model.

The modelled last glacial maximum ice sheet possess a central divide located at
about 3\,500\,\unit{m} elevation along the spine of the Rocky Mountains. The
distribution of surface velocities shows a clear asymmetry between the
maritime, western flank, and the continental, eastern flank of the ice sheet
(Fig.~\ref{fig:lgm}).

% ----------------------------------------------------------------------
\section{Discussion}
\label{sec:discussion}
% ----------------------------------------------------------------------

\subsection{Ice-sheet scale glaciation dynamics}

During the last glacial cycle, geomorphological evidence has allowed to
identify two major advances of the Cordilleran ice sheet. These correspond to
the last glacial maximum \aref, and the older xxx glaciation, found in the
northernmost parts of the ice sheet \aref. It should be noted that all
simulations presented here reproduce these two major events, which constitutes
a first-order agreement between model results and geomorphological evidence.

However, the timing of these events depends highly on the choice of temperature
time-series used to force the model. For instance, the timing of the last
glacial maximum is best reproduced by the Antarctic records, while simulations
driven by the NGRIP and oceanic records pre-date the last glacial maximum by
several thousands of years. Regarding the oceanic records, the early volume
maxima can be explain by an early warming in the palaeo-temperature record.
This early warming has been previously documented, and explained as a
weakening of the California current, precisely associated with the presence of
the Cordilleran ice sheet \aref. This precocious glacial maximum illustrates
the complexity of climate-ice-sheet feedbacks on a regional scale, and shows
that, although closely located to the modelling domain, the ODP~1012 and
ODP~1020 proxy records records can't be reliably used as a temperature forcing
to model the last Cordilleran ice sheet.

During most of the glacial cycle, ice cover is restricted to high mountain
areas. This is particularly true when using the Greenland-based forcing
(Fig~\ref{fig:duration}). Most of the glacial maxima margin has a transient
character, showing that the ice sheet is far out of balance with climate. There
persists a central ice dome over the Skeena mountains over the entire glacial
period. This ice dome serves as a nucleation centre at the initiation of the
advance phase preceding the last glacial maximum. It is here because of more
widespread precipitation over the Skeena mountains than across other East-West
transects towards south or north of this region. We interpret this as an
orographic precipitation effect resulting from a "window" through the sharp
range of the Coast Mountains at this latitude (Fig?).

\subsection{Deglaciation patterns}

Deglaciation happens first in the southern half and then in the northern half
of the ice-covered area (Fig.~\ref{fig:deglac}). Retreat patterns are different
than reconstructed over central BC, however they match Martin's reconstruction
of Liard lowland "unzipping" pretty well. In the GRIP-driven simulation, the
patterns of de-glaciation and Younger Dryas re-advance match with recent
cosmogenic exposure dates.

% ----------------------------------------------------------------------
\conclusions
\label{sec:concl}
% ----------------------------------------------------------------------

Numerical simulations of the Cordilleran ice sheet through the last glacial
cycle presented in this study consistently produce two glaciation maxima during
MIS~4 (61.9--55.4\,\unit{kyr}) and MIS~2 (29.5--16.9\,\unit{kyr}), independently
of the choice of palaeo-temperature record used to force the model. This result
can be seen as a first-order agreement between the model and geomorphological
evidence, which has documented palaeo-glaciation corresponding to these two
stages. However, the timing of glaciation peaks highly depends on which record
is used to drive the model. The timing of the last glacial maximum is best
reproduced by the Antarctic record, and occurs too early in all simulations
that were driven by other records. The mismatch is greatest when forcing the
model with oceanic records from the Pacific Northwest, which are affected by a
weakening of the California current during the last glacial maximum.

In all simulations presented here, ice cover is limited to disjoint mountain ice
caps during most of the glacial cycle. This agrees with earlier interpretations
of the geomorphological evidence preceding last glacial maximum. However, our
simulations produce a persistent ice cover over the Skeena mountains during
the entire glacial cycle. During time periods when the Cordilleran ice sheet did
not exist, the Skeena ice cap appear to be fed by an eastwards precipitation
intrusion through a topographic window in the Coast Mountains. The ice cap act
as a nucleation centre at the onset of the last glacial maximum re-advance, and
may have contributed to the spectacular glacial erosional imprint of the Skeena
Mountains landscape.

Concerning the deglaciation phase, when most geological evidence is available,
none of the simulations presented here produce an ideal picture. The timing of
timing of the last glacial maximum and early deglaciation is generally best
reproduced by forcing the model with a palaeo-temperature record from
Antarctica. However, a Younger Dryas re-advance is best reproduced at places
where it has been documented using temperature forcing from the GRIP record.
Nonetheless, the patterns of deglaciation are consistent between different
simulations. They show a rapid deglaciation of the southern half of the ice
sheet, followed by unzipping of the Liard Lowland, and centripetal retract of
the ice margin towards the last palaeo-ice cover in the Skeena Mountains
during the early Holocene (10.9--9.5\,\unit{kyr}).

One must keep in mind, however, that these results are only accurate for our
choices of ice-sheet model (PISM), surface mass balance model (PDD), study
object (the Cordilleran ice sheet). Most importantly, our simplistic
palaeo-climate forcing does not include precipitation corrections in response
to the presence of the ice sheet, likely leading to overestimates of glacial
extent and volume. Nevertheless, our results identify the largely understudied
Skeena Mountains as a key area to understanding glacial dynamics of the
Cordilleran ice sheet, highlighting the need for further geological
investigation of this region.

% Acknowledgements
%\begin{acknowledgements}
  % Author contributions
  %\hack{\noindent}\textit{Author contributions.}
%\end{acknowledgements}

% References
%\input{references.bbl}
\newpage

% ----------------------------------------------------------------------
% Floats
% ----------------------------------------------------------------------

% tab:records
\begin{table*}[t]
  \caption{Palaeo-temperature proxy records and scaling parameters used to
           prepare temperature offset time-series used to force the ice sheet
           model through the last 120\,\unit{kyr}. $T_{[32;22]}$ refers to the
           mean temperature anomaly during the period -32 to~-22~\unit{kyr} after
           scaling.}
  \label{tab:records}
  {\begin{tabular}{lcccc}
    \tophline
    Record & Proxy & Scaling factor & $T_{[32;22]}$ & Source\\
    \middlehline
    GRIP     & \chem{\delta^{18}O} & ?\% & -5.8{\degree}C & \aref \\
    NGRIP    & \chem{\delta^{18}O} & ?\% & -6.1{\degree}C & \aref \\
    EPICA    & \chem{\delta^{18}O} & ?\% & -5.6{\degree}C & \aref \\
    Vostok   & \chem{\delta^{18}O} & ?\% & -5.6{\degree}C & \aref \\
    ODP~1012 & \chem{U^{K'}_{37}}  & ?\% & -5.8{\degree}C & \aref \\
    ODP~1020 & \chem{U^{K'}_{37}}  & ?\% & -5.8{\degree}C & \aref \\
    \bottomhline
  \end{tabular}}
  \belowtable{}
\end{table*}

% fig:locmap
\begin{figure}
  \includegraphics{locmap}
  \caption{Location map. In construction.}
  \label{fig:locmap}
\end{figure}

% fig:atm
\begin{figure}
  \includegraphics{atm}
  \caption{Monthly mean near-surface air temperature, precipitation and
           standard deviation of daily mean temperature for January and July
           months from the North American Regional Reanalysis (NARR)
           climatology, used as input fields to the ice sheet model. Note the
           strong contrasts in seasonality, timing of the precipitation peak,
           and temperature variability over the model domain, notably between
           the maritime and continental regions.}
  \label{fig:atm}
\end{figure}

% fig:timeseries
\begin{figure}
  \includegraphics{timeseries}
  \caption{\textbf{(top)} temperature offset time-series from ice and sediment
           core records used as palaeo-climate forcing for the ice sheet model
           (Table~\ref{tab:records}). \textbf{(bottom)} Modelled ice volume
           through the last 120\,\unit{kyr}, expressed in meters of sea-level
           equivalent. The grey spans indicate the ranges of modelled times of
           glacial extrema corresponding to MIS~4 (61.9--55.4\,\unit{kyr}), 3
           (52.2--45.6\,\unit{kyr}), and 2 (last glacial maximum,
           29.5--16.9\,\unit{kyr}).
           \note{The dotted line represents an ongoing 5km-resolution
           simulation, which may be eventually be used in
           Figs.~\ref{fig:duration} and \ref{fig:deglac}.}}
  \label{fig:timeseries}
\end{figure}

% fig:snapshots
\begin{figure}
  \includegraphics{snapshots}
  \caption{Snapshots of modelled ice surface topography (500\,\unit{m}
           contours) at modelled times of glacial
           extrema, corresponding to MIS~4, 3, and 2.}
  \label{fig:snapshots}
\end{figure}

% fig:lgm
\begin{figure}
  \includegraphics{lgm}
  \caption{Modelled ice surface topography (100\,\unit{m} contours) and
           velocity (\unit{m\,yr^{-1}}) corresponding to the last glacial
           volume maximum.}
  \label{fig:lgm}
\end{figure}

% fig:duration
\begin{figure}
  \includegraphics{duration}
  \caption{Modelled duration of ice cover during the last 120\,\unit{kyr}.
           Aside the location of present-day ice caps, note the persistence of
           a central ice dome over the Skeena mountains during most of the
           simulation. On the other hand, the maximal extent of the ice sheet
           corresponds to short durations of ice cover along most of the
           margin.
           \note{I am trying to improve the readability of low ice cover
           duration values on this figure.}}
  \label{fig:duration}
\end{figure}

% fig:deglac
\begin{figure}
  \includegraphics{deglac}
  \caption{Modelled age of the last deglaciation. Areas where the MIS~4 glacial
           advance exceeded the last glacial maximum advanced are marked in
           green. Hatches denote the Younger Dryas re-advance, which is most
           significant in the GRIP-driven simulation.}
  \label{fig:deglac}
\end{figure}

% ----------------------------------------------------------------------
\end{document}
\endinput
% ----------------------------------------------------------------------
