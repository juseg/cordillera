% cordillera-climate.tex
% ----------------------------------------------------------------------

% Copernicus manuscript
\documentclass[tc, ms]{copernicus}

% Copernicus final print
%\documentclass[tc]{copernicus}

% Copernicus discussion paper
%\documentclass[tcd, hvmath]{copernicus_discussions}

% Copernicus-like latex2rtf compatible
%% copernicus_rtf.tex
% ------------------

% Base class and packages
\documentclass{article}
\usepackage{color}
\usepackage{geometry}
\usepackage{graphicx}
\usepackage{setspace}
\onehalfspacing

% Replacements for bibtex commands
\newcommand{\citep}[1]{(\textcolor{blue}{#1})}
\newcommand{\citet}[1]{\textcolor{blue}{#1}}

% Replacements for Copernicus commands
\newcommand{\introduction}[0]{\section{Introduction}}
\newcommand{\conclusions}[0]{\section{Conclusions}}
\newcommand{\tophline}[0]{\hline}
\newcommand{\middlehline}[0]{\hline}
\newcommand{\bottomhline}[0]{\hline}
\newcommand{\unit}[1]{\ensuremath{\mathrm{#1}}}
\newcommand{\degree}[0]{\ensuremath{^{\circ}}}

% Ignore other Copernicus commands
\newcommand{\runningtitle}[1]{}
\newcommand{\runningauthor}[1]{}
\newcommand{\received}[1]{}
\newcommand{\correspondence}[1]{}
\newcommand{\pubdiscuss}[1]{}
\newcommand{\revised}[1]{}
\newcommand{\accepted}[1]{}
\newcommand{\published}[1]{}



% Coloured hyperlinks
\usepackage[colorlinks]{hyperref}

% Encoding
\usepackage[T1]{fontenc}

% Figure directory
\graphicspath{{figures/}}

% My commands
\def\idea#1{\textcolor[rgb]{0,0.5,0}{\textbf{[IDEA: #1]}}}
\def\todo#1{\textcolor[rgb]{0.5,0,0}{\textbf{[TODO: #1]}}}
\def\aref{\textcolor[rgb]{0.5,0,0}{\textbf{[REF]}}}

% ----------------------------------------------------------------------
\begin{document}\hack{\sloppy}
% ----------------------------------------------------------------------

% Title
\title{Numerical simulation of the Cordilleran ice sheet
       through the last glacial cycle}

% Authors
\author[1,2]{J.~Seguinot}
\runningauthor{J.~Seguinot et~al.}
\correspondence{J.~Seguinot (julien.seguinot@natgeo.su.se)}

% Running title
\runningtitle{Climate forcing for Cordilleran ice sheet simulations}

% Affiliations
\affil[1]{Department of Physical Geography and Quaternary Geology and the
          Bolin Centre for Climate Research, Stockholm University,
          Stockholm, Sweden}
\affil[2]{Helmholtz Centre Potsdam, GFZ German Research Centre for Geosciences,
          Potsdam, Germany}

% For Copernicus
\received{}
\accepted{}
\published{}

% Title
\firstpage{1}
\maketitle

% Abstract
\begin{abstract}

  Despite more than a century of geological observations, the Cordilleran ice
  sheet of North America remains poorly understood in terms of its former
  extent, volume and dynamics. Although geomorphological evidence is abundant,
  its complexity is such that whole ice-sheet reconstructions of advance and
  retreat patterns are lacking. Here we use a numerical ice sheet model
  calibrated against field-based evidence to attempt a quantitative
  reconstruction of the Cordilleran ice sheet history through the last glacial
  cycle. A series of simulations is driven by time-dependent temperature
  offsets from six proxy records located around the globe. Although this
  approach reveals large variations in model response to evolving atmospheric
  forcing, all simulations produce two major glaciations events during MIS~4
  (61.9--55.4\,\unit{kyr}) and MIS~2 (29.5--16.9\,\unit{kyr}). The timing of
  glaciation is
  better reproduced using temperature forcing derived from Greenland and
  Antarctic ice-core than from regional ocean sediment cores. During most of
  the last glacial cycle, the modelled ice cover is discontinuous and
  restricted to high mountain areas. However, widespread precipitation over the
  Skeena mountains favours the persistence of a central ice dome throughout the
  glacial cycle. It acts as a nucleation centre before the last glacial maximum
  and hosts the last remain of Cordilleran ice during the
  early Holocene (10.9--9.5\,\unit{kyr}).

\end{abstract}


% ----------------------------------------------------------------------
\introduction
\label{sec:intro}
% ----------------------------------------------------------------------

During the last glacial cycle, glaciers and ice caps of the North American
Cordillera have been more extensive than today. At the peak of glaciation, a
contiguous blanket of ice extended from Denali Peak in the north to the Puget
Sound in the south. This blanket is known as the former Cordilleran ice sheet.

For more than a century, exploration and geological investigation of the
Cordillera lead to a large collection of evidence of the former ice cover. This
evidence consists of mapped boundaries of the former ice extent, direction of
past sliding velocities and location of former melt-water streams, as well as
carbon dating and cosmogenic dating of glacier retreat.

Field-based evidence allowed to reconstruct, with a high degree of confidence,
the maximal extent attained by the Cordilleran ice sheet during the last
glacial cycle. However, former ice thickness, and in turn the ice sheet's
contribution to the last glacial maximum sea-level low-stand, remains poorly
constrained. Moreover, our understanding of the deglaciation phase is
restricted to a regional level, while little is known about the ice sheet
history preceding the last glacial maximum extent. Although time-evolving,
whole ice-sheet reconstructions of glacial advance and retreat patters are
available for the Laurentide and Eurasian ice sheet, this is not the case for
the Cordilleran ice sheet.

The present study aims to use a numerical ice sheet model calibrated against
field-based evidence to attempt a quantitative reconstruction of the
Cordilleran ice sheet history through the last glacial cycle. Although
numerical modelling has been identified as a tool to improve our understanding
of Cordilleran ice sheet more than twenty years ago, the ubiquitously
mountainous topography of the region has presented a major challenge to its
application, which is only overcome by recent development of numerical ice
sheet models and underlying scientific tools. In addition to the topographic
complexity, strong climatic contrasts characteristic of the North American
Cordillera require the use of high-resolution temperature and precipitation
forcing as input data to the ice sheet model.

Here we use the Parallel Ice Sheet Model (PISM), forced by a high-resolution
climatology from the North American Regional Reanalysis, to model the
Cordilleran ice sheet through the last glacial cycle. Because past climate
is mostly unknown, palaeoclimate forcing is applied in the simplified form of
six time-dependent temperature offset time-series. These time-series were
obtained by scaling six different palaeo-temperature reconstructions from
proxy records around the globe, including two \chem{\delta^{18}O} records from
Greenland ice cores, two \chem{\delta^{18}O} records from Antarctic ice cores,
and two alkenone unsaturation index records from Northwest Pacific oceanic
sediment cores. The output from the model is compared to geomorphological
evidence in terms of timing of glaciation, extent of glaciation and patterns
of deglaciation.

% Greenland Ice Core Project (GRIP)
% North Greenland Ice Core Project (NGRIP)
% European Project for Ice Coring in Antarctica (EPICA)
% Lake Vostok (Vostok)
% Ocean Drilling Program (ODP)

Fig.~\ref{fig:locmap} about here.

% ----------------------------------------------------------------------
\section{Model setup}
\label{sec:model}
% ----------------------------------------------------------------------

\subsection{Ice thermodynamics}

SSA+SIA, pseudo-plastic sliding, temperature-dependent flow.

\subsection{Bedrock response}

Thermal model and deformation model.

\subsection{Surface mass balance}

Positive degree-day model including spatially and seasonally variable standard
deviation (Fig.~\ref{fig:atm}).

\subsection{Atmospheric forcing}

Input climatology from North American Regional Reanalysis according to
sensitivity study from the previous paper (Fig.~\ref{fig:atm}).

% ----------------------------------------------------------------------
\section{Results}
\label{sec:results}
% ----------------------------------------------------------------------

\subsection{Palaeo-climate sensitivity test}

All six simulations consistently produce two major glaciations during MIS~4 and
MIS~2 (Figs.~\ref{fig:timeseries} and~\ref{fig:snapshots}).
However the timing is much different. Notably the oceanic core forcing result
in much too early last glacial maximum.

\subsection{High-resolution simulations}

We re-run the GRIP (?) and Vostok (?) core simulations at higher resolution.
Results are
consistent with lower resolution but differences appear due to small-scale
topography better capturing mass-balance lapse-rate effects. In general the
Greenland-based simulation show much more variability.

% ----------------------------------------------------------------------
\section{Discussion}
\label{sec:discussion}
% ----------------------------------------------------------------------

\subsection{Timing of glaciation}

Timing of glaciation is better reproduced by Greenland and Antarctic cores than
nearby oceanic records. This is because the oceanic cores are affected by
variations in strength of a local current (the California current) associated
with the presence of the ice sheet. The timing of the LGM is better reproduced
by Antarctic than Greenland records.

\subsection{Extent of glaciation}

During most of the glacial cycle, ice cover is restricted to high mountain
areas. This is particularly true when using the Greenland-based forcing
(Fig~\ref{fig:duration}). Most of the glacial maxima margin have a transient
character, showing that the ice sheet is far out of balance with climate. There
persist a central ice dome over the Skeena mountains over the entire glacial
period. This ice dome serves as a nucleation centre at the initiation of the
advance phase preceding the last glacial maximum. It is here because of more
widespread precipitation over the Skeena mountains than across other East-West
transects towards south or north of this region. We interpret this as an
orographic precipitation effect resulting from a "window" through the sharp
range of the Coast Mountains at this latitude (Fig?).

\subsection{Deglaciation patterns}

Deglaciation happens first in the southern half and then in the northern half
of the ice-covered area (Fig.~\ref{fig:deglac}). Retreat patterns are different
than reconstructed over central BC, however they match Martin's reconstruction
of Liard lowland "unzipping" pretty well. In the GRIP-driven simulation, the
patterns of de-glaciation and Younger Dryas re-advance match with recent
cosmogenic exposure dates.

% ----------------------------------------------------------------------
\conclusions
\label{sec:concl}
% ----------------------------------------------------------------------

% Acknowledgements
%\begin{acknowledgements}
  % Author contributions
  %\hack{\noindent}\textit{Author contributions.}
%\end{acknowledgements}

% References
%\input{references.bbl}
\newpage

% ----------------------------------------------------------------------
% Floats
% ----------------------------------------------------------------------

% tab:records
\begin{table*}[t]
  \caption{Palaeo-temperature proxy records and scaling parameters used to
           prepare temperature offset time-series used to force the ice sheet
           model through the last 120\,\unit{kyr}. $T_{[32;22]}$ refers to the
           mean temperature anomaly during the period -32 to~-22~\unit{kyr} after
           scaling.}
  \label{tab:records}
  {\begin{tabular}{lcccc}
    \tophline
    Record & Proxy & Scaling factor & $T_{[32;22]}$ & Source\\
    \middlehline
    GRIP     & \chem{\delta^{18}O} & ?\% & -5.8{\degree}C & \aref \\
    NGRIP    & \chem{\delta^{18}O} & ?\% & -6.0{\degree}C & \aref \\
    EPICA    & \chem{\delta^{18}O} & ?\% & -5.6{\degree}C & \aref \\
    Vostok   & \chem{\delta^{18}O} & ?\% & -5.6{\degree}C & \aref \\
    ODP~1012 & \chem{U^{K'}_{37}}  & ?\% & -5.8{\degree}C & \aref \\
    ODP~1020 & \chem{U^{K'}_{37}}  & ?\% & -5.8{\degree}C & \aref \\
    \bottomhline
  \end{tabular}}
  \belowtable{}
\end{table*}

% fig:locmap
\begin{figure}
  \includegraphics{locmap}
  \caption{Location map. In construction.}
  \label{fig:locmap}
\end{figure}

% fig:atm
\begin{figure}
  \includegraphics{atm}
  \caption{Monthly mean near-surface air temperature, precipitation and
           standard deviation of daily mean temperature for January and July
           months from the North American Regional Reanalysis (NARR)
           climatology, used as input fields to the ice sheet model. Note the
           strong contrasts in seasonality, timing of the precipitation peak,
           and temperature variability over the model domain, notably between
           the maritime and continental regions.}
  \label{fig:atm}
\end{figure}

% fig:timeseries
\begin{figure}
  \includegraphics{timeseries}
  \caption{\textbf{(top)} temperature offset time-series from ice and sediment
           core records used as palaeo-climate forcing for the ice sheet model
           (Table~\ref{tab:records}). \textbf{(bottom)} Modelled ice volume
           through the last 120\,\unit{kyr}, expressed in meters of sea-level
           equivalent. The grey spans indicate the ranges of modelled times of
           glacial extrema corresponding to MIS~4 (61.9--55.4\,\unit{kyr}), 3
           (52.2--45.6\,\unit{kyr}), and 2 (last glacial maximum,
           29.5--16.9\,\unit{kyr}).}
  \label{fig:timeseries}
\end{figure}

% fig:snapshots
\begin{figure}
  \includegraphics{snapshots}
  \caption{Snapshots of modelled ice surface topography (1\,\unit{km} contours)
           and velocity (\unit{m\,yr^{-1}}) at modelled times of glacial
           extrema, corresponding to MIS~4, 3, and 2.}
  \label{fig:snapshots}
\end{figure}

% fig:duration
\begin{figure}
  \includegraphics{duration}
  \caption{Modelled duration of ice cover during the last 120\,\unit{kyr}.
           Aside the location of present-day ice caps, note the persistence of
           a central ice dome over the Skeena mountains during most of the
           simulation. On the other hand, the maximal extent of the ice sheet
           corresponds to short durations of ice cover along most of the margin.}
  \label{fig:duration}
\end{figure}

% fig:deglac
\begin{figure}
  \includegraphics{deglac}
  \caption{Modelled age of the last deglaciation. Areas where the MIS~4 glacial
           advance exceeded the last glacial maximum advanced are marked in
           green. Hatches denote the Younger Dryas re-advance, which is most
           significant in the GRIP-driven simulation.}
  \label{fig:deglac}
\end{figure}

% ----------------------------------------------------------------------
\end{document}
\endinput
% ----------------------------------------------------------------------
