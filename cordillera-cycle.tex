% cordillera-climate.tex
% =============================================================================

% Copernicus manuscript
\documentclass[tc, manuscript]{copernicus}

% Copernicus final print
%\documentclass[tc]{copernicus}

% Copernicus discussion paper
%\documentclass[tcd, hvmath, online]{copernicus}

% Copernicus-like latex2rtf compatible
%% copernicus_rtf.tex
% ------------------

% Base class and packages
\documentclass{article}
\usepackage{color}
\usepackage{geometry}
\usepackage{graphicx}
\usepackage{setspace}
\onehalfspacing

% Replacements for bibtex commands
\newcommand{\citep}[1]{(\textcolor{blue}{#1})}
\newcommand{\citet}[1]{\textcolor{blue}{#1}}

% Replacements for Copernicus commands
\newcommand{\introduction}[0]{\section{Introduction}}
\newcommand{\conclusions}[0]{\section{Conclusions}}
\newcommand{\tophline}[0]{\hline}
\newcommand{\middlehline}[0]{\hline}
\newcommand{\bottomhline}[0]{\hline}
\newcommand{\unit}[1]{\ensuremath{\mathrm{#1}}}
\newcommand{\degree}[0]{\ensuremath{^{\circ}}}

% Ignore other Copernicus commands
\newcommand{\runningtitle}[1]{}
\newcommand{\runningauthor}[1]{}
\newcommand{\received}[1]{}
\newcommand{\correspondence}[1]{}
\newcommand{\pubdiscuss}[1]{}
\newcommand{\revised}[1]{}
\newcommand{\accepted}[1]{}
\newcommand{\published}[1]{}



% Coloured hyperlinks
\hypersetup{colorlinks, citecolor=blue}

% Figure directory
\graphicspath{{figures/}}

% My commands
\newcommand{\renote}[1]{\footnote{\textbf{Comment reply}: #1}}
\newcommand{\todo}[1]{\emph{[\textbf{Todo:} #1]}}
\newcommand{\aref}[0]{\textbf{[ref.]}}

% =============================================================================
\begin{document}
\linenumbers
% =============================================================================

% Title
\title{Numerical simulations of the Cordilleran ice sheet
       through the last glacial cycle}

% Authors
\Author[1,2]{J.}{Seguinot}
\Author[2]{I.}{Rogozhina}
\Author[3]{M.}{Margold}
\Author[1]{J.}{Kleman}
\Author[1]{A.~P.}{Stroeven}
\runningauthor{J.~Seguinot et~al.}
\correspondence{J.~Seguinot (julien.seguinot@natgeo.su.se)}

% Running title
\runningtitle{Climate forcing for Cordilleran ice sheet simulations}

% Affiliations
\affil[1]{Department of Physical Geography and Quaternary Geology and the
          Bolin Centre for Climate Research, Stockholm University,
          Stockholm, Sweden}
\affil[2]{Helmholtz Centre Potsdam, GFZ German Research Centre for Geosciences,
          Potsdam, Germany}
\affil[3]{Department of Geography, Durham University, UK}

% For Copernicus
\received{}
\pubdiscuss{}
\revised{}
\accepted{}
\published{}

% Title
\firstpage{1}
\maketitle

% Abstract
\begin{abstract}

  Despite more than a century of geological observations, the Cordilleran ice
  sheet of North America remains poorly understood in terms of its former
  extent, volume and dynamics. Although geomorphological evidence is abundant,
  its complexity is such that whole ice-sheet reconstructions of advance and
  retreat patterns are lacking. Here we use a numerical ice sheet model
  calibrated against field-based evidence to attempt a quantitative
  reconstruction of the Cordilleran ice sheet history through the last glacial
  cycle. A series of simulations is driven by time-dependent temperature
  offsets from six proxy records located around the globe. Although this
  approach reveals large variations in model response to evolving atmospheric
  forcing, all simulations produce two major glaciation events during
  marine isotope stages~4 (61.9--55.4\,kyr) and~2
  (29.5--16.9\,kyr). The timing of glaciation is
  better reproduced using temperature reconstructions from Greenland and
  Antarctic ice cores than from regional oceanic sediment cores. During most of
  the last glacial cycle, the modelled ice cover is discontinuous and
  restricted to high mountain areas. However, widespread precipitation over the
  Skeena Mountains favours the persistence of a central ice dome throughout the
  glacial cycle. It acts as a nucleation centre before the last glacial maximum
  and hosts the last remains of Cordilleran ice during the
  early Holocene (10.9--9.5\,kyr).

\end{abstract}

% =============================================================================
\introduction
\label{sec:intro}
% =============================================================================

During the last glacial cycle, glaciers and ice caps of the North American
Cordillera have been more extensive than today. At the Last Glacial
Maximum (LGM), a continuous blanket of ice, the Cordilleran ice sheet\renote{
    In \citep{Seguinot.etal.2014}, The Cryosphere changed all my occurences of
    Cordilleran (Greenland etc) \textbf{I}ce \textbf{S}heet to lowercase
    \textbf{i}ce \textbf{s}heet. Let's keep it so for now for consistency
    within the thesis (kappa + papers).}
\citep{Dawson.1888}, extended from the Alaska Range in the north to the
North Cascades in the south. In addition, it extended offshore, where it calved
into the Pacific Ocean, and merged with the western margin of its much larger
neighbour, the Laurentide ice sheet, east of the Rocky Mountains
(Fig.~\ref{fig:locmap}).

More than a century of exploration and geological investigations of the
Cordilleran mountains have led to many observations in support of the former
ice sheet
    \citep{Jackson.Clague.1991}.
Despite the lack of documented end moraines offshore, in the zone of confluence
with the Laurentide ice sheet, and in areas swept by the Missoula floods
    \citep{Carrara.etal.1996},
moraines that demarcate the south-western and north-eastern margins provide key
constraints that allow reasonable reconstructions of maximum ice sheet extents
    (\citealp{Prest.etal.1968}; \citealp[Fig. 1.12]{Clague.1989};
     \citealp{Duk-Rodkin.1999};
     \citealp{Booth.etal.2003}; \citealp{Dyke.2004}).
The LGM Cordilleran ice sheet maximum extent was short-lived, as indicated by
field evidence from radiocarbon dating
    \citep{Clague.etal.1980, Clague.1985, Clague.1986, Porter.Swanson.1998,
           Menounos.etal.2008},
cosmogenic exposure dating
    \citep{Stroeven.etal.2010, Stroeven.etal.2014, Margold.etal.2014},
bedrock deformation in response to former ice loads
    \citep{Clague.James.2002, Clague.etal.2005},
and offshore sedimentary records
    \citep{Cosma.etal.2008, Davies.etal.2011}.
However, former ice thicknesses and, therefore, the ice sheet's contribution to
the LGM sea level low stand
    \citep{Carlson.Clark.2012, Clark.Mix.2002}
remain uncertain.

Our understanding of the Cordilleran glaciation history prior to the LGM is
even more fragmentary
    \citep{Barendregt.Irving.1998, Kleman.etal.2010, Rutter.etal.2012},
although it is clear that maximum glaciation of the Cordilleran ice sheet
predates the last glacial cycle
    \citep{Hidy.etal.2013}.
In parts of the Yukon Territory and Alaska, the distribution of tills
    \citep{Turner.etal.2013}
and dated glacial erratics indicate an extensive Marine Oxygen Isotope Stage
(MIS)~4 glaciation
    \citep{Ward.etal.2007, Ward.etal.2008, Briner.Kaufman.2008,
           Stroeven.etal.2010, Stroeven.etal.2014},
yet it is not known whether other regions in the study area were affected.
Landforms in the interior regions include flow sets that are likely
older than the LGM
    \citep[Fig.~2]{Kleman.etal.2010},
but their absolute age remains uncertain.

In contrast, evidence for the deglaciation history of the Cordilleran
ice sheet since the LGM is considerable, albeit mostly at a regional scale.
Geomorphological evidence from south-central British Columbia indicates a rapid
deglaciation, including an early emergence of elevated areas while thin
stagnant ice still covered surrounding lowlands
    \citep{Fulton.1967, Fulton.1991, Margold.etal.2011, Margold.etal.2013a}.
This model, although credible, may not apply in all areas of the Cordilleran
ice sheet
    \citep{Margold.etal.2013}.
Although solid evidence for late-glacial glacier re-advances have been found in
the Coast, Columbia and Rocky mountains
    \citep{Clague.etal.1997, Friele.Clague.2002, Friele.Clague.2002a,
           Kovanen.2002, Kovanen.Easterbrook.2002, Lakeman.etal.2008,
           Menounos.etal.2008},
it appears to be more sparse than for formerly glaciated regions surrounding
the North Atlantic \aref, thus allowing for considerable uncertainty concerning
the possibility of a regional late glacial cold reversal.

In general, the topographic complexity of the North American Cordillera and its
effect on glacial history have inhibited the construction of ice sheet-wide
glacial advance and retreat patterns such as
those available for the Fennoscandian and Laurentide ice sheets
     \citep{Boulton.etal.2001, Dyke.Prest.1987, Dyke.etal.2003,
            Kleman.etal.1997, Kleman.etal.2010}.
Here, we use a numerical ice sheet model \citep{PISM-authors.2014},
calibrated against field-based evidence, to perform a quantitative
reconstruction of the Cordilleran ice sheet history through the last glacial
cycle, and
analyse some of the long-standing questions related to its evolution:

\begin{itemize}
  \item How much ice was locked in the Cordilleran ice sheet during the
    last glacial maximum?
  \item What was the scale of glaciation prior to the last glacial maximum?
  \item Which were its primary dispersal centres? Do they reflect stable or
    ephemeral configurations?
  \item How rapid was the last deglaciation? Did it include late glacial
    stand-stills or re-advances?
\end{itemize}

Although numerical ice sheet modelling has been established as a useful tool to
improve our understanding of the Cordilleran ice sheet
    (\citealp[p.~227]{Jackson.Clague.1991}; \citealp{Robert.1991},
     \citealp{Marshall.etal.2000})
the ubiquitously mountainous
topography of the region has presented two major challenges to its application.
First, only recent development of numerical ice sheet models and underlying
scientific computing tools \citep{Bueler.Brown.2009, Balay.etal.2014} allows
high-resolution numerical modelling of palaeo-glaciers on mountainous terrain
over millenial time scales \citep[e.g.,][]{Golledge.etal.2012}. Second, the
complex
topography of the North American Cordillera also induces strong gradients in
seasonality and distinct patterns of precipition, thus requiring the use of
high-resolution atmospheric forcing fields as input to the ice sheet model
\citep{Seguinot.etal.2014}.

Because climate conditions over the last glacial cycle are subject to
considerable uncertainty, our palaeoclimate forcing is a simplistic approach
including temperature and precipitation fields derived from a
present-day atmospheric reanalysis \citep{Mesinger.etal.2006,
Seguinot.etal.2014} supplemented by a lapse-rate correction
and temperature offset time series. The latter are obtained by scaling six
different palaeo-temperature reconstructions from proxy records around the
globe, including two oxygen isotopes records from Greenland ice cores
\citep{Dansgaard.etal.1993, Andersen.etal.2004}, two oxygen isotopes
records from Antarctic ice cores \citep{Petit.etal.1999,Jouzel.etal.2007},
and two alkenone unsaturation index records from Northwest Pacific ocean
sediment cores \citep{Herbert.etal.2001}. Model output is compared to
geomorphological evidence in terms of the timing and extent of glaciation and
patterns of deglaciation.


% =============================================================================
\section{Model setup}
\label{sec:model}
% =============================================================================

% -----------------------------------------------------------------------------
\subsection{Overview}
% -----------------------------------------------------------------------------

\todo{Move parameter values into a separate table.}

The simulations presented here were run using the Parallel Ice Sheet Model
(PISM, development version~11b0a7f and stable version~0.6.1), an open source,
finite difference, shallow ice sheet model \citep{PISM-authors.2014}. The model
requires input on basal topography, sea level, geothermal heat flux and
atmospheric\renote{
    Changed here from ``climate'' to ``atmospheric''. Although I have used
    ``climate forcing'' in the previous paper, I now think that ``atmospheric
    forcing'' is more appropriate: ``climate'' may also encompass the oceanic
    component, not really assessed here.}
forcing, and computes the evolution of ice extent
and thickness over time, the thermal and dynamic\renote{
    Thermal = temperature, dynamic = force + movement. If I meant movement
    alone, I would use ``cinematic''. Here we could use ``ice temperatures,
    stress and deformation (i.e. velocities)'', but that becomes a bit clumsy,
    I feel.}
states of the ice sheet, and the associated lithospheric response.

Basal topography is derived from the ETOPO1 combined topography and bathymetry
dataset with a~resolution of 1\,arc-min \citep{Amante.Eakins.2009}. Sea level
is lowered as a function of time based on the Spectral Mapping Project\renote{
    I had no time to check for later products, but if you have suggestions for
    something more up-to-date, I can try to include it in later simulations
    (everything will have to be re-run anyway).}
\citep[SPECMAP;][]{Imbrie.etal.1989} time scale. Geothermal heat flux
is applied as a constant value of 70\,\unit{mW\,m^{-2}} at 3\,km depth
(Sect.~\ref{sec:icedyn}). Atmospheric forcing is provided by a monthly
climatology from the North American Regional Reanalysis
\citep[NARR;][]{Mesinger.etal.2006} perturbated by time-dependent and
lapse-rate temperature corrections (Sect.~\ref{sec:atm}).

Each simulation starts from assumed ice-free conditions at 120\,000 years ago
(120\,ka), and runs to present. Our modelling domain of 1\,500 by 3\,000\,km
encompasses the entire area covered by the Cordilleran ice sheet at the LGM
(Fig.~\ref{fig:locmap}). The simulations were run on two distinct grids, using
a lower horizontal resolution of 10\,km, and a higher horizontal resolution of
6\,km. These computations were performed on 16 to 128 computing cores at the
Swedish National Supercomputing Centre.

% -----------------------------------------------------------------------------
\subsection{Ice thermodynamics}
\label{sec:icedyn}
% -----------------------------------------------------------------------------

Ice sheet dynamics are typically modelled using a combination of internal
deformation and basal sliding. PISM is a~shallow ice sheet model, which implies
that the balance of stresses is approximated based on their predominant
components. The Shallow Shelf Approximation (SSA) is used as a ``sliding law''
for the Shallow Ice Approximation (SIA) by adding both velocity solutions
\citep[Eqns.~7--9 and 15]{Bueler.Brown.2009, Winkelmann.etal.2011}\renote{
    Arjen, I don't agree with your comment. Reference to equation number(s)
    show the reader where to look. \citet{Winkelmann.etal.2011} is a rather
    long paper that also treats many other aspects of the model. Their SIA+SSA
    approach builds on \citep{Bueler.Brown.2009}, but it is slightly different.
    It is the one in use in PISM, and in my simulations.}.
Ice rheology depends on temperature and water content through an enthalpy
formulation \citep{Aschwanden.etal.2012}. Surface air temperature derived from
the atmospheric forcing (Sect.~\ref{sec:atm}) provides the upper boundary
condition to the ice enthalpy model. Temperature is computed subglacially to
a~depth of 3\,km\renote{
    Yes, I changed this from 1 to 3\,km as we now model longer period of times
    than in the previous paper. The penetration depth of temperature anomalies
    (i.e. skin depth) depends on the period of temperature oscillations at the
    surface. The longer the period (in our case 100\,kyr), the deeper the
    temperature change. A good way to estimate what this parameter should be is
    to solve the 1-D temperature diffusion equation in the case of periodic
    surface forcing, using rock heat capacity and rock thermal conductivity
    values similar as in the numerical model. This basically forms the basis
    for my choice of the 3\,km value.},
where it is conditioned by a lower boundary geothermal heat flux of
70\,\unit{mW\,m^{-2}}. Although this uniform value does not account for the
high spatial geothermal variability in the region
\citep{Blackwell.Richards.2004}, it is, on average, representative of available
heat flow measurements. In the low-resolution simulations, the vertical grid
consists of to 51~enthalpy layers in the ice sheet and 31~temperature layers in
the bedrock. In the high-resolution simulations, up to 101~ice layers, and
61~bedrock layers are used.

A~pseudo-plastic sliding law \citep{Bueler.Pelt.2014} relates the
bed-parallel shear stress to the sliding velocity. The yield stress, $\tau_c$,
is modelled using the Mohr--Coulomb criterion,
\begin{equation}
   \tau_c = c_0 + N\,\tan{\phi} \,,
\end{equation}
where cohesion, $c_0$, is assumed to be zero. The friction angle, $\phi$,
varies from 15 to 45{\degree}\renote{
    Different from the previous paper, but fairly similar in effect (see
    $\alpha$ and $\delta$ parameter values and corresponding equations in the
    kappa, basal sliding section). These values are is based on a quick
    sensitivity test on Greenland.}.
It is taken to be a~function of modern bed elevation, with lower values at
lower elevations (<200\,m above sea level), thus accounting for a~weakening of
till associated with the presence of marine sediments. Effective pressure, $N$,
is related to the ice overburden stress, $\rho gh$, and the modelled amount of
subglacial water, using a formula derived from laboratory experiments with till
extracted from an Antarctic ice stream \citep{Tulaczyk.etal.2000,
Bueler.Pelt.2014},
\begin{equation}
    N = \delta \rho gh \, 10^{(e_0/C_c) (1 - (W/W_{max}))} \,,
\end{equation}
where $delta$ is choosed as 0.95, $e_0$ is a measured reference void ratio and
$C_c$ a measured compressibility coefficient. The amount of water at the base,
$W$, varies from zero to $W_{max}=2$\,m, a threshold above which instantaneous
drainage is assumed\renote{
    Regarding a parameter table, it is in the kappa, I am skipping it for now
    but will try to include it soon.}.
Finally, the bedrock topography responds to ice load\renote{
    I disagree with writing ``the isostatic response''. The main point in
    using this model is that computes a non-isostatic response.}
following a bedrock deformation model that includes point-wise isostasy,
elastic lithosphere flexure and viscous mantle deformation in the lower
half-space\renote{
    I had put an emphasis on semi-infinite at this is the key assumption of
    that model, allowing effective solution in Fourier domain. This is the
    whole point of Ed's paper and probably the reason why he picked it for PISM
    But yes, I guess that ``lower half-space'' is fine, too...}
\citep{Lingle.Clark.1985,Bueler.etal.2007}.

% -----------------------------------------------------------------------------
\subsection{Surface mass balance}
% -----------------------------------------------------------------------------

\renote{
    Surface mass balance is the balance of mass fluxes at the surface,
    consisting of ice accumulation and ice ablation, which I introduce straight
    away. I don't know what kind of ``opening sentence'' can be added here.}
Ice surface accumulation and ablation are computed from monthly mean
near-surface air temperature $T_m$, monthly standard deviation of near-surface
air temperature $\sigma$, and monthly precipitation $P_m$, using
a~temperature-index model \citep[e.g.,][]{Hock.2003}. Accumulation is equal to
precipitation when air temperatures are below 0\,\unit{{\degree}C}, and
decreases to zero linearly with temperatures between 0 and
2\,\unit{{\degree}C}. Ablation is computed from the number of positive
degree-days (PDD), defined as the integral of temperatures above
0\,\unit{{\degree}C} in one year.

The PDD computation accounts for stochastic temperature variations by assuming
a normal temperature distribution of standard deviation $\sigma$ aroung the
expected value $T_m$. It is expressed by an error-function formulation
\citep{Calov.Greve.2005},
\begin{equation}
    \label{eqn:calovgreve}
    \mathrm{PDD} = \int_{t_1}^{t_2} \mathrm{d}t
        \left[\frac{\sigma}{\sqrt{2\pi}}
                \exp\left({-\frac{T_{m}^2}{2\sigma^2}}\right)
              + \frac{T_{m}}{2} \, \mathrm{erfc}
                \left(-\frac{T_{m}}{\sqrt{2}\sigma}\right)\right] \,,
\end{equation}
which is numerically approximated using week-long sub-intervals. In order to
account for the effects of spatial and seasonal variations of temperature
variability \citep{Seguinot.2013}, $\sigma$ is computed from daily temperature
values from the North American Regional Reanalysis
\citep[NARR,][]{Mesinger.etal.2006}, after excluding variability associated
with the seasonal cycle itself \citep[cf.][]{Seguinot.Rogozhina.2014}\renote{
    I may add a figure in the kappa if I find time, but I don't think that we
    should do it here. The subject is too marginal for the present study.}.
The ablation model incorporates degree-day factors of
3.04\,\unit{mm\,{\degree}C^{-1}\,day^{-1}} for snow and
4.59\,\unit{mm\,{\degree}C^{-1}\,day^{-1}} for ice, as derived from
mass-balance measurements on contemporary glaciers from the Coast Mountains and
Rocky Mountains in British Columbia \citep{Shea.etal.2009}.

% -----------------------------------------------------------------------------
\subsection{Atmospheric forcing}
\label{sec:atm}
% -----------------------------------------------------------------------------

Atmospheric forcing of the model consists of a present-day monthly climatology,
$\{T_{m0}, P_{m0}\}$, where temperatures are modified by offset time series,
${\Delta}T_{TS}$, and lapse-rate corrections, ${\Delta}T_{LR}$:
\begin{align}
    T_m(t, x, y) &= T_{m0}(x, y) + {\Delta}T_{TS}(t)
                    + {\Delta}T_{LR}(t, x, y) \,, \\
    P_m(t, x, y) &= P_{m0}(x, y) \,.
\end{align}

The present-day monthly climatology, $\{T_{m0}, P_{m0}\}$, was computed from
near-surface air temperature and precipitation rate fields of the NARR
\citep{Mesinger.etal.2006}, averaged between 1979 and 2000. Modern climate of the
North American Cordillera is characterised by strong geographic variations in
temperature seasonality, timing of the maximum annual precipitation, and
daily temperature variability (Fig.~\ref{fig:atm}).
Our choice of data from the NARR is motivated by the need for an accurate,
high-resolution precipitation forcing, as identified in a previous sensitivity
study \citep{Seguinot.etal.2014}.

Temperature offset time-series, ${\Delta}T_{TS}$, are derived from
palaeo-temperature proxy records from
the Greenland Ice Core Project \citep[GRIP,][]{Dansgaard.etal.1993}, the
North Greenland Ice Core Project \citep[NGRIP,][]{Andersen.etal.2004},
the European Project for Ice Coring in Antarctica \citep[EPICA,][]
{Jouzel.etal.2007}, the Vostok ice core \citep{Petit.etal.1999}, and Ocean
Drilling Program (ODP) sites 1012 and 1020, both located off the Californian
coast \citep{Herbert.etal.2001}. Palaeo-temperatures from the GRIP and NGRIP
records were calculated from oxygen isotope (\chem{\delta^{18}O}) measurements
using a quadratic equation \citep{Johnsen.etal.1995},
\begin{equation}
    {\Delta}T_{TS}(t) = -11.88 [\chem{\delta^{18}O}(t)
                               -\chem{\delta^{18}O}(0)]
                        -0.1925[\chem{\delta^{18}O}(t)^2
                                -\chem{\delta^{18}O}(0)^2] \,,
\end{equation}
while temperature reconstructions from Antarctic and oceanic cores were
provided as such. All records were scaled linearly in
order to simulate realistic and comparable ice limits at the last
glacial maximum (Table~\ref{tab:records}, Fig.~\ref{fig:timeseries}).
In the rest of this paper, we occasionally refer to different model
runs by the name of the proxy record used for the palaeo-temperature forcing.

Finally, lapse-rate corrections, ${\Delta}T_{LR}$, are computed as a function
of ice surface elevation, $s$, using the NARR surface geopotential height
invariant field as a reference topography, $b_{ref}$, by
\begin{align}
    {\Delta}T_{LR}(t, x, y) &= -\gamma [s(t, x, y)-b_{ref}] \\
                            &= -\gamma [h(t, x, y)+b(t, x, y)-b_{ref}],
\end{align}

thus accounting for the evolution of ice thickness, ${h=s-b}$\renote{
    Small change in notation to match with the kappa},
on the one hand, and
for differences between the the ice flow model basal topography, $b$, and the
NARR reference topography, $b_{ref}$, on the other hand. All simulations use an
annual temperature lapse rate of $\gamma = 6\,\unit{{\degree}C\,km^{-1}}$.

% =============================================================================
\section{Sensitivity to atmospheric forcing time-series}
\label{sec:results}
% =============================================================================

% -----------------------------------------------------------------------------
\subsection{Evolution of ice volume}
% -----------------------------------------------------------------------------

Despite large differences in the input atmospheric forcing (temperature offset
time-series; Fig.~\ref{fig:timeseries}), model output presents consistent
features that can be observed across the range of forcing data used. In all
simulations, modelled ice volumes remain relatively low during most of the
glacial cycle, except during two major glacial events which occur between 61.9
and 55.4\,ka during Marine Oxygen Isotope Stage (MIS)~4, and between 29.5 and
16.9\,ka during MIS~2 (Fig.~\ref{fig:timeseries}). An ice volume minimum is
consistently attained between 52.2 and 45.6\,ka during MIS~3. However, the
magnitude and precise timing of these three events depend significantly on the
choice of proxy record used for time-dependent atmospheric forcing
(Table~\ref{tab:extrema})\renote{
    Hopefully this new table will assess several of your comments at once.}.

Simulations forced by the Greenland ice core palaeo-temperature\renote{
    From here on, I prepend ``palaeo-temperature'' to ``record(s)'' in place of
    ``climate'' as you (Arjen) originally suggested. If we are to precise what
    type of ``records'' we use, let's be clear about it.}
records (GRIP, NGRIP) produce the highest variability in modelled ice volume
throughout the last glacial cycle. In contrast, simulations driven by oceanic
(ODP~1012, ODP~1020) and Antarctic (EPICA, Vostok) palaeo-temperature records
generally result in smaller\renote{
    Until we run out of other ajectives, I don't think that ``modest'' and
    ``generous'' are adequate terms for describing ice volumes.}
modelled ice volumes during MIS~4 and larger ice volumes during MIS~3.
More broadly speaking, they produce lower ice volume variability
throughout the simulation length. The NGRIP atmospheric forcing is the only one
that results in a larger ice volume during MIS~4 than during the last glacial
maximum (MIS~2).

While simulations driven by GRIP and the two Antarctic palaeo-temperature
records attain a last ice volume maximum between 19.5 and 16.9\,ka, those
informed by NGRIP and the two oceanic palaeo-temperature records attain their
maximum ice volume thousands of years earlier. Moreover, the ODP~1012 run
yields a rapid deglaciation of the modelled area prior to 20\,ka. The ODP~1020
simulation predicts an early maximum in ice volume at 29.5\,ka, followed by
slower deglaciation than modelled using other palaeo-temperature records.
Finally, model runs forced by Antarctic palaeo-temperature records produce a
rapid an uninterrupted deglaciation after the last glacial maximum, whereas the
simulation driven by the GRIP palaeo-temperature record also yields a rapid
deglaciation but in three steps, interrupted by two stand-stills and readvances
of the ice margin.

% -----------------------------------------------------------------------------
\subsection{Extreme configurations}
% -----------------------------------------------------------------------------

Despite such differences in the timing of attained volume maxima and mimima,
snapshots of model output show relatively consistent patterns of glaciation
(Fig.~\ref{fig:snapshots}). As a result of scaling factors applied to the
different palaeo-temperature proxy records (Table~\ref{tab:records}), modelled
ice sheet geometries during the last glacial maximum (MIS~2;
Fig.~\ref{fig:snapshots}, lower panels) appear highly similar from one
simulation to the next. They include a ca. 1\,500\,km-long central divide
located about 3\,500\,m\ above sea level (a.s.l.) along the spine of the Rocky
Mountains. The volumes of these modelled last glacial maximum ice sheets are
contained within a close range from 8.24 to 8.66\,m of sea-level equivalent
(s.l.e; Table~\ref{tab:extrema}).
A central ice cap persists in all simulations over the Skeena
Mountains during the MIS~3 ice volume minima (Fig.~\ref{fig:snapshots}, middle
panels). However, the size of the Skeena Mountains ice cap, and the dimensions
of the Cordilleran ice sheet during the MIS~4 glaciation, depend sensitively on
the choice of palaeo-temperature record used to drive the model.

\todo{Add general statements on the scale of glaciation.}

% =============================================================================
\section[Comparison to the geologic record]
        {Comparison to the geologic record\renote{
    Arjen, you have proposed 2.1 Evaluation of model runs agains field
    evidence, 2.2 Palaeoglaciology of the Cordilleran ice sheet, and 2.3 The
    last deglaciation; but I don't think these headers work. In my vision of
    the paper, this entire ``discussion'' part is all about evaluation of model
    runs agains field evidence and palaeoglaciology. Because of this, and
    because we also present new ``results'', or rather different views of the
    same results, I propose to rename it to a more explicit ``Comparison to the
    geologic record'' (or similar). I have also renamed the ``results'' to
    ``Sensitivity to atmospheric forcing time-series'', which is actually what
    that section is about: raw model output, no discussion against the geologic
    record.}}
\label{sec:discussion}
% =============================================================================

% -----------------------------------------------------------------------------
\subsection{Glacial maxima}
% -----------------------------------------------------------------------------

\subsubsection{Timing of glaciation}

Independently of the choice of palaeo-temperature record\renote{
    In my view, ``atmospheric/climate forcing'' would be far too broad here. We
    could possibly apply a much wilder choice of atmospheric forcing
    techniques, leading just as wild differences in model output
    \citep[e.g.,][but possibly much more than that]{Seguinot.etal.2014}.}
used to force the ice sheet model, our simulations consistently produce two
glacial maxima during the last glacial cycle. The first maximum configuration
is obtained during MIS~4 (61.9--55.4\,ka) and the second during MIS~2
(29.5--16.9\,ka). These events broadly correspond in timing to the Gladstone
(MIS~4) and McConnell (MIS~2) glaciations documented by geological evidence for
the northern sector of the Cordilleran ice sheet
    \citep{Duk-Rodkin.etal.1996, Ward.etal.2007,
           Stroeven.etal.2010, Stroeven.etal.2014},
and to the Fraser Glaciation (MIS~2) documented for its southern sector
    \citep{Porter.Swanson.1998, Margold.etal.2014}.
There is patchy stratigraphical evidence for glaciations older than the Fraser
Glaciation \citep{Clague.Ward.2011} in British Columbia, and their extent and
timing are therefore still highly conjectural
    \citep[perhaps MIS~4 or early MIS~3; e.g.,][]{Cosma.etal.2008}.

However, the exact timing of modelled MIS~2 maxima depend strongly on the
choice of applied palaeo-temperature record, which allows for a more in-depth
comparison with geological evidence for the timing of maximum Cordilleran ice
sheet extent. In the Puget lowland (Fig.~\ref{fig:locmap}), the LGM advance of
the southern Cordilleran ice sheet margin has been constrained by radiocarbon
dating on wood between 17.4 and 16.4\,\unit{\chem{^{14}C}\,cal\,kyr\,BP}
\citep{Porter.Swanson.1998}.
These dates have been confirmed by the offshore sediment record, which shows an
increase of glaciomarine sedimentation between 19.5 and
16.2\,\unit{\chem{^{14}C}\,cal\,kyr\,BP} \citep{Cosma.etal.2008}. Radiocarbon
dating of the northern Cordilleran ice sheet margin is much less constrained
but straddles presented constraints from the southern margin. However,
cosmogenic exposure dating by \citep{Stroeven.etal.2010, Stroeven.etal.2014}
places the timing of maximum CIS extent during the McConnell glaciation close
to 17\,ka.

Among the simulations presented here, only those forced with the GRIP, EPICA
and Vostok palaeo-temperature records yield Cordilleran ice sheet maximum
extent that may be compatible with these field constraints
(Fig.~\ref{fig:timeseries}, lower panel; Fig.~\ref{fig:snapshots}, lowest
panels). Simulations driven by the NGRIP, ODP~1012 and ODP~1020
palaeo-temperature records, on the contrary, yield MIS~2 maximum Cordilleran
ice sheet extents that pre-date field-based constraints by several thousands of
years. Concerning the simulations driven by oceanic records, this early
deglaciation is caused by an early warming present in the alkenone
palaeo-temperature reconstructions (Fig.~\ref{fig:timeseries}, upper panel;
\citealp[Fig.3]{Herbert.etal.2001}). However, this
early warming is a local effect, corresponding to a weakening of the California
current \citep[Fig.~3]{Herbert.etal.2001}. The California current, driving cold
waters southwards along the south-western coast of North America\renote{
    Not only USA but also Baja California in Mexico.},
has been shown to have weakened during each peak global glaciation (in SPECMAP)
durint the past 550\,ka, resulting in paradoxically warmer sea-surface
temperatures at the locations of the ODP~1012 and ODP~1020 sites also during
the LGM \citep{Herbert.etal.2001}.

Because most of the marine margine of the Cordilleran ice sheet terminated in a
sector of the Pacific Ocean unaffected by variations in the California current,
it probably remained unaffected by this early warming. However, the above paradox
illustrates the complexity of ice-sheet feedbacks on regional climate, and
demonstrates that, although located in the neighbourhood of the modelling
domain, the ODP~1012 and ODP~1020 palaeo-temperature records cannot be
used as a realistic forcing to model the Cordilleran ice sheet
through the last glacial cycle.

\todo{say something about NGRIP.}

... Hence, we focus the rest of our analysis on simulations forced by palaeo-temperature records from GRIP and EPICA ice cores (Fig.~\ref{fig:timeseries}, lower panel, dotted lines).

\todo{avoid repetition with the next paragraph.}

These results indicate that a palaeo-temperature proxy record located in the
direct vicinity of the former Cordilleran ice sheet would potentially lead to
improved understanding of the history of advance and retreat of the western ice
sheet margin. Because, to our knowledge, no
such record yet exists, we focus the rest of our analysis on simulations forced
by palaeo-temperature records from GRIP and EPICA ice cores
(Fig.~\ref{fig:timeseries}, lower panel, dotted lines).

\subsubsection{Ice configuration during MIS~2}

\todo{Add MIS~2 ice topo and velocity map. Include two paragraphs from Martin.}

\subsubsection{Ice configuration during MIS~4}

\todo{Add MIS~4 ice topo and velocity map. Discuss that some of the margins
      are more extensive during MIS~4 than during MIS~2, in the GRIP simu.}

% -----------------------------------------------------------------------------
\subsection{Nucleation centres}
% -----------------------------------------------------------------------------

\subsubsection{Transient ice sheet states}

\todo{Put more equal weight to the northern sector}

\todo{Add some point, refer back to Fig.~\ref{fig:atm}.}

\todo{Mention the major nucleation centres. Growth of the CIS from the
      highlands.}

Although snapshots from the model output (Fig.~\ref{fig:snapshots}) yield a
clear spatial picture of modelled glaciation patterns at specific time frames,
they often can not be directly compared to geomorphological evidence, which is
by nature time-transgressive. In this section, we use high-resolution
simulations driven by the GRIP and EPICA records to locate the major nucleation
centres of the Cordilleran ice sheet and assess their longevity during the last
glacial cycle.

\todo{Rewrite this opening paragraph. Fig.~\ref{fig:snapshots} can actually be
      compared to geological evidence.}

During most of the glacial cycle, modelled ice cover is restricted to disjoint
ice caps centred on major mountain ranges of the North American Cordillera
(Fig.~\ref{fig:duration}). A 2\,500\,km-long continuous ice cover spanning from
the Alaska Range in the northeast to the Rocky Mountains in the southwest only
exists\renote{
    I chosed to write about model output in the present form to discern in from
    the reality.}
for at most 29\,kyr in total over the entire last glacial cycle. However,
except for its marine margin and the northern foothills of the Alaska Range,
the maximum extent of the ice sheet is attained for an even more brief period
of a few thousand years (Figs.~\ref{fig:timeseries}--{fig:duration}). This
result illustrates that the maximum extents of the modelled ice sheet during
MIS~4 and 2 were both short-lived and therefore out of balance with
contemporary climate, which confirms previous inferences of a short-lived
Cordilleran ice sheet based on the geological record \citep{Clague.etal.1980,
Stroeven.etal.2010}.

\subsubsection{Major ice-dispersal centres}

Pehaps one of the most persistent landscape feature is the Skeena Mountains ice
cap, which persisted throughout the entire last glacial cycle
(Figs.~\ref{fig:snapshots} and~\ref{fig:duration}). Regardless of the applied
forcing (Fig.~\ref{fig:snapshots}, middle panels), this ice cap appears to
survive MIS~3, and serves as a nucleation centre at the onset of the glacial
readvance towards the LGM (MIS~2). The importance of residual ice for Noeth
American glacial history leading up to the LGM has been illustrated by the
MIS~3 residual ice bodies in northern and eastern Canada as nucleation centres
for a much more extensive MIS~2 configuration \citep{Kleman.etal.2010}.

The presence of a Skeena Mountains ice cap during most of the last glacial
cycle can be explained by a more widespread\renote{
    I mean widespread, precipitation there is lower than elsewhere.},
current winter precipitation for that region there than for other parts of the
modelling domain (Fig.~\ref{fig:atm}). Along most of the north-western coast of
North America, coastal mountain ranges form a pronounced topographic barrier
for westerly winds, capturing atmospheric moisture in the form of orographic
precipitation, and resulting in arid interior lowlands. However, near the
centre of our modelling domain, this barrier is less pronounced than elsewhere,
allowing westerly winds to carry moisture\renote{
    I do not agree with ``carry precipitation''. When reading this I get the
    picture of water drops falling diagonaly under action of the wind. I do not
    deny that transport of precipitation happens and
    is an important process for understanding orographic
    precipitation (think of why it is raining so much in the Swedish mountains
    even when the wind comes from the west), but it is not what I am trying to
    describe here. I wrote ``moisture'' and this is what I mean.}
further inland, until it is progressively captured by the extensive, but mildly
elevated group of the Skeena Mountains in north-central British Columbia.

\subsubsection{Potential imprint on the landscape}

\todo{Explain why it would not be good with general statements on thermal
      evolution of ice cover.}

[In addition, the erosional landscape of the Skeena mountain bears one of the
strongest glacial imprint found within the area formerly covered by the
Cordilleran ice sheet \aref%
\footnote{\textbf{Arjen, Martin, Johan,} can we use a reference here to support
    this point (observations, mapping, shaded relief)? Alternatively, we could
    maybe use a photograph, if any of you has one. It would add a bit of
    variation into the figure list. \textbf{Johan}, you seemed to support this
    this point during our discussion, any idea?}.
Here we suggest that persistent ice cover (Fig.~\ref{fig:duration})
associated with basal ice temperatures at the pressure-melting point
(Fig.~\ref{fig:warmfrac} and ~\ref{fig:warmbase}%
\footnote{Fig.~\ref{fig:erosion} is tricky because it may be significantly
    affected by the constant-geothermal heat flux assumption, which is not
    valid for this region. Arjen, Johan and I decided to leave it for my kappa
    for now. Regarding Figs.~\ref{fig:warmfrac} and ~\ref{fig:warmbase}, I
    still wonder if it is useful to show both. Basically we have
    Fig.~\ref{fig:warmbase}\,=\,Fig.~\ref{fig:duration}\,*\,Fig.~\ref{fig:warmfrac}.
    Maybe the reader can perform this multiplication visually without need of
    Fig.~\ref{fig:warmbase}? Tell me what you think of it.})
may partly explain this spectacular landscape.]

A correlation is observed between the modelled duration of warm based ice cover
(Fig.~\ref{fig:warmbase}) and the degree of glacial modification of the
landscape (mainly in terms of the development of deep glacial valleys and
troughs). We find evidence for this on both the southwestern and northeastern
slopes of the southern Coast Mountains, on the western slopes of the Columbia
Mountains, on the western and eastern slopes of the northern Coast Mountains
and the Saint-Elias Mountains, and radiating off the Skeena Mountains
(Figs.~\ref{fig:duration}, \ref{fig:warmbase} and~X\renote{
    Yes it could be interesting to compare basal flow patterns with Fig.~2 of
    \citet{Kleman.etal.2010}, but I don't think that I have the time for this
    right now.}).
The Skeena Mountains, for example, indeed bear a strong glacial imprint that
indicates ice drainage in a system of distinct glacial troughs emanating in a
radial pattern from the centre of the mountain range (Fig. X). We suggest that
persistent ice cover (Fig.~\ref{fig:duration}) associated with basal ice
temperatures at the pressure-melting point (Figs.~\ref{fig:warmfrac}
and~\ref{fig:warmbase}) explains the large-scale glacial erosional imprint on
the landscape. In contrast, a well-developed network of glacial valleys west of
the MacKenzie Mountains \citep[Fig.~X;][]{Kleman.etal.2010,
Stroeven.etal.2010}, which is modelled to have hosted warm-based ice
(Fig.~\ref{fig:warmfrac}) has only been glaciated for a short duration
(Fig.~\ref{fig:duration}) during the last glacial cycle according to our
results (Figs.~\ref{fig:duration} and~\ref{fig:warmbase}). We therefore infer
that either our model does not perform reliably in this part of the domain or
the observed landscape pattern originates from older glacial cycles and
indicates an increased relative importance of this ice dispersal centre prior
to the Late Pleistocene \citep[cf.][]{Ward.etal.2008, Demuro.etal.2012}.

\todo{Paragraph on model uncertainties related to warm-based ice.}

% -----------------------------------------------------------------------------
\subsection{The last deglaciation}
% -----------------------------------------------------------------------------

\subsubsection{Pace and patterns of deglaciation}

In the North American Cordillera alike other glaciated regions, the large
majority of the glacial geologic record relates to the last few millennia of
glacier cover, most of the older evidence having been overridden by ice retreat
retreat during the deglaciation\footnote{\textbf{Johan,} can we refer to one
of your papers here?}.

\todo{add 1 sentence to connect to the next paragraph: that is easy:D}

The timing of peak ice volume during the last glacial maximum and the pacing of
deglaciation depend cirticially on the choice of atmospheric forcing
(Figs.~\ref{fig:timeseries} and~\ref{fig:deglacseries}). Adopting the EPICA
atmospheric forcing yields peak ice volume at 17.3\,ka and an uninterupted
deglaciation until 10.6\,ka (Fig.~\ref{fig:deglacseries}, lower panel, red
curves). The simulation driven by the GRIP palaeo-temperature record yields
peak ice volume at 19.1\,ka and a deglaciation interupted by two standstills or
readvances until 9.4\,ka. The first interuption occurs between 16.6 and
14.5\,ka, and the second between 12.6 and 11.6\,ka
(Fig.~\ref{fig:deglacseries}, lower panel, blue curve). Hence, the two model
runs, while similar in overall timing compared to runs with other climate
drivers, differ in detail in that the EPICA depicts peak glaciation almost
2\,ka later and shows a faster uninterrupted deglaciation which yields ice-free
conditions more than 1\,ka earlier.

\todo{From the timing of max glaciation and the duration of the deglaciation,
is there any evidence to prefer one model output above the other? If so, state
that, if not, state that too. The behavior of both models is perhaps reasonable
within the uncertainty envelope that the field evidence provides...?}

\subsubsection{Possibility for a late-glacial readvance}

The possibility of late glacial readvances in the North American Cordillera has
been debated for some time \citep{Osborn.Gerloff.1997}, and locally these have
been reconstructed and dated. Radiocarbon-dated end moraines in the Fraser and
Squamish valleys off the southern tip of the Coast Mountains, indicates
consecutive glacier maximal (or stillstands while in overall retreat), one of
which one corresponds to the Younger Dryas chronozone \citep{Clague.etal.1997,
Friele.Clague.2002, Friele.Clague.2002a, Kovanen.2002,
Kovanen.Easterbrook.2002}. In the Finlay River area of northern Rocky
Mountains, sharp-crested moraines indicate a late-glacial readvance of alpine
glaciers and their interaction with the main body of the decaying Cordilleran
ice sheet \citep{Lakeman.etal.2008}. Additional evidence for late-glacial
alpine glacier readvances includes moraines in the eastern Coast Mountains,
Rocky Mountains and the Columbia Mountains \citep{Osborn.Gerloff.1997,
Menounos.etal.2008}. Although further work is needed to constrain the timing of
the late-glacial readvance, to assess its extents and geographical
distribution, and to identify the potential climatic triggers
\citep{Menounos.etal.2008}, it is interesting to note that the simulation
driven by the GRIP record produces a late-glacial readvance in the Coast
Mountains, Rocky Mountains and the Finlay River area, corresponding to where it
has been identified in the geological record (Fig.~\ref{fig:deglac}, left
panel). In contrast the EPICA-driven simulation produces a nearly-continuous
deglaciation with only a highly restricted glacial readvance
(Fig.~\ref{fig:deglac}, right panel).

\todo{Note that most of the evidence for re-advance is for small glaciers,
      whereas the model output indicates that there was still a big body of ice
      at that time, as in \citet{Lakeman.etal.2008}.}

Predicted patterns of ice sheet retreat are relatively consistent between the
two simulations (Fig.~\ref{fig:deglac}). The southern sector of the modelling
domain, including the Puget Lowland, the Coast and the Rocky mountains, and the
Interior Plateau of British Columbia, becomes completely deglaciated by 12\,ka,
whereas a significant ice cover remains over the Skeena, Selwyn, MacKenzie,
Wrangell and Saint-Elias mountains in the northern sector of the modelling
domain. After 12\,ka, deglaciation continues to proceed across the Liard
Lowland with a radial ice margin retreat towards the surrounding mountain
ranges, and inward flow towards the depression, consistent with the regional
melt water record of the last deglaciation \citep{Margold.etal.2013}
\todo{rephrase with ``unzip''?}. Remaining ice continues to decay by retreating
towards the Selwyn and Skeena mountains. The last remnants of the Cordilleran
ice sheet finally disappears from the Skeena mountains around 10.6\,ka (EPICA)
and 9.4\,ka (GRIP).

\subsubsection{Deglacial flow directions}

Because a general tenant in glacial geomorphology is that the majority of
landforms (lineations and eskers) are part of the deglacial envelop
\citep[terminology from]{Kleman.etal.2006}, that is that these were formed
close inside the retreating margin of ice sheets \citep{Boulton.Clark.1990,
Kleman.etal.1997, Kleman.etal.2010}, we present a map of basal flow directions
immediately preceding deglaciation or at the time of cessation of sliding
inside the cold-base retreating margin (Fig.~\ref{fig:lastflow})
\todo{description of the general trends that the map shows}. Patterns of
glacial lineation formed in the northern and southern sectors of the
Cordilleran ice sheet and in the Liard Lowland (\citealp{Prest.etal.1968};
\citealp[Fig.~1.12]{Clague.1989}; \citealp[Fig.~2]{Kleman.etal.2010};
\citealp[Fig.~2]{Margold.etal.2013}), similarities with the patterns of
deglacial ice flow from modelling (Fig.~\ref{fig:lastflow}). However, this is
not the case for the Interior Plateau of British Columbia, where both
simulations predict negligible basal sliding during deglaciation
(Fig.~\ref{fig:lastflow}), but where an impressive set of glacial delineations
indicate a substantial eastward flow component of the Cordilleran ice sheet
\citep{Prest.etal.1968, Kleman.etal.2010}.

The Interior Plateau lineation swarm could thus present a smoking gun for the
reliability of the presented model results. One explanation for the incongruent
results could therefore be that the modelled LGM ice sheet is too thick, or
that the ice divide is positioned too far to the east \todo{descript the
problem for the incongruent results of these two conditions}. Because feedback
mechanisms between ice sheet topography and regional climate are absent in our
model, including wind redirection, orographic precipitation effects, and latent
warming of moisture-depleted air, we regard misfeeds in ice thickness and ice
devide location as plausible \citep{Seguinot.etal.2014}. Another explanation
for the incongruent results could be that the Interior Plateau lineation swarm
predates deglaciation and that deglacitation landforms are largely absent. The
modelled deglaciation of the Interior Plateau, central British Columbia,
consists of a rapid northwards retreat (Fig.~\ref{fig:deglac}) of
southwards-flowing (Fig.~\ref{fig:lastflow}) non-sliding ice lobes positioned
in-between deglaciated (ice-free) mountain ranges
(Figs.~\ref{fig:profiles-grip} and~\ref{fig:profiles-epica}). This result
appears compatible with the prevailing conceptual model of deglaciation of
central British Columbia, in which mountain ranges emerge from the ice before
the plateau \citep[Fig.~7]{Fulton.1991}. If it is valid, it may suggest that
glacial lineations on the Interior Plateau of British Columbia may be of older
age than the LGM, and may have remained intact throughout the deglaciation.


% =============================================================================
\conclusions
\label{sec:concl}
% =============================================================================

Numerical simulations of the Cordilleran ice sheet through the last glacial
cycle presented in this study consistently produce two glacial maxima during
MIS~4 (61.9--55.4\,ka) and MIS~2 (29.5--16.9\,ka), two periods
corresponding to documented extensive glaciations. This result is
independent of the choice of the palaeo-temperature record used to approximate
the past climate evolution, and thus
can be seen as a first-order agreement between the model and the geological
evidence. However, the timing of the two glaciation peaks depends sensitively
on which record
is used to drive the model. The timing of the LGM is best
reproduced by the Antarctic record, and occurs too early in all simulations
that are driven by the other records. The mismatch is greatest when using
oceanic records from the Pacific Northwest, which are affected by the
weakening of the California current during the LGM.

In all simulations presented here, ice cover is limited to disjoint mountain ice
caps during most of the glacial cycle, confirming previous inferences from the
geological evidence preceding the LGM. However, our
simulations produce persistent ice cover on the Skeena Mountains during
the entire glacial cycle. At the time when a full-size Cordilleran ice sheet is
absent, the Skeena ice cap appears to be fed by the eastwards precipitation
intrusion through a topographic window in the Coast Mountains. The ice cap acts
as a nucleation centre at the onset of the LGM readvance, and appears
consistent with the distinct glacial imprint of the Skeena Mountains landscape.

During deglaciation, none of the palaeo-temperature records used produces a
close agreement between the model results and the geological evidence. Although
the EPICA record yields a more realistic timing of the LGM and early
deglaciation, only the GRIP record produces a late-glacial readvance in areas
where it has been documented in the literature. Nonetheless, the
general patterns of deglaciation are consistent between both simulations driven
by the GRIP and EPICA records, and show a rapid deglaciation of the southern
half of the ice sheet, including a rapid northwards retreat across the Interior
Plateau of central British Columbia. This is followed by an opening of the ice
margin in the Liard Lowland, and a final retreat of the margin of the remaining
ice caps towards the Selwyn and, later, the Skeena mountains, which host the
last remnant of the ice sheet
during the early Holocene (10.9--9.5\,kyr).

These results are strongly dependent on selected ice-sheet model (PISM),
surface
mass balance model (PDD) and climate forcing. Most importantly, our simplistic
palaeo-climate forcing does not include precipitation corrections in response
to the presence of an ice cover, potentially leading to overestimated glacial
extent and volume in continental regions. Nevertheless, our results identify
the Skeena Mountains as a key area to understanding
glacial dynamics of the Cordilleran ice sheet, highlighting the need for
further geological investigation of this region.

% Author contributions
\section*{Author contributions}
\dots

% Acknowledgements
\begin{acknowledgements}
We are very thankful to Constantine Khroulev, Ed Bueler and Andy Aschwanden for
providing constant help and development on PISM. This work was supported by the
Swedish Research Council~(VR) grant no. 2008-3449 to A.~P.~Stroeven, by the
German Academic Exchange Service~(DAAD) grant no.~50015537 and a Knut and Alice
Wallenberg Foundation grant to J.~Seguinot.
Computer resources were provided by the Swedish National
Infrastructure for Computing (SNIC) allocation no. 2013/1-159 to A.~P.~Stroeven
at the National Supercomputing Center (NSC).
\end{acknowledgements}

% References
\bibliographystyle{copernicus}
\bibliography{refs/references.bib}
\newpage

% =============================================================================
% Floats
% =============================================================================

% tab:records
\begin{table*}[t]
  \caption{Palaeo-temperature proxy records and scaling parameters yielding
           temperature offset time-series used to force the ice sheet model
           through the last glacial cycle (Fig.~\ref{fig:timeseries}). $f$
           corresponds to the scaling factor adopted to yield last glacial
           maximum ice limits in the vicinity of mapped end moraines, and
           $T_{[32, 22]}$ refers to the resulting mean temperature anomaly
           during the period -32 to~-22~ka after scaling.}
  \label{tab:records}
  {\begin{tabular}{l|ccc|ccc|l}
    \tophline
    Record & Latitude & Longitude & Elev. & Proxy & $f$ & $T_{[32, 22]}$
           & Reference\\
    & & & (m~a.s.l) & & & (K) & \\
    \middlehline
    GRIP     &  72{\degree} 35' N  % 72.58 (decimal)
             &  37{\degree} 38' W  % 37.64 (decimal)
             & 3238\,m
             & \chem{\delta^{18}O}
             & 0.35 & -5.8{\degree}C  % -16.4126 (before scaling)
             & \citet{Dansgaard.etal.1993} \\

    NGRIP    &  75{\degree} 06' N  % 75.10
             &  42{\degree} 19' W  % 42.32
             & 2917\,m
             & \chem{\delta^{18}O}
             & 0.22 & -6.1{\degree}C  % -26.7098
             & \citet{Andersen.etal.2004} \\

    EPICA    &  75{\degree} 06' S  % 75.1
             & 123{\degree} 21' E  % 123.35
             & 3233\,m
             & \chem{\delta^{18}O}
             & 0.60 & -5.6{\degree}C  % -9.2055
             & \citet{Jouzel.etal.2007} \\

    Vostok   &  78{\degree} 28' S  % 78.8
             & 106{\degree} 50' E  % 106.8
             & 3488\,m
             & \chem{\delta^{18}O}
             & 0.70 & -5.6{\degree}C  % -7.9550
             & \citet{Petit.etal.1999} \\

    ODP~1012 &  32{\degree} 17' N
             & 118{\degree} 23' W
             & -1772\,m
             & \chem{U^{K'}_{37}}
             & 1.53 & -5.8{\degree}C  % -3.7889
             & \citet{Herbert.etal.2001} \\

    ODP~1020 &  41{\degree} 00' N
             & 126{\degree} 26' W
             & -3038\,m
             & \chem{U^{K'}_{37}}
             & 1.16 & -5.8{\degree}C  % -5.0000
             & \citet{Herbert.etal.2001} \\
    \bottomhline
  \end{tabular}}
  \belowtable{}
\end{table*}

% tab:records
\begin{table*}[t]
  \caption{Extrema of ice volume and extent corresponding to MIS~4, 3 and 2 for
           each of the low-resolution simulations (Fig.~\ref{fig:timeseries}).}
  \label{tab:extrema}
  {\begin{tabular}{l*{3}{|ccc}}
    \tophline
             & \multicolumn{3}{c}{Age (ka)}
             & \multicolumn{3}{c}{Ice extent (\unit{10^6\,km^2})}
             & \multicolumn{3}{c}{Ice volume (m~s.l.e.)} \\
    Record   &  MIS~4 &  MIS~3 &  MIS~2
             &  MIS~4 &  MIS~3 &  MIS~2
             &  MIS~4 &  MIS~3 &  MIS~2 \\
    \middlehline
    GRIP     & -57.58 & -49.24 & -19.52
             &   1.98 &   0.46 &   2.13
             &   7.52 &   0.89 &   8.52 \\
    NGRIP    & -60.26 & -50.16 & -22.85
             &   2.16 &   0.50 &   2.09
             &   8.69 &   0.93 &   8.24 \\
    EPICA    & -61.87 & -45.57 & -17.10
             &   1.57 &   0.95 &   2.08
             &   5.20 &   2.44 &   8.35 \\
    Vostok   & -60.87 & -49.68 & -16.87
             &   1.55 &   0.86 &   2.14
             &   5.10 &   2.01 &   8.66 \\
    ODP 1012 & -55.41 & -47.08 & -23.21
             &   1.44 &   0.85 &   2.13
             &   4.50 &   2.06 &   8.46 \\
    ODP 1020 & -60.16 & -52.24 & -29.46
             &   1.32 &   0.70 &   2.08
             &   3.88 &   1.52 &   8.32 \\
    \middlehline
    Minimum  & -61.87 & -52.24 & -29.46
             &   1.32 &   0.46 &   2.08
             &   3.88 &   0.89 &   8.24 \\
    Maximum  & -55.41 & -45.57 & -16.87
             &   2.16 &   0.95 &   2.14
             &   8.69 &   2.44 &   8.66 \\
    \bottomhline
  \end{tabular}}
  \belowtable{}
\end{table*}

% fig:locmap
\begin{figure}
  \includegraphics{locmap}
  \caption{Relief map of northern North America showing a reconstruction of the
           areas once covered by the Cordilleran (CIS), Laurentide (LIS),
           Innuitian (IIS) and Greenland (GIS) ice sheets during the last
           18\,\unit{\chem{^{14}C}\,kyr\,BP} (21.4\,cal\,kyr\,BP)
           \citep{Dyke.2004}. The rectangular box denotes the location of the
           modelling domain used in this study. Major mountain ranges covered
           by the ice sheet include the Alaska Range (AR), the Wrangell and
           St.-Elias mountains (WSE), the Selwyn and MacKenzie mountains (SMK),
           the Skeena Mountains (SM), the Coast Mountains (CM), the Rocky
           Mountains (RM) and the North Cascades (NC). The background
           map consists of ETOPO1 \citep{Amante.Eakins.2009} and Natural Earth
           Data \citep{Patterson.Kelso.2014}.
           \todo{Mark palaeo-ice sheets and mountain ranges on the map.
                 Mark location of the Puget Lowland.}}
  \label{fig:locmap}
\end{figure}

% fig:atm
\begin{figure}
  \includegraphics{atm}
  \caption{Monthly mean near-surface air temperature, precipitation and
           standard deviation of daily mean temperature for January and July
           months from the North American Regional Reanalysis (NARR)
           climatology, used to force the ice sheet model. Note the
           strong contrasts in seasonality, timing of the precipitation peak,
           and temperature variability over the model domain, notably between
           the maritime and continental regions.
           \todo{Dash boundaries between ice sheets?}}
  \label{fig:atm}
\end{figure}

% fig:timeseries
\begin{figure*}
  \includegraphics{timeseries}
  \caption{Temperature offset time-series from ice core and sediment core
           records (Table~\ref{tab:records}) used as palaeo-climate forcing for
           the ice sheet model \textbf{(top)}, and modelled ice volume
           through the last 120\,kyr, expressed in meters of sea-level
           equivalent \textbf{(bottom)}. Gray spans indicate Marine Isotope
           Stages (MIS) according to a global compilation of benthic
           \chem{\delta^{18}O} records \citep{Lisiecki.Raymo.2005}. Hatched
           rectangles highlight modelled ice volume extrema corresponding to
           MIS~4 (61.9--55.4\,kyr), MIS~3 (52.2--45.6\,kyr), and
           MIS~2 (last glacial maximum, 29.5--16.9\,kyr). Dotted lines
           correspond to the GRIP and EPICA 6\,km-resolution runs.}
  \label{fig:timeseries}
\end{figure*}

% fig:snapshots
\begin{figure*}
  \includegraphics{snapshots}
  \caption{Snapshots of modelled surface topography (500\,m contours)
           corresponding to the ice volume extrema indicated on
           Fig.~\ref{fig:timeseries}. Note the occurence of spatial similarities
           despite large differences in timing.
           \todo{Perhaps indicate the location of the Skeena Mountains.}}
  \label{fig:snapshots}
\end{figure*}

% fig:duration
\begin{figure*}
  \includegraphics{duration}
  \caption{Modelled duration of ice cover during the last 120\,kyr.
           Note the irregular colour scale. A contiguous ice cover spanning
           from the Alaska Range (AR) to the southern Coast Mountains (CM) and
           Rocky Mountains (RM) exists for about 29\,kyr in both
           simulations. A central
           ice cover persists over the Skeena Mountains (SM) during most of the
           simulation. On the other hand, the maximal extent of the ice sheet
           generally corresponds to relatively short durations of ice cover.}
  \label{fig:duration}
\end{figure*}

% fig:warmfrac
\begin{figure*}
  \includegraphics{warmfrac}
  \caption{Modelled fraction of warm-based ice cover during the ice-covered
           period. Note the dominance of warm-based conditions on the
           continental shelf and major glacial troughs of the coastal ranges.
           Hatches indicate areas that were covered by cold ice only.
           \todo{indicate location of the Skeena Mountains.}}
  \label{fig:warmfrac}
\end{figure*}

% fig:warmbase
\begin{figure*}
  \includegraphics{warmbase}
  \caption{Modelled duration of warm-based ice cover during the last
           120\,kyr. Long ice cover durations combined with basal
           temperatures at the pressure-melting point may explain the strong
           glacial erosional imprint of the Skeena Mountains (SM) landscape.
           Hatches indicate areas that were covered by cold ice only.
           \todo{indicate location of the Skeena Mountains.}}
  \label{fig:warmbase}
\end{figure*}

% fig:erosion
\begin{figure*}
  \includegraphics{erosion}
  \caption{Modelled cumulative basal displacement (integrand of basal velocity)
           over the last 120\,kyr.
           \todo{remove this figure (see footnote in text).}}
  \label{fig:erosion}
\end{figure*}

% fig:deglacseries
\begin{figure}
  \includegraphics{deglacseries}
  \caption{Temperature offset time-series from the GRIP and EPICA ice core
           records (Table~\ref{tab:records}) \textbf{(top)}, and modelled ice
           volume during the deglaciation, expressed in meters of sea-level
           equivalent \textbf{(bottom)}.}
  \label{fig:deglacseries}
\end{figure}

% fig:deglac
\begin{figure*}
  \includegraphics{deglac}
  \caption{Modelled age of the last deglaciation. Areas where the MIS~4 glacial
           advance exceeded the last glacial maximum advanced are marked in
           green. Hatches denote re-advance of mountain-centred ice caps and
           and the decaying ice sheet between 14 and 10\,kyr., which is more
           pronounced in the GRIP-driven simulation.}
  \label{fig:deglac}
\end{figure*}

% fig:deglacshots-grip
\begin{figure*}
  \includegraphics{deglacshots-grip}
  \caption{Snapshots of modelled surface topography (200\,m contours)
           and surface velocity (colour mapping) from the GRIP simulation,
           corresponding to the last glacial ice volume maximum (19.1\,kyr) and
           the last deglaciation.}
  \label{fig:deglacshots-grip}
\end{figure*}

% fig:deglacshots-epica
\begin{figure*}
  \includegraphics{deglacshots-epica}
  \caption{Snapshots of modelled surface topography (200\,m contours)
           and surface velocity (colour mapping) from the EPICA simulation,
           corresponding to the last glacial ice volume maximum (17.3\,kyr) and
           the last deglaciation.}
  \label{fig:deglacshots-epica}
\end{figure*}


% fig:profiles-grip
\begin{figure}
  \includegraphics{profiles-grip}
  \caption{Modelled bedrock (black) and ice surface (blue) topography profiles
           during deglaciation (22.0--8.0\,kyr) in the GRIP 6\,km
           simulation, corresponding to the four transects indicated in
           Fig.~\ref{fig:deglac}.}
  \label{fig:profiles-grip}
\end{figure}

% fig:profiles-epica
\begin{figure}
  \includegraphics{profiles-epica}
  \caption{Modelled bedrock (black) and ice surface (blue) topography profiles
           during deglaciation (22.0--8.0\,kyr) in the EPICA 6\,km
           simulation, corresponding to the four transects indicated in
           Fig.~\ref{fig:deglac}.}
  \label{fig:profiles-epica}
\end{figure}

% fig:lastflow
\begin{figure*}
  \includegraphics{lastflow}
  \caption{Modelled directions of the deglacial basal flow velocities. Hatches
           indicate areas that remain non-sliding throughout deglaciation
           (22.0--8.0\,kyr), notably including the Interior Plateau of central
           British Columbia.
           Sliding grid cells were distinguished from non-sliding grid cells
           using a velocity threshold of 1\,\unit{m\,yr^{-1}}.}
  \label{fig:lastflow}
\end{figure*}

% =============================================================================
\end{document}
\endinput
% =============================================================================
