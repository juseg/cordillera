% cordillera-climate.tex
% =============================================================================

% Copernicus manuscript
\documentclass[tc, manuscript]{copernicus}

% Copernicus final print
%\documentclass[tc]{copernicus}

% Copernicus discussion paper
%\documentclass[tcd, hvmath, online]{copernicus}

% Copernicus-like latex2rtf compatible
%% copernicus_rtf.tex
% ------------------

% Base class and packages
\documentclass{article}
\usepackage{color}
\usepackage{geometry}
\usepackage{graphicx}
\usepackage{setspace}
\onehalfspacing

% Replacements for bibtex commands
\newcommand{\citep}[1]{(\textcolor{blue}{#1})}
\newcommand{\citet}[1]{\textcolor{blue}{#1}}

% Replacements for Copernicus commands
\newcommand{\introduction}[0]{\section{Introduction}}
\newcommand{\conclusions}[0]{\section{Conclusions}}
\newcommand{\tophline}[0]{\hline}
\newcommand{\middlehline}[0]{\hline}
\newcommand{\bottomhline}[0]{\hline}
\newcommand{\unit}[1]{\ensuremath{\mathrm{#1}}}
\newcommand{\degree}[0]{\ensuremath{^{\circ}}}

% Ignore other Copernicus commands
\newcommand{\runningtitle}[1]{}
\newcommand{\runningauthor}[1]{}
\newcommand{\received}[1]{}
\newcommand{\correspondence}[1]{}
\newcommand{\pubdiscuss}[1]{}
\newcommand{\revised}[1]{}
\newcommand{\accepted}[1]{}
\newcommand{\published}[1]{}



% Coloured hyperlinks
\hypersetup{colorlinks, citecolor=blue}

% Figure directory
\graphicspath{{figures/}}

% My commands
\newcommand{\renote}[1]{\footnote{\textbf{Comment reply}: #1}}
\newcommand{\todo}[1]{\emph{[\textbf{Todo:} #1]}}
\newcommand{\aref}[0]{\textbf{[ref.]}}

% =============================================================================
\begin{document}
\linenumbers
% =============================================================================

% Title
\title{Numerical simulations of the Cordilleran ice sheet
       through the last glacial cycle}

% Authors
\Author[1,2]{J.}{Seguinot}
\Author[2]{I.}{Rogozhina}
\Author[3]{M.}{Margold}
\Author[1]{J.}{Kleman}
\Author[1]{A.~P.}{Stroeven}
\runningauthor{J.~Seguinot et~al.}
\correspondence{J.~Seguinot (julien.seguinot@natgeo.su.se)}

% Running title
\runningtitle{Climate forcing for Cordilleran ice sheet simulations}

% Affiliations
\affil[1]{Department of Physical Geography and Quaternary Geology and the
          Bolin Centre for Climate Research, Stockholm University,
          Stockholm, Sweden}
\affil[2]{Helmholtz Centre Potsdam, GFZ German Research Centre for Geosciences,
          Potsdam, Germany}
\affil[3]{Department of Geography, Durham University, UK}

% For Copernicus
\received{}
\pubdiscuss{}
\revised{}
\accepted{}
\published{}

% Title
\firstpage{1}
\maketitle

% Abstract
\begin{abstract}

  Despite more than a century of geological observations, the Cordilleran ice
  sheet of North America remains poorly understood in terms of its former
  extent, volume and dynamics. Although geomorphological evidence is abundant,
  its complexity is such that whole ice-sheet reconstructions of advance and
  retreat patterns are lacking. Here we use a numerical ice sheet model
  calibrated against field-based evidence to attempt a quantitative
  reconstruction of the Cordilleran ice sheet history through the last glacial
  cycle. A series of simulations is driven by time-dependent temperature
  offsets from six proxy records located around the globe. Although this
  approach reveals large variations in model response to evolving atmospheric
  forcing, all simulations produce two major glaciation events during
  marine isotope stages~4 (61.9--55.4\,kyr) and~2
  (29.5--16.9\,kyr). The timing of glaciation is
  better reproduced using temperature reconstructions from Greenland and
  Antarctic ice cores than from regional oceanic sediment cores. During most of
  the last glacial cycle, the modelled ice cover is discontinuous and
  restricted to high mountain areas. However, widespread precipitation over the
  Skeena Mountains favours the persistence of a central ice dome throughout the
  glacial cycle. It acts as a nucleation centre before the last glacial maximum
  and hosts the last remains of Cordilleran ice during the
  early Holocene (10.9--9.5\,kyr).

\end{abstract}

% =============================================================================
\introduction
\label{sec:intro}
% =============================================================================

During the last glacial cycle, glaciers and ice caps of the North American
Cordillera have been more extensive than today. At the Last Glacial
Maximum (LGM\renote{
    I decided to adopt ``LGM'', which is very common, but not ``CIS'', for the
    sake of readability. This corresponds to abbreviations used in
    \citet{Seguinot.etal.2014} (and now, in the kappa).}),
a continuous blanket of ice, the Cordilleran ice sheet\renote{
    In \citet{Seguinot.etal.2014}, The Cryosphere changed all my occurences of
    Cordilleran (Greenland etc) \textbf{I}ce \textbf{S}heet to lowercase
    \textbf{i}ce \textbf{s}heet. Let's keep it so for now for consistency
    within the thesis (kappa + papers).}
\citep{Dawson.1888}, extended from the Alaska Range in the north to the
North Cascades in the south. In addition, it extended offshore, where it calved
into the Pacific Ocean, and merged with the western margin of its much larger
neighbour, the Laurentide ice sheet, east of the Rocky Mountains
(Fig.~\ref{fig:locmap}).

More than a century of exploration and geological investigations of the
Cordilleran mountains have led to many observations in support of the former
ice sheet
    \citep{Jackson.Clague.1991}.
Despite the lack of documented end moraines offshore, in the zone of confluence
with the Laurentide ice sheet, and in areas swept by the Missoula floods
    \citep{Carrara.etal.1996},
moraines that demarcate the south-western and north-eastern margins provide key
constraints that allow reasonable reconstructions of maximum ice sheet extents
    (\citealp{Prest.etal.1968}; \citealp[Fig. 1.12]{Clague.1989};
     \citealp{Duk-Rodkin.1999};
     \citealp{Booth.etal.2003}; \citealp{Dyke.2004}).
The LGM Cordilleran ice sheet maximum extent was short-lived, as indicated by
field evidence from radiocarbon dating
    \citep{Clague.etal.1980, Clague.1985, Clague.1986, Porter.Swanson.1998,
           Menounos.etal.2008},
cosmogenic exposure dating
    \citep{Stroeven.etal.2010, Stroeven.etal.2014, Margold.etal.2014},
bedrock deformation in response to former ice loads
    \citep{Clague.James.2002, Clague.etal.2005},
and offshore sedimentary records
    \citep{Cosma.etal.2008, Davies.etal.2011}.
However, former ice thicknesses and, therefore, the ice sheet's contribution to
the LGM sea level low stand
    \citep{Carlson.Clark.2012, Clark.Mix.2002}
remain uncertain.

Our understanding of the Cordilleran glaciation history prior to the LGM is
even more fragmentary
    \citep{Barendregt.Irving.1998, Kleman.etal.2010, Rutter.etal.2012},
although it is clear that maximum glaciation of the Cordilleran ice sheet
predates the last glacial cycle
    \citep{Hidy.etal.2013}.
In parts of the Yukon Territory and Alaska, the distribution of tills
    \citep{Turner.etal.2013}
and dated glacial erratics
    \citep{Ward.etal.2007, Ward.etal.2008, Briner.Kaufman.2008,
           Stroeven.etal.2010, Stroeven.etal.2014}
indicate an extensive Marine Oxygen Isotope Stage (MIS)~4 glaciation,
yet it is not known whether other regions in the study area were affected.
Landforms in the interior regions include flow sets that are likely
older than the LGM
    \citep[Fig.~2]{Kleman.etal.2010},
but their absolute age remains uncertain.

In contrast, evidence for the deglaciation history of the Cordilleran
ice sheet since the LGM is considerable, albeit mostly at a regional scale.
Geomorphological evidence from south-central British Columbia indicates a rapid
deglaciation, including an early emergence of the elevated areas while thin,
stagnant ice still covered the surrounding lowlands
    \citep{Fulton.1967, Fulton.1991, Margold.etal.2011, Margold.etal.2013a}.
This model, although credible, may not apply in all areas of the Cordilleran
ice sheet
    \citep{Margold.etal.2013}.
Although solid evidence for late-glacial glacier re-advances have been found in
the Coast, Columbia and Rocky mountains
    \citep{Clague.etal.1997, Friele.Clague.2002, Friele.Clague.2002a,
           Kovanen.2002, Kovanen.Easterbrook.2002, Lakeman.etal.2008,
           Menounos.etal.2008},
it appears to be more sparse than for formerly glaciated regions surrounding
the North Atlantic \aref\footnote{
    Do we need to add a couple of refs. from Britain, Scandinavia, the Alps,
    Eastern America here? Or a review paper? Could yuo help me with this?},
thus allowing for considerable uncertainty concerning
the possibility of a regional late glacial cold reversal.

In general, the topographic complexity of the North American Cordillera and its
effect on glacial history have inhibited the construction of ice sheet-wide
glacial advance and retreat patterns such as
those available for the Fennoscandian and Laurentide ice sheets
     \citep{Boulton.etal.2001, Dyke.Prest.1987, Dyke.etal.2003,
            Kleman.etal.1997, Kleman.etal.2010}.
Here, we use a numerical ice sheet model \citep{PISM-authors.2014},
calibrated against field-based evidence, to perform a quantitative
reconstruction of the Cordilleran ice sheet history through the last glacial
cycle, and
analyse some of the long-standing questions related to its evolution:

\begin{itemize}
  \item How much ice was locked in the Cordilleran ice sheet during the
    last glacial maximum?
  \item What was the scale of glaciation prior to the last glacial maximum?
  \item Which were the primary dispersal centres? Do they reflect stable or
    ephemeral configurations?
  \item How rapid was the last deglaciation? Did it include late glacial
    standstills or readvances?
\end{itemize}

Although numerical ice sheet modelling has been established as a useful tool to
improve our understanding of the Cordilleran ice sheet
    (\citealp[p.~227]{Jackson.Clague.1991}; \citealp{Robert.1991},
     \citealp{Marshall.etal.2000}),
the ubiquitously mountainous
topography of the region has presented two major challenges to its application.
First, only recent development of numerical ice sheet models and underlying
scientific computing tools \citep{Bueler.Brown.2009, Balay.etal.2014} has
allowed for high-resolution numerical modelling of glaciers and ice sheets on
mountainous terrain
over millenial time scales \citep[e.g.,][]{Golledge.etal.2012}. Second, the
complex
topography of the North American Cordillera also induces strong gradients in
seasonality and distinct patterns of precipition, thus requiring the use of
high-resolution atmospheric forcing fields as an input to the ice sheet model
\citep{Seguinot.etal.2014}.

Because climate conditions over the last glacial cycle are subject to
considerable uncertainty, our palaeo-climate forcing is a simplistic approach
including temperature and precipitation fields derived from a
present-day atmospheric reanalysis \citep{Mesinger.etal.2006,
Seguinot.etal.2014} supplemented by lapse-rate corrections
and temperature offset time series. The latter are obtained by scaling six
different palaeo-temperature reconstructions from proxy records around the
globe, including two oxygen isotope records from Greenland ice cores
\citep{Dansgaard.etal.1993, Andersen.etal.2004}, two oxygen isotope
records from Antarctic ice cores \citep{Petit.etal.1999,Jouzel.etal.2007},
and two alkenone unsaturation index records from Northwest Pacific ocean
sediment cores \citep{Herbert.etal.2001}. We then proceed to compare the model
output to geological evidence and discuss the timing and extent of glaciation
and the patterns of deglaciation.


% =============================================================================
\section{Model setup}
\label{sec:model}
% =============================================================================

% -----------------------------------------------------------------------------
\subsection{Overview}
\label{sec:overview}
% -----------------------------------------------------------------------------

The simulations presented here were run using the Parallel Ice Sheet Model
(PISM, development version~11b0a7f and stable version~0.6.1), an open source,
finite difference, shallow ice sheet model \citep{PISM-authors.2014}. The model
requires input on basal topography, sea level, geothermal heat flux and
atmospheric\renote{
    Changed here from ``climate'' to ``atmospheric''. Although I have used
    ``climate forcing'' in the previous paper, I now think that ``atmospheric
    forcing'' is more appropriate: ``climate'' may also encompass the oceanic
    component, which is not assessed here.}
forcing. It computes the evolution of ice extent
and thickness over time, the thermal and dynamic\renote{
    Thermal = temperature, dynamic = force + movement. If I meant movement
    alone, I would use ``cinematic''. Here we could use ``ice temperatures,
    stress and deformation (i.e. velocities)'', but that becomes a bit clumsy,
    I feel.}
states of the ice sheet, and the associated lithospheric response.

Basal topography is derived from the ETOPO1 combined topography and bathymetry
dataset with a~resolution of 1\,arc-min \citep{Amante.Eakins.2009}. Sea level
is lowered as a function of time based on the Spectral Mapping Project\renote{
    I had no time to check for later products, but if you have suggestions for
    something more up-to-date, I can try to include it in later simulations
    (everything will have to be re-run anyway).}
\citep[SPECMAP;][]{Imbrie.etal.1989} time scale. Geothermal heat flux
is applied as a constant value of 70\,\unit{mW\,m^{-2}} at 3\,km depth
(Sect.~\ref{sec:icedyn}). Surface mass balance is computed using a positive
degree-day (PDD) model (Sect.~\ref{sec:surface}). Atmospheric forcing is
provided by a monthly climatology from the North American Regional Reanalysis
\citep[NARR;][]{Mesinger.etal.2006} perturbated by time-dependent and
lapse-rate temperature corrections (Sect.~\ref{sec:atm}).

Each simulation starts from assumed ice-free conditions at 120000 years ago
(120\,ka), and runs to the present. Our modelling domain of 1500 by 3000\,km
encompasses the entire area covered by the Cordilleran ice sheet at the LGM
(Fig.~\ref{fig:locmap}). The simulations were run on two distinct grids, using
a lower horizontal resolution of 10\,km, and a higher horizontal resolution of
6\,km. These computations were performed on 16 to 128 computing cores at the
Swedish National Supercomputing Centre.

% -----------------------------------------------------------------------------
\subsection{Ice thermodynamics}
\label{sec:icedyn}
% -----------------------------------------------------------------------------

Ice sheet dynamics are typically modelled using a combination of internal
deformation and basal sliding. PISM is a~shallow ice sheet model, which implies
that the balance of stresses is approximated based on their predominant
components. The Shallow Shelf Approximation (SSA) is used as a ``sliding law''
for the Shallow Ice Approximation (SIA) by adding velocity solutions of the
two approximation
\citep[Eqns.~7--9 and 15]{Bueler.Brown.2009, Winkelmann.etal.2011}\renote{
    Arjen, I don't agree with your comment. Reference to equation number(s)
    show the reader where to look. \citet{Winkelmann.etal.2011} is a rather
    long paper that also treats many other aspects of the model. Their SIA+SSA
    approach builds on \citep{Bueler.Brown.2009}, but it is slightly different.
    It is the one in use in PISM, and in my simulations.}.
Ice rheology depends on temperature and water content through an enthalpy
formulation \citep{Aschwanden.etal.2012}. Surface air temperature derived from
the atmospheric forcing (Sect.~\ref{sec:atm}) provides the upper boundary
condition to the ice enthalpy model. Temperature is computed subglacially to
a~depth of 3\,km\renote{
    Yes, I changed this from 1 to 3\,km as we now model longer period of times
    than in the previous paper. The penetration depth of temperature anomalies
    (i.e. skin depth) depends on the period of temperature oscillations at the
    surface. The longer the period (in our case 100\,kyr), the deeper the
    temperature change. A good way to estimate what this parameter should be is
    to solve the 1-D temperature diffusion equation in the case of periodic
    surface forcing, using rock heat capacity and rock thermal conductivity
    values similar as in the numerical model. This basically forms the basis
    for my choice of the 3\,km value.},
where it is conditioned by a lower boundary geothermal heat flux of
70\,\unit{mW\,m^{-2}}. Although this uniform value does not account for the
high spatial geothermal variability in the region
\citep{Blackwell.Richards.2004}, it is, on average, representative of available
heat flow measurements. In the low-resolution simulations, the vertical grid
consists of up to 51~enthalpy layers in the ice sheet and 31~temperature layers
in the bedrock. In the high-resolution simulations, up to 101~ice layers, and
61~bedrock layers are used.

A~pseudo-plastic sliding law \citep{Bueler.Pelt.2014} relates the
bed-parallel shear stresses to the sliding velocity\footnote{
    Should I pull in this equation, too?}.
The yield stress, $\tau_c$,
is modelled using the Mohr--Coulomb criterion,
\begin{equation}
   \tau_c = c_0 + N\,\tan{\phi} \,,
\end{equation}
where cohesion, $c_0$, is assumed to be zero. The friction angle, $\phi$,
varies from 15 to 45{\degree}\renote{
    Different from the previous paper, but fairly similar in effect (see
    $\alpha$ and $\delta$ parameter values and corresponding equations in the
    kappa, basal sliding section). These values are is based on a quick
    sensitivity test on Greenland.}.
It is taken to be a piecewise-linear function of modern bed elevation, with
the lowest value below modern sea level (0\,m above sea level, m~a.s.l) and the
highest value above the approximate elevation of highest shorelines
(200\,m~a.s.l), thus accounting for a~weakening of
till associated with the presence of marine sediments. Effective pressure, $N$,
is related to the ice overburden stress, $\rho gh$, and the modelled amount of
subglacial water, using a formula derived from laboratory experiments with till
extracted from an Antarctic ice stream \citep{Tulaczyk.etal.2000,
Bueler.Pelt.2014},
\begin{equation}
    N = \delta \rho gh \, 10^{(e_0/C_c) (1 - (W/W_{max}))} \,,
\end{equation}
where $delta$ is chosen as 0.02, $e_0$ is a measured reference void ratio and
$C_c$ a measured compressibility coefficient (Table~\ref{tab:params}). The
amount of water at the base, $W$, varies from zero to $W_{max}=2$\,m, a
threshold above which instantaneous drainage is assumed.
Finally, the bedrock topography responds to ice load\renote{
    I disagree with writing ``the isostatic response''. The main point in
    using this model is that computes a non-isostatic response.}
following a bedrock deformation model that includes point-wise isostasy,
elastic lithosphere flexure and viscous mantle deformation in the lower
half-space\renote{
    I had put an emphasis on semi-infinite as this is, from a mathematical
    point of view, the key assumption of that model, allowing effective
    solution in Fourier domain. This is the whole point of Ed's paper and
    probably the reason why he picked it for PISM. But yes, I guess that
    ``lower half-space'' is fine, too...}
\citep[Table~\ref{tab:params};][]{Lingle.Clark.1985,Bueler.etal.2007}.

% -----------------------------------------------------------------------------
\subsection{Surface mass balance}
\label{sec:surface}
% -----------------------------------------------------------------------------

\renote{
    Surface mass balance is the balance of mass fluxes at the surface,
    consisting of ice accumulation and ice ablation, which I introduce straight
    away. I don't see what kind of ``opening sentence'' can be added here.}
Ice surface accumulation and ablation are computed from monthly mean
near-surface air temperature, $T_m$, monthly standard deviation of near-surface
air temperature, $\sigma$, and monthly precipitation, $P_m$, using
a~temperature-index model \citep[e.g.,][]{Hock.2003}. Accumulation is equal to
precipitation when air temperatures are below 0\,\unit{{\degree}C}, and
decreases to zero linearly with temperatures between 0 and
2\,\unit{{\degree}C}. Ablation is computed from the number of positive
degree-days (PDD), defined as the integral of temperatures above
0\,\unit{{\degree}C} in one year.

The PDD computation accounts for stochastic temperature variations by assuming
a normal temperature distribution of standard deviation $\sigma$ aroung the
expected value $T_m$. It is expressed by an error-function formulation
\citep{Calov.Greve.2005},
\begin{equation}
    \label{eqn:calovgreve}
    \mathrm{PDD} = \int_{t_1}^{t_2} \mathrm{d}t
        \left[\frac{\sigma}{\sqrt{2\pi}}
                \exp\left({-\frac{T_{m}^2}{2\sigma^2}}\right)
              + \frac{T_{m}}{2} \, \mathrm{erfc}
                \left(-\frac{T_{m}}{\sqrt{2}\sigma}\right)\right] \,,
\end{equation}
which is numerically approximated using week-long sub-intervals. In order to
account for the effects of spatial and seasonal variations of temperature
variability \citep{Seguinot.2013}, $\sigma$ is computed from daily temperature
values from the North American Regional Reanalysis
\citep[NARR,][]{Mesinger.etal.2006}, after excluding variability associated
with the seasonal cycle itself \citep[cf.][]{Seguinot.Rogozhina.2014}\renote{
    I may add a figure in the kappa if I find time, but I don't think that we
    should do it here. The subject is too marginal for the present study.}.
Degree-day factors for snow and ice melt are derived from
mass-balance measurements on contemporary glaciers from the Coast Mountains and
Rocky Mountains in British Columbia
\citep[Table~\ref{tab:params};][]{Shea.etal.2009}.

% -----------------------------------------------------------------------------
\subsection{Atmospheric forcing}
\label{sec:atm}
% -----------------------------------------------------------------------------

Atmospheric forcing of the model consists of a present-day monthly climatology%
\renote{
    ``Climatology'' is a common term in meteorology, used to design a set
    of variables aggregated over several decades, and assumed to represent
    some climate state.},
$\{T_{m0}, P_{m0}\}$, where temperatures are modified by offset time series,
${\Delta}T_{TS}$, and lapse-rate corrections, ${\Delta}T_{LR}$:
\begin{align}
    T_m(t, x, y) &= T_{m0}(x, y) + {\Delta}T_{TS}(t)
                    + {\Delta}T_{LR}(t, x, y) \,, \\
    P_m(t, x, y) &= P_{m0}(x, y) \,.
\end{align}

The present-day monthly climatology, $\{T_{m0}, P_{m0}\}$, was computed from
near-surface air temperature and precipitation rate fields from the NARR
\citep{Mesinger.etal.2006}, averaged between 1979 and 2000. Modern climate of the
North American Cordillera is characterised by strong geographic variations in
temperature seasonality, timing of the maximum annual precipitation, and
daily temperature variability (Fig.~\ref{fig:atm}).
Our choice of data from the NARR is motivated by the need for an accurate,
high-resolution precipitation forcing, as identified in a previous sensitivity
study \citep{Seguinot.etal.2014}.

Temperature offset time-series, ${\Delta}T_{TS}$, are derived from
palaeo-temperature proxy records from
the Greenland Ice Core Project \citep[GRIP,][]{Dansgaard.etal.1993}, the
North Greenland Ice Core Project \citep[NGRIP,][]{Andersen.etal.2004},
the European Project for Ice Coring in Antarctica \citep[EPICA,][]
{Jouzel.etal.2007}, the Vostok ice core \citep{Petit.etal.1999}, and Ocean
Drilling Program (ODP) sites 1012 and 1020, both located off the coast of
California \citep{Herbert.etal.2001}. Palaeo-temperatures from the GRIP and
NGRIP
records were calculated from oxygen isotope (\chem{\delta^{18}O}) measurements
using a quadratic equation \citep{Johnsen.etal.1995},
\begin{equation}
    {\Delta}T_{TS}(t) = -11.88 [\chem{\delta^{18}O}(t)
                               -\chem{\delta^{18}O}(0)]
                        -0.1925[\chem{\delta^{18}O}(t)^2
                                -\chem{\delta^{18}O}(0)^2] \,,
\end{equation}
while temperature reconstructions from Antarctic and oceanic cores were
provided as such. All records were scaled linearly (Table~\ref{tab:records}) in
order to simulate realistic and comparable ice extents at the LGM
(Fig.~\ref{fig:timeseries}).

Finally, lapse-rate corrections, ${\Delta}T_{LR}$, are computed as a function
of ice surface elevation, $s$, using the NARR surface geopotential height
invariant field as a reference topography, $b_{ref}$, by
\begin{align}
    {\Delta}T_{LR}(t, x, y) &= -\gamma [s(t, x, y)-b_{ref}] \\
                            &= -\gamma [h(t, x, y)+b(t, x, y)-b_{ref}],
\end{align}

thus accounting for the evolution of ice thickness, ${h=s-b}$\renote{
    Small change in notation to match with the kappa},
on the one hand, and for differences between the basal topography of the ice
flow model, $b$, and the
NARR reference topography, $b_{ref}$, on the other hand. All simulations use an
annual temperature lapse rate of $\gamma = 6\,\unit{{\degree}C\,km^{-1}}$.
In the rest of this paper, we refer to different model runs by the name of the
proxy record used for the palaeo-temperature forcing.

% =============================================================================
\section{Sensitivity to atmospheric forcing time-series}
\label{sec:results}
% =============================================================================

% -----------------------------------------------------------------------------
\subsection{Evolution of ice volume}
% -----------------------------------------------------------------------------

Despite large differences in the input atmospheric forcing
(Fig.~\ref{fig:timeseries}, upper panel), model output presents consistent
features that can be observed across the range of forcing data used. In all
simulations, modelled ice volumes remain relatively low during most of the
glacial cycle, except during two major glacial events which occur between 61.9
and 55.4\,ka during MIS~4, and between 29.5 and 16.9\,ka during MIS~2
(Fig.~\ref{fig:timeseries}, lower panel). An ice volume minimum is
consistently reached between 52.2 and 45.6\,ka during MIS~3. However, the
magnitude and precise timing of these three events depend significantly on the
choice of proxy record used for time-dependent atmospheric forcing
(Table~\ref{tab:extrema})\renote{
    This new table contains modelled ages, volumes and extents at MIS~4--2}.

Simulations forced by the Greenland ice core palaeo-temperature\renote{
    From here on, I prepend ``palaeo-temperature'' to ``record(s)'' in place of
    ``climate'' as you (Arjen) originally suggested. If we are to precise what
    type of ``records'' we use, let's be clear about it.}
records (GRIP, NGRIP) produce the highest variability in modelled ice volume
throughout the last glacial cycle. In contrast, simulations driven by oceanic
(ODP~1012, ODP~1020) and Antarctic (EPICA, Vostok) palaeo-temperature records
generally result in smaller\renote{
    Until we run out of other ajectives, I don't think that ``modest'' and
    ``generous'' are necessary terms for describing ice volumes.}
modelled ice volumes during MIS~4 and larger ice volumes during MIS~3.
More broadly speaking, they produce lower ice volume variability
throughout the simulation length. The NGRIP atmospheric forcing is the only one
that results in a larger ice volume during MIS~4 than during the MIS~2.

While simulations driven by the GRIP and the two Antarctic palaeo-temperature
records attain a last ice volume maximum between 19.5 and 16.9\,ka, those
informed by the NGRIP and the two oceanic palaeo-temperature records attain their
maximum ice volume thousands of years earlier. Moreover, the ODP~1012 run
yields a rapid deglaciation of the modelled area prior to 20\,ka. The ODP~1020
simulation predicts an early maximum in ice volume at 29.5\,ka, followed by
slower deglaciation than modelled using the other palaeo-temperature records.
Finally, model runs forced by Antarctic palaeo-temperature records result in a
rapid and uninterrupted deglaciation after the last glacial maximum, whereas the
simulation driven by the GRIP palaeo-temperature record also results in a rapid
deglaciation but in three steps, separated by two ice volume standstills.

% -----------------------------------------------------------------------------
\subsection{Extreme configurations}
% -----------------------------------------------------------------------------

Despite such differences in the timing of attained volume maxima and mimima
(Table~\ref{tab:extrema}), all the model runs show relatively consistent
patterns of glaciation (Fig.~\ref{fig:snapshots}, upper panels). During MIS~4,
all simulation produce an extensive ice sheet, covering an area of at least
half that attained during MIS~4. However, the corresponding maximum ice volumes
vary differ significantly between model runs, and cary between 3.88 and~8.69\,m
sea-level equivalent (m.~s.l.e; Table~\ref{tab:extrema}).

During the MIS~3 ice volume minima, a central ice cap persists in all
simulations over the Skeena Mountains (Fig.~\ref{fig:snapshots}, middle
panels). Although this ice cap is present in all simulations, its dimensions
depend sensitively on the choice of palaeo-temperature record used to drive the
model. Modelled ice volume minima vary between 0.89 and 2.44\,m~s.l.e.
(Table~\ref{tab:extrema}).

The modelled ice sheet geometries during the last glacial maximum (MIS~2;
Fig.~\ref{fig:snapshots}, lower panels) include a ca. 1500\,km-long central
divide located about 3500\,m~a.s.l. along the spine of the Rocky Mountains and
appear highly similar from one simulation to the next. Although the similatity
in modelled ice extent is a direct result from the choice of scaling factors
applied to the different palaeo-temperature proxy records
(Table~\ref{tab:records}), it is interesting to note that modelled maximum ice
volumes are also contained within a close range of 8.24 to 8.66\,m~s.l.e.
(Table~\ref{tab:extrema}).

% =============================================================================
\section[Comparison to the geologic record]
        {Comparison to the geologic record\renote{
    Arjen has proposed: 2.1 Evaluation of model runs agains field evidence, 2.2
    Palaeoglaciology of the Cordilleran ice sheet, and 2.3 The last
    deglaciation; but I don't think that these headers work. In my view, this
    entire ``discussion'' part is all about evaluation of model runs agains
    field evidence and palaeoglaciology. It is necessary that this comparison
    starts here, in order to ``kick out'' the records that perform badly (ODP*,
    NGRIP) and justify that the rest of the analysis is performed on two
    (high-resolution) simulations only. Such figures as
    Figs.~\ref{fig:duration}--\ref{fig:warmfrac} and Fig.~\ref{fig:deglac},
    which I consider as different views of the results rather than new results
    (apart the fact that they use higher resolution) are drawn for the purpose
    of comparing model results against geological evidence. I propose to rename
    this section to a more explicit ``Comparison to the geologic record'' (or
    similar). I have also renamed the ``results'' to ``Sensitivity to
    atmospheric forcing time-series'', which is actually what that section is
    about: raw model output, no discussion against the geologic record. This
    new discussion is divided in three parts: 4.1 Glacial maxima, where we
    study the maximum stages, relatively well-documented by the geology; 4.2
    Nucleation centres, where we study ``intermediate'' states, for which less
    is known, but the model can say something; and 4.3 The last deglaciation,
    which deserves a separate section because many people are interested in
    this period, and much geological evidence is available.}}
\label{sec:discussion}
% =============================================================================

% -----------------------------------------------------------------------------
\subsection{Glacial maxima}
% -----------------------------------------------------------------------------

\subsubsection{Timing of glaciation}
\label{sec:timing}

Independently of the choice of palaeo-temperature record\renote{
    In my view, ``atmospheric/climate forcing'' would be far too broad here. We
    could possibly apply a much wilder choice of atmospheric forcing
    techniques, leading just as wild differences in model output
    \citep[e.g.,][but possibly much more than that]{Seguinot.etal.2014}.}
used to force the ice sheet model, our simulations consistently produce two
glacial maxima during the last glacial cycle. The first maximum configuration
is obtained during MIS~4 (61.9--55.4\,ka) and the second during MIS~2
(29.5--16.9\,ka; Figs.~\ref{fig:timeseries} and~\ref{fig:snapshots};
Table~\ref{tab:extrema}). These events broadly correspond in timing to the
Gladstone
(MIS~4) and McConnell (MIS~2) glaciations documented by geological evidence for
the northern sector of the Cordilleran ice sheet
    \citep{Duk-Rodkin.etal.1996, Ward.etal.2007,
           Stroeven.etal.2010, Stroeven.etal.2014},
and to the Fraser Glaciation (MIS~2) documented for its southern sector
    \citep{Porter.Swanson.1998, Margold.etal.2014}.
There is patchy stratigraphical evidence for glaciations older than the Fraser
Glaciation \citep{Clague.Ward.2011} in British Columbia, and their extent and
timing are therefore still highly conjectural
    \citep[perhaps MIS~4 or early MIS~3; e.g.,][]{Cosma.etal.2008}.

However, the exact timing of modelled MIS~2 maxima depends strongly on the
choice of applied palaeo-temperature record, which allows for a more in-depth
comparison with geological evidence for the timing of maximum Cordilleran ice
sheet extent. In the Puget lowland (Fig.~\ref{fig:locmap}), the LGM advance of
the southern Cordilleran ice sheet margin has been constrained by radiocarbon
dating on wood between 17.4 and 16.4\,\unit{\chem{^{14}C}\,cal\,ka\,BP}
\citep{Porter.Swanson.1998}.
These dates have been confirmed by the offshore sedimentary record, which shows
an increase of glaciomarine sedimentation between 19.5 and
16.2\,\unit{\chem{^{14}C}\,cal\,ka\,BP} \citep{Cosma.etal.2008}. Radiocarbon
dating of the northern Cordilleran ice sheet margin is much less constrained
but straddles presented constraints from the southern margin. However,
cosmogenic exposure dating
places the timing of maximum CIS extent during the McConnell glaciation close
to 17\,ka \citep{Stroeven.etal.2010, Stroeven.etal.2014}.

Among the simulations presented here, only those forced with the GRIP, EPICA
and Vostok palaeo-temperature records yield Cordilleran ice sheet maximum
extent that may be compatible with these field constraints
(Fig.~\ref{fig:timeseries}, lower panel; Table~\ref{tab:extrema}).
Simulations driven by the NGRIP, ODP~1012 and ODP~1020
palaeo-temperature records, on the contrary, yield MIS~2 maximum Cordilleran
ice sheet extents that pre-date field-based constraints by several thousands of
years. Concerning the simulations driven by oceanic records, this early
deglaciation is caused by an early warming present in the alkenone
palaeo-temperature reconstructions (Fig.~\ref{fig:timeseries}, upper panel;
\citealp[Fig.~3]{Herbert.etal.2001}). However, this
early warming is a local effect, corresponding to a weakening of the California
current \citep{Herbert.etal.2001}. The California current, driving cold
waters southwards along the south-western coast of North America\renote{
    Not only USA but also Baja California in Mexico.},
has been shown to have weakened during each peak global glaciation (in SPECMAP)
durint the past 550\,ka, including the LGM, resulting in paradoxically warmer
sea-surface temperatures at the locations of the ODP~1012 and ODP~1020 sites
\citep{Herbert.etal.2001}.

Because most of the marine margine of the Cordilleran ice sheet terminated in a
sector of the Pacific Ocean unaffected by variations in the California current,
it probably remained unaffected by this early warming. However, the above paradox
illustrates the complexity of ice-sheet feedbacks on regional climate, and
demonstrates that, although located in the neighbourhood of the modelling
domain, the ODP~1012 and ODP~1020 palaeo-temperature records cannot be
used as a realistic forcing to model the Cordilleran ice sheet
through the last glacial cycle.

\todo{Write something about NGRIP.}

Hence, we focus the rest of our analysis on simulations forced by
palaeo-temperature records from GRIP and EPICA ice cores. To allow for a more
detailed comparison against the geologic record, these two simulations were
re-run on the high-resolution grid (Fig.~\ref{fig:timeseries}, lower panel,
dotted lines).


\subsubsection{Ice configuration during MIS~2}
\label{sec:mis2}

During maximum glaciation, the main meridional ice divide is positioned over
the western flank of the Rocky Mountains (Fig.~\ref{fig:snapshots}, lower
panels; Fig.~\ref{fig:mis2}). This result appears to contrast with
palaeoglaciological reconstructions for central and southern British Columbia
that inferred the ice divide to have been situated in a more westerly position,
over the western margin of the Interior Plateau \citep{Ryder.etal.1991,
Stumpf.etal.2000, Kleman.etal.2010, Clague.Ward.2011, Margold.etal.2013a}. A
latitudinal saddle has been inferred to have connected ice dispersal centres in
the Columbia Mountains with the main ice divide \citep{Ryder.etal.1991,
Kleman.etal.2010, Clague.Ward.2011, Margold.etal.2013a}. A latitudinal saddle
does indeed features in our modelling results, however, in an inverse
configuration between the main ice divide over the Columbia Mountains and a
secondary divide over the southern Coast Mountains (Fig.~\ref{fig:mis2}).

On the one hand, this could reflect the fact that our model does not include
feedback mechanisms between ice sheet topography and the regional climate.
Firstly, during the build-up phase preceding the LGM, rapid accumulation over
the Coast Mountains enhanced the topographic barrier formed by these mountain
ranges, which likely resulted in a decrease of precipitation and, therefore,
accumulation in the interior. Secondly, latent warming of the moisture-depleted
air parcels flowing over this enhanced topography could have resulted in an
inflow of warmer air over the eastern flank of the ice sheet, increasing melt
along the advancing margin \citep[cf.][]{Langen.etal.2012}. Because these two
processes, both inhibiting of ice-sheet growth, are absent from our model, the
eastern margin of the ice sheet and the position of the main meridional ice
divide are certainly biased towards the east in our simulations
\citep{Seguinot.etal.2014}.

On the other hand, field-based palaeoglaciological reconstructions have
struggled to reconcile the more westerly-centred ice divide in south-central
British Columbia with evidence in the Rocky Mountains and beyond, that the
Cordilleran ice sheet invaded the western Interior Plains, where it merged with
the southwestern margin of the Laurentide ice sheet and was deflected to the
south \citep{Jackson.etal.1997, Bednarski.Smith.2007, Kleman.etal.2010,
Margold.etal.2013, Margold.etal.2013a}. Ice geometries from our model runs do
not have this problem, because the position and elevation of the ice divide
results in significant ice drainage across the Rocky Mountains at the LGM
(Fig.~\ref{fig:mis2}).


\subsubsection{Ice configuration during MIS~4}
\label{sec:mis4}

The modelled ice sheet configurations corresponding to ice volume maxima during
MIS~4 are more variable (Fig.~\ref{fig:snapshots}, upper panels;
Fig.~\ref{fig:mis4}). The GRIP simulation (Fig.~\ref{fig:mis4}, left panel)
results in a modelled maximum ice sheet that closely resemble that obtained
during MIS~2, with the only major differences being a slightly less developed
northern and eastern sectors. However, the EPICA simulations produce a reduced
ice volume maximum (Fig.~\ref{fig:timeseries}), which translates in the
modelled ice sheet geometry into a much less developed southern sector, more
restricted ice cover on the northern and eastern flanks, and generally lower
ice surface elevations in the interior (Fig.~\ref{fig:mis4}, left panel).

In the GRIP simulation, some sections of the MIS~4 maximum ice margin are
slightly more extensive than their MIS~2 counterparts. This is notably the case
in the Puget Lowland, along parts of the marine margin, and parts of the north
slope of the Alaska Range. However, these differences generally correspond to a
few model grid cells. Their interpretation would be hazardous considering
the simplicity of our atmospheric forcing, and the scale of our simulations.


% -----------------------------------------------------------------------------
\subsection{Nucleation centres}
% -----------------------------------------------------------------------------

\subsubsection{Transient ice sheet states}

The accuracy of palaeo-glaciological reconstructions is generally biased
towards the stages of maximum extent, for which the geological evidence is
generally more detailed and more complete than for other periods. However,
these maximum stages are, by nature, extreme configurations, which do not
necessarily represent the dominant patterns of glaciation throughout the period
of ice cover.

For the Cordilleran ice sheet, geological evidence from radiocarbon dating
    \citep{Clague.etal.1980, Clague.1985, Clague.1986, Porter.Swanson.1998,
           Menounos.etal.2008},
cosmogenic exposure dating
    \citep{Stroeven.etal.2010, Stroeven.etal.2014, Margold.etal.2014},
bedrock deformation in response to former ice loads
    \citep{Clague.James.2002, Clague.etal.2005},
and offshore sedimentary records
    \citep{Cosma.etal.2008, Davies.etal.2011}
indicate that the LGM maximum extent was short-lived. To compare this finding
to our simulations, we use numerical modelling output covering the whole
simulation period to compute durations of ice cover throughout the last glacial
cycle (Fig.~\ref{fig:duration}).

The resulting maps show that, during most of the glacial cycle, modelled ice
cover is restricted to disjoint ice caps centred on major mountain ranges of
the North American Cordillera (Fig.~\ref{fig:duration}, blue areas). A
2500\,km-long continuous ice cover spanning from the Alaska Range in the
northeast to the Rocky Mountains in the southwest only exists\renote{
    I chosed to write about model output in the present form to discern in from
    the reality.}
for at most 29\,ka in total over the entire last glacial cycle
(Fig.~\ref{fig:duration}, hatched areas). However,
except for its marine margin and the northern foothills of the Alaska Range,
the maximum extent of the ice sheet is attained for an even more brief period
of a few thousand years (Figs.~\ref{fig:duration}, red areas). This
result illustrates that the maximum extents of the modelled ice sheet during
MIS~4 and 2 were both short-lived and therefore out of balance with
contemporary climate.

A notable exception to the transient character of the LGM Cordilleran ice sheet
is the northern slope of the Alaska Range, where modelled glaciers are confined
to the foothills during the entire simulation period (Fig.~\ref{fig:duration},
AR). This apparent insensitivity of modelled glacial extent to temperature
fluctuations results from a combination of low precipitations, high summer
temperatures and high temperature standard deviations in this sector of the
modelling domain (Fig.~\ref{fig:atm}). This result could potentially explain
the local distribution of glacial deposits, which indicates that these glaciers
have remained small throughout the Pleistocene \citep{Kaufman.Manley.2004}.

\subsubsection{Major ice-dispersal centres}

It is generally believed that the Cordilleran ice sheet formed by the
coalescence of of several mountain-centred ice caps \citep{Davis.Mathews.1944}.
In our simulations, major ice-dispersal centres, visible on the modelled ice
cover duration maps (Fig.~\ref{fig:duration}), are located over the Coast
Mountains (CM), the Columbia and Rocky mountains (CR), the Skeena Mountains
(SM), and the Selwyn and MacKenzie Mountains (SMK). Although the Coast,
Columbia, Rocky and Skeena Mountains (CM, CR, SM) are covered by mountain
glaciers for most of the model time, providing durable nucleation centre for
the initiation of ice sheet growth, this is not the case for the Selwyn and
MacKenzie mountains (SMK), where ice cover on the highest peaks is limited to a
fraction of the last glacial cycle. In other words, the Selwyn and MacKenzie
mountains appear as a secondary ice-dispersal centre which ``activates'' only
during the coldest periods of the last glacial cycle. The Northern Rocky
Mountains (Fig.~\ref{fig:duration}, NR), do not act as a nucleation centre,
but rather as a pinning point for the ice margin coming from the west.

Perhaps the most striking feature displayed by the distributions of modelled
ice cover is the persistence of the Skeena Mountains ice cap throughout the
entire last glacial cycle and its predominance over other ice-dispersal centres
(Figs.~\ref{fig:snapshots} and~\ref{fig:duration}). Regardless of the applied
forcing, this ice cap appears to survive MIS~3 (Fig.~\ref{fig:snapshots},
middle panels), and serves as a nucleation centre at the onset of the glacial
readvance towards the LGM (MIS~2). The importance of residual ice for North
American glacial history leading up to the LGM has been illustrated by the
MIS~3 residual ice bodies in northern and eastern Canada as nucleation centres
for a much more extensive MIS~2 configuration \citep{Kleman.etal.2010}\renote{
    Does this sentence really fit here? I feel that this sudden mention of the
    Laurentide ice sheet perturbates a bit the flow. Moreover, knowing that
    the Laurentide ice sheet is probably the main driver for the 100\,ka
    sea-level cycle, it would have been surprising if it had completely
    vanished at MIS~3. But for the Cordilleran ice sheet, usually thought of
    as fast-responding and intermittent, I feel that the presence of residual
    ice is less trivial.}.

The presence of a Skeena Mountains ice cap during most of the last glacial
cycle can be explained by a currently more widespread\renote{
    I mean widespread, precipitation there is lower than elsewhere.}
winter precipitation for that region there than for other parts of the
modelling domain (Fig.~\ref{fig:atm}). Along most of the north-western coast of
North America, coastal mountain ranges form a pronounced topographic barrier
for westerly winds, capturing atmospheric moisture in the form of orographic
precipitation, and resulting in arid interior lowlands. However, near the
centre of our modelling domain, this barrier is less pronounced than elsewhere,
allowing westerly winds to carry moisture\renote{
    I do not agree with ``carry precipitation''. When reading this I get the
    picture of water drops falling diagonaly under action of the wind. I do not
    deny that transport of precipitation happens and
    is an important process for understanding orographic
    precipitation (think of why it is raining so much in the Swedish mountains
    even when the wind comes from the west), but it is not what I am trying to
    describe here. I wrote ``moisture'' and this is what I mean.}
further inland, until it is progressively captured by the extensive, but mildly
elevated group of the Skeena Mountains in north-central British Columbia.

\subsubsection{Potential imprint on the landscape}

A correlation is observed between the modelled duration of warm based ice cover
(Fig.~\ref{fig:warmbase}) and the degree of glacial modification of the
landscape (mainly in terms of the development of deep glacial valleys and
troughs). We find evidence for this on both the southwestern and northeastern
slopes of the southern Coast Mountains, on the western slopes of the Columbia
Mountains, on the western and eastern slopes of the northern Coast Mountains
and the Saint-Elias Mountains, and radiating off the Skeena Mountains
(Figs.~\ref{fig:duration} and \ref{fig:warmbase};
\citealp[Fig.~2]{Kleman.etal.2010})\renote{
    Yes, I agree that it would be good to eventually reproduce here the basal
    flow patterns from Johan's map \citep[Fig.~2]{Kleman.etal.2010}, but I
    don't think that I will have the time for this before thesis submission}.
The Skeena Mountains, for example, indeed bear a strong glacial imprint that
indicates ice drainage in a system of distinct glacial troughs emanating in a
radial pattern from the centre of the mountain range
\citep[Fig.~2]{Kleman.etal.2010}. We suggest that
persistent ice cover (Fig.~\ref{fig:duration}) associated with basal ice
temperatures at the pressure-melting point (Figs.~\ref{fig:warmfrac}
and~\ref{fig:warmbase}) explains the large-scale glacial erosional imprint on
the landscape. A well-developed network of glacial valleys west of
the MacKenzie Mountains (\citealp[Fig.~2]{Kleman.etal.2010}; \citealp[Fig.~8]
{Stroeven.etal.2010}) is modelled to have hosted warm-based ice
(Fig.~\ref{fig:warmfrac}) but it has only been glaciated for a short fraction
of the last glacial cycle (Fig.~\ref{fig:duration}) according to our
results. This perhaps indicate that\renote{
    I don't think that we have to infer that ``our model does not perform
    reliably'' here. I am actually quite happy to discover that the model
    puts warm-based ice where there is (another) region of strong glacial
    erosion. I would rather say that computing the ``duration of
    warm-based ice cover'' as I did for is a very simple approach and that we
    can expect it to give a qualitative proxy for glacial erosion but not a
    quantitative one.}
the observed landscape pattern originates from older glacial cycles and
witnesses an increased relative importance of this ice dispersal centre prior
to the Late Pleistocene \citep[cf.][]{Ward.etal.2008, Demuro.etal.2012}.

The modelled distribution of warm-based ice cover (Figs.~\ref{fig:warmbase}
and~\ref{fig:warmfrac}) is inevitably affected by our assumption of a constant,
70\,\unit{mW\,m^{-2}} geothermal heat flux at 3\,km depth
(Sect.~\ref{sec:overview}). However, both the Skeena Mountain area and the area
west of the MacKenzie Mountains are areas of higher-than-average geothermal
heat flux \citep{Blackwell.Richards.2004}. We can therefore expect a higher
frequency of warm-based ice cover in this areas if we were to include
spatially-variable geothermal forcing in our simulations.


% -----------------------------------------------------------------------------
\subsection{The last deglaciation}
% -----------------------------------------------------------------------------

\subsubsection{Pace and patterns of deglaciation}

In the North American Cordillera alike other glaciated regions, the large
majority of the glacial geologic record relates to the last few millennia of
glaciation, most of the older evidence having been overridden by ice retreat
during the deglaciation \citep{Kleman.1994, Kleman.etal.2010}. From a
numerical modelling perspective, phases of glacier retreat are more challenging
than phases of growth, because they involve more rapid fluctuations of the ice
margin, increased flow velocities and longitudinal stress gradients, and poorly
understood hydrological processes, typically included in the models through
simple parametrisations \citep[e.g.][]{Clason.etal.2012, Clason.etal.2014,
Bueler.Pelt.2014}, if included at all. However, after mapped moraines
indicative of maximal extents (Sect.~\ref{sec:mis2} and~\ref{sec:mis4}),
geomorphologically-reconstructed patterns of the last deglaciation provide the
second best source of evidence for validation of our simulations.

In the North American Cordillera, the presence of lateral melt-water channels
at high elevation \citep{Margold.etal.2011, Margold.etal.2013a,
Margold.etal.2014}, and abundant esker systems at low elevation
\citep{Burke.etal.2012, Burke.etal.2012a, Perkins.etal.2013,
Margold.etal.2013}\footnote{
    Is this a good choice of references?}
indicate a rapid deglaciation. The southern and northern margins of the
Cordilleran ice sheet reached their last glacial maximum extent around
17\,ka \citep[Sect.~\ref{sec:timing};][]{Porter.Swanson.1998, Cosma.etal.2008,
Stroeven.etal.2010, Stroeven.etal.2014}, putting an older bound to the
onset of ice retreat. However, he timing of the late deglaciation is less well
constrained, but recent cosmogenic dates from north-central British Columbia
indicate that a seizable ice cap emanating from the central Coast Mountains or
the Skeena Mountains persisted into the Younger Dryas chronozone, at least
until 12.4\,ka \citep{Margold.etal.2014}.

In our simulations, the timing of peak ice volume during the last glacial
maximum and the pacing of deglaciation depend cirticially on the choice of
atmospheric forcing (Table~\ref{tab:extrema}, Figs.~\ref{fig:timeseries}
and~\ref{fig:deglacseries}). Adopting the EPICA
atmospheric forcing yields peak ice volume at 17.3\,ka and an uninterupted
deglaciation until 10.6\,ka (Fig.~\ref{fig:deglacseries}, lower panel, red
curves). The simulation driven by the GRIP palaeo-temperature record yields
peak ice volume at 19.1\,ka and a deglaciation interrupted by two ice volume
standstills until 9.4\,ka. The first interuption occurs between 16.6 and
14.5\,ka, and the second between 12.6 and 11.6\,ka
(Fig.~\ref{fig:deglacseries}, lower panel, blue curve). Hence, the two model
runs, while similar in overall timing compared to runs with other climate
drivers, differ in detail in that the EPICA depicts peak glaciation almost
2\,ka later than the GRIP, in closer agreement to dated maximum extents, and
shows a faster, uninterrupted deglaciation which yields ice-free conditions
more than 1\,ka earlier.

Modelled patterns of ice sheet retreat are relatively consistent between the
two simulations (Figs.~\ref{fig:deglacshots} and.~\ref{fig:deglac}). The
southern sector of the modelling domain, including the Puget Lowland, the Coast
and Rocky mountains, and the Interior Plateau of British Columbia, becomes
completely deglaciated by 12\,ka, whereas a significant ice cover remains over
the Skeena, Selwyn, MacKenzie, Wrangell and Saint-Elias mountains in the
northern sector of the modelling domain. After 12\,ka, deglaciation continues
to proceed across the Liard Lowland with a radial ice margin retreat towards
the surrounding mountain ranges, consistent with the regional melt water record
of the last deglaciation \citep{Margold.etal.2013}. Remaining ice continues to
decay by retreating towards the Selwyn and Skeena mountains. The last remnants
of the Cordilleran ice sheet finally disappear from the Skeena mountains around
10.6\,ka (EPICA) and 9.4\,ka (GRIP).


\subsubsection{Late-glacial readvance}

The possibility of late glacial readvances in the North American Cordillera has
been debated for some time \citep{Osborn.Gerloff.1997}, and locally these have
been reconstructed and dated. Radiocarbon-dated end moraines in the Fraser and
Squamish valleys, off the southern tip of the Coast Mountains, indicate
consecutive glacier maxima, or standstills while in overall retreat, one of
which corresponding to the Younger Dryas chronozone \citep{Clague.etal.1997,
Friele.Clague.2002, Friele.Clague.2002a, Kovanen.2002,
Kovanen.Easterbrook.2002}. Although these moraines characterise independent
valley glaciers, not necessarily connected to an ice sheet, the Finlay River
area of the northern Rocky Mountains presents a different kind of evidence.
There, sharp-crested moraines indicate a late-glacial readvance of local alpine
glaciers and, more importantly, their interaction with larger, lingering
remnants of the main body of the Cordilleran ice sheet in the valleys
\citep{Lakeman.etal.2008}. Additional evidence for late-glacial
alpine glacier readvances includes moraines in the eastern Coast Mountains,
Rocky Mountains and the Columbia Mountains \citep{Osborn.Gerloff.1997,
Menounos.etal.2008}.

Although further work is needed to constrain the timing of the late-glacial
readvance, to assess its extents and geographical distribution, and to identify
the potential climatic triggers \citep{Menounos.etal.2008}, it is interesting
to note that the simulation driven by the GRIP record produces a late-glacial
readvance in the Coast Mountains, Rocky Mountains and the Finlay River area,
corresponding to where it has been identified in the geological record
(Fig.~\ref{fig:deglac}, left panel). In addition to matching the location of
local readvances, the GRIP-driven simulation shows that larger ice bodies
emanating from the Skeena and MacKenzie mountains may still have exist at the
time of this late-glacial readvance\renote{
    Thanks Arjen for your suggestion to reformulate. Good point.}.
In contrast, the EPICA-driven simulation produces a nearly-continuous
deglaciation with only a tightly restricted glacial readvance
(Fig.~\ref{fig:deglac}, right panel).


\subsubsection{Deglacial flow directions}

Because a general tenant in glacial geomorphology is that the majority of
landforms (lineations and eskers) are part of the deglacial envelop
\citep[terminology from]{Kleman.etal.2006}, that is that these were formed
close inside the retreating margin of ice sheets \citep{Boulton.Clark.1990,
Kleman.etal.1997, Kleman.etal.2010}, we present a map of basal flow directions
immediately preceding deglaciation or at the time of cessation of sliding
inside the cold-base retreating margin (Fig.~\ref{fig:lastflow})
\todo{description of the general trends that the map shows}. Patterns of
glacial lineation formed in the northern and southern sectors of the
Cordilleran ice sheet and in the Liard Lowland (\citealp{Prest.etal.1968};
\citealp[Fig.~1.12]{Clague.1989}; \citealp[Fig.~2]{Kleman.etal.2010};
\citealp[Fig.~2]{Margold.etal.2013}), similarities with the patterns of
deglacial ice flow from modelling (Fig.~\ref{fig:lastflow}). However, this is
not the case for the Interior Plateau of British Columbia, where both
simulations predict negligible basal sliding during deglaciation
(Fig.~\ref{fig:lastflow}), but where an impressive set of glacial delineations
indicate a substantial eastward flow component of the Cordilleran ice sheet
\citep{Prest.etal.1968, Kleman.etal.2010}.

The Interior Plateau lineation swarm could thus present a smoking gun for the
reliability of the presented model results. One explanation for the incongruent
results could therefore be that the modelled LGM ice sheet is too thick, or
that the ice divide is positioned too far to the east \todo{descript the
problem for the incongruent results of these two conditions}. Because feedback
mechanisms between ice sheet topography and regional climate are absent in our
model, including wind redirection, orographic precipitation effects, and latent
warming of moisture-depleted air, we regard misfeeds in ice thickness and ice
devide location as plausible \citep{Seguinot.etal.2014}. Another explanation
for the incongruent results could be that the Interior Plateau lineation swarm
predates deglaciation and that deglacitation landforms are largely absent. The
modelled deglaciation of the Interior Plateau, central British Columbia,
consists of a rapid northwards retreat (Fig.~\ref{fig:deglac}) of
southwards-flowing (Fig.~\ref{fig:lastflow}) non-sliding ice lobes positioned
in-between deglaciated (ice-free) mountain ranges
(Figs.~\ref{fig:profiles-grip} and~\ref{fig:profiles-epica}). This result
appears compatible with the prevailing conceptual model of deglaciation of
central British Columbia, in which mountain ranges emerge from the ice before
the plateau \citep[Fig.~7]{Fulton.1991}. If it is valid, it may suggest that
glacial lineations on the Interior Plateau of British Columbia may be of older
age than the LGM, and may have remained intact throughout the deglaciation.


% =============================================================================
\conclusions
\label{sec:concl}
% =============================================================================

Numerical simulations of the Cordilleran ice sheet through the last glacial
cycle presented in this study consistently produce two glacial maxima during
MIS~4 (61.9--55.4\,ka) and MIS~2 (29.5--16.9\,ka), two periods
corresponding to documented extensive glaciations. This result is
independent of the choice of the palaeo-temperature record used to approximate
the past climate evolution, and thus
can be seen as a first-order agreement between the model and the geological
evidence. However, the timing of the two glaciation peaks depends sensitively
on which record
is used to drive the model. The timing of the LGM is best
reproduced by the Antarctic record, and occurs too early in all simulations
that are driven by the other records. The mismatch is greatest when using
oceanic records from the Pacific Northwest, which are affected by the
weakening of the California current during the LGM.

In all simulations presented here, ice cover is limited to disjoint mountain ice
caps during most of the glacial cycle, confirming previous inferences from the
geological evidence preceding the LGM. However, our
simulations produce persistent ice cover on the Skeena Mountains during
the entire glacial cycle. At the time when a full-size Cordilleran ice sheet is
absent, the Skeena ice cap appears to be fed by the eastwards precipitation
intrusion through a topographic window in the Coast Mountains. The ice cap acts
as a nucleation centre at the onset of the LGM readvance, and appears
consistent with the distinct glacial imprint of the Skeena Mountains landscape.

During deglaciation, none of the palaeo-temperature records used produces a
close agreement between the model results and the geological evidence. Although
the EPICA record yields a more realistic timing of the LGM and early
deglaciation, only the GRIP record produces a late-glacial readvance in areas
where it has been documented in the literature. Nonetheless, the
general patterns of deglaciation are consistent between both simulations driven
by the GRIP and EPICA records, and show a rapid deglaciation of the southern
half of the ice sheet, including a rapid northwards retreat across the Interior
Plateau of central British Columbia. This is followed by an opening of the ice
margin in the Liard Lowland, and a final retreat of the margin of the remaining
ice caps towards the Selwyn and, later, the Skeena mountains, which host the
last remnant of the ice sheet
during the early Holocene (10.9--9.5\,kyr).

These results are strongly dependent on selected ice-sheet model (PISM),
surface
mass balance model (PDD) and climate forcing. Most importantly, our simplistic
palaeo-climate forcing does not include precipitation corrections in response
to the presence of an ice cover, potentially leading to overestimated glacial
extent and volume in continental regions. Nevertheless, our results identify
the Skeena Mountains as a key area to understanding
glacial dynamics of the Cordilleran ice sheet, highlighting the need for
further geological investigation of this region.

% Author contributions
\section*{Author contributions}
\dots

% Acknowledgements
\begin{acknowledgements}
We are very thankful to Constantine Khroulev, Ed Bueler and Andy Aschwanden for
providing constant help and development on PISM. This work was supported by the
Swedish Research Council~(VR) grant no. 2008-3449 to A.~P.~Stroeven, by the
German Academic Exchange Service~(DAAD) grant no.~50015537 and a Knut and Alice
Wallenberg Foundation grant to J.~Seguinot.
Computer resources were provided by the Swedish National
Infrastructure for Computing (SNIC) allocation no. 2013/1-159 to A.~P.~Stroeven
at the National Supercomputing Center (NSC).
\end{acknowledgements}

% References
\bibliographystyle{copernicus}
\bibliography{refs/references.bib}
\newpage

% =============================================================================
% Floats
% =============================================================================

% tab:params
\begin{table*}
  \centering
  \caption{Main parameters of the ice sheet model.
           \todo{At the moment, this is pretty extensive (does not fit on one
                 page using the standard font size). There are probably a
                 couple of things that are not needed here, like for instance
                 the very standard things (gravity, gas constant, etc). Many
                 parameters are not defined in the text. Do we need to
                 introduce them, or should we skip them? I wonder if the last
                 column is needed, in which case it needs to be completed.}}
  \label{tab:params}
  \footnotesize
  {\begin{tabular}{llrll}
    \tophline
    Not.    & Name & Value & Unit & Source \\
    \middlehline

    \multicolumn{2}{l}{\emph{Ice rheology}} \\

    $\rho$  & Ice density
            & 910
            & \unit{kg\,m^{-3}}
            & \citet{Aschwanden.etal.2012} \\

    $g$     & Standard gravity
            & 9.81
            & \unit{m\,s^{-2}}
            & \citet{Aschwanden.etal.2012} \\

    $n$     & Glen exponent
            & 3
            & -
            & \aref \\

    $A_c$   & Ice hardness coefficient cold
            & $3.61\times10^{-13}$
            & \unit{Pa^{-3}\,s^{-1}}
            & \citet{Paterson.Budd.1982} \\

    $A_w$   & Ice hardness coefficient warm
            & $1.73\times10^3$
            & \unit{Pa^{-3}\,s^{-1}}
            & \citet{Paterson.Budd.1982} \\

    $Q_c$   & Flow law activation energy cold
            & $6.0\times10^4$
            & \unit{J\,mol^{-1}}
            & \citet{Paterson.Budd.1982} \\

    $Q_w$   & Flow law activation energy warm
            & $13.9\times10^4$
            & \unit{J\,mol^{-1}}
            & \citet{Paterson.Budd.1982} \\

    $R$     & Ideal gas constant
            & 8.31441
            & \unit{J\,mol^{-1}\,K^{-1}}
            & \aref \\

    $T_c$   & Flow law critical temperature
            & 263.15
            & \unit{K}
            & \citet{Paterson.Budd.1982} \\

    $f$     & Flow law water fraction coeff.
            & 181.25
            & -
            & \citet{Lliboutry.Duval.1985} \\

    $\beta$ & Clapeyron constant
            & $7.9\times10^{-8}$
            & \unit{K\,Pa^{-1}}
            & \citet{Luthi.etal.2002} \\

    $c_i$   & Ice specific heat capacity
            & 2009
            & \unit{J\,kg^{-1}\,K^{-1}}
            & \citet{Aschwanden.etal.2012} \\

    $c_w$   & Water specific heat capacity
            & 4170
            & \unit{J\,kg^{-1}\,K^{-1}}
            & \citet{Aschwanden.etal.2012} \\

    $k$     & Ice thermal conductivity
            & 2.10
            & \unit{J\,m^{-1}\,K^{-1}\,s^{-1}}
            & \citet{Aschwanden.etal.2012} \\

    $L$     & Water latent heat of fusion
            & $3.34\times10^5$
            & \unit{J\,kg^{-1}\,K^{-1}}
            & \citet{Aschwanden.etal.2012} \\

    \multicolumn{2}{l}{\emph{Basal sliding}} \\

    $q$     & Pseudo-plastic sliding exponent
            & 0.25
            & -
            & - \\

    $v_{th}$& Pseudo-plastic threshold velocity
            & 100.0
            & \unit{m\,yr^{-1}}
            & - \\

    $c_0$   & Till cohesion
            & 0.0
            & Pa
            & - \\

    $\delta$& Effective pressure coefficient
            & 0.02
            & -
            & - \\

    $e_0$   & Till reference void ratio
            & 0.69
            & -
            & \citet{Tulaczyk.etal.2000} \\

    $C_c$   & Till compressibility coefficient
            & 0.12
            & -
            & \citet{Tulaczyk.etal.2000} \\

    $W_{max}$ & Maximal till water thickness
            & 2.0
            & m
            & - \\

    \multicolumn{2}{l}{\emph{Bedrock and lithosphere}} \\

    $\rho_b$& Bedrock density
            & 3300
            & \unit{kg\,m^{-3}}
            & \aref \\

    $c_b$   & Bedrock specific heat capacity
            & 1000
            & \unit{J\,kg^{-1}\,K^{-1}}
            & \citet{Ritz.1997} \\

    $k_b$   & Bedrock thermal conductivity
            & 3.0
            & \unit{J\,m^{-1}\,K^{-1}\,s^{-1}}
            & \citet{Ritz.1997} \\

    $\nu_m$ & Mantle viscosity
            & $1\times10^21$
            & \unit{Pa\,s}
            & \citet{Lingle.Clark.1985} \\

    $\rho_l$& Lithosphere density
            & 3300
            & \unit{kg\,m^{-3}}
            & \citet{Lingle.Clark.1985} \\

    $D$     & Lithosphere flexural rigidity
            & $5.0\times10^24$
            & \unit{N}
            & \citet{Lingle.Clark.1985} \\

    \multicolumn{2}{l}{\emph{Surface and atmosphere}} \\

    $T_s$   & Temperature of snow precipitation
            & 273.15
            & \unit{K}
            & \aref \\

    $T_r$   & Temperature of rain precipitation
            & 275.15
            & \unit{K}
            & \aref \\

    $F_s$   & Degree-day factor for ice
            & $3.04\times10^{-3}$
            & \unit{m\,K^{-1}\,day^{-1}}
            & \citet{Shea.etal.2009} \\

    $F_i$   & Degree-day factor for ice
            & $4.59\times10^{-3}$
            & \unit{m\,K^{-1}\,day^{-1}}
            & \citet{Shea.etal.2009} \\

    $\gamma$& Air temperature lapse-rate
            & $6\times10^{-3}$
            & \unit{K\,m{-1}}
            & - \\

    \bottomhline
  \end{tabular}}
  \belowtable{}
\end{table*}

% tab:records
\begin{table*}[t]
  \caption{Palaeo-temperature proxy records and scaling parameters yielding
           temperature offset time-series used to force the ice sheet model
           through the last glacial cycle (Fig.~\ref{fig:timeseries}). $f$
           corresponds to the scaling factor adopted to yield last glacial
           maximum ice limits in the vicinity of mapped end moraines, and
           $T_{[32, 22]}$ refers to the resulting mean temperature anomaly
           during the period -32 to~-22~ka after scaling.}
  \label{tab:records}
  {\begin{tabular}{l|ccc|ccc|l}
    \tophline
    Record & Latitude & Longitude & Elev. & Proxy & $f$ & $T_{[32, 22]}$
           & Reference\\
    & & & (m~a.s.l) & & & (K) & \\
    \middlehline
    GRIP     &  72{\degree} 35' N  % 72.58 (decimal)
             &  37{\degree} 38' W  % 37.64 (decimal)
             & 3238\,m
             & \chem{\delta^{18}O}
             & 0.35 & -5.8{\degree}C  % -16.4126 (before scaling)
             & \citet{Dansgaard.etal.1993} \\

    NGRIP    &  75{\degree} 06' N  % 75.10
             &  42{\degree} 19' W  % 42.32
             & 2917\,m
             & \chem{\delta^{18}O}
             & 0.22 & -6.1{\degree}C  % -26.7098
             & \citet{Andersen.etal.2004} \\

    EPICA    &  75{\degree} 06' S  % 75.1
             & 123{\degree} 21' E  % 123.35
             & 3233\,m
             & \chem{\delta^{18}O}
             & 0.60 & -5.6{\degree}C  % -9.2055
             & \citet{Jouzel.etal.2007} \\

    Vostok   &  78{\degree} 28' S  % 78.8
             & 106{\degree} 50' E  % 106.8
             & 3488\,m
             & \chem{\delta^{18}O}
             & 0.70 & -5.6{\degree}C  % -7.9550
             & \citet{Petit.etal.1999} \\

    ODP~1012 &  32{\degree} 17' N
             & 118{\degree} 23' W
             & -1772\,m
             & \chem{U^{K'}_{37}}
             & 1.53 & -5.8{\degree}C  % -3.7889
             & \citet{Herbert.etal.2001} \\

    ODP~1020 &  41{\degree} 00' N
             & 126{\degree} 26' W
             & -3038\,m
             & \chem{U^{K'}_{37}}
             & 1.16 & -5.8{\degree}C  % -5.0000
             & \citet{Herbert.etal.2001} \\
    \bottomhline
  \end{tabular}}
  \belowtable{}
\end{table*}

% tab:extrema
\begin{table*}[t]
  \caption{Extrema of ice volume and extent corresponding to MIS~4, 3 and 2 for
           each of the low-resolution simulations (Fig.~\ref{fig:timeseries}).}
  \label{tab:extrema}
  {\begin{tabular}{l*{3}{|ccc}}
    \tophline
             & \multicolumn{3}{c}{Age (ka)}
             & \multicolumn{3}{c}{Ice extent (\unit{10^6\,km^2})}
             & \multicolumn{3}{c}{Ice volume (m~s.l.e.)} \\
    Record   &  MIS~4 &  MIS~3 &  MIS~2
             &  MIS~4 &  MIS~3 &  MIS~2
             &  MIS~4 &  MIS~3 &  MIS~2 \\
    \middlehline
    GRIP     & -57.58 & -49.24 & -19.52
             &   1.98 &   0.46 &   2.13
             &   7.52 &   0.89 &   8.52 \\
    NGRIP    & -60.26 & -50.16 & -22.85
             &   2.16 &   0.50 &   2.09
             &   8.69 &   0.93 &   8.24 \\
    EPICA    & -61.87 & -45.57 & -17.10
             &   1.57 &   0.95 &   2.08
             &   5.20 &   2.44 &   8.35 \\
    Vostok   & -60.87 & -49.68 & -16.87
             &   1.55 &   0.86 &   2.14
             &   5.10 &   2.01 &   8.66 \\
    ODP 1012 & -55.41 & -47.08 & -23.21
             &   1.44 &   0.85 &   2.13
             &   4.50 &   2.06 &   8.46 \\
    ODP 1020 & -60.16 & -52.24 & -29.46
             &   1.32 &   0.70 &   2.08
             &   3.88 &   1.52 &   8.32 \\
    \middlehline
    Minimum  & -61.87 & -52.24 & -29.46
             &   1.32 &   0.46 &   2.08
             &   3.88 &   0.89 &   8.24 \\
    Maximum  & -55.41 & -45.57 & -16.87
             &   2.16 &   0.95 &   2.14
             &   8.69 &   2.44 &   8.66 \\
    \bottomhline
  \end{tabular}}
  \belowtable{}
\end{table*}

% avoid "too many unprocessed floats" errors
\clearpage

% fig:locmap
\begin{figure}
  \includegraphics{locmap}
  \caption{Relief map of northern North America showing a reconstruction of the
           areas once covered by the Cordilleran (CIS), Laurentide (LIS),
           Innuitian (IIS) and Greenland (GIS) ice sheets during the last
           18\,\unit{\chem{^{14}C}\,kyr\,BP} (21.4\,cal\,kyr\,BP)
           \citep{Dyke.2004}. The rectangular box denotes the location of the
           modelling domain used in this study. Major mountain ranges covered
           by the ice sheet include the Alaska Range (AR), the Wrangell and
           St.-Elias mountains (WSE), the Selwyn and MacKenzie mountains (SMK),
           the Skeena Mountains (SM), the Coast Mountains (CM), the Rocky
           Mountains (RM) and the North Cascades (NC). The background
           map consists of ETOPO1 \citep{Amante.Eakins.2009} and Natural Earth
           Data \citep{Patterson.Kelso.2014}.
           \todo{Mark palaeo-ice sheets and mountain ranges on the map.
                 Mark location of the Puget Lowland.}}
  \label{fig:locmap}
\end{figure}

% fig:atm
\begin{figure}
  \includegraphics{atm}
  \caption{Monthly mean near-surface air temperature, precipitation and
           standard deviation of daily mean temperature for January and July
           months from the North American Regional Reanalysis (NARR)
           climatology, used to force the ice sheet model. Note the
           strong contrasts in seasonality, timing of the precipitation peak,
           and temperature variability over the model domain, notably between
           the maritime and continental regions.
           \todo{Dash boundaries between ice sheets?}}
  \label{fig:atm}
\end{figure}

% fig:timeseries
\begin{figure*}
  \includegraphics{timeseries}
  \caption{Temperature offset time-series from ice core and sediment core
           records (Table~\ref{tab:records}) used as palaeo-climate forcing for
           the ice sheet model (top panel), and modelled ice volume
           through the last 120\,kyr, expressed in meters of sea-level
           equivalent (bottom panel). Gray spans indicate Marine Isotope
           Stages (MIS) according to a global compilation of benthic
           \chem{\delta^{18}O} records \citep{Lisiecki.Raymo.2005}. Hatched
           rectangles highlight modelled ice volume extrema corresponding to
           MIS~4 (61.9--55.4\,kyr), MIS~3 (52.2--45.6\,kyr), and
           MIS~2 (last glacial maximum, 29.5--16.9\,kyr). Dotted lines
           correspond to the GRIP and EPICA 6\,km-resolution runs.}
  \label{fig:timeseries}
\end{figure*}

% fig:snapshots
\begin{figure*}
  \includegraphics{snapshots}
  \caption{Snapshots of modelled surface topography (500\,m contours)
           corresponding to the ice volume extrema indicated on
           Fig.~\ref{fig:timeseries}. Note the occurence of spatial similarities
           despite large differences in timing.
           \todo{Perhaps indicate the location of the Skeena Mountains.}}
  \label{fig:snapshots}
\end{figure*}

% fig:mis2
\begin{figure*}
  \includegraphics{icemaps-mis2}
  \caption{Modelled surface topography (200\,m contours) and surface velocity
           (colour mapping) corresponding to the maximum ice volume during
           MIS~2 in the GRIP and EPICA high-resolution simulations.}
  \label{fig:mis2}
\end{figure*}

% fig:mis4
\begin{figure*}
  \includegraphics{icemaps-mis4}
  \caption{Modelled surface topography (200\,m contours) and surface velocity
           (colour mapping) corresponding to the maximum ice volume during
           MIS~4 in the GRIP and EPICA high-resolution simulations.}
  \label{fig:mis4}
\end{figure*}

% fig:duration
\begin{figure*}
  \includegraphics{duration}
  \caption{Modelled duration of ice cover during the last 120\,ka.
           Note the irregular colour scale. A contiguous ice cover spanning
           from the Alaska Range (AR) to the southern Coast Mountains (CM) and
           Rocky Mountains (RM) exists for about 29\,ka in both
           simulations. A central
           ice cover persists over the Skeena Mountains (SM) during most of the
           simulation. On the other hand, the maximal extent of the ice sheet
           generally corresponds to relatively short durations of ice cover.
           \todo{Add WSE, SMK, NR, NC mountain ranges, and changes kyr to ka}}
  \label{fig:duration}
\end{figure*}

% fig:warmbase
\begin{figure*}
  \includegraphics{warmbase}
  \caption{Modelled duration of warm-based ice cover during the last
           120\,kyr. Long ice cover durations combined with basal
           temperatures at the pressure-melting point may explain the strong
           glacial erosional imprint of the Skeena Mountains (SM) landscape.
           Hatches indicate areas that were covered by cold ice only.
           \todo{indicate location of the Skeena Mountains.}}
  \label{fig:warmbase}
\end{figure*}

% fig:warmfrac
\begin{figure*}
  \includegraphics{warmfrac}
  \caption{Modelled fraction of warm-based ice cover during the ice-covered
           period. Note the dominance of warm-based conditions on the
           continental shelf and major glacial troughs of the coastal ranges.
           Hatches indicate areas that were covered by cold ice only.
           \todo{indicate location of the Skeena Mountains.}}
  \label{fig:warmfrac}
\end{figure*}

% fig:deglacseries
\begin{figure}
  \includegraphics{deglacseries}
  \caption{Temperature offset time-series from the GRIP and EPICA ice core
           records (Table~\ref{tab:records}) (top panel), and modelled ice
           volume during the deglaciation, expressed in meters of sea-level
           equivalent (bottom panel).}
  \label{fig:deglacseries}
\end{figure}

% fig:deglacshots
\begin{figure*}
  \includegraphics{deglacshots}
  \caption{Snapshots of modelled surface topography (200\,m contours)
           and surface velocity (colour mapping) during the last deglaciation
           from the GRIP (top panels) and EPICA (bottom panels) simulations.}
  \label{fig:deglacshots}
\end{figure*}

% fig:deglac
\begin{figure*}
  \includegraphics{deglac}
  \caption{Modelled age of the last deglaciation. Areas where the MIS~4 glacial
           advance exceeded the last glacial maximum advanced are marked in
           green. Hatches denote re-advance of mountain-centred ice caps and
           and the decaying ice sheet between 14 and 10\,kyr., which is more
           pronounced in the GRIP-driven simulation.}
  \label{fig:deglac}
\end{figure*}

% fig:profiles-grip
\begin{figure}
  \includegraphics{profiles-grip}
  \caption{Modelled bedrock (black) and ice surface (blue) topography profiles
           during deglaciation (22.0--8.0\,kyr) in the GRIP 6\,km
           simulation, corresponding to the four transects indicated in
           Fig.~\ref{fig:deglac}.}
  \label{fig:profiles-grip}
\end{figure}

% fig:profiles-epica
\begin{figure}
  \includegraphics{profiles-epica}
  \caption{Modelled bedrock (black) and ice surface (blue) topography profiles
           during deglaciation (22.0--8.0\,kyr) in the EPICA 6\,km
           simulation, corresponding to the four transects indicated in
           Fig.~\ref{fig:deglac}.}
  \label{fig:profiles-epica}
\end{figure}

% fig:lastflow
\begin{figure*}
  \includegraphics{lastflow}
  \caption{Modelled directions of the deglacial basal flow velocities. Hatches
           indicate areas that remain non-sliding throughout deglaciation
           (22.0--8.0\,kyr), notably including the Interior Plateau of central
           British Columbia.
           Sliding grid cells were distinguished from non-sliding grid cells
           using a velocity threshold of 1\,\unit{m\,yr^{-1}}.}
  \label{fig:lastflow}
\end{figure*}

% =============================================================================
\end{document}
\endinput
% =============================================================================
