\documentclass[tc, manuscript]{copernicus}
\graphicspath{{figures/}}
\hypersetup{hidelinks}

\begin{document}\hack{\sloppy}


\title{Numerical simulations of the Cordilleran ice sheet
       through the last glacial cycle}

\Author[1,2,3]{J.}{Seguinot}
\Author[3]{I.}{Rogozhina}
\Author[2]{A.~P.}{Stroeven}
\Author[2]{M.}{Margold}
\Author[2]{J.}{Kleman}

\runningauthor{J.~Seguinot et~al.}

\correspondence{J.~Seguinot (seguinot@vaw.baug.ethz.ch)}

\runningtitle{Numerical simulations of the Cordilleran ice sheet
              through the last glacial cycle}

\affil[1]{Laboratory of Hydraulics, Hydrology and Glaciology, ETH Z\"{u}rich, Z\"{u}rich, Switzerland}
\affil[2]{Department of Physical Geography and the Bolin Centre for Climate Research, Stockholm University, Stockholm, Sweden}
\affil[3]{Helmholtz Centre Potsdam, GFZ German Research Centre for Geosciences, Potsdam,~Germany}


\hack{\allowdisplaybreaks}

\received{21~June 2015}
\accepted{25~June 2015}
\published{}

\firstpage{1}

\maketitle


\begin{abstract}
Despite more than a century of geological observations, the
Cordilleran ice sheet of North America remains poorly understood in
terms of its former extent, volume and dynamics. Although
geomorphological evidence is abundant, its complexity is such that
whole ice-sheet reconstructions of advance and retreat patterns are
lacking. Here we use a numerical ice sheet model calibrated against
field-based evidence to attempt a quantitative reconstruction of the
Cordilleran ice sheet history through the last glacial
cycle. A series of simulations is driven by time-dependent
temperature offsets from six proxy records located around the
globe. Although this approach reveals large variations in model
response to evolving climate forcing, all simulations produce two
major glaciations during marine oxygen isotope stages 4
(62.2--56.9\,\unit{ka}) and 2 (23.1--16.8\,\unit{ka}). The timing of
glaciation is better reproduced using temperature reconstructions
from Greenland and Antarctic ice cores than from regional oceanic
sediment cores. During most of the last glacial cycle, the modelled
ice cover is discontinuous and restricted to high mountain
areas. However, widespread precipitation over the Skeena Mountains
favours the persistence of a central ice dome throughout the glacial
cycle. It acts as a nucleation centre before the Last Glacial
Maximum and hosts the last remains of Cordilleran ice until the
middle Holocene (6.7\,\unit{ka}).
\end{abstract}



\introduction
\label{sec:intro}

      During the last glacial cycle, glaciers and ice caps of the North
      American Cordillera have been more extensive than today. At the Last
      Glacial Maximum (LGM), a~continuous blanket of ice, the Cordilleran
      ice sheet \citep{Dawson.1888}, stretched from the Alaska Range in the
      north to the North Cascades in the south (Fig.~\ref{fig:locmap}). In
      addition, it extended offshore, where it calved into the Pacific
      Ocean, and merged with the western margin of its much larger
      neighbour, the Laurentide ice sheet, east of the Rocky Mountains.

      More than a~century of exploration and geological investigation of the
      Cordilleran mountains have led to many observations in support of the
      former ice sheet \citep{Jackson.Clague.1991}. Despite the lack of
      documented end moraines offshore, in the zone of confluence with the
      Laurentide ice sheet, and in areas swept by the Missoula floods
      \citep{Carrara.etal.1996}, moraines that demarcate the northern and
      south-western margins provide key constraints that allow reasonable
      reconstructions of maximum ice sheet extents
      (\citealp{Prest.etal.1968}; \citealp[Fig.~1.12]{Clague.1989};
      \citealp{Duk-Rodkin.1999}; \citealp{Booth.etal.2003};
      \citealp{Dyke.2004}). As indicated by field evidence from radiocarbon
      dating \citep{Clague.etal.1980, Clague.1985, Clague.1986,
      Porter.Swanson.1998, Menounos.etal.2008}, cosmogenic exposure dating
      \citep{Stroeven.etal.2010, Stroeven.etal.2014, Margold.etal.2014},
      bedrock deformation in response to former ice loads
      \citep{Clague.James.2002, Clague.etal.2005}, and offshore sedimentary
      records \citep{Cosma.etal.2008, Davies.etal.2011}, the LGM Cordilleran
      ice sheet extent was short-lived. However, former ice thicknesses and,
      therefore, the ice sheet's contribution to the LGM sea level lowstand
      \citep{Carlson.Clark.2012, Clark.Mix.2002} remain uncertain.

      Our understanding of the Cordilleran glaciation history prior to the
      LGM is even more fragmentary \citep{Barendregt.Irving.1998,
      Kleman.etal.2010, Rutter.etal.2012}, although it is clear that the
      Pleistocene maximum extent of the Cordilleran ice sheet predates the
      last glacial cycle \citep{Hidy.etal.2013}. In parts of the Yukon
      Territory and Alaska, and in the Puget Lowland, the distribution of
      tills \citep{Turner.etal.2013, Troost.2014} and dated glacial erratics
      \citep{Ward.etal.2007, Ward.etal.2008, Briner.Kaufman.2008,
      Stroeven.etal.2010, Stroeven.etal.2014} indicate an extensive Marine
      Oxygen Isotope Stage (MIS)~4 glaciation. Landforms in the interior
      regions include flow sets that are likely older than the LGM
      \citep[Fig.~2]{Kleman.etal.2010}, but their absolute age remains
      uncertain.

      In contrast, evidence for the deglaciation history of the Cordilleran
      ice sheet since the LGM is considerable, albeit mostly at a~regional
      scale. Geomorphological evidence from south-central British Columbia
      indicates a~rapid deglaciation, including an early emergence of
      elevated areas while thin, stagnant ice still covered the surrounding
      lowlands \citep{Fulton.1967, Fulton.1991, Margold.etal.2011,
      Margold.etal.2013a}. This model, although credible, may not apply in
      all areas of the Cordilleran ice sheet \citep{Margold.etal.2013}.
      Although solid evidence for late-glacial glacier re-advances has been
      found in the Coast, Columbia and Rocky mountains
      \citep{Reasoner.etal.1994, Osborn.Gerloff.1997,
      Clague.etal.1997, Friele.Clague.2002, Friele.Clague.2002a,
      Kovanen.2002, Kovanen.Easterbrook.2002, Lakeman.etal.2008,
      Menounos.etal.2008}, it appears to be sparser than for formerly
      glaciated regions surrounding the North Atlantic
      \citep[e.g.,][]{Sissons.1979, Lundqvist.1987, Ivy-Ochs.etal.1999,
      Stea.etal.2011}. Nevertheless, recent oxygen isotope measurements from
      Gulf of Alaska sediments reveal a~climatic evolution highly correlated
      to that of Greenland during this period, including a~distinct Late
      Glacial cold reversal between 14.1 and 11.7\,\unit{ka}
      \citep{Praetorius.Mix.2014}.

      In general, the topographic complexity of the North American
      Cordillera and its effect on glacial history have inhibited the
      reconstruction of ice sheet-wide glacial advance and retreat patterns
      such as those available for the Fennoscandian and Laurentide ice
      sheets \citep{Boulton.etal.2001, Dyke.Prest.1987, Dyke.etal.2003,
      Kleman.etal.1997, Kleman.etal.2010, Stroeven.etal.inreview}. Here, we
      use a~numerical ice sheet model \citep{PISM-authors.2015}, calibrated
      against field-based evidence, to perform a~quantitative reconstruction
      of the Cordilleran ice sheet evolution through the last glacial cycle,
      and analyse some of the long-standing questions related to its
      evolution:
\begin{itemize}
  \item How much ice was locked in the Cordilleran ice sheet during the LGM?
  \item What was the scale of glaciation prior to the LGM?
  \item Which were the primary dispersal centres? Do they reflect stable or
    ephemeral configurations?
  \item How rapid was the last deglaciation? Did it include Late Glacial
    standstills or readvances?
\end{itemize}
      Although numerical ice sheet modelling has been established as
      a~useful tool to improve our understanding of the Cordilleran ice
      sheet (\citealp[p.~227]{Jackson.Clague.1991}; \citealp{Robert.1991};
      \citealp{Marshall.etal.2000}), the ubiquitously mountainous topography
      of the region has presented two major challenges to its application.
      First, only recent developments in numerical ice sheet models and
      underlying scientific computing tools \citep{Bueler.Brown.2009,
      Balay.etal.2015} have allowed for high-resolution numerical modelling
      of glaciers and ice sheets on mountainous terrain over millennial time
      scales \citep[e.g.,][]{Golledge.etal.2012}. Second, the complex
      topography of the North American Cordillera also induces strong
      geographic variations in temperature and precipitation
      \citep{Jarosch.etal.2012}, thus requiring
      the use of high-resolution climate forcing fields as an input to an
      ice sheet model \citep{Seguinot.etal.2014}. However, evolving climate
      conditions over the last glacial cycle are subject to considerable
      uncertainty and still lie beyond the computational reach of atmosphere
      circulation models.

      Our palaeo-climate forcing therefore includes spatial temperature and
      precipitation grids derived from a~present-day atmospheric reanalysis
      \citep{Mesinger.etal.2006} that was previously tested against
      observational data and shown to best reproduce the steep precipitation
      gradients previously identified as necessary to model the LGM extent
      of the Cordilleran ice sheet in agreement with its geological imprint
      among four atmospheric reanalyses available over the study area
      \citep{Seguinot.etal.2014}. To mimic climate evolution through the
      last glacial cycle, these grids are simply supplemented by lapse-rate
      corrections and temperature offset time series. The latter are
      obtained by scaling six different palaeo-temperature reconstructions
      from proxy records around the globe, including two oxygen isotope
      records from Greenland ice cores \citep{Dansgaard.etal.1993,
      Andersen.etal.2004}, two oxygen isotope records from Antarctic ice
      cores \citep{Petit.etal.1999,Jouzel.etal.2007}, and two alkenone
      unsaturation index records from Northwest Pacific ocean sediment cores
      \citep{Herbert.etal.2001},

      Although these proxy records were all obtained outside the model
      domain, more regional palaeo-temperature reconstructions spanning over
      the last glacial cycle are lacking. For instance, the Mount Logan ice
      core oxygen isotope record covers only the last 30\,000\,\unit{years}
      (30\,\unit{ka}) and has been interpreted as a proxy for source region
      rather than for palaeo-temperature \citep{Fisher.etal.2004,
      Fisher.etal.2008}. Sea-surface temperatures have been reconstructed
      offshore Vancouver Island from alkenone unsaturation indices over the
      last 16\,\unit{\chem{^{14}C}\,cal\,ka} \citep{Kienast.McKay.2001}, and
      from the Mg/Ca ratio in planktonic foraminifera over the period from
      10 to ca.~50\,\unit{\chem{^{14}C}\,cal\,ka} \citep{Taylor.etal.2014,
      Taylor.etal.2015}, but these records cover only parts of the last
      glacial cycle.

      After testing the model sensitivity to
      these climate forcing time-series (Sect.~\ref{sec:results}) and to
      some of the most influential ice flow parameters
      (Sect.~\ref{sec:sens}), we then proceed to compare the model output
      to geological evidence and discuss the timing and extent of glaciation
      and the patterns of deglaciation, based on which we use the
      applicability of different records to modelling the history of the
      Cordilleran ice sheet (Sect.~\ref{sec:discussion}).


\section{Model setup}
\label{sec:model}

\subsection{Overview}
\label{sec:overview}%

      The simulations presented here were run using the Parallel Ice Sheet
      Model (PISM, development version~8ff7cbe), an open source, finite
      difference, shallow ice sheet model \citep{PISM-authors.2015}. The
      model requires input on basal topography, sea level, geothermal heat
      flux and climate forcing. It computes the evolution of ice extent and
      thickness over time, the thermal and dynamic states of the ice sheet,
      and the associated lithospheric response.

      Basal topography is bilinearly interpolated from the ETOPO1 combined
      topography and bathymetry dataset with a~resolution of 1\,arc-min
      \citep{Amante.Eakins.2009} to the model grids. Sea level is lowered as
      a~function of time based on the Spectral Mapping Project
      \citep[SPECMAP,][]{Imbrie.etal.1989} time scale. Ice deformation
      follows temperature and water-content dependent creep
      (Sect.~\ref{sec:icedyn}). Basal sliding follows
      a pseudo-plastic law where the yield stress accounts for till
      deconsolidation under high water pressure (Sect.~\ref{sec:sliding}).
      Ice shelf calving is calculated based on ice thickness and principle
      strain rates (Sect.~\ref{sec:sliding}). Surface mass balance is
      computed using a~positive degree-day (PDD) model
      (Sect.~\ref{sec:surface}). Default parameter values are given in
      Table~\ref{tab:params}. Climate forcing is provided by a~monthly
      climatology averaged from 1979 to 2000 from the North American
      Regional Reanalysis \citep[NARR,][]{Mesinger.etal.2006}, perturbed
      by time-dependent offsets and lapse-rate temperature corrections
      (Sect.~\ref{sec:atm}, Table~\ref{tab:records}). A sensitivity study
      to some of the most influential ice rheology (Sect~\ref{sec:icedyn})
      and basal sliding (Sect.~\ref{sec:sliding}) parameters was performed
      (Table~\ref{tab:sens_params}).

      Each simulation starts from assumed ice-free conditions at
      120\,\unit{ka}, and runs to the present.
      Our modelling domain of 1\,500 by 3\,000\,\unit{km} encompasses
      the entire area covered by the Cordilleran ice sheet at the LGM
      (Fig.~\ref{fig:locmap}). The simulations were run on two distinct
      grids, using a~lower horizontal resolution of 10\,\unit{km}, and
      a~higher horizontal resolution of 5\,\unit{km}. These computations
      were performed on 16 to 128 computing cores at the Swedish National
      Supercomputing Centre.

\subsection{Ice rheology}
\label{sec:icedyn}

      Ice sheet dynamics are typically modelled using a~combination of
      internal deformation and basal sliding. PISM is a~shallow ice sheet
      model, which implies that the balance of stresses is approximated
      based on their predominant components. The Shallow Shelf Approximation
      (SSA) is combined with the Shallow Ice Approximation
      (SIA) by adding velocity solutions of the two approximations
      \citep[Eqs.~7--9 and 15]{Winkelmann.etal.2011}. The computational
      efficiency of this hybrid scheme nevertheless comes at the cost of
      violating stress conservation in the transition zone where
      gravitational stresses have a double contribution to computed SIA and
      SSA velocities, leading to a positive bias of total ice velocity that
      remains to be quantified.

      Ice deformation is governed by the constitutive law for ice
      \citep{Glen.1952, Nye.1953},
%
\begin{align}
&\label{eqn:glenslaw}
&\vec{\dot{\epsilon}} = A\,\tau_{\mathrm{e}}^{n-1}\,\vec{\tau} \,.
\end{align}
%
      where $\vec{\dot{\epsilon}}$ is the strain-rate tensor,
      $\vec{\tau}$ the deviatoric stress tensor, and $\tau_{\mathrm{e}}$ the
      equivalent stress defined by ${\tau_{\mathrm{e}}}^2 = \frac{1}{2}
      \mathrm{tr}(\vec{\tau}^2)$. The ice softness coefficient, $A$, depends
      on ice temperature, $T$, pressure, $p$, and water content, $\omega$,
      through a piece-wise Arrhenius-type law
      \citep[Eqs.~63--65]{Aschwanden.etal.2012},
%
\begin{align}
&\label{eqn:softness}
&A = E\cdot
\begin{cases}
A_{\mathrm{c}} \,e^\frac{-Q_{\mathrm{c}}}{RT_{\text{pa}}}            & \text{if}\ T_{\text{pa}}  <  T_{\mathrm{c}} \,, \\
A_{\mathrm{w}} (1+f\omega)\,e^\frac{-Q_{\mathrm{w}}}{RT_{\text{pa}}} & \text{if}\ T_{\text{pa}} \ge T_{\mathrm{c}} \,,
\end{cases}
\end{align}
%
      where $T_{\text{pa}}$ is the pressure-adjusted ice temperature
      calculated using the Clapeyron relation, ${T_{\text{pa}} = T - \beta
      p}$. $R=8.31441$\,\unit{J\,mol^{-1}\,K^{-1}} is the ideal gas
      constant, and $A_{\mathrm{c}}$, $A_{\mathrm{w}}$, $Q_{\mathrm{c}}$ and
      $Q_{\mathrm{w}}$, are constant parameters corresponding to values
      measured below and above a critical temperature threshold
      $T_{\mathrm{c}}=-10$\,\unit{{\degree}C}
      \citep[p.~72]{Paterson.Budd.1982,Cuffey.Paterson.2010}. The water
      fraction, $\omega$, is capped at a maximum value of 0.01, above which
      no measurements are available \citep[Eq.~5.7]{Lliboutry.Duval.1985,
      Greve.1997}. Finally, $E$ is a non-dimensional enhancement factor
      which can take different values, $E_{\text{SIA}}$, in the Shallow Ice
      Approximation (SIA) and $E_{\text{SSA}}$, in the Shallow Shelf
      Approximation (SSA).

      In all our simulations, we set constant the power-law exponent, $n=3$,
      according to \citet[p.~55--57]{Cuffey.Paterson.2010}, the Clapeyron
      constant, $\beta=7.9\times 10^{-8}$\,\unit{K\,Pa^{-1}}, according to
      \citet{Luthi.etal.2002}, the water fraction coefficient, $f=181.25$,
      according to \citet{Lliboutry.Duval.1985}, and the SSA enhancement
      factor, $E_{\text{SSA}}=1$, according to
      \citet[p.~77]{Cuffey.Paterson.2010}. These fixed parameter values are
      summarized in Table~\ref{tab:params}.

      On the other hand, we test the model sensitivity
      (Sect.~\ref{sec:sens}) to different values for the two creep
      parameters, $A_{\mathrm{c}}$ and $A_{\mathrm{w}}$, the two activation
      energies, $Q_{\mathrm{c}}$ and $Q_{\mathrm{w}}$, and the SIA
      enhancement factor, $E_{\text{SIA}}$, as follow.
%
\begin{itemize}
  \item Our \emph{default} configuration used in the control run of the
    sensitivity study and all other simulations in the paper include
    rheological parameters,
    $A_{\mathrm{c}}$, $A_{\mathrm{w}}$, $Q_{\mathrm{c}}$ and
    $Q_{\mathrm{w}}$, derived from \citet{Paterson.Budd.1982} and given in
    \citet[Eqn.~5]{Bueler.Brown.2009}, and $E_{\text{SIA}}=1$.
  \item Our \emph{hard ice} configuration include rheological parameters,
    $A_{\mathrm{c}}$, $A_{\mathrm{w}}$, $Q_{\mathrm{c}}$ and
    $Q_{\mathrm{w}}$, derived from \citet[p.~72 and
    76]{Cuffey.Paterson.2010}, and $E_{\text{SIA}}=1$, which correspond to a
    stiffer rheology than that used in the control run.
  \item Our \emph{soft ice} configuration include rheological parameters
    from \citet{Cuffey.Paterson.2010}, and $E_{\text{SIA}}=5$, the
    recommended value for ice age polar ice
    \citep[p.~77]{Cuffey.Paterson.2010}.
\end{itemize}
%
      An additional simulation using the ice rheology from
      \citet{Cuffey.Paterson.2010} and $E_{\text{SIA}}=2$, the recommended
      value for Holocene polar ice \citep[p.~77]{Cuffey.Paterson.2010} was
      performed, but its results were very similar to that of our default
      run and thus not presented here.

      Actual parameter values for $A_{\mathrm{c}}$, $A_{\mathrm{w}}$,
      $Q_{\mathrm{c}}$, $Q_{\mathrm{w}}$ and $E_{\text{SIA}}$ used in our
      simulations are given in Table~\ref{tab:sens_params}, while the effect
      of the three different parametrisations on temperature-dependent ice
      softness, $A$, is illustrated in Fig.~\ref{fig:sens_plot_rheo}.

      Surface air
      temperature derived from the climate forcing (Sect.~\ref{sec:atm})
      provides the upper boundary condition to the ice enthalpy
      model. Temperature is computed in the ice and in the bedrock to
      a~depth of 3\,\unit{km} below the ice-bedrock interface, where it is
      conditioned by a~lower boundary geothermal heat flux of
      $q_{\mathrm{G}}=70$\,\unit{mW\,m^{-2}}. Although this uniform value
      does
      not account for the high spatial geothermal variability in the region
      \citep{Blackwell.Richards.2004}, it is, on average, representative of
      available heat flow measurements. In the low-resolution simulations,
      the vertical grid consists of 31~temperature layers in the bedrock and
      up to 51~enthalpy layers in the ice sheet, corresponding to a~vertical
      resolution of 100\,\unit{m}. The high-resolution simulations use
      61~bedrock layers and up to 101~ice layers with a~vertical resolution
      of 50\,\unit{m}.

\subsection{Basal sliding}
\label{sec:sliding}

      A~pseudo-plastic sliding law,
%
\begin{align}
&\label{eqn:pseudoplastic}
    \vec{\tau}_{\mathrm{b}} = -\tau_{\mathrm{c}} \frac{\vec{v}_{\mathrm{b}}}{{v_{\text{th}}}^q\,|\vec{v}_{\mathrm{b}}|^{1-q}} \,,
\end{align}
%
      relates the bed-parallel shear stresses, $\vec{\tau}_{\mathrm{b}}$, to
      the sliding velocity, $\vec{v}_{\mathrm{b}}$.
      The yield stress, $\tau_{\mathrm{c}}$, is modelled using the
      Mohr--Coulomb criterion,
%
\begin{align}
&\tau_{\mathrm{c}} = c_0 + N\,\tan{\phi} \,,
\end{align}
%
      where cohesion, $c_0$, is assumed to be zero. Effective pressure, $N$,
      is related to the ice overburden stress, $P_0=\rho gh$, and the
      modelled amount of
      subglacial water, using a~formula derived from laboratory experiments
      with till extracted from the base of Ice Stream B in West Antarctica
      \citep{Tulaczyk.etal.2000, Bueler.Pelt.2015},
%
\begin{align}
&\label{eqn:ntil}
&N = \delta P_0 \, 10^{(e_0/C_{\mathrm{c}}) (1 - (W/W_{\text{max}}))} \,,
\end{align}
%
      where $\delta$ sets the minimum ratio between the effective and
      overburden pressures, $e_0$ is a~reference void ratio and
      $C_{\mathrm{c}}$ is the till compressibility coefficient
      \citep{Tulaczyk.etal.2000}. The amount of water at the base, $W$,
      varies from zero to $W_{\text{max}}$, a~threshold above which
      additional melt water is assumed to drain off instantaneously.

      In all our sumlations, we set constant the pseudo-plastic sliding
      exponent, $q=0.25$, and the threshold velocity,
      $v_{\text{th}}=100$\,\unit{m\,a^{-1}}, according to values used by
      \citet{Aschwanden.etal.2013}, the till cohesion, $c_0=0$, whose
      measured values are consistently negligible
      \citep[p.~268]{Tulaczyk.etal.2000, Cuffey.Paterson.2010}, the till
      reference void ratio, $e_0=0.69$, and the till compressibility
      coefficient, $C_{\mathrm{c}}=0.12$, according to the only measurements
      available to our knowledge, published by \citep{Tulaczyk.etal.2000}.
      These fixed parameter values are summarized in Table~\ref{tab:params}.

      We also use a constant spatial distribution of the till friction
      angle, $\phi$, whose values vary from 15 to 45\unit{\degree} as a
      a~piecewise-linear function of modern bed elevation, with the lowest
      value occuring below modern sea level (0\,\unit{m} above sea level,
      \unit{m\,a.s.l.}) and the highest value occuring above the generalised
      elevation of the highest shorelines
      \citep[200\,\unit{m\,a.s.l.},][Fig.~5]{Clague.1981}. This range of
      values span over the range of measured values for glacial till of 18
      to 40\unit{\degree} \citep[p.~268]{Cuffey.Paterson.2010}. It accounts
      for frictional basal conditions associated with discontinuous till
      cover at high elevations, and for a~weakening of till associated with
      the presence of marine sediments (cf. \citealp{Martin.etal.2011};
      \citealp[Supplement]{Aschwanden.etal.2013};
      \citealp{PISM-authors.2015}).

      An additional simulation with a constant till friction angle,
      $\phi=30$\unit{\degree}, corresponding to the average value in
      \citet[p.~268]{Cuffey.Paterson.2010}, was actually performed, but the
      induced variability was small and thus not presented here.

      On the other hand, we test the model sensitivity
      (Sect.~\ref{sec:sens}) to different values for the minimum ratio
      between the effective and overburden pressures, $\delta$, and the
      maximum water height in the till, $W_{\text{max}}$, as follow.
%
\begin{itemize}
  \item Our \emph{default} configuration used in the control run of the
    sensitivity study and all and all other simulations in the paper include
    $\delta=0.02$ and $W_{\text{max}}=2$\,\unit{m} as in
    \citet{Bueler.Pelt.2015}.
  \item Our \emph{soft bed} configuration use $\delta=0.01$ and
    $W_{\text{max}}=1$\,\unit{m}.
  \item Our \emph{hard bed} configuration use $\delta=0.05$ and
    $W_{\text{max}}=5$\,\unit{m}.
\end{itemize}
%
      The effect of the three different parametrisations on the effective
      pressure on the till, $N$, in response to water content, $W$, is
      illustrated in Fig.~\ref{fig:sens_plot_ntil}. All parameter choices
      are listed in Table~\ref{tab:sens_params}.

      Finally, the bedrock topography responds to ice load following
      a~bedrock deformation model that includes local isostasy, elastic
      lithosphere flexure and viscous astenosphere deformation in an
      infinite half-space \citep{Lingle.Clark.1985,Bueler.etal.2007}.
      A~relatively low viscosity value of $\nu_{\mathrm{m}} =
      1\times10^{19}$\,\unit{Pa\,s} is used for the astenosphere
      (Table~\ref{tab:params}) in accordance with the results from regional
      glacial isostatic adjustment modelling at the northern Cascadia
      subduction zone \citep{James.etal.2009}.

\subsection{Ice shelf calving}
\label{sec:calving}

      Ice shelf calving is computed using a~double criterion. First,
      a~physically-realistic calving flux is computed based on eigenvalues
      of the horizontal strain rate tensor \citep{Winkelmann.etal.2011,
      Levermann.etal.2012}. This allows floating ice to advance in confined
      embayments, but prevents the formation of extensive ice shelves in the
      open ocean. Second, floating ice thinner than 50\,\unit{m} is
      systematically calved off. A~subgrid scheme by
      \citet{Albrecht.etal.2011} allows for a~continuous migration of the
      calving front. This formulation of calving has been applied to the
      Antarctic ice sheet and has shown to produce a~realistic calving front
      positon for many of the present-day ice shelves
      \citep{Martin.etal.2011}.

\subsection{Surface mass balance}
\label{sec:surface}

      Ice surface accumulation and ablation are computed from monthly mean
      near-surface air temperature, $T_{\mathrm{m}}$, monthly standard
      deviation of near-surface air temperature, $\sigma$, and monthly
      precipitation, $P_{\mathrm{m}}$, using a~temperature-index model
      \citep[e.g.,][]{Hock.2003}. Accumulation is equal to precipitation
      when air temperatures are below 0\,\unit{{\degree}C}, and decreases to
      zero linearly with temperatures between 0 and 2\,\unit{{\degree}C}.
      Ablation is computed from PDD, defined as an integral of temperatures
      above 0\,\unit{{\degree}C} in one year.

      The PDD computation accounts for stochastic temperature variations by
      assuming a~normal temperature distribution of standard deviation
      $\sigma$ around the expected value $T_{\mathrm{m}}$. It is expressed
      by an error-function formulation \citep{Calov.Greve.2005},
\begin{align}
&\label{eqn:calovgreve}
    {\text{PDD}} = \int_{t_1}^{t_2} \mathrm{d}t
        \left[\frac{\sigma}{\sqrt{2\pi}}
                \exp\left({-\frac{T_{\mathrm{m}}^2}{2\sigma^2}}\right)
              + \frac{T_{\mathrm{m}}}{2} \, {\text{erfc}}
                \left(-\frac{T_{\mathrm{m}}}{\sqrt{2}\sigma}\right)\right] \,,
\end{align}
      which is numerically approximated using week-long sub-intervals. In
      order to account for the effects of spatial and seasonal variations of
      temperature variability \citep{Seguinot.2013}, $\sigma$ is computed
      from NARR daily temperature values from 1979 to 2000
      \citep{Mesinger.etal.2006}, including variability associated with the
      seasonal cycle, and bilinearly interpolated to the model grids
      (Fig.~\ref{fig:atm}). Degree-day factors for snow and
      ice melt are derived from mass-balance measurements on contemporary
      glaciers from the Coast Mountains and Rocky Mountains in British
      Columbia \citep[Table~\ref{tab:params};][]{Shea.etal.2009}.

\subsection{Climate forcing}
\label{sec:atm}%

      Climate forcing driving ice sheet simulations consists of
      a~present-day monthly climatology, $\{T_{\mathrm{m}0},
      P_{\mathrm{m}0}\}$, where temperatures are modified by offset time
      series, ${\Delta}T_{\text{TS}}$, and lapse-rate corrections,
      ${\Delta}T_{\text{LR}}$:
\begin{align}
&T_{\mathrm{m}}(t, x, y) = T_{\mathrm{m}0}(x, y) + {\Delta}T_{\text{TS}}(t)
                    + {\Delta}T_{\text{LR}}(t, x, y) \,, \\
&    P_{\mathrm{m}}(t, x, y) = P_{\mathrm{m}0}(x, y) \,.
\end{align}
      The present-day monthly climatology was bilinearly interpolated from
      near-surface air
      temperature and precipitation rate fields from the NARR, averaged from
      1979 to 2000. Modern climate of the North American Cordillera is
      characterised by strong geographic variations in temperature
      seasonality, timing of the maximum annual precipitation, and daily
      temperature variability (Fig.~\ref{fig:atm}). Although the ability
      of NARR to reproduce these steep climatic gradients is limited by its
      spatial resolution of 32\,\unit{km} \citep{Jarosch.etal.2012}, it was
      tested against observational data in a~previous sensitivity study and
      identified as yielding modelled Cordilleran ice sheet LGM extent in
      closer agreement to geological reconstructions than other atmospheric
      reanalyses \citep{Seguinot.etal.2014}.

      Temperature offset time-series, ${\Delta}T_{\text{TS}}$, are derived
      from palaeo-temperature proxy records from the Greenland Ice Core
      Project \citep[GRIP,][]{Dansgaard.etal.1993}, the North Greenland Ice
      Core Project \citep[NGRIP,][]{Andersen.etal.2004}, the European
      Project for Ice Coring in Antarctica \citep[EPICA,][]
      {Jouzel.etal.2007}, the Vostok ice core \citep{Petit.etal.1999}, and
      Ocean Drilling Program (ODP) sites 1012 and 1020, both located off the
      coast of California \citep{Herbert.etal.2001}. Palaeo-temperature
      anomalies from the GRIP and NGRIP records were calculated from oxygen
      isotope (\chem{\delta^{18}O}) measurements using a~quadratic equation
      \citep{Johnsen.etal.1995},
\begin{align}
{\Delta}T_{\text{TS}}(t) ={~}&-11.88 [\chem{\delta^{18}O}(t)
                                -\chem{\delta^{18}O}(0)] \nonumber \\
                        &-0.1925[\chem{\delta^{18}O}(t)^2
                                 -\chem{\delta^{18}O}(0)^2] \,,
\end{align}
      while temperature reconstructions from Antarctic and oceanic cores
      were provided as such. For each proxy record used and each of the
      configurations used in the sensitivity tests, palaeo-temperature
      anomalies were scaled linearly (Tables~\ref{tab:records}
      and~\ref{tab:sens_params}) in order to simulate comparable ice extents
      at the LGM (Table~\ref{tab:extrema}) and realistic outlines
      (Fig.~\ref{fig:lr_maps}).

      Finally, lapse-rate corrections, ${\Delta}T_{\text{LR}}$, are computed
      as a~function of ice surface elevation, $s$, using the NARR surface
      geopotential height invariant field as a~reference topography,
      $b_{\text{ref}}$:
\begin{align}
{\Delta}T_{\text{LR}}(t, x, y) &= -\gamma [s(t, x, y)-b_{\text{ref}}] \\
                            &= -\gamma [h(t, x, y)+b(t, x, y)-b_{\text{ref}}],
\end{align}
      thus accounting for the evolution of ice thickness, ${h=s-b}$, on the
      one hand, and for differences between the basal topography of the ice
      flow model, $b$, and the NARR reference topography, $b_{\text{ref}}$,
      on the other hand. All simulations use an annual temperature lapse
      rate of $\gamma = 6\,\unit{K\,km^{-1}}$. In the rest of this paper, we
      refer to different model runs by the name of the proxy record used for
      the palaeo-temperature forcing.


\section{Sensitivity to climate forcing time-series}
\label{sec:results}

\subsection{Evolution of ice volume}

      Despite large differences in the input climate forcing
      (Fig.~\ref{fig:lr_ts}, upper panel), model output presents consistent
      features that can be observed across the range of forcing data used.
      In all simulations, modelled ice volumes remain relatively low during
      most of the glacial cycle, except during two major glacial events
      which occur between 62.2 and 56.9\,\unit{ka} during MIS~4, and between
      23.1 and 16.8\,\unit{ka} during MIS~2 (Fig.~\ref{fig:lr_ts}, lower
      panel). An ice volume minimum is consistently reached between 56.0 and
      42.9\,\unit{ka} during MIS~3. However, the magnitude and precise
      timing of these three events depend significantly on the choice of
      proxy record used to derive a~time-dependent climate forcing
      (Table~\ref{tab:extrema}).

      Simulations forced by the Greenland ice core palaeo-temperature
      records (GRIP, NGRIP) produce the highest variability in modelled ice
      volume throughout the last glacial cycle. In contrast, simulations
      driven by oceanic (ODP~1012, ODP~1020) and Antarctic (EPICA, Vostok)
      palaeo-temperature records generally result in lower ice volume
      variability throughout the simulation length, resulting in lower
      modelled ice volumes during MIS~4 and larger ice volumes during MIS~3.
      The NGRIP climate forcing is the only one that results in a~larger ice
      volume during MIS~4 than during MIS~2.

      While simulations driven by the GRIP and the two Antarctic
      palaeo-temperature records attain a~last ice volume maximum between
      19.1 and 16.8\,\unit{ka}, those informed by the NGRIP and the two
      oceanic palaeo-temperature records attain their maximum ice volumes
      thousands of years earlier. Moreover, the ODP~1012 run yields a~rapid
      deglaciation of the modelled area prior to 17\,\unit{ka}. The ODP~1020
      simulation predicts an early maximum in ice volume at 21.0\,\unit{ka},
      followed by slower deglaciation than modelled using the other
      palaeo-temperature records. Finally, whereas model runs forced by the
      Antarctic palaeo-temperature records result in a~rapid and
      uninterrupted deglaciation after the LGM, the simulation driven by the
      GRIP palaeo-temperature record also results in a~rapid deglaciation
      but in three steps, separated by two periods of ice sheet regrowth
      (Fig.~\ref{fig:lr_ts}).

\subsection{Extreme configurations}

      Despite large differences in the timing of attained volume extrema
      (Table~\ref{tab:extrema}), all model runs show relatively consistent
      patterns of glaciation. During MIS~4, all simulations produce an
      extensive ice sheet, covering an area of at least half of that
      attained during MIS~2 (Table~\ref{tab:extrema};
      Fig.~\ref{fig:lr_maps}, upper panels). Corresponding maximum ice
      volumes also differ significantly between model runs, and vary between
      3.54 and~8.71\,\unit{m} sea level equivalents (m\,s.l.e.;
      Table~\ref{tab:extrema}).

      In the MIS~3 ice volume minimum reconstructions, a~central ice cap
      persists over the Skeena Mountains (Fig.~\ref{fig:lr_maps}, middle
      panels). Although this ice cap is present in all simulations, its
      dimensions depend sensitively on the choice of the applied
      palaeo-temperature record. Modelled ice volume minima spread over
      a~wide range between 1.54 and 2.94\,\unit{m}\,s.l.e.
      (Table~\ref{tab:extrema}).

      Modelled ice sheet geometries during the LGM (MIS~2;
      Fig.~\ref{fig:lr_maps}, lower panels) invariably include a~ca.
      1500\,\unit{km}-long central divide above 3000\,\unit{m\,a.s.l.}
      located along the spine of the Rocky Mountains. As an indirect result
      of the choice of scaling factors applied to different
      palaeo-temperature proxy records (Table~\ref{tab:records}), modelled
      maximum ice volumes also fall within a~tight range of 8.39 to
      9.07\,\unit{m}\,s.l.e. (Table~\ref{tab:extrema}).


\section{Sensitivity to ice flow parameters}
\label{sec:sens}

      Using the GRIP ice core palaeo-temperature record as climate forcing
      time-series, the model shows a significant sensitivity to selected ice
      rheological and basal sliding parameters (Fig.~\ref{fig:sens_ts};
      Table~\ref{tab:sens_extrema}) in terms of modelled ice extent, and
      even more so in terms of modelled ice volumes.

      As a direct result of the different scaling factors applied
      (Table~\ref{tab:sens_params}), the resulting simulations show very
      little difference in the modelled glaciated areas corresponding to
      maximum ice volumes during MIS~2, but also during MIS~4
      (Table~\ref{tab:sens_extrema}). However, such can not be said of the
      modelled glaciated area corresponding to minimum ice volume during
      MIS~3. In fact, the extent of the remnant ice cap which persists over
      the Skeena Mountains during this stage shows a significant sensitivity
      to ice rheology of 31\,\unit{\%}, and an even more important
      sensitivity to basal sliding parameters of 62\,\unit{\%}
      (Table~\ref{tab:sens_extrema}).

      The modelled sea-level relevant ice volumes show more variability than
      the modelled glaciated areas (Table~\ref{tab:sens_extrema},
      Fig.~\ref{fig:sens_ts}). As one could expect, softer ice and weaker
      till both result in a thinner ice sheet, while harder ice and stronger
      till result in a thicker ice sheet. For instance, peak ice volume
      during the MIS~2 (LGM) varies by 30\,\unit{\%} between the two
      parametrisations of ice rheology used, and by 21\,\unit{\%} between
      the two parametrisations of basal sliding used. The differences in
      sea-level relevant ice volume are greatest during the MIS~3
      (Table~\ref{tab:sens_extrema}, Fig.~\ref{fig:sens_ts}) where both the
      areal and thickness contributions add up.


\section{Comparison with the geologic record}
\label{sec:discussion}

      Large variations in the model responses to evolving climate forcing
      reveal its sensitivity to the choice of palaeo-temperature proxy
      record. To distinguish between different records, geological evidence
      of former glaciations provides a~basis for validation of our runs,
      while the results from numerical modelling can perhaps help to analyse
      some of the complexity of this evidence. In this section, we compare
      model outputs to the geologic record, in terms of timing and
      configuration of the maximum stages, location and lifetime of major
      nucleation centres, and patterns of ice retreat during the last
      deglaciation.

\subsection{Glacial maxima}

\subsubsection{Timing of glaciation}
\label{sec:timing}

      Independently of the palaeo-temperature records used to force the ice
      sheet model, our simulations consistently produce two glacial maxima
      during the last glacial cycle. The first maximum configuration is
      obtained during MIS~4 (62.2--56.9\,\unit{ka}) and the second during
      MIS~2 (23.1--16.8\,\unit{ka}; Figs.~\ref{fig:lr_ts},
      \ref{fig:lr_maps}; Table~\ref{tab:extrema}). These events broadly
      correspond in timing to the Gladstone (MIS~4) and McConnell (MIS~2)
      glaciations documented by geological evidence for the northern sector
      of the Cordilleran ice sheet \citep{Duk-Rodkin.etal.1996,
      Ward.etal.2007, Stroeven.etal.2010, Stroeven.etal.2014}, and to the
      Fraser Glaciation (MIS~2) documented for its southern sector
      \citep{Porter.Swanson.1998, Margold.etal.2014}. There is
      stratigraphical evidence for an MIS~4 glaciation in British Columbia
      \citep{Clague.Ward.2011} and in the Puget Lowland \citep{Troost.2014},
      but their extent and timing are still highly conjectural
      \citep[perhaps MIS~4 or early MIS~3; e.g.,][]{Cosma.etal.2008}.

      The exact timing of modelled MIS~2 maximum ice volume depends strongly
      on the choice of applied palaeo-temperature record, which allows for
      a~more in-depth comparison with geological evidence for the timing of
      maximum Cordilleran ice sheet extent. In the Puget Lowland
      (Fig.~\ref{fig:locmap}), the LGM advance of the southern Cordilleran
      ice sheet margin has been constrained by radiocarbon dating on wood
      between 17.4 and 16.4\,\unit{\chem{^{14}C}\,cal\,ka}
      \citep{Porter.Swanson.1998}. These dates are consistent with
      radiocarbon dates from the offshore sedimentary record, which reveals
      an increase of glaciomarine sedimentation between 19.5 and
      16.2\,\unit{\chem{^{14}C}\,cal\,ka} \citep{Cosma.etal.2008,
      Taylor.etal.2014}. Radiocarbon dating of the northern Cordilleran ice
      sheet margin is much less constrained but straddles presented
      constraints from the southern margin. However, cosmogenic exposure
      dating places the timing of maximum northern ice sheet margin extent
      during the McConnell
      glaciation close to 17\,\unit{\chem{^{10}Be}\,ka}
      \citep{Stroeven.etal.2010, Stroeven.etal.2014}. A~sharp transition in
      the sediment record of the Gulf of Alaska indicates a~retreat of
      regional outlet glaciers onto land at
      14.8\,\unit{\chem{^{14}C}\,cal\,ka} \citep{Davies.etal.2011}.

      Among the simulations presented here, only those forced with the GRIP,
      EPICA and Vostok palaeo-temperature records yield Cordilleran ice
      sheet maximum extents that may be compatible with these field
      constraints (Fig.~\ref{fig:lr_ts}, lower panel;
      Table~\ref{tab:extrema}). Simulations driven by the NGRIP, ODP~1012
      and ODP~1020 palaeo-temperature records, on the contrary, yield MIS~2
      maximum Cordilleran ice sheet volumes that pre-date field-based
      constraints by several thousands of years (about 6, 6 and 4\,\unit{ka}
      respectively). Concerning the simulations driven by oceanic records,
      this early deglaciation is caused by an early warming present in the
      alkenone palaeo-temperature reconstructions (Fig.~\ref{fig:lr_ts},
      upper panel; \citealp[Fig.~3]{Herbert.etal.2001}). However, this early
      warming is a~local effect, corresponding to a~weakening of the
      California current \citep{Herbert.etal.2001}. The California current,
      driving cold waters southwards along the south-western coast of North
      America, has been shown to have weakened during each peak of global
      glaciation (in SPECMAP) during the past 550\,\unit{ka}, including the
      LGM, resulting in paradoxically warmer sea-surface temperatures at the
      locations of the ODP~1012 and ODP~1020 sites
      \citep{Herbert.etal.2001}.

      Because most of the marine margin of the Cordilleran ice sheet
      terminated in a~sector of the Pacific Ocean unaffected by variations
      in the California current, it probably remained insensitive to this
      local phenomenum. However, the above paradox illustrates the
      complexity of ice-sheet feedbacks on regional climate, and
      demonstrates that, although located in the neighbourhood of the
      modelling domain, the ODP~1012 and ODP~1020 palaeo-temperature records
      cannot be used as a~realistic forcing to model the Cordilleran ice
      sheet through the last glacial cycle. Similarly, the simulation using
      the NGRIP palaeo-temperature record depicts an early onset of
      deglaciation (Fig.~\ref{fig:lr_ts}) following its last glacial volume
      maximum (22.9\,\unit{ka}, Table~\ref{tab:extrema}) attained about
      6\,\unit{ka} earlier than dated evidence of the LGM advance. There is
      a~fair agreement between the EPICA and Vostok palaeo-temperature
      records, resulting in only small differences between the simulations
      driven by those records. These differences are not subject to further
      analysis here; instead we focus on simulations forced by
      palaeo-temperature records from the GRIP and EPICA ice cores that
      appear to produce the most realistic reconstructions of regional
      glaciation history, yet bearing significant disparities in model
      output. To allow for a~more detailed comparison against the geological
      record, these two simulations were re-run using a~higher-resolution
      grid (Sect.~\ref{sec:model}; Fig.~\ref{fig:lr_ts}, lower panel, dotted
      lines).


\subsubsection{Ice configuration during MIS~2}
\label{sec:mis2}

      During maximum glaciation, both simulations position the main
      meridional ice divide over the western flank of the Rocky Mountains
      (Figs.~\ref{fig:lr_maps}, lower panels and \ref{fig:hr_maps_mis2}).
      This result appears to contrast with palaeoglaciological
      reconstructions for central and southern British Columbia with ice
      divides in a~more westerly position, over the western margin of the
      Interior Plateau \citep{Ryder.etal.1991, Stumpf.etal.2000,
      Kleman.etal.2010, Clague.Ward.2011, Margold.etal.2013a}. These
      indicate that a~latitudinal saddle connected ice dispersal centres in
      the Columbia Mountains with the main ice divide
      \citep{Ryder.etal.1991, Kleman.etal.2010, Clague.Ward.2011,
      Margold.etal.2013a}. A~latitudinal saddle does indeed feature in our
      modelling results, however, in an inverse configuration between the
      main ice divide over the Columbia Mountains and a~secondary divide
      over the southern Coast Mountains (Fig.~\ref{fig:hr_maps_mis2}).

      Such deviation from the geological inferences could reflect the fact
      the NARR has difficulties with simulating orographic processes in some
      areas of steep topography \citep{Jarosch.etal.2012}. In a previous
      study \citep{Seguinot.etal.2014}, we have evaluated the performance of
      NARR in forcing constant-climate simulations of the Cordilleran ice
      sheet against that of an observation-based data set
      \citep[WorldClim,][]{Hijmans.etal.2005}. Indeed, the use of NARR in
      these simulations produced slightly different patterns of glaciation
      relative to WorldClim, including more extensive ice cover on the
      Columbia and Rocky mountains \citep[Figs.~6--7]{Seguinot.etal.2014}.
      Our simulations have shown that these differences are mainly caused by
      disparities in the precipitation fields of the two data sets
      \citep[Figs.13--14]{Seguinot.etal.2014}. Over the southern part of our
      model domain, it was shown that the implementation of temperature and
      precipitation downscaling can address these limitations
      \citep[e.g.,][]{Jarosch.etal.2012}. However, extending this
      downscaling method to the entire model domain used in our study could
      be challenging, because the northern part of the model domain is
      characterized by stronger precipitation gradients (Fig.~\ref{fig:atm})
      and includes much fewer weather stations than the south
      \citep{Hijmans.etal.2005}.

      The westerly position of the modelled ice divide and western ice sheet
      margin may also reflect the fact that our model assumes fixed
      modern-day spatial patterns for temperature and precipitation and thus
      does not include feedback mechanisms between ice sheet topography and
      the regional climate. Regarding temperature, it is reasonable to think
      that the cooling was greater inland than near the coast, prohibiting
      melt at the western margin. However, our simulations already produce
      an excess of ice inland. Including such a temperature continentality
      gradient in the model while keeping the precipitation pattern constant
      would thus cause an even greater mismatch between the model results
      and the geologically reconstructed Last Glacial Maximum (LGM) ice
      margins.

      Consequently, the mismatch we observe between the modelled and
      reconstructed LGM ice margins let us think that the assumption of
      fixed modern-day precipitation patterns is more critical than the
      assumption of fixed modern-day temperature patterns.
      Firstly, during the build-up
      phase preceding the LGM, rapid accumulation over the Coast Mountains
      enhanced the topographic barrier formed by these mountain ranges,
      which likely resulted in a~decrease of precipitation and, therefore,
      a~decrease of accumulation in the interior. Secondly, latent warming
      of the moisture-depleted air parcels flowing over this enhanced
      topography could have resulted in an inflow of potentially warmer air
      over the eastern flank of the ice sheet, counterbalancing the potential
      continentality gradient discussed above through increasing melt along the
      advancing margin \citep[cf.][]{Langen.etal.2012}. Because these two
      processes, both with a~tendency to limit ice-sheet growth, are absent
      from our model, the eastern margin of the ice sheet and the position
      of the main meridional ice divide are certainly biased towards the
      east in our simulations \citep{Seguinot.etal.2014}.

      However, field-based palaeoglaciological reconstructions have
      struggled to reconcile the more westerly-centred ice divide in
      south-central British Columbia with evidence in the Rocky Mountains
      and beyond, that the Cordilleran ice sheet invaded the western
      Interior Plains, where it merged with the southwestern margin of the
      Laurentide ice sheet and was deflected to the south
      \citep{Jackson.etal.1997, Bednarski.Smith.2007, Kleman.etal.2010,
      Margold.etal.2013, Margold.etal.2013a}. Ice geometries from our model
      runs do not have this problem, because the position and elevation of
      the ice divide ensure significant ice drainage across the Rocky
      Mountains at the LGM (Fig.~\ref{fig:hr_maps_mis2}).

      During MIS~2, the modelled total ice volume peaks at
      8.62\,\unit{m}\,s.l.e. (19.1\,\unit{ka}) in the GRIP simulation and at
      8.56\,\unit{m}\,s.l.e. (17.4\,\unit{ka}) in the EPICA simulation.
      However, these numers are subject to uncertainties on physics
      embedded in the model. The range of parameter values tested in our
      sensitivity study on the lower-resolution GRIP simulation
      (Sect.~\ref{sec:sens}) yielded relative errors of 30\,\unit{\%}
      regarding ice rheological parameters and 21\,\unit{\%} regarding basal
      sliding parameters (Fig.~\ref{fig:sens_ts};
      Table~\ref{tab:sens_extrema}).


\subsubsection{Ice configuration during MIS~4}
\label{sec:mis4}

      The modelled ice sheet configurations corresponding to ice volume
      maxima during MIS~4 are more sensitive to the choice of atmospheric
      forcing than those corresponding to ice volume maxima during MIS~2
      (Figs.~\ref{fig:lr_maps}, upper panels and \ref{fig:hr_maps_mis4}).
      The GRIP simulation (Fig.~\ref{fig:hr_maps_mis4}, left panel) results
      in a~modelled maximum ice sheet extent that closely resembles that
      obtained during MIS~2, with the only major difference of being
      slightly less extensive across northern and eastern sectors. In
      contrast, the EPICA simulation produces a~lower ice volume maximum
      (Fig.~\ref{fig:lr_ts}), which translates in the modelled ice sheet
      geometry into a~significantly reduced southern sector, more restricted
      ice cover in northern and eastern sectors, and generally lower ice
      surface elevations in the interior (Fig.~\ref{fig:hr_maps_mis4}, right
      panel). Thus, only the GRIP simulation can explain the presence of
      MIS~4 glacial deposits in the Puget Lowland \citep{Troost.2014} and
      that of ice-rafted debris in the marine sediment record offshore
      Vancouver Island at ca.~47\,\unit{\chem{^{14}C}\,cal\,ka}
      \citep{Cosma.etal.2008}.

      During MIS~4, the modelled total ice volume peaks at
      7.43\,\unit{m}\,s.l.e. (57.6\,\unit{ka}) in the GRIP simulation and at
      4.84\,\unit{m}\,s.l.e. (61.9\,\unit{ka}) in the EPICA simulation,
      corresponding to respectively 86 and 57\,\unit{\%} of modelled MIS~2
      ice volumes.
      These estimates appear little sensitive to the range of parameter
      values tested in our sensitivity study (Sect.~\ref{sec:sens}), which
      yielded relative errors of 17\,\unit{\%} to ice rheological parameters
      and 21\,\unit{\%} to basal sliding parameters
      (Fig.~\ref{fig:sens_ts}; Table~\ref{tab:sens_extrema}).

\subsection{Nucleation centres}

\subsubsection{Transient ice sheet states}

      Palaeo-glaciological reconstructions are generally more robust for
      maximum ice sheet extents and late ice sheet configurations than for
      intermediate or minimum ice sheet extents and older ice sheet
      configurations \citep{Kleman.etal.2010}. However, these maximum stages
      are, by nature, extreme configurations, which do not necessarily
      represent the dominant patterns of glaciation throughout the period of
      ice cover \citep{Porter.1989, Kleman.Stroeven.1997, Kleman.etal.2008,
      Kleman.etal.2010}.

      For the Cordilleran ice sheet, geological evidence from radiocarbon
      dating \citep{Clague.etal.1980, Clague.1985, Clague.1986,
      Porter.Swanson.1998, Menounos.etal.2008}, cosmogenic exposure dating
      \citep{Stroeven.etal.2010, Stroeven.etal.2014, Margold.etal.2014},
      bedrock deformation in response to former ice loads
      \citep{Clague.James.2002, Clague.etal.2005}, and offshore sedimentary
      records \citep{Cosma.etal.2008, Davies.etal.2011} indicate that the
      LGM maximum extent was short-lived. To compare this finding to our
      simulations, we use numerical modelling output to compute durations of
      ice cover throughout the last glacial cycle
      (Fig.~\ref{fig:hr_geom_duration}).

      The resulting maps show that, during most of the glacial cycle,
      modelled ice cover is restricted to disjoint ice caps centred on major
      mountain ranges of the North American Cordillera
      (Fig.~\ref{fig:hr_geom_duration}, blue areas). A~2\,500\,\unit{km}-long
      continuous expanse of ice, extending from the Alaska Range in the
      north-east to the Rocky Mountains in the south-west, is only in
      operation for at most 34\,\unit{ka}, which is about a~third of the
      timespan of the last glacial cycle (Fig.~\ref{fig:hr_geom_duration},
      hatched areas). However, except for its margins in the Pacific Ocean
      and in the northern foothills of the Alaska Range, the maximum extent
      of the ice sheet is attained for a~much shorter period of time of only
      few thousand years (Fig.~\ref{fig:hr_geom_duration}, red areas). This
      result illustrates that the maximum extents of the modelled ice sheet
      during MIS~4 and MIS~2 were both short-lived and therefore out of
      balance with contemporary climate.

      A~notable exception to the transient character of the maximum extent
      of Cordilleran ice sheet is the northern slope of the Alaska Range,
      where modelled glaciers are confined to its foothills during the
      entire simulation period (Fig.~\ref{fig:hr_geom_duration}, AR). This
      apparent insensitivity of modelled glacial extent to temperature
      fluctuations results from a~combination of low precipitation, high
      summer temperature and large temperature standard deviation (PDD SD)
      in the plains of the Alaska Interior (Fig.~\ref{fig:atm}) which
      confines glaciation to the foothills of the mountains. This result
      could potentially explain the local distribution of glacial deposits,
      which indicates that glaciers flowing on the northern slope of the
      Alaska Range have remained small throughout the Pleistocene
      \citep{Kaufman.Manley.2004}.

\subsubsection{Major ice-dispersal centres}

      It is generally believed that the Cordilleran ice sheet formed by the
      coalescence of several mountain-centred ice caps
      \citep{Davis.Mathews.1944}. In our simulations, major ice-dispersal
      centres, visible on the modelled ice cover duration maps
      (Fig.~\ref{fig:hr_geom_duration}), are located over the Coast
      Mountains (CM), the Columbia and Rocky mountains (CRM), the Skeena
      Mountains (SM), and the Selwyn and Mackenzie mountains (SMKM).

      The location of modelled ice-dispersal centres is potentially biased
      by present day ice volumes contained in the ETOPO1 basal topography
      data. Although a reconstruction exist for the thickness of western
      Canadian glaciers south of 60{\degree}N \citep{Clarke.etal.2013}, the
      most problematic part of the model domain in this respect is by far
      that of the Wrangell and Saint Elias mountains, where ice thicknesses
      up to 1200\,m have been measured by low-frequency radar
      \citep{Rignot.etal.2013}. In this area, located over the USA Canada
      border just north of 60{\degree}N, temperate ice, surge dynamics and
      deep subglacial depressions in the icefield interior pose a
      fundamental challenge to reconstructing basal topography for the
      entire ice cap \citep{Rignot.etal.2013}. Although this causes our
      simulations to overestimate ice surface topography in this region, we
      are not aware of bed topography reconstructions that could be used to
      force the ice sheet model. However, this drawback seem to have
      little effect on modelled Cordilleran ice sheet dynamics. In fact, the
      Wrangell and Saint Elias mountains, heavily glacierized at present,
      host an ice cap for the entire length of both simulations, but that
      ice cap does not appear to be a~major feed to the Cordilleran ice
      sheet (Fig.~\ref{fig:hr_geom_duration}, WSEM). Moreover, with this
      exception of the Wrangell and Saint Elias mountains ice field,
      present-day ice volumes are small relative to the ice volumes
      concerned in our study.

      Although the Coast,
      Skeena and Columbia and Rocky mountains (CM, SM, CRM) are covered by
      mountain glaciers for most of the last glacial cycle, providing
      durable nucleation centres for an ice sheet, this is not the case for
      the Selwyn and Mackenzie mountains (SMKM), where ice cover on the
      highest peaks is limited to a~small fraction of the last glacial
      cycle. In other words, the Selwyn and Mackenzie mountains only appear
      as a~secondary ice-dispersal centre during the coldest periods of the
      last glacial cycle. The Northern Rocky Mountains
      (Fig.~\ref{fig:hr_geom_duration}, NRM) do not act as a~nucleation
      centre, but rather as a~pinning point for the Cordilleran ice sheet
      margin coming from the west.

      Perhaps the most striking feature displayed by the distributions of
      modelled ice cover is the persistence of the Skeena Mountains ice cap
      throughout the entire last glacial period (ca.~100--10\,\unit{ka}) and
      its predominance over the other ice-dispersal centres
      (Figs.~\ref{fig:lr_maps} and~\ref{fig:hr_geom_duration}, SM).
      Regardless of the applied forcing, this ice cap appears to survive
      MIS~3 (Fig.~\ref{fig:lr_maps}, middle panels), and serves as
      a~nucleation centre at the onset of the glacial readvance towards the
      LGM (MIS~2). This situation appear similar to the neighbouring
      Laurentide ice sheet, for which the importance of residual ice for the
      glacial history leading up to the LGM has been illustrated by the
      MIS~3 residual ice bodies in northern and eastern Canada as nucleation
      centres for a~much more extensive MIS~2 configuration
      \citep{Kleman.etal.2010}.

      The presence of a~Skeena Mountains ice cap during most of the last
      glacial cycle can be explained by meteorological conditions more
      favourable for ice growth there than elsewhere. In fact, reanalysed
      atmospheric fields used to force the surface mass balance model show
      that high winter precipitations are mainly confined to the western
      slope of the Coast Mountains, except in the centre of the modelling
      domain where they also occur further inland than along other east-west
      transects (Fig.~\ref{fig:atm}). In fact, along most of the
      north-western coast of North America, coastal mountain ranges form
      a~pronounced topographic barrier for westerly winds, capturing
      atmospheric moisture in the form of orographic precipitation, and
      resulting in arid interior lowlands. However, near the centre of our
      modelling domain, this barrier is less pronounced than elsewhere,
      allowing westerly winds to carry moisture further inland, until it is
      captured by the extensive Skeena Mountains in north-central British
      Columbia, thus resulting in a~more widespread distribution of winter
      precipitation (Fig.~\ref{fig:atm}).

      The modelled total ice volume corresponding to these persistent
      ice-dispersal centres attains a~minimum of 1.54\,\unit{m}\,s.l.e.
      (42.9\,\unit{ka}) in the GRIP simulation and of 2.55\,\unit{m}\,s.l.e.
      (52.4\,\unit{ka}) in the EPICA simulation, corresponding to
      respectively 18 and 30\,\unit{\%} of the MIS~2 ice volumes.
      However, these numbers should be considered with caution as our
      sensitivity study (Sect.~\ref{sec:sens}) shows that minimum ice volume
      during MIS~3 is highly sensitive to ice flow parameters, with relative
      errors of 57\,\unit{\%} to the range of ice rheological parameters
      tested and 120\,\unit{\%} to the range of basal sliding parameters
      tested (Fig.~\ref{fig:sens_ts}; Table~\ref{tab:sens_extrema}).

\subsubsection{Erosional imprint on the landscape}

      A~correlation is observed between the modelled duration of warm based
      ice cover (Fig.~\ref{fig:hr_geom_warmbase}) and the degree of glacial
      modification of the landscape (mainly in terms of the development of
      deep glacial valleys and troughs). We find evidence for this on the
      slopes of the Coast Mountains, the Columbia and Rocky mountains,
      the Wrangell and Saint Elias mountains, and radiating off the Skeena
      Mountains (Figs.~\ref{fig:hr_geom_duration} and
      \ref{fig:hr_geom_warmbase}; \citealp[Fig.~2]{Kleman.etal.2010}). The
      Skeena Mountains, for example, indeed bear a~strong glacial imprint
      that indicates ice drainage in a~system of distinct glacial troughs
      emanating in a~radial pattern from the centre of the mountain range
      \citep[Fig.~2]{Kleman.etal.2010}. We suggest that persistent ice cover
      (Fig.~\ref{fig:hr_geom_duration}) associated with basal ice
      temperatures at the pressure melting point
      (Figs.~\ref{fig:hr_geom_warmbase} and~\ref{fig:hr_geom_warmfrac})
      explains the large-scale glacial erosional imprint on the landscape.
      A~well-developed network of glacial valleys running to the north-west
      on the west slope of the Selwyn and Mackenzie Mountains
      (\citealp[Fig.~2]{Kleman.etal.2010}; \citealp[Fig.~8]
      {Stroeven.etal.2010}) is modelled to have hosted predominantly
      warm-based ice (Fig.~\ref{fig:hr_geom_warmfrac}). However, because it
      is only glaciated for a~short fraction of the last glacial cycle in
      our simulations (Fig.~\ref{fig:hr_geom_duration}), this perhaps
      indicates that the observed landscape pattern originates from multiple
      glacial cycles and witnesses an increased relative importance of the
      Selwyn and Mackenzie mountains ice dispersal centre
      (Fig.~\ref{fig:hr_geom_duration}, SMKM), prior to the Late Pleistocene
      \citep[cf.][]{Ward.etal.2008, Demuro.etal.2012}.

      The modelled distribution of warm-based ice cover
      (Figs.~\ref{fig:hr_geom_warmbase} and~\ref{fig:hr_geom_warmfrac}) is
      inevitably affected by our assumption of a~constant,
      70\,\unit{mW\,m^{-2}} geothermal heat flux at 3\,\unit{km} depth
      (Sect.~\ref{sec:icedyn}). However, the Skeena Mountains and the area
      west of the Mackenzie Mountains experience higher-than-average
      geothermal heat flux with measured values of ca.~80 and
      ca.~100\,\unit{mW\,m^{-2}} \citep{Blackwell.Richards.2004}. We can
      therefore expect even longer durations of warm-based ice cover for
      these areas if we were to include spatially variable geothermal
      forcing in our Cordilleran ice sheet simulations.


\subsection{The last deglaciation}

\subsubsection{Pace and patterns of deglaciation}

      Similarly to other glaciated regions, most glacial traces in the North
      American Cordillera relate to the last few millennia of glaciation,
      because most of the older evidence has been overprinted by warm-based
      ice retreat during the last deglaciation \citep{Kleman.1994,
      Kleman.etal.2010}. From a~numerical modelling perspective, phases of
      glacier retreat are more challenging than phases of growth, because
      they involve more rapid fluctuations of the ice margin, increased flow
      velocities and longitudinal stress gradients, and poorly understood
      hydrological processes. The latter are typically included in the
      models through simple parametrisations
      \citep[e.g.][]{Clason.etal.2012, Clason.etal.2014, Bueler.Pelt.2015},
      if included at all. However, next after the mapping of maximum ice
      sheet extents during MIS~2 and MIS~4 (Sects.~\ref{sec:mis2}
      and~\ref{sec:mis4}), geomorphologically-based reconstructions of
      patterns of ice sheet retreat during the last deglaciation provide the
      second best source of evidence for the validation of our simulations.

      In the North American Cordillera, the presence of lateral meltwater
      channels at high elevation \citep{Margold.etal.2011,
      Margold.etal.2013a, Margold.etal.2014}, and abundant esker systems at
      low elevation \citep{Burke.etal.2012, Burke.etal.2012a,
      Perkins.etal.2013, Margold.etal.2013} indicate that meltwater was
      produced over large portions of the ice sheet surface during
      deglaciation. The southern and northern margins of the Cordilleran ice
      sheet reached their last glacial maximum extent around 17\,\unit{ka}
      \citep[Sect.~\ref{sec:timing};][]{Porter.Swanson.1998,
      Cosma.etal.2008, Stroeven.etal.2010, Stroeven.etal.2014}, which we
      take as a~limiting age for the onset of ice retreat. The timing of
      final deglaciation is less well constrained, but recent cosmogenic
      dates from north-central British Columbia indicate that a~seizable ice
      cap emanating from the central Coast Mountains or the Skeena Mountains
      persisted into the Younger Dryas chronozone, at least until
      12.4\,\unit{ka} \citep{Margold.etal.2014}.

      In our simulations, the timing of peak ice volume during the LGM and
      the pacing of deglaciation depend critically on the choice of climate
      forcing (Table~\ref{tab:extrema}; Figs.~\ref{fig:lr_ts}
      and~\ref{fig:hr_ts_deglac}). Adopting the EPICA climate forcing yields
      peak ice volume at 17.4\,\unit{ka} and an uninterrupted deglaciation
      until about 9\,\unit{ka} (Fig.~\ref{fig:hr_ts_deglac}, lower panel,
      red curves). On the contrary, the simulation driven by the GRIP
      palaeo-temperature record yields peak ice volume at 19.1\,\unit{ka}
      and a~deglaciation interrupted by two phases of regrowth until about
      8\,\unit{ka}. The first interruption occurs between 16.6 and
      14.5\,\unit{ka}, and the second between 13.1 and 11.6\,\unit{ka}
      (Fig.~\ref{fig:hr_ts_deglac}, lower panel, blue curve).

      Hence, the two model runs, while similar in overall timing compared to
      the runs with other climate drivers, differ in detail. On the one
      hand, the EPICA run depicts peak glaciation about 2\,\unit{ka} later
      than the GRIP run, in closer agreement with dated maximum extents, and
      shows a~faster, uninterrupted deglaciation which yields sporadic ice
      cover more than 1\,\unit{ka} earlier. On the other hand, the GRIP run
      yields a~deglaciation in three steps, compatible with marine sediment
      sequences offshore Vancouver Island, where the distribution of
      ice-rafted debris indicates an ice margin retreat from the Georgia
      Strait in two phases that are contemporary with warming oceanic
      temperatures from 17.2 to 16.5 and from 15.5 to
      14.0\,\unit{\chem{^{14}C}\,cal\,ka} \citep{Taylor.etal.2014}.

      Modelled patterns of ice sheet retreat are relatively consistent
      between the two simulations (Figs.~\ref{fig:hr_maps_deglac}
      and~\ref{fig:hr_geom_deglacage}). The southern sector of the modelling
      domain, including the Puget Lowland, the Coast Mountains, the Columbia
      and and Rocky mountains, and the Interior Plateau of British Columbia,
      becomes completely deglaciated by 10\,\unit{ka}, whereas a~significant
      ice cover remains over the Skeena, the Selwyn and Mackenzie, and the
      Wrangell and Saint Elias mountains in the northern sector of the
      modelling domain. After 10\,\unit{ka}, deglaciation continues to
      proceed across the Liard Lowland with a~radial ice margin retreat
      towards the surrounding mountain ranges, consistent with the regional
      melt water record of the last deglaciation \citep{Margold.etal.2013}.
      Remaining ice continues to decay by retreating towards the Selwyn and
      MacKenzie, and Skeena mountains. The last remnants of the Cordilleran
      ice sheet finally disappear from the Skeena Mountains at
      6.7\,\unit{ka} in both simulations.

\subsubsection{Late-glacial readvance}

      The possibility of late glacial readvances in the North American
      Cordillera has been debated for some time \citep{Luckman.Osborn.1979,
      Reasoner.etal.1994, Osborn.Gerloff.1997, Osborn.etal.1995},
      and locally these have been reconstructed and dated. Radiocarbon-dated
      end moraines in the Fraser and Squamish valleys, off the southern tip
      of the Coast Mountains, indicate consecutive glacier maxima, or
      standstills while in overall retreat, one of which corresponds to the
      Younger Dryas chronozone \citep{Clague.etal.1997, Friele.Clague.2002,
      Friele.Clague.2002a, Kovanen.2002, Kovanen.Easterbrook.2002}. Although
      most of these moraines characterise independent valley glaciers, that
      may have been disconnected from the waning Cordilleran ice sheet, the
      Finlay River area in the Omenica Mountains
      (Fig.~\ref{fig:hr_geom_deglacage}, OM) presents a~different kind of
      evidence. There, sharp-crested moraines indicate a~late-glacial
      readvance of local alpine glaciers and, more importantly, their
      interaction with larger, lingering remnants of the main body of the
      Cordilleran ice sheet in the valleys \citep{Lakeman.etal.2008}.
      Additional evidence for late-glacial alpine glacier readvances
      includes moraines in the eastern Coast Mountains, and the Columbia and
      Rocky mountains \citep{Reasoner.etal.1994, Osborn.Gerloff.1997,
      Menounos.etal.2008}.

      Although further work is needed to constrain the timing of the
      late-glacial readvance, to assess its extents and geographical
      distribution, and to identify the potential climatic triggers
      \citep{Menounos.etal.2008}, it is interesting to note that the
      simulation driven by the GRIP record produces a~late-glacial readvance
      in the Coast Mountains and in the Columbia and Rocky Mountains
      (Fig.~\ref{fig:hr_geom_deglacage}, left panel). In addition to
      matching the location of some local readvances, the GRIP-driven
      simulation shows that a~large remnant of the decaying ice sheet may
      still have existed at the time of this late-glacial readvance. In
      contrast, the EPICA-driven simulation produces a~nearly-continuous
      deglaciation with only a~tightly restricted late-glacial readvance on
      the western slopes of the Saint Elias and the Coast mountains
      (Fig.~\ref{fig:hr_geom_deglacage}, right panel).


\subsubsection{Deglacial flow directions}

      Because a~general conjecture in glacial geomorphology is that the majority
      of landforms (lineations and eskers) are part of the deglacial
      envelope \citep[terminology from][]{Kleman.etal.2006}, having been
      formed close inside the retreating margin of ice sheets
      \citep{Boulton.Clark.1990, Kleman.etal.1997, Kleman.etal.2010}, we
      present maps of basal flow directions immediately preceding
      deglaciation or at the time of cessation of sliding inside
      a~cold-based retreating margin (Fig.~\ref{fig:hr_geom_lastflow}). The
      modelled deglacial flow patterns are mostly consistent between the
      GRIP and EPICA simulations. They depict an active ice sheet retreat in
      the peripheral areas, followed by stagnant ice decay in some of the
      interior regions. Several parts of the modelling domain do not
      experience any basal sliding throughout the deglaciation phase
      (Fig.~\ref{fig:hr_geom_lastflow}, hatched areas). This notably
      includes parts of the Interior Plateau in British Columbia, major
      portions of the Alaskan sector of the ice sheet, and a~tortuous ribbon
      running from the Northern Rocky Mountains over the Skeena and Selwyn
      Mountains and into the Mackenzie Mountains.

      Patterns of glacial lineations formed in the northern and southern
      sectors of the Cordilleran ice sheet (\citealp{Prest.etal.1968};
      \citealp[Fig.~1.12]{Clague.1989}; \citealp[Fig.~2]{Kleman.etal.2010})
      show similarities with the patterns of deglacial ice flow from
      numerical modelling (Fig.~\ref{fig:hr_geom_lastflow}). In the northern
      half of the modelling domain, modelled deglacial flow directions
      depict an active downhill flow as the last remnants of the ice sheet
      retreat towards mountain ranges. Converging deglacial flow patterns in
      the Liard Lowland, for instance (Fig.~\ref{fig:hr_geom_lastflow}),
      closely resemble the pattern indicated by glacial lineations
      \citep[Fig.~2]{Margold.etal.2013}.

      On the Interior Plateau of south-central British Columbia, both
      simulations produce a~retreat of the ice margin towards the north-east
      (Fig.~\ref{fig:hr_geom_deglacage}), a~pattern which is validated by
      the geomorphological and stratigraphical record for ancient
      pro-glacial lakes dammed by the retreating ice sheet
      \citep{Perkins.Brennand.2014}. However, the two simulations differ in
      the mode of retreat. The GRIP simulation yields an active retreat with
      basal sliding towards the ice margin to the south, whereas the EPICA
      simulation produces negligible basal sliding on the plateau during
      deglaciation (Fig.~\ref{fig:hr_geom_lastflow}). Yet, the Interior
      Plateau also hosts an impressive set of glacial lineations which
      indicate a~substantial eastwards flow component of the Cordilleran ice
      sheet \citep{Prest.etal.1968, Kleman.etal.2010} not present in any of
      the two simulations (Fig.~\ref{fig:hr_geom_lastflow}).
      One explanation for the
      incongruent results could be that the missing feedback mechanisms
      between ice sheet topography and regional climate resulted in
      a~modelled ice divide of the LGM ice sheet being too far to the east
      (Sect.~\ref{sec:mis2}; Fig.~\ref{fig:hr_maps_mis2};
      \citealp{Seguinot.etal.2014}). A~more westerly-located LGM ice divide
      would certainly result in a~different deglacial flow pattern over the
      Interior Plateau. However, a~more westerly-positioned LGM ice divide
      would certainly be associated with an LGM ice sheet less extensive to
      the west, and therefore thinner ice on the Interior Plateau during
      deglaciation than
      modelled here. Decreased ice thickness would not promote warm-based
      conditions but, on the contrary, enlarge the region of negligible
      basal sliding (Fig.~\ref{fig:hr_geom_lastflow}). Thus, a~second
      explanation for the incongruent results could be that the Interior
      Plateau lineation system predates deglaciation ice flow, as perhaps
      indicated by some eskers that appear incompatible with these glacial
      lineations \citep[Fig.~9]{Margold.etal.2013a}. Finally, a~third
      explanation could be that local geothermal heat associated with
      volcanic activity on the Interior Plateau could have triggered the
      basal sliding \citep[cf. Greenland ice
      sheet;][]{Fahnestock.etal.2001}.

      The modelled deglaciation of the Interior Plateau of British Columbia
      consists of a~rapid northwards retreat
      (Fig.~\ref{fig:hr_geom_deglacage}) of southwards-flowing non-sliding
      ice lobes (Fig.~\ref{fig:hr_geom_lastflow}) positioned in-between
      deglaciated mountain ranges (Figs.~\ref{fig:hr_pf_grip}
      and~\ref{fig:hr_pf_epica}). This result appears compatible with the
      prevailing conceptual model of deglaciation of central British
      Columbia, in which mountain ranges emerge from the ice before the
      plateau \citep[Fig.~7]{Fulton.1991}. However, due to different
      topographic and climatic conditions, our simulations produce different
      deglaciation patterns in the northern half of the model domain,
      indicating that this conceptual model may not be applied to the entire
      area formerly covered by the Cordilleran ice sheet.


\conclusions
\label{sec:concl}

      Numerical simulations of the Cordilleran ice sheet through the last
      glacial cycle presented in this study consistently produce two glacial
      maxima during MIS~4 (62.2--56.9\,\unit{ka},
      3.5--8.7\,\unit{m}\,s.l.e.) and MIS~2 (23.1--16.8\,\unit{ka},
      8.4--9.1\,\unit{m}\,s.l.e.), two periods corresponding to documented
      extensive glaciations. This result is independent of the
      palaeo-temperature record used among the six selected for this study,
      and thus can be regarded as a~robust model output, which broadly
      matches geological evidence. However, the timing of the two glaciation
      peaks depends sensitively on which climate record is used to drive the
      model. The timing of the LGM is best reproduced by the EPICA and
      Vostok Antarctic ice core records. It occurs about 2\,\unit{ka} too
      early in the simulation forced by the GRIP ice core record, and occurs
      even earlier in all other simulations. The mismatch is largest for the
      two Northwest Pacific ODP palaeo-temperature records, which are
      affected by the weakening of the California current during the LGM.
      Nevertheless, the fact that Cordilleran ice sheet dynamics are here
      modelled in best agreement to geological reconstructions using
      palaeo-temperature records from the distant Greenland and Antarctic
      ice sheets, and the significant differences remaining between our two
      preffered simulations highlight the need for more regional
      palaeo-climate reconstructions of the last glacial cycle in and around
      the North American Cordillera.

      In all simulations presented here, ice cover is limited to disjoint
      mountain ice caps during most of the glacial cycle. The most
      persistent nucleation centres are located in the Coast Mountains, the
      Columbia and Rocky mountains, the Selwyn and Mackenzie mountains, and
      most importantly, in the Skeena Mountains. Throughout the modelled
      last glacial cycle, the Skeena Mountains host an ice cap which appears
      to be fed by the moisture intruding inland from the west through
      a~topographic breach in the Coast Mountains. The Skeena ice cap acts
      as the main nucleation centre for the glacial reavance towards the LGM
      configuration. As indicated by persistent, warm-based ice in the
      model, this ice cap perhaps explains the distinct glacial erosional
      imprint observed on the landscape of the Skeena Mountains.

      During deglaciation, none of the climate records used can be selected
      as producing an optimal agreement between the model results and the
      geological evidence. Although the EPICA-driven simulation yields the
      most realistic timing of the LGM and, therefore, start of
      deglaciation, only the GRIP-driven simulation produces late glacial
      readvances in areas where these have been documented. Nonetheless, the
      patterns of ice sheet retreat are consistent between the two
      simulations, and show a~rapid deglaciation of the southern sector of
      the ice sheet, including a~rapid northwards retreat across the
      Interior Plateau of central British Columbia. The GRIP-driven
      simulation then produces a~late-glacial readvance of local ice caps
      and of the main body of the decaying Cordilleran ice sheet primarily
      in the Coast and the Columbia and Rocky Mountains. In both
      simulations, this is followed by an opening of the Liard Lowland, and
      a~final retreat of the remaining ice caps towards the Selwyn and,
      finally, the Skeena mountains, which hosts the last remnant of the ice
      sheet during the middle Holocene (6.7\,\unit{ka}). Our results
      identify the Skeena Mountains as a~key area to understanding glacial
      dynamics of the Cordilleran ice sheet, highlighting the need for
      further geological investigation of this region.

\Supplementary{zip}

\authorcontribution{%
      J.~Seguinot ran the simulations; I.~Rogozhina guided experiment
      design; A.~P.~Stroeven, M.~Margold and J.~Kleman took part in the
      interpretation and comparison of model results against geological
      evidence. All authors contributed to the text.}





\begin{acknowledgements}
      Foremost, we would like to thank Shawn Marshall for providing
      a~detailed, constructive analysis of this study during J.~Seguinot's
      PhD defence (September 2014). His comments were used to improve the
      model set-up. We are very thankful to Constantine Khroulev, Ed Bueler,
      and Andy Aschwanden for providing constant help and development with
      PISM. This work was supported by the Swedish Research Council~(VR)
      grant no.~2008-3449 to A.~P.~Stroeven and by the German Academic
      Exchange Service~(DAAD) grant no.~50015537 and a~Knut and Alice
      Wallenberg Foundation grant to J.~Seguinot. Computer resources were
      provided by the Swedish National Infrastructure for Computing (SNIC)
      allocation no.~2013/1-159 and~2014/1-159 to A.~P.~Stroeven at the
      National Supercomputing Center (NSC).
\end{acknowledgements}


\begin{thebibliography}{114}

\bibitem[{Albrecht et~al.(2011)Albrecht, Martin, Haseloff, Winkelmann, and
  Levermann}]{Albrecht.etal.2011}
 Albrecht,~T., Martin,~M., Haseloff,~M., Winkelmann,~R., and Levermann,~A.: Parameterization for subgrid-scale motion of ice-shelf calving fronts, The Cryosphere, 5, 35--44,
doi:\href{http://dx.doi.org/10.5194/tc-5-35-2011}{10.5194/tc-5-35-2011}, 2011.


\bibitem[{Amante and Eakins(2009)}]{Amante.Eakins.2009}
Amante,~C. and Eakins,~B.~W.: ETOPO1 1 arc-minute global relief model: procedures, data sources and analysis, NOAA technical memorandum NESDIS NGDC-24, Natl. Geophys. Data Center, NOAA, Boulder, CO,
doi:\href{http://dx.doi.org/10.7289/V5C8276M}{10.7289/V5C8276M}, 2009.


\bibitem[{Andersen et~al.(2004)Andersen, Azuma, Barnola, Bigler, Biscaye, Caillon, Chappellaz, Clausen, Dahl-Jensen, Fischer, Fl\"uckiger, Fritzsche, Fujii, Goto-Azuma, Gr{\o}nvold, Gundestrup, Hansson, Huber, Hvidberg, Johnsen, Jonsell, Jouzel, Kipfstuhl, Landais, Leuenberger, Lorrain, Masson-Delmotte, Miller, Motoyama, Narita, Popp, Rasmussen, Raynaud, Rothlisberger, Ruth, Samyn, Schwander, Shoji, Siggard-Andersen, Steffensen, Stocker, Sveinbj\"ornsd\'ottir, Svensson, Takata, Tison, Thorsteinsson, Watanabe, Wilhelms, and White}]{Andersen.etal.2004}
Andersen,~K.~K., Azuma,~N., Barnola,~J.-M., Bigler,~M., Biscaye,~P., Caillon,~N., Chappellaz,~J., Clausen,~H.~B., Dahl-Jensen,~D., Fischer,~H., Fl\"uckiger,~J., Fritzsche,~D., Fujii,~Y., Goto-Azuma,~K., Gr{\o}nvold,~K., Gundestrup,~N.~S., Hansson,~M., Huber,~C., Hvidberg,~C.~S., Johnsen,~S.~J., Jonsell,~U., Jouzel,~J., Kipfstuhl,~S., Landais,~A., Leuenberger,~M., Lorrain,~R., Masson-Delmotte,~V., Miller,~H., Motoyama,~H., Narita,~H., Popp,~T., Rasmussen,~S.~O., Raynaud,~D., Rothlisberger,~R., Ruth,~U., Samyn,~D., Schwander,~J., Shoji,~H., Siggard-Andersen,~M.-L., Steffensen,~J.~P., Stocker,~T., Sveinbj\"ornsd\'ottir,~A.~E., Svensson,~A., Takata,~M., Tison,~J.-L., Thorsteinsson,~T., Watanabe,~O., Wilhelms,~F., and White,~J.~W.~C.: High-resolution record of Northern Hemisphere climate extending into the last interglacial period, Nature, 431, 147--151,
doi:\href{http://dx.doi.org/10.1038/nature02805}{10.1038/nature02805}, data archived at the World Data Center for Paleoclimatology, Boulder, Colorado, USA, 2004.


\bibitem[{Aschwanden et~al.(2012)Aschwanden, Bueler, Khroulev, and Blatter}]{Aschwanden.etal.2012}
Aschwanden,~A., Bueler,~E., Khroulev,~C., and Blatter,~H.: An enthalpy formulation for glaciers and ice sheets,~J. Glaciol., 58, 441--457,
doi:\href{http://dx.doi.org/10.3189/2012JoG11J088}{10.3189/2012JoG11J088}, 2012.


\bibitem[{Aschwanden et~al.(2013)Aschwanden, A{\dh}algeirsd\'{o}ttir, and Khroulev}]{Aschwanden.etal.2013}
 Aschwanden,~A., A{\dh}algeirsd\'{o}ttir,~G., and Khroulev,~C.: Hindcasting to measure ice sheet model sensitivity to   initial states, The Cryosphere, 7, 1083--1093,
doi:\href{http://dx.doi.org/10.5194/tc-7-1083-2013}{10.5194/tc-7-1083-2013}, 2013.


\bibitem[{Balay et~al.(2015)Balay, Abhyankar, Adams, Brown, Brune, Buschelman, Eijkhout, Gropp, Kaushik, Knepley, McInnes, Rupp, Smith, and Zhang}]{Balay.etal.2015}
Balay,~S., Abhyankar,~S., Adams,~M.~F., Brown,~J., Brune,~P., Buschelman,~K.,
Eijkhout,~V., Gropp,~W.~D., Kaushik,~D., Knepley,~M.~G., McInnes,~L.~C.,
Rupp,~K., Smith,~B.~F., and Zhang,~H.: {PETS}c {W}eb page, available at:
\url{http://www.mcs.anl.gov/petsc} (last access: 2015), 2015.


\bibitem[{Barendregt and Irving(1998)}]{Barendregt.Irving.1998}
Barendregt,~R.~W. and Irving,~E.: Changes in the extent of North American ice sheets during the late Cenozoic, Can.~J. Earth Sci., 35, 504--509,
doi:\href{http://dx.doi.org/10.1139/e97-126}{10.1139/e97-126}, 1998.


\bibitem[{Bednarski and Smith(2007)}]{Bednarski.Smith.2007}
Bednarski,~J.~M. and Smith,~I.~R.: Laurentide and montane glaciation along the Rocky Mountain Foothills of northeastern British Columbia, Can.~J. Earth Sci., 44, 445--457,
doi:\href{http://dx.doi.org/10.1139/e06-095}{10.1139/e06-095}, 2007.


\bibitem[{Blackwell and Richards(2004)}]{Blackwell.Richards.2004}
Blackwell,~D.~D. and Richards,~M.: Geothermal Map of North America, Am. Assoc. Petr. Geol., Tulsa, OK, 2004.


\bibitem[{Booth et~al.(2003)Booth, Troost, Clague, and Waitt}]{Booth.etal.2003}
Booth,~D.~B., Troost,~K.~G., Clague,~J.~J., and Waitt,~R.~B.: The Cordilleran ice sheet, in: The Quaternary Period in the United States, edited by: Gillespie,~A., Porter,~S., and Atwater,~B., vol.~1 of Dev. Quaternary Sci., Elsevier, Amsterdam, 17--43,
doi:\href{http://dx.doi.org/10.1016/s1571-0866(03)01002-9}{10.1016/s1571-0866(03)01002-9}, 2003.


\bibitem[{Boulton and Clark(1990)}]{Boulton.Clark.1990}
Boulton,~G.~S. and Clark,~C.~D.: A highly mobile Laurentide ice sheet revealed by satellite images of glacial lineations, Nature, 346, 813--817,
doi:\href{http://dx.doi.org/10.1038/346813a0}{10.1038/346813a0}, 1990.


\bibitem[{Boulton et~al.(2001)Boulton, Dongelmans, Punkari, and Broadgate}]{Boulton.etal.2001}
Boulton,~G.~S., Dongelmans,~P., Punkari,~M., and Broadgate,~M.: Palaeoglaciology of an ice sheet through a glacial cycle: the European ice sheet through the Weichselian, Quaternary Res., 20, 591--625,
doi:\href{http://dx.doi.org/10.1016/s0277-3791(00)00160-8}{10.1016/s0277-3791(00)00160-8}, 2001.


\bibitem[{Briner and Kaufman(2008)}]{Briner.Kaufman.2008}
Briner,~J.~P. and Kaufman,~D.~S.: Late Pleistocene mountain glaciation in Alaska: key chronologies,~J. Quaternary Sci., 23, 659--670,
doi:\href{http://dx.doi.org/10.1002/jqs.1196}{10.1002/jqs.1196}, 2008.


\bibitem[{Bueler and Brown(2009)}]{Bueler.Brown.2009}
Bueler,~E. and Brown,~J.: Shallow shelf approximation as a ``sliding law'' in a~thermodynamically coupled ice sheet model,~J. Geophys. Res., 114, F03008,
doi:\href{http://dx.doi.org/10.1029/2008JF001179}{10.1029/2008JF001179}, 2009.


\bibitem[{Bueler and van Pelt(2015)}]{Bueler.Pelt.2015}
 Bueler,~E. and van~Pelt,~W.: Mass-conserving subglacial hydrology in the Parallel Ice Sheet Model version 0.6, Geosci. Model Dev., 8, 1613--1635,
doi:\href{http://dx.doi.org/10.5194/gmd-8-1613-2015}{10.5194/gmd-8-1613-2015}, 2015.


\bibitem[{Bueler et~al.(2007)Bueler, Lingle, and Brown}]{Bueler.etal.2007}
Bueler,~E., Lingle,~C.~S., and Brown,~J.: Fast computation of a viscoelastic deformable Earth model for ice-sheet simulations, Ann. Glaciol., 46, 97--105,
doi:\href{http://dx.doi.org/10.3189/172756407782871567}{10.3189/172756407782871567}, 2007.


\bibitem[{Burke et~al.(2012{\natexlab{a}})Burke, Brennand, and Perkins}]{Burke.etal.2012}
Burke,~M.~J., Brennand,~T.~A., and Perkins,~A.~J.: Transient subglacial hydrology of a thin ice sheet: insights from the Chasm esker, British Columbia, Canada, Quaternary Res., 58, 30--55,
doi:\href{http://dx.doi.org/10.1016/j.quascirev.2012.09.004}{10.1016/j.quascirev.2012.09.004}, 2012{\natexlab{a}}.


\bibitem[{Burke et~al.(2012{\natexlab{b}})Burke, Brennand, and Perkins}]{Burke.etal.2012a}
Burke,~M.~J., Brennand,~T.~A., and Perkins,~A.~J.: Evolution of the subglacial hydrologic system beneath the rapidly decaying Cordilleran Ice Sheet caused by ice-dammed lake drainage: implications for meltwater-induced ice acceleration, Quaternary Res., 50, 125--140,
doi:\href{http://dx.doi.org/10.1016/j.quascirev.2012.07.005}{10.1016/j.quascirev.2012.07.005}, 2012{\natexlab{b}}.


\bibitem[{Calov and Greve(2005)}]{Calov.Greve.2005}
Calov,~R. and Greve,~R.: A~semi-analytical solution for the positive degree-day model with stochastic temperature variations,~J. Glaciol., 51, 173--175,
doi:\href{http://dx.doi.org/10.3189/172756505781829601}{10.3189/172756505781829601}, 2005.


\bibitem[{Carlson and Clark(2012)}]{Carlson.Clark.2012}
Carlson,~A.~E. and Clark,~P.~U.: Ice sheet sources of sea level rise and
freshwater discharge during the last deglaciation, Rev. Geophys., 50, RG4007,
doi:\href{http://dx.doi.org/10.1029/2011rg000371}{10.1029/2011rg000371},
2012.


\bibitem[{Carrara et~al.(1996)Carrara, Kiver, and Stradling}]{Carrara.etal.1996}
Carrara,~P.~E., Kiver,~E.~P., and Stradling,~D.~F.: The southern limit of Cordilleran ice in the Colville and Pend Oreille valleys of northeastern Washington during the Late Wisconsin glaciation, Can.~J. Earth Sci., 33, 769--778,
doi:\href{http://dx.doi.org/10.1139/e96-059}{10.1139/e96-059}, 1996.


\bibitem[{Clague et~al.(1980)Clague, Armstrong, and Mathews}]{Clague.etal.1980}
Clague,~J., Armstrong,~J., and Mathews,~W.: Advance of the late Wisconsin Cordilleran Ice Sheet in southern British Columbia since 22\,000~yr~BP, Quaternary Res., 13, 322--326,
doi:\href{http://dx.doi.org/10.1016/0033-5894(80)90060-5}{10.1016/0033-5894(80)90060-5}, 1980.


\bibitem[{Clague(1981)}]{Clague.1981}
Clague,~J.~J.: Late Quaternary Geology and Geochronology of British Columbia Part 2: Summary and Discussion of Radiocarbon-Dated Quaternary History, Geol. Surv. of Can., Ottawa, ON, Paper 80-35,
doi:\href{http://dx.doi.org/10.4095/119439}{10.4095/119439}, 1981.


\bibitem[{Clague(1985)}]{Clague.1985}
Clague,~J.~J.: Delaciation of the Prince Rupert -- Kitimat area, British Columbia, Can.~J. Earth Sci., 22, 256--265,
doi:\href{http://dx.doi.org/10.1139/e85-022}{10.1139/e85-022}, 1985.


\bibitem[{Clague(1986)}]{Clague.1986}
Clague,~J.~J.: The Quaternary stratigraphic record of British Columbia --- evidence for episodic sedimentation and erosion controlled by glaciation, Can.~J. Earth Sci., 23, 885--894,
doi:\href{http://dx.doi.org/10.1139/e86-090}{10.1139/e86-090}, 1986.


\bibitem[{Clague(1989)}]{Clague.1989}
Clague,~J.~J.: Character and distribution of Quaternary deposits (Canadian Cordillera), in: Quaternary Geology of Canada and Greenland, edited by: Fulton,~R.~J., vol.~1 of Geology of Canada, Geol. Surv. of Can., Ottawa, ON, 34--48,
doi:\href{http://dx.doi.org/10.4095/127905}{10.4095/127905}, 1989.


\bibitem[{Clague and James(2002)}]{Clague.James.2002}
Clague,~J.~J. and James,~T.~S.: History and isostatic effects of the last ice sheet in southern British Columbia, Quaternary Res., 21, 71--87,
doi:\href{http://dx.doi.org/10.1016/s0277-3791(01)00070-1}{10.1016/s0277-3791(01)00070-1}, 2002.


\bibitem[{Clague and Ward(2011)}]{Clague.Ward.2011}
Clague,~J.~J. and Ward,~B.: Pleistocene Glaciation of British
Columbia, in:   Dev. Quaternary Sci., vol.~15, edited by: Ehlers,~J., Gibbard,~P.~L., and Hughes,~P.~D., Elsevier, Amsterdam, 563--573,
doi:\href{http://dx.doi.org/10.1016/b978-0-444-53447-7.00044-1}{10.1016/b978-0-444-53447-7.00044-1}, 2011.


\bibitem[{Clague et~al.(1997)Clague, Mathewes, Guilbault, Hutchinson, and Ricketts}]{Clague.etal.1997}
Clague,~J.~J., Mathewes,~R.~W., Guilbault,~J.-P., Hutchinson,~I., and Ricketts,~B.~D.: Pre-Younger Dryas resurgence of the southwestern margin of the Cordilleran ice sheet, British Columbia, Canada, Boreas, 26, 261--278,
doi:\href{http://dx.doi.org/10.1111/j.1502-3885.1997.tb00855.x}{10.1111/j.1502-3885.1997.tb00855.x}, 1997.


\bibitem[{Clague et~al.(2005)Clague, Froese, Hutchinson, James, and Simon}]{Clague.etal.2005}
Clague,~J.~J., Froese,~D., Hutchinson,~I., James,~T.~S., and Simon,~K.~M.: Early growth of the last Cordilleran ice sheet deduced from glacio-isostatic depression in southwest British Columbia, Canada, Quaternary Res., 63, 53--59,
doi:\href{http://dx.doi.org/10.1016/j.yqres.2004.09.007}{10.1016/j.yqres.2004.09.007}, 2005.


\bibitem[{Clark and Mix(2002)}]{Clark.Mix.2002}
Clark,~P.~U. and Mix,~A.~C.: Ice sheets and sea level of the Last Glacial Maximum, Quaternary Res., 21, 1--7,
doi:\href{http://dx.doi.org/10.1016/s0277-3791(01)00118-4}{10.1016/s0277-3791(01)00118-4}, 2002.


\bibitem[{Clason et~al.(2012)Clason, Mair, Burgess, and Nienow}]{Clason.etal.2012}
Clason,~C., Mair,~D.~W., Burgess,~D.~O., and Nienow,~P.~W.: Modelling the delivery of supraglacial meltwater to the ice/bed interface: application to southwest Devon Ice Cap, Nunavut, Canada,~J. Glaciol., 58, 361--374,
doi:\href{http://dx.doi.org/10.3189/2012jog11j129}{10.3189/2012jog11j129}, 2012.


\bibitem[{Clarke et~al.(2013)Clarke, Anslow, Jarosch, Radi{\'{c}}, Menounos, Bolch, and Berthier}]{Clarke.etal.2013}
Clarke, G. K.~C., Anslow, F.~S., Jarosch, A.~H., Radi{\'{c}}, V., Menounos, B., Bolch, T., and Berthier, E.: Ice Volume and Subglacial Topography for Western Canadian Glaciers from Mass Balance Fields, Thinning Rates, and a Bed Stress Model, J. Climate, 26, 4282--4303,
doi:\href{http://dx.doi.org/10.1175/jcli-d-12-00513.1}{10.1175/jcli-d-12-00513.1}, 2013.


\bibitem[{Clason et~al.(2014)Clason, Applegate, and Holmlund}]{Clason.etal.2014}
Clason,~C., Applegate,~P., and Holmlund,~P.: Modelling Late Weichselian evolution of the Eurasian ice sheets forced by surface meltwater-enhanced basal sliding,~J. Glaciol., 60, 29--40,
doi:\href{http://dx.doi.org/10.3189/2014jog13j037}{10.3189/2014jog13j037}, 2014.


\bibitem[{Cosma et~al.(2008)Cosma, Hendy, and Chang}]{Cosma.etal.2008}
Cosma,~T., Hendy,~I., and Chang,~A.: Chronological constraints on Cordilleran Ice Sheet glaciomarine sedimentation from core MD02-2496 off Vancouver Island (western Canada), Quaternary Sci. Rev., 27, 941--955,
doi:\href{http://dx.doi.org/10.1016/j.quascirev.2008.01.013}{10.1016/j.quascirev.2008.01.013}, 2008.


\bibitem[{Cuffey and Paterson(2010)}]{Cuffey.Paterson.2010}
Cuffey, K.~M. and Paterson, W. S.~B.: The physics of glaciers, Elsevier, Amsterdam, 2010.


\bibitem[{Dansgaard et~al.(1993)Dansgaard, Johnsen, Clausen, Dahl-Jensen, Gundestrup, Hammer, Hvidberg, Steffensen, Sveinbj\"ornsdottir, Jouzel, and Bond}]{Dansgaard.etal.1993}
Dansgaard,~W., Johnsen,~S.~J., Clausen,~H.~B., Dahl-Jensen,~D., Gundestrup,~N.~S., Hammer,~C.~U., Hvidberg,~C.~S., Steffensen,~J.~P., Sveinbj\"ornsdottir,~A.~E., Jouzel,~J., and Bond,~G.: Evidence for general instability of past climate from a 250-kyr ice-core record, Nature, 364, 218--220,
doi:\href{http://dx.doi.org/10.1038/364218a0}{10.1038/364218a0}, data archived at the World Data Center for Paleoclimatology, Boulder, Colorado, USA., 1993.


\bibitem[{Davies et~al.(2011)Davies, Mix, Stoner, Addison, Jaeger, Finney, and Wiest}]{Davies.etal.2011}
Davies,~M.~H., Mix,~A.~C., Stoner,~J.~S., Addison,~J.~A., Jaeger,~J., Finney,~B., and Wiest,~J.: The deglacial transition on the southeastern Alaska Margin: Meltwater input, sea level rise, marine productivity, and sedimentary anoxia, Paleoceanography, 26, PA2223,
doi:\href{http://dx.doi.org/10.1029/2010pa002051}{10.1029/2010pa002051}, 2011.


\bibitem[{Davis and Mathews(1944)}]{Davis.Mathews.1944}
Davis,~N.~F.~G. and Mathews,~W.~H.: Four phases of glaciation with illustrations from Southwestern British Columbia,~J. Geol., 52, 403--413,
doi:\href{http://dx.doi.org/10.1086/625236}{10.1086/625236}, 1944.


\bibitem[{Dawson(1888)}]{Dawson.1888}
Dawson,~G.~M.: III. -- Recent observations on the glaciation of British Columbia and adjacent regions, Geol. Mag., 5, 347--350,
doi:\href{http://dx.doi.org/10.1017/s0016756800182159}{10.1017/s0016756800182159}, 1888.


\bibitem[{Demuro et~al.(2012)Demuro, Froese, Arnold, and Roberts}]{Demuro.etal.2012}
Demuro,~M., Froese,~D.~G., Arnold,~L.~J., and Roberts,~R.~G.: Single-grain OSL dating of glaciofluvial quartz constrains Reid glaciation in NW Canada to MIS 6, Quaternary Res., 77, 305--316,
doi:\href{http://dx.doi.org/10.1016/j.yqres.2011.11.009}{10.1016/j.yqres.2011.11.009}, 2012.


\bibitem[{Duk-Rodkin(1999)}]{Duk-Rodkin.1999}
Duk-Rodkin,~A.: Glacial limits map of Yukon Territory, Open File 3694, Geol. Surv. of Can., Ottawa, ON,
doi:\href{http://dx.doi.org/10.4095/210739}{10.4095/210739}, 1999.


\bibitem[{Duk-Rodkin et~al.(1996)Duk-Rodkin, Barendregt, Tarnocai, and Phillips}]{Duk-Rodkin.etal.1996}
Duk-Rodkin,~A., Barendregt,~R.~W., Tarnocai,~C., and Phillips,~F.~M.: Late Tertiary to late Quaternary record in the Mackenzie Mountains, Northwest Territories, Canada: stratigraphy, paleosols, paleomagnetism, and chlorine-36, Can.~J. Earth Sci., 33, 875--895,
doi:\href{http://dx.doi.org/10.1139/e96-066}{10.1139/e96-066}, 1996.


\bibitem[{Dyke(2004)}]{Dyke.2004}
Dyke,~A.~S.: An outline of North American deglaciation with emphasis
on central and northern Canada, in: Dev. Quaternary Sci., vol.~2,
edited by: Ehlers,~J. and Gibbard,~P.~L., Elsevier, Amsterdam, 373--424,
doi:\href{http://dx.doi.org/10.1016/S1571-0866(04)80209-4}{10.1016/S1571-0866(04)80209-4}, 2004.


\bibitem[{Dyke and Prest(1987)}]{Dyke.Prest.1987}
Dyke,~A.~S. and Prest,~V.~K.: Late wisconsinan and holocene history of the laurentide ice sheet, G{e}ogr. Phys. Quatern., 41, 237--263,
doi:\href{http://dx.doi.org/10.7202/032681ar}{10.7202/032681ar}, 1987.


\bibitem[{Dyke et~al.(2003)Dyke, Moore, and Robertson}]{Dyke.etal.2003}
Dyke,~A.~S., Moore,~A., and Robertson,~L.: Deglaciation of North America, Open File 1547, Geol. Surv. of Can., Ottawa, ON, 2003.


\bibitem[{Ehlers and Gibbard(2004)}]{Ehlers.Gibbard.2004}
Ehlers,~J. and Gibbard,~P.~L. (Eds.): Dev. Quaternary Sci., vol.~2, Elsevier, Amsterdam, 2004.


\bibitem[{Ehlers et~al.(2011)Ehlers, Gibbard, and Hughes}]{Ehlers.etal.2011}
Ehlers,~J., Gibbard,~P.~L., and Hughes,~P.~D. (Eds.):  Dev. Quaternary
Sci., vol.~15, Elsevier, Amsterdam, 2011.


\bibitem[{Fahnestock et~al.(2001)Fahnestock, Abdalati, Joughin, Brozena, and Gogineni}]{Fahnestock.etal.2001}
Fahnestock,~M., Abdalati,~W., Joughin,~I., Brozena,~J., and Gogineni,~P.: High geothermal heat flow, basal melt, and the origin of rapid ice flow in central greenland, Sience, 294, 2338--2342,
doi:\href{http://dx.doi.org/10.1126/science.1065370}{10.1126/science.1065370}, 2001.


\bibitem[{Fisher et~al.(2008)Fisher, Osterberg, Dyke, Dahl-Jensen, Demuth, Zdanowicz, Bourgeois, Koerner, Mayewski, Wake, Kreutz, Steig, Zheng, Yalcin, Goto-Azuma, Luckman, and Rupper}]{Fisher.etal.2008}
Fisher, D., Osterberg, E., Dyke, A., Dahl-Jensen, D., Demuth, M., Zdanowicz, C., Bourgeois, J., Koerner, R.~M., Mayewski, P., Wake, C., Kreutz, K., Steig, E., Zheng, J., Yalcin, K., Goto-Azuma, K., Luckman, B., and Rupper, S.: The Mt Logan Holocene--late Wisconsinan isotope record: tropical Pacific--Yukon connections, The Holocene, 18, 667--677,
doi:\href{http://dx.doi.org/10.1177/0959683608092236}{10.1177/0959683608092236}, 2008.


\bibitem[{Fisher et~al.(2004)Fisher, Wake, Kreutz, Yalcin, Steig, Mayewski, Anderson, Zheng, Rupper, Zdanowicz, Demuth, Waszkiewicz, Dahl-Jensen, Goto-Azuma, Bourgeois, Koerner, Sekerka, Osterberg, Abbott, Finney, and Burns}]{Fisher.etal.2004}
Fisher, D.~A., Wake, C., Kreutz, K., Yalcin, K., Steig, E., Mayewski, P., Anderson, L., Zheng, J., Rupper, S., Zdanowicz, C., Demuth, M., Waszkiewicz, M., Dahl-Jensen, D., Goto-Azuma, K., Bourgeois, J.~B., Koerner, R.~M., Sekerka, J., Osterberg, E., Abbott, M.~B., Finney, B.~P., and Burns, S.~J.: Stable Isotope Records from Mount Logan, Eclipse Ice Cores and Nearby Jellybean Lake. Water Cycle of the North Pacific Over 2000 Years and Over Five Vertical Kilometres: Sudden Shifts and Tropical Connections, G\'{e}ogr. phys. Quatern., 58, 337,
doi:\href{http://dx.doi.org/10.7202/013147ar}{10.7202/013147ar}, 2004.


\bibitem[{Friele and Clague(2002{\natexlab{a}})}]{Friele.Clague.2002}
Friele,~P.~A. and Clague,~J.~J.: Readvance of glaciers in the British Columbia Coast Mountains at the end of the last glaciation, Quatern. Int., 87, 45--58,
doi:\href{http://dx.doi.org/10.1016/s1040-6182(01)00061-1}{10.1016/s1040-6182(01)00061-1}, 2002{\natexlab{a}}.


\bibitem[{Friele and Clague(2002{\natexlab{b}})}]{Friele.Clague.2002a}
Friele,~P.~A. and Clague,~J.~J.: Younger Dryas readvance in Squamish river valley, southern Coast mountains, British Columbia, Quaternary Res., 21, 1925--1933,
doi:\href{http://dx.doi.org/10.1016/s0277-3791(02)00081-1}{10.1016/s0277-3791(02)00081-1}, 2002{\natexlab{b}}.


\bibitem[{Fulton(1967)}]{Fulton.1967}
Fulton,~R.~J.: Deglaciation studies in Kamloops region, an area of moderate relief, British Columbia, vol. 154 of Bull., Geol. Surv. of Can., Ottawa, ON,
doi:\href{http://dx.doi.org/10.4095/101467}{10.4095/101467}, 1967.


\bibitem[{Fulton(1991)}]{Fulton.1991}
Fulton,~R.~J.: A Conceptual model for growth and decay of the Cordilleran Ice Sheet, G{e}ogr. Phys. Quatern., 45, 281--286,
doi:\href{http://dx.doi.org/10.7202/032875ar}{10.7202/032875ar}, 1991.


\bibitem[{Glen(1952)}]{Glen.1952}
Glen, J.: Experiments on the deformation of ice, J. Glaciol., 2, 111--114, 1952.


\bibitem[{Golledge et~al.(2012)Golledge, Mackintosh, Anderson, Buckley, Doughty, Barrell, Denton, Vandergoes, Andersen, and Schaefer}]{Golledge.etal.2012}
Golledge,~N.~R., Mackintosh,~A.~N., Anderson,~B.~M., Buckley,~K.~M., Doughty,~A.~M., Barrell,~D.~J., Denton,~G.~H., Vandergoes,~M.~J., Andersen,~B.~G., and Schaefer,~J.~M.: Last Glacial Maximum climate in {N}ew {Z}ealand inferred from a modelled {S}outhern {A}lps icefield, Quaternary Res., 46, 30--45,
doi:\href{http://dx.doi.org/10.1016/j.quascirev.2012.05.004}{10.1016/j.quascirev.2012.05.004}, 2012.


\bibitem[{Greve(1997)}]{Greve.1997}
Greve, R.: A continuum-mechanical formulation for shallow polythermal ice sheets, Philos. T. R. Soc. A, 355, 921--974,
doi:\href{http://dx.doi.org/10.1098/rsta.1997.0050}{10.1098/rsta.1997.0050}, 1997.


\bibitem[{Herbert et~al.(2001)Herbert, Schuffert, Andreasen, Heusser, Lyle, Mix, Ravelo, Stott, and Herguera}]{Herbert.etal.2001}
Herbert,~T.~D., Schuffert,~J.~D., Andreasen,~D., Heusser,~L., Lyle,~M., Mix,~A., Ravelo,~A.~C., Stott,~L.~D., and Herguera,~J.~C.: Collapse of the California current during glacial maxima linked to climate change on land, Sience, 293, 71--76,
doi:\href{http://dx.doi.org/10.1126/science.1059209}{10.1126/science.1059209}, data archived at the World Data Center for Paleoclimatology, Boulder, Colorado, USA, 2001.


\bibitem[{Hijmans et~al.(2005)Hijmans, Cameron, Parra, Jones, and Jarvis}]{Hijmans.etal.2005}
Hijmans, R.~J., Cameron, S.~E., Parra, J.~L., Jones, P.~G., and Jarvis, A.: Very high resolution interpolated climate surfaces for global land areas, Int. J. Climatol., 25, 1965--1978,
doi:\href{http://dx.doi.org/10.1002/joc.1276}{10.1002/joc.1276}, 2005.


\bibitem[{Hidy et~al.(2013)Hidy, Gosse, Froese, Bond, and Rood}]{Hidy.etal.2013}
Hidy,~A.~J., Gosse,~J.~C., Froese,~D.~G., Bond,~J.~D., and Rood,~D.~H.: A latest Pliocene age for the earliest and most extensive Cordilleran Ice Sheet in northwestern Canada, Quaternary Res., 61, 77--84,
doi:\href{http://dx.doi.org/10.1016/j.quascirev.2012.11.009}{10.1016/j.quascirev.2012.11.009}, 2013.


\bibitem[{Hock(2003)}]{Hock.2003}
Hock,~R.: Temperature index melt modelling in mountain areas,~J. Hydrol., 282, 104--115,
doi:\href{http://dx.doi.org/10.1016/S0022-1694(03)00257-9}{10.1016/S0022-1694(03)00257-9}, 2003.


\bibitem[{Imbrie et~al.(1989)Imbrie, McIntyre, and Mix}]{Imbrie.etal.1989}
Imbrie,~J., McIntyre,~A., and Mix,~A.: Oceanic response to orbital forcing in the late quaternary: observational and experimental strategies, in: Climate and Geo-Sciences, edited by: Berger,~A., Schneider,~S., and Duplessy,~J., vol. 285 of NATO ASI Series C, Kluwer, Norwell, MA, 121--164,
doi:\href{http://dx.doi.org/10.1007/978-94-009-2446-8_7}{10.1007/978-94-009-2446-8\_7}, 1989.


\bibitem[{Ivy-Ochs et~al.(1999)Ivy-Ochs, Schluchter, Kubik, and Denton}]{Ivy-Ochs.etal.1999}
Ivy-Ochs,~S., Schluchter,~C., Kubik,~P.~W., and Denton,~G.~H.: Moraine
exposure dates imply synchronous Younger Dryas Glacier advances in the
European Alps and in the Southern Alps of New Zealand, Geogr. Ann. A, 81,
313--323,
doi:\href{http://dx.doi.org/10.1111/1468-0459.00060}{10.1111/1468-0459.00060},
1999.


\bibitem[{Jackson and Clague(1991)}]{Jackson.Clague.1991}
Jackson,~L.~E. and Clague,~J.~J.: The Cordilleran ice sheet: one hundred and fifty years of exploration and discovery, G{e}ogr. Phys. Quatern., 45, 269--280,
doi:\href{http://dx.doi.org/10.7202/032874ar}{10.7202/032874ar}, 1991.


\bibitem[{Jackson et~al.(1997)Jackson, Phillips, Shimamura, and Little}]{Jackson.etal.1997}
Jackson,~L.~E., Phillips,~F.~M., Shimamura,~K., and Little,~E.~C.: Cosmogenic $^{36}$Cl dating of the Foothills erratics train, Alberta, Canada, Geology, 25, 195,
doi:\href{http://dx.doi.org/10.1130/0091-7613(1997)025<0195:ccdotf>2.3.co;2}{10.1130/0091-7613(1997)025\textless0195:ccdotf\textgreater2.3.co;2}, 1997.


\bibitem[{James et~al.(2009)James, Gowan, Wada, and Wang}]{James.etal.2009}
James,~T.~S., Gowan,~E.~J., Wada,~I., and Wang,~K.: Viscosity of the
asthenosphere from glacial isostatic adjustment and subduction dynamics at
the northern Cascadia subduction zone, British Columbia, Canada,~J. Geophys.
Res., 114, B04405,
doi:\href{http://dx.doi.org/10.1029/2008jb006077}{10.1029/2008jb006077},
2009.


\bibitem[{Jarosch et~al.(2012)Jarosch, Anslow, and Clarke}]{Jarosch.etal.2012}
Jarosch, A.~H., Anslow, F.~S., and Clarke, G. K.~C.: High-resolution precipitation and temperature downscaling for glacier models, Clim. Dynam., 38, 391--409,
doi:\href{http://dx.doi.org/10.1007/s00382-010-0949-1}{10.1007/s00382-010-0949-1}, 2012.


\bibitem[{Johnsen et~al.(1995)Johnsen, Dahl-Jensen, Dansgaard, and Gundestrup}]{Johnsen.etal.1995}
Johnsen,~S.~J., Dahl-Jensen,~D., Dansgaard,~W., and Gundestrup,~N.: Greenland palaeotemperatures derived from GRIP bore hole temperature and ice core isotope profiles, Tellus B, 47, 624--629,
doi:\href{http://dx.doi.org/10.1034/j.1600-0889.47.issue5.9.x}{10.1034/j.1600-0889.47.issue5.9.x}, 1995.


\bibitem[{Jouzel et~al.(2007)Jouzel, Masson-Delmotte, Cattani, Dreyfus, Falourd, Hoffmann, Minster, Nouet, Barnola, Chappellaz, Fischer, Gallet, Johnsen, Leuenberger, Loulergue, Luethi, Oerter, Parrenin, Raisbeck, Raynaud, Schilt, Schwander, Selmo, Souchez, Spahni, Stauffer, Steffensen, Stenni, Stocker, Tison, Werner, and Wolff}]{Jouzel.etal.2007}
Jouzel,~J., Masson-Delmotte,~V., Cattani,~O., Dreyfus,~G., Falourd,~S., Hoffmann,~G., Minster,~B., Nouet,~J., Barnola,~J.~M., Chappellaz,~J., Fischer,~H., Gallet,~J.~C., Johnsen,~S., Leuenberger,~M., Loulergue,~L., Luethi,~D., Oerter,~H., Parrenin,~F., Raisbeck,~G., Raynaud,~D., Schilt,~A., Schwander,~J., Selmo,~E., Souchez,~R., Spahni,~R., Stauffer,~B., Steffensen,~J.~P., Stenni,~B., Stocker,~T.~F., Tison,~J.~L., Werner,~M., and Wolff,~E.~W.: Orbital and millennial antarctic climate variability over the past 800\,000~years, Sience, 317, 793--796,
doi:\href{http://dx.doi.org/10.1126/science.1141038}{10.1126/science.1141038}, data archived at the World Data Center for Paleoclimatology, Boulder, Colorado, USA., 2007.


\bibitem[{Kaufman and Manley(2004)}]{Kaufman.Manley.2004}
Kaufman,~D.~S. and Manley,~W.~F.: Pleistocene maximum and late wisconsinan glacier extents across Alaska, {USA}, in: Dev. Quaternary Sci., vol.~2,
edited by: Ehlers,~J. and Gibbard,~P.~L., Elsevier, Amsterdam, 9--27,
doi:\href{http://dx.doi.org/10.1016/S1571-0866(04)80182-9}{10.1016/S1571-0866(04)80182-9}, 2004.


\bibitem[{Kienast and McKay(2001)}]{Kienast.McKay.2001}
Kienast, S.~S. and McKay, J.~L.: Sea surface temperatures in the subarctic northeast Pacific reflect millennial-scale climate oscillations during the last 16 kyrs, Geophys. Res. Lett., 28, 1563--1566,
doi:\href{http://dx.doi.org/10.1029/2000GL012543}{10.1029/2000GL012543}, 2001.


\bibitem[{Kleman(1994)}]{Kleman.1994}
Kleman,~J.: Preservation of landforms under ice sheets and ice caps, Geomorphology, 9, 19--32,
doi:\href{http://dx.doi.org/10.1016/0169-555x(94)90028-0}{10.1016/0169-555x(94)90028-0}, 1994.


\bibitem[{Kleman and Stroeven(1997)}]{Kleman.Stroeven.1997}
Kleman,~J. and Stroeven,~A.~P.: Preglacial surface remnants and Quaternary glacial regimes in northwestern Sweden, Geomorphology, 19, 35--54,
doi:\href{http://dx.doi.org/10.1016/s0169-555x(96)00046-3}{10.1016/s0169-555x(96)00046-3}, 1997.


\bibitem[{Kleman et~al.(1997)Kleman, H{\"a}ttestrand, Borgstr{\"o}m, and Stroeven}]{Kleman.etal.1997}
Kleman,~J., H{\"a}ttestrand,~C., Borgstr{\"o}m,~I., and Stroeven,~A.: Fennoscandian paleoglaciology reconstructed using a glacial geological inversion model,~J. Glaciol., 43, 283--299, 1997.


\bibitem[{Kleman et~al.(2006)Kleman, H{\"a}ttestrand, Stroeven, Jansson, De~Angelis, and Borgstr{\"o}m}]{Kleman.etal.2006}
Kleman,~J., H{\"a}ttestrand,~C., Stroeven,~A.~P., Jansson,~K.~N., De~Angelis,~H., and Borgstr{\"o}m,~I.: Reconstruction of palaeo-ice sheets -- inversion of their glacial geomorphological record, in: Glacier Science and Environmental Change, edited by: Knight,~P.~G., Blackwell, Malden, MA, 192--198,
doi:\href{http://dx.doi.org/10.1002/9780470750636.ch38}{10.1002/9780470750636.ch38}, 2006.


\bibitem[{Kleman et~al.(2008)Kleman, Stroeven, and Lundqvist}]{Kleman.etal.2008}
Kleman,~J., Stroeven,~A.~P., and Lundqvist,~J.: Patterns of quaternary ice sheet erosion and deposition in Fennoscandia and a theoretical framework for explanation, Geomorphology, 97, 73--90,
doi:\href{http://dx.doi.org/10.1016/j.geomorph.2007.02.049}{10.1016/j.geomorph.2007.02.049}, 2008.


\bibitem[{Kleman et~al.(2010)Kleman, Jansson, De~Angelis, Stroeven, H\"{a}ttestrand, Alm, and Glasser}]{Kleman.etal.2010}
Kleman,~J., Jansson,~K., De~Angelis,~H., Stroeven,~A., H\"{a}ttestrand,~C., Alm,~G., and Glasser,~N.: North American ice sheet build-up during the last glacial cycle, 115--21\,kyr, Quaternary Sci. Rev., 29, 2036--2051,
doi:\href{http://dx.doi.org/10.1016/j.quascirev.2010.04.021}{10.1016/j.quascirev.2010.04.021}, 2010.


\bibitem[{Kovanen(2002)}]{Kovanen.2002}
Kovanen,~D.~J.: Morphologic and stratigraphic evidence for Aller{\o}d and Younger Dryas age glacier fluctuations of the Cordilleran Ice Sheet, British Columbia, Canada and Northwest Washington,~USA, Boreas, 31, 163--184,
doi:\href{http://dx.doi.org/10.1111/j.1502-3885.2002.tb01064.x}{10.1111/j.1502-3885.2002.tb01064.x}, 2002.


\bibitem[{Kovanen and Easterbrook(2002)}]{Kovanen.Easterbrook.2002}
Kovanen,~D.~J. and Easterbrook,~D.~J.: Timing and extent of Aller{\o}d and Younger Dryas age (ca. 12\,500--10\,000 $^{14}$C~yr~BP) oscillations of the Cordilleran Ice Sheet in the Fraser Lowland, Western North America, Quaternary Res., 57, 208--224,
doi:\href{http://dx.doi.org/10.1006/qres.2001.2307}{10.1006/qres.2001.2307}, 2002.


\bibitem[{Lakeman et~al.(2008)Lakeman, Clague, and Menounos}]{Lakeman.etal.2008}
Lakeman,~T.~R., Clague,~J.~J., and Menounos,~B.: Advance of alpine glaciers during final retreat of the Cordilleran ice sheet in the Finlay River area, northern British Columbia, Canada, Quaternary Res., 69, 188--200,
doi:\href{http://dx.doi.org/10.1016/j.yqres.2008.01.002}{10.1016/j.yqres.2008.01.002}, 2008.


\bibitem[{Langen et~al.(2012)Langen, Solgaard, and Hvidberg}]{Langen.etal.2012}
Langen,~P.~L., Solgaard,~A.~M., and Hvidberg,~C.~S.: Self-inhibiting growth of the Greenland ice sheet, Geophys. Res. Lett., 39, L12502,
doi:\href{http://dx.doi.org/10.1029/2012GL051810}{10.1029/2012GL051810}, 2012.


\bibitem[{Levermann et~al.(2012)Levermann, Albrecht, Winkelmann, Martin, Haseloff, and Joughin}]{Levermann.etal.2012}
 Levermann,~A., Albrecht,~T., Winkelmann,~R., Martin,~M.~A., Haseloff,~M., and Joughin,~I.: Kinematic first-order calving law implies potential for  abrupt ice-shelf retreat, The Cryosphere, 6, 273--286,
doi:\href{http://dx.doi.org/10.5194/tc-6-273-2012}{10.5194/tc-6-273-2012}, 2012.


\bibitem[{Lingle and Clark(1985)}]{Lingle.Clark.1985}
Lingle,~C.~S. and Clark,~J.~A.: A~numerical model of interactions between a~marine ice sheet and the Solid Earth: application to a~West Antarctic ice stream,~J. Geophys. Res., 90, 1100--1114,
doi:\href{http://dx.doi.org/10.1029/JC090iC01p01100}{10.1029/JC090iC01p01100}, 1985.


\bibitem[{Lisiecki and Raymo(2005)}]{Lisiecki.Raymo.2005}
Lisiecki,~L.~E. and Raymo,~M.~E.: A Pliocene-Pleistocene stack of 57 globally distributed benthic $\delta^{18}$O records, Paleoceanography, 20, PA1003,
doi:\href{http://dx.doi.org/10.1029/2004pa001071}{10.1029/2004pa001071}, 2005.


\bibitem[{Lliboutry and Duval(1985)}]{Lliboutry.Duval.1985}
Lliboutry, L.~A. and Duval, P.: Various isotropic and anisotropic ices found in glaciers and polar ice caps and their corresponding rheologies, Ann. Geophys., 3, 207--224, 1985.


\bibitem[{Luckman and Osborn(1979)}]{Luckman.Osborn.1979}
Luckman, B. and Osborn, G.: Holocene glacier fluctuations in the middle Canadian Rocky Mountains, Quaternary Res., 11, 52--77,
doi:\href{http://dx.doi.org/10.1016/0033-5894(79)90069-3}{10.1016/0033-5894(79)90069-3}, 1979.


\bibitem[{Lundqvist(1987)}]{Lundqvist.1987}
Lundqvist,~J.: Glaciodynamics of the Younger Dryas Marginal zone in Scandinavia: implications of a revised glaciation model, Geogr. Ann. A, 69, 305,
doi:\href{http://dx.doi.org/10.2307/521191}{10.2307/521191}, 1987.


\bibitem[{L{\"u}thi et~al.(2002)L{\"u}thi, Funk, Iken, Gogineni, and Truffer}]{Luthi.etal.2002}
L{\"u}thi, M., Funk, M., Iken, A., Gogineni, S., and Truffer, M.: Mechanisms of fast flow in {J}akobshavns {I}sbr{\ae}, {G}reenland; {P}art {III}: measurements of ice deformation, temperature and cross-borehole conductivity in boreholes to the bedrock, J. Glaciol., 48, 369--385,
doi:\href{http://dx.doi.org/10.3189/172756502781831322}{10.3189/172756502781831322}, 2002.


\bibitem[{Margold et~al.(2011)Margold, Jansson, Kleman, and Stroeven}]{Margold.etal.2011}
Margold,~M., Jansson,~K.~N., Kleman,~J., and Stroeven,~A.~P.: Glacial meltwater landforms of central British Columbia,~J. Maps, 7, 486--506,
doi:\href{http://dx.doi.org/10.4113/jom.2011.1205}{10.4113/jom.2011.1205}, 2011.


\bibitem[{Margold et~al.(2013{\natexlab{a}})Margold, Jansson, Kleman, and Stroeven}]{Margold.etal.2013}
Margold,~M., Jansson,~K.~N., Kleman,~J., and Stroeven,~A.~P.: Lateglacial ice dynamics of the Cordilleran Ice Sheet in northern British Columbia and southern Yukon Territory: retreat pattern of the Liard Lobe reconstructed from the glacial landform record,~J. Quaternary Sci., 28, 180--188,
doi:\href{http://dx.doi.org/10.1002/jqs.2604}{10.1002/jqs.2604}, 2013{\natexlab{a}}.


\bibitem[{Margold et~al.(2013{\natexlab{b}})Margold, Jansson, Kleman, Stroeven, and Clague}]{Margold.etal.2013a}
Margold,~M., Jansson,~K.~N., Kleman,~J., Stroeven,~A.~P., and Clague,~J.~J.: Retreat pattern of the Cordilleran Ice Sheet in central British Columbia at the end of the last glaciation reconstructed from glacial meltwater landforms, Boreas, 42, 830--847,
doi:\href{http://dx.doi.org/10.1111/bor.12007}{10.1111/bor.12007}, 2013{\natexlab{b}}.


\bibitem[{Margold et~al.(2014)Margold, Stroeven, Clague, and Heyman}]{Margold.etal.2014}
Margold,~M., Stroeven,~A.~P., Clague,~J.~J., and Heyman,~J.: Timing of terminal Pleistocene deglaciation at high elevations in southern and central British Columbia constrained by $^{10}$Be exposure dating, Quaternary Res., 99, 193--202,
doi:\href{http://dx.doi.org/10.1016/j.quascirev.2014.06.027}{10.1016/j.quascirev.2014.06.027}, 2014.


\bibitem[{Marshall et~al.(2000)Marshall, Tarasov, Clarke, and Peltier}]{Marshall.etal.2000}
Marshall,~S.~J., Tarasov,~L., Clarke,~G.~K., and Peltier,~W.~R.: Glaciological reconstruction of the Laurentide Ice Sheet: physical processes and modelling challenges, Can.~J. Earth Sci., 37, 769--793,
doi:\href{http://dx.doi.org/10.1139/e99-113}{10.1139/e99-113}, 2000.


\bibitem[{Martin et~al.(2011)Martin, Winkelmann, Haseloff, Albrecht, Bueler, Khroulev, and Levermann}]{Martin.etal.2011}
 Martin,~M.~A., Winkelmann,~R., Haseloff,~M., Albrecht,~T., Bueler,~E., Khroulev,~C., and Levermann,~A.: The Potsdam Parallel Ice Sheet Model (PISM-PIK) -- Part 2: Dynamic equilibrium simulation of the Antarctic ice sheet, The Cryosphere, 5, 727--740,
doi:\href{http://dx.doi.org/10.5194/tc-5-727-2011}{10.5194/tc-5-727-2011}, 2011.


\bibitem[{Menounos et~al.(2008)Menounos, Osborn, Clague, and Luckman}]{Menounos.etal.2008}
Menounos,~B., Osborn,~G., Clague,~J.~J., and Luckman,~B.~H.: Latest Pleistocene and Holocene glacier fluctuations in western Canada, Quaternary Sci. Rev., 28, 2049--2074,
doi:\href{http://dx.doi.org/10.1016/j.quascirev.2008.10.018}{10.1016/j.quascirev.2008.10.018}, 2008.


\bibitem[{Mesinger et~al.(2006)Mesinger, DiMego, Kalnay, Mitchell, Shafran, Ebisuzaki, Jovi\'{c}, Woollen, Rogers, Berbery, Ek, Fan, Grumbine, Higgins, Li, Lin, Manikin, Parrish, and Shi}]{Mesinger.etal.2006}
Mesinger,~F., DiMego,~G., Kalnay,~E., Mitchell,~K., Shafran,~P.~C., Ebisuzaki,~W., Jovi\'{c},~D., Woollen,~J., Rogers,~E., Berbery,~E.~H., Ek,~M.~B., Fan,~Y., Grumbine,~R., Higgins,~W., Li,~H., Lin,~Y., Manikin,~G., Parrish,~D., and Shi,~W.: North American regional reanalysis,~B. Am. Meteorol. Soc., 87, 343--360,
doi:\href{http://dx.doi.org/10.1175/BAMS-87-3-343}{10.1175/BAMS-87-3-343}, 2006.


\bibitem[{Nye(1953)}]{Nye.1953}
Nye, J.~F.: The Flow Law of Ice from Measurements in Glacier Tunnels, Laboratory Experiments and the Jungfraufirn Borehole Experiment, Proc. R. Soc. London, Ser. A, 219, 477--489, 1953.


\bibitem[{Osborn and Gerloff(1997)}]{Osborn.Gerloff.1997}
Osborn,~G. and Gerloff,~L.: Latest pleistocene and early Holocene fluctuations of glaciers in the Canadian and northern American Rockies, Quatern. Int., 38--39, 7--19,
doi:\href{http://dx.doi.org/10.1016/s1040-6182(96)00026-2}{10.1016/s1040-6182(96)00026-2}, 1997.


\bibitem[{Osborn et~al.(1995)Osborn, Clapperton, Davis, Reasoner, Rodbell, Seltzer, and Zielinski}]{Osborn.etal.1995}
Osborn, G., Clapperton, C., Davis, P., Reasoner, M., Rodbell, D.~T., Seltzer, G.~O., and Zielinski, G.: Potential glacial evidence for the younger dryas event in the Cordillera of North and South America, Quaternary Sci. Rev., 14, 823--832,
doi:\href{http://dx.doi.org/10.1016/0277-3791(95)00064-x}{10.1016/0277-3791(95)00064-x}, 1995.


\bibitem[{Paterson and Budd(1982)}]{Paterson.Budd.1982}
Paterson, W. S.~B. and Budd, W.~F.: Flow parameters for ice sheet modeling, Cold Reg. Sci. Technol., 6, 175--177, 1982.


\bibitem[{Patterson and Kelso(2015)}]{Patterson.Kelso.2015}
Patterson,~T. and Kelso,~N.~V.: {N}atural {E}arth, {F}ree vector and raster
map data, available at: \url{http://naturalearthdata.com} (last access:
2015), 2015.


\bibitem[{Perkins and Brennand(2014)}]{Perkins.Brennand.2014}
Perkins,~A.~J. and Brennand,~T.~A.: Refining the pattern and style of Cordilleran Ice Sheet retreat: palaeogeography, evolution and implications of lateglacial ice-dammed lake systems on the southern Fraser Plateau, British Columbia, Canada, Boreas, 44, 319--342,
doi:\href{http://dx.doi.org/10.1111/bor.12100}{10.1111/bor.12100}, 2014.


\bibitem[{Perkins et~al.(2013)Perkins, Brennand, and Burke}]{Perkins.etal.2013}
Perkins,~A.~J., Brennand,~T.~A., and Burke,~M.~J.: Genesis of an esker-like ridge over the southern Fraser Plateau, British Columbia: implications for paleo-ice sheet reconstruction based on geomorphic inversion, Geomorphology, 190, 27--39,
doi:\href{http://dx.doi.org/10.1016/j.geomorph.2013.02.005}{10.1016/j.geomorph.2013.02.005}, 2013.


\bibitem[{Petit et~al.(1999)Petit, Jouzel, Raynaud, Barkov, Barnola, Basile, Bender, Chappellaz, Davis, Delaygue, Delmotte, Kotlyakov, Legrand, Lipenkov, Lorius, P{\'e}pin, Ritz, Saltzman, and Stievenard}]{Petit.etal.1999}
Petit,~J.~R., Jouzel,~J., Raynaud,~D., Barkov,~N.~I., Barnola,~J.-M., Basile,~I., Bender,~M., Chappellaz,~J., Davis,~M., Delaygue,~G., Delmotte,~M., Kotlyakov,~V.~M., Legrand,~M., Lipenkov,~V.~Y., Lorius,~C., P{\'e}pin,~L., Ritz,~C., Saltzman,~E., and Stievenard,~M.: Climate and atmospheric history of the past 420,000 years from the Vostok ice core, Antarctica, Nature, 399, 429--436,
doi:\href{http://dx.doi.org/10.1038/20859}{10.1038/20859}, data archived at the World Data Center for Paleoclimatology, Boulder, Colorado, USA., 1999.


\bibitem[{Porter(1989)}]{Porter.1989}
Porter,~S.~C.: Some geological implications of average Quaternary glacial conditions, Quaternary Res., 32, 245--261,
doi:\href{http://dx.doi.org/10.1016/0033-5894(89)90092-6}{10.1016/0033-5894(89)90092-6}, 1989.


\bibitem[{Porter and Swanson(1998)}]{Porter.Swanson.1998}
Porter,~S.~C. and Swanson,~T.~W.: Radiocarbon age constraints on rates of advance and retreat of the {P}uget {L}obe of the {C}ordilleran {I}ce {S}heet during the last glaciation, Quaternary Res., 50, 205--213,
doi:\href{http://dx.doi.org/10.1006/qres.1998.2004}{10.1006/qres.1998.2004}, 1998.


\bibitem[{Praetorius and Mix(2014)}]{Praetorius.Mix.2014}
Praetorius,~S.~K. and Mix,~A.~C.: Synchronization of North Pacific and Greenland climates preceded abrupt deglacial warming, Sience, 345, 444--448,
doi:\href{http://dx.doi.org/10.1126/science.1252000}{10.1126/science.1252000}, 2014.


\bibitem[{Prest et~al.(1968)Prest, Grant, and Rampton}]{Prest.etal.1968}
Prest,~V.~K., Grant,~D.~R., and Rampton,~V.~N.: Glacial map of Canada, ``A'' Series Map 1253A, Geol. Surv. of Can., Ottawa, ON,
doi:\href{http://dx.doi.org/10.4095/108979}{10.4095/108979}, 1968.


\bibitem[{Reasoner et~al.(1994)Reasoner, Osborn, and Rutter}]{Reasoner.etal.1994}
Reasoner, M.~A., Osborn, G., and Rutter, N.~W.: Age of the Crowfoot advance in the Canadian Rocky Mountains: A glacial event coeval with the Younger Dryas oscillation, Geology, 22, 439--442,
doi:\href{http://dx.doi.org/10.1130/0091-7613(1994)022<0439:AOTCAI>2.3.CO;2}{10.1130/0091-7613(1994)022<0439:AOTCAI>2.3.CO;2}, 1994.


\bibitem[{Rignot et~al.(2013)Rignot, Mouginot, Larsen, Gim, and Kirchner}]{Rignot.etal.2013}
Rignot, E., Mouginot, J., Larsen, C.~F., Gim, Y., and Kirchner, D.: Low-frequency radar sounding of temperate ice masses in Southern Alaska, Geophys. Res. Lett., 40, 5399--5405,
doi:\href{http://dx.doi.org/10.1002/2013gl057452}{10.1002/2013gl057452}, 2013.


\bibitem[{Robert(1991)}]{Robert.1991}
Robert,~B.~L.: Modeling the Cordilleran ice sheet, G{e}ogr. Phys. Quatern., 45, 287--299,
doi:\href{http://dx.doi.org/10.7202/032876ar}{10.7202/032876ar}, 1991.


\bibitem[{Rutter et~al.(2012)Rutter, Coronato, Helmens, Rabassa, and Z{\'a}rate}]{Rutter.etal.2012}
Rutter,~N., Coronato,~A., Helmens,~K., Rabassa,~J., and Z{\'a}rate,~M.: Glaciations in North and South America from the Miocene to the Last Glacial Maximum: Comparisons, Linkages and Uncertainties, Springer Briefs in Earth System Sciences, Springer, Dordrecht,
doi:\href{http://dx.doi.org/10.1007/978-94-007-4399-1}{10.1007/978-94-007-4399-1}, 2012.


\bibitem[{Ryder et~al.(1991)Ryder, Fulton, and Clague}]{Ryder.etal.1991}
Ryder,~J.~M., Fulton,~R.~J., and Clague,~J.~J.: The Cordilleran ice sheet and the glacial geomorphology of southern and central British Colombia, G{e}ogr. Phys. Quatern., 45, 365--377,
doi:\href{http://dx.doi.org/10.7202/032882ar}{10.7202/032882ar}, 1991.


\bibitem[{Seguinot(2013)}]{Seguinot.2013}
Seguinot,~J.: Spatial and seasonal effects of temperature variability in a positive degree-day glacier surface mass-balance model,~J. Glaciol., 59, 1202--1204,
doi:\href{http://dx.doi.org/10.3189/2013JoG13J081}{10.3189/2013JoG13J081}, 2013.


\bibitem[{Seguinot et~al.(2014)Seguinot, Khroulev, Rogozhina, Stroeven, and Zhang}]{Seguinot.etal.2014}
 Seguinot,~J., Khroulev,~C., Rogozhina,~I., Stroeven,~A.~P., and Zhang,~Q.: The effect of climate forcing on numerical simulations of the Cordilleran ice sheet at the Last Glacial Maximum, The Cryosphere, 8, 1087--1103,
doi:\href{http://dx.doi.org/10.5194/tc-8-1087-2014}{10.5194/tc-8-1087-2014}, 2014.


\bibitem[{Shea et~al.(2009)Shea, Moore, and Stahl}]{Shea.etal.2009}
Shea,~J.~M., Moore,~R.~D., and Stahl,~K.: Derivation of melt factors from glacier mass-balance records in western Canada,~J. Glaciol., 55, 123--130,
doi:\href{http://dx.doi.org/10.3189/002214309788608886}{10.3189/002214309788608886}, 2009.


\bibitem[{Sissons(1979)}]{Sissons.1979}
Sissons,~J.~B.: The Loch Lomond Stadial in the British Isles, Nature, 280, 199--203,
doi:\href{http://dx.doi.org/10.1038/280199a0}{10.1038/280199a0}, 1979.


\bibitem[{Stea et~al.(2011)Stea, Seaman, Pronk, Parkhill, Allard, and Utting}]{Stea.etal.2011}
Stea,~R.~R., Seaman,~A.~A., Pronk,~T., Parkhill,~M.~A., Allard,~S., and Utting,~D.: The Appalachian Glacier Complex in Maritime Canada, in:   Dev. Quaternary Sci., vol.~15, edited by: Ehlers,~J., Gibbard,~P.~L., and Hughes,~P.~D., Elsevier, Amsterdam, 631--659,
doi:\href{http://dx.doi.org/10.1016/b978-0-444-53447-7.00048-9}{10.1016/b978-0-444-53447-7.00048-9}, 2011.


\bibitem[{Stroeven et~al.(2010)Stroeven, Fabel, Codilean, Kleman, Clague, Miguens-Rodriguez, and Xu}]{Stroeven.etal.2010}
Stroeven,~A.~P., Fabel,~D., Codilean,~A.~T., Kleman,~J., Clague,~J.~J., Miguens-Rodriguez,~M., and Xu,~S.: Investigating the glacial history of the northern sector of the Cordilleran ice sheet with cosmogenic $^{10}$Be concentrations in quartz, Quaternary Sci. Rev., 29, 3630--3643,
doi:\href{http://dx.doi.org/10.1016/j.quascirev.2010.07.010}{10.1016/j.quascirev.2010.07.010}, 2010.


\bibitem[{Stroeven et~al.(2014)Stroeven, Fabel, Margold, Clague, and Xu}]{Stroeven.etal.2014}
Stroeven,~A.~P., Fabel,~D., Margold,~M., Clague,~J.~J., and Xu,~S.: Investigating absolute chronologies of glacial advances in the {NW} sector of the Cordilleran ice sheet with terrestrial in situ cosmogenic nuclides, Quaternary Sci. Rev., 92, 429--443,
doi:\href{http://dx.doi.org/10.1016/j.quascirev.2013.09.026}{10.1016/j.quascirev.2013.09.026}, 2014.


\bibitem[{Stroeven et~al.(2015)Stroeven, H{\"a}ttestrand, Kleman, Heyman, Fabel, Fredin, Goodfellow, Harbor, Jansen, Olsen, Caffee, Fink, Lundqvist, Rosqvist, Str\"omberg, and Jansson}]{Stroeven.etal.inreview}
Stroeven,~A.~P., H{\"a}ttestrand,~C., Kleman,~J., Heyman,~J., Fabel,~D.,
Fredin,~O., Goodfellow,~B.~W., Harbor,~J.~M., Jansen,~J.~D., Olsen,~L.,
Caffee,~M.~W., Fink,~D., Lundqvist,~J., Rosqvist,~G.~C., Str\"omberg,~B., and
Jansson,~K.~N.: Deglaciation of Fennoscandia, Quaternary Sci. Rev., in
review, 2015.


\bibitem[{Stumpf et~al.(2000)Stumpf, Broster, and Levson}]{Stumpf.etal.2000}
Stumpf,~A.~J., Broster,~B.~E., and Levson,~V.~M.: Multiphase flow of the late Wisconsinan Cordilleran ice sheet in western Canada, Geol. Soc. Am. Bull., 112, 1850--1863,
doi:\href{http://dx.doi.org/10.1130/0016-7606(2000)112<1850:mfotlw>2.0.co;2}{10.1130/0016-7606(2000)112\textless1850:mfotlw\textgreater2.0.co;2}, 2000.


\bibitem[{Taylor et~al.(2014)Taylor, Hendy, and Pak}]{Taylor.etal.2014}
Taylor,~M., Hendy,~I., and Pak,~D.: Deglacial ocean warming and marine margin retreat of the Cordilleran Ice Sheet in the North Pacific Ocean, Earth Planet. Sc. Lett., 403, 89--98,
doi:\href{http://dx.doi.org/10.1016/j.epsl.2014.06.026}{10.1016/j.epsl.2014.06.026}, 2014.


\bibitem[{Taylor et~al.(2015)Taylor, Hendy, and Pak}]{Taylor.etal.2015}
Taylor, M.~A., Hendy, I.~L., and Pak, D.~K.: The California Current System as a transmitter of millennial scale climate change on the northeastern Pacific margin from 10 to 50 ka, Paleoceanography, 30, 1168--1182,
doi:\href{http://dx.doi.org/10.1002/2014pa002738}{10.1002/2014pa002738}, 2015.


\bibitem[{the PISM~authors(2015)}]{PISM-authors.2015}
the PISM~authors: {PISM}, a~{P}arallel {I}ce {S}heet {M}odel, available at:
\url{http://www.pism-docs.org} (last access: 2015), 2015.


\bibitem[{Troost(2014)}]{Troost.2014}
Troost,~K.~G.: The penultimate glaciation and mid-to late-pleistocene
stratigraphy in the Central Puget Lowland, Washington, in: 2014 GSA Annual
Meeting, Vancouver, 19--22 October 2014, abstract no. 138-9, 2014.


\bibitem[{Tulaczyk et~al.(2000)Tulaczyk, Kamb, and Engelhardt}]{Tulaczyk.etal.2000}
Tulaczyk,~S., Kamb,~W.~B., and Engelhardt,~H.~F.: Basal mechanics of Ice Stream B, west Antarctica: 1. Till mechanics,~J. Geophys. Res., 105, 463,
doi:\href{http://dx.doi.org/10.1029/1999jb900329}{10.1029/1999jb900329}, 2000.


\bibitem[{Turner et~al.(2013)Turner, Ward, Bond, Jensen, Froese, Telka, Zazula, and Bigelow}]{Turner.etal.2013}
Turner,~D.~G., Ward,~B.~C., Bond,~J.~D., Jensen,~B.~J., Froese,~D.~G., Telka,~A.~M., Zazula,~G.~D., and Bigelow,~N.~H.: Middle to Late Pleistocene ice extents, tephrochronology and paleoenvironments of the White River area, southwest Yukon, Quaternary Res., 75, 59--77,
doi:\href{http://dx.doi.org/10.1016/j.quascirev.2013.05.011}{10.1016/j.quascirev.2013.05.011}, 2013.


\bibitem[{Ward et~al.(2007)Ward, Bond, and Gosse}]{Ward.etal.2007}
Ward,~B.~C., Bond,~J.~D., and Gosse,~J.~C.: Evidence for a 55--50 ka (early Wisconsin) glaciation of the Cordilleran ice sheet, Yukon Territory, Canada, Quaternary Res., 68, 141--150,
doi:\href{http://dx.doi.org/10.1016/j.yqres.2007.04.002}{10.1016/j.yqres.2007.04.002}, 2007.


\bibitem[{Ward et~al.(2008)Ward, Bond, Froese, and Jensen}]{Ward.etal.2008}
Ward,~B.~C., Bond,~J.~D., Froese,~D., and Jensen,~B.: Old Crow tephra ($140\pm10$\,ka) constrains penultimate Reid glaciation in central Yukon Territory, Quaternary Res., 27, 1909--1915,
doi:\href{http://dx.doi.org/10.1016/j.quascirev.2008.07.012}{10.1016/j.quascirev.2008.07.012}, 2008.


\bibitem[{Winkelmann et~al.(2011)Winkelmann, Martin, Haseloff, Albrecht, Bueler, Khroulev, and Levermann}]{Winkelmann.etal.2011}
 Winkelmann,~R., Martin,~M.~A., Haseloff,~M., Albrecht,~T., Bueler,~E., Khroulev,~C., and Levermann,~A.: The Potsdam Parallel Ice Sheet Model (PISM-PIK) -- Part 1: Model description, The Cryosphere, 5, 715--726,
doi:\href{http://dx.doi.org/10.5194/tc-5-715-2011}{10.5194/tc-5-715-2011}, 2011.


\end{thebibliography}


\clearpage{}  % new page for tables


\begin{table*}
\caption{%
      Default parameter values used in the ice sheet model.}
\label{tab:params}
\scalebox{.85}[.85]
{\begin{tabular}{llrll}
    \tophline
    Not.    & Name & Value & Unit & Source \\
    \middlehline
    \multicolumn{2}{l}{{Ice rheology}} \\
    \cline{1-5}

    $\rho$  & Ice density
            & 910
            & \unit{kg\,m^{-3}}
            & \citet{Aschwanden.etal.2012} \\

    $g$     & Standard gravity
            & 9.81
            & \unit{m\,s^{-2}}
            & \citet{Aschwanden.etal.2012} \\

    $n$     & Glen exponent
            & 3
            & --
            & \citet{Cuffey.Paterson.2010} \\

    $A_{\mathrm{c}}$   & Ice hardness coefficient cold$^1$
            & $3.61\times10^{-13}$
            & \unit{Pa^{-3}\,s^{-1}}
            & \citet{Paterson.Budd.1982} \\

    $A_{\mathrm{w}}$   & Ice hardness coefficient warm$^1$
            & $1.73\times10^3$
            & \unit{Pa^{-3}\,s^{-1}}
            & \citet{Paterson.Budd.1982} \\

    $Q_{\mathrm{c}}$   & Flow law activation energy cold$^1$
            & $6.0\times10^4$
            & \unit{J\,mol^{-1}}
            & \citet{Paterson.Budd.1982} \\

    $Q_{\mathrm{w}}$   & Flow law activation energy warm$^1$
            & $13.9\times10^4$
            & \unit{J\,mol^{-1}}
            & \citet{Paterson.Budd.1982} \\

    $T_{\mathrm{c}}$   & Flow law critical temperature
            & 263.15
            & \unit{K}
            & \citet{Paterson.Budd.1982} \\

    $f$     & Flow law water fraction coeff.
            & 181.25
            & --
            & \citet{Lliboutry.Duval.1985} \\

    $R$     & Ideal gas constant
            & 8.31441
            & \unit{J\,mol^{-1}\,K^{-1}}
            & -- \\

    $\beta$ & Clapeyron constant
            & $7.9\times10^{-8}$
            & \unit{K\,Pa^{-1}}
            & \citet{Luthi.etal.2002} \\

    $c_{\mathrm{i}}$   & Ice specific heat capacity
            & 2009
            & \unit{J\,kg^{-1}\,K^{-1}}
            & \citet{Aschwanden.etal.2012} \\

    $c_{\mathrm{w}}$   & Water specific heat capacity
            & 4170
            & \unit{J\,kg^{-1}\,K^{-1}}
            & \citet{Aschwanden.etal.2012} \\

    $k$     & Ice thermal conductivity
            & 2.10
            & \unit{J\,m^{-1}\,K^{-1}\,s^{-1}}
            & \citet{Aschwanden.etal.2012} \\

    $L$     & Water latent heat of fusion
            & $3.34\times10^5$
            & \unit{J\,kg^{-1}\,K^{-1}}
            & \citet{Aschwanden.etal.2012} \\

    \cline{1-5}
    \multicolumn{2}{l}{{Basal sliding}} \\
    \cline{1-5}

    $q$     & Pseudo-plastic sliding exponent
            & 0.25
            & --
            & \citet{Aschwanden.etal.2013} \\

    $v_{\text{th}}$& Pseudo-plastic threshold velocity
            & 100.0
            & \unit{m\,yr^{-1}}
            & \citet{Aschwanden.etal.2013} \\

    $c_0$   & Till cohesion
            & 0.0
            & Pa
            & \citet{Tulaczyk.etal.2000} \\
    $e_0$   & Till reference void ratio
            & 0.69
            & --
            & \citet{Tulaczyk.etal.2000} \\

    $C_{\mathrm{c}}$   & Till compressibility coefficient
            & 0.12
            & --
            & \citet{Tulaczyk.etal.2000} \\

    $\delta$& Minimum effective pressure ratio$^1$
            & 0.02
            & --
            & \citet{Bueler.Pelt.2015} \\

    $W_{\text{max}}$ & Maximal till water thickness$^1$
            & 2.0
            & m
            & \citet{Bueler.Pelt.2015} \\

    $b_0$   & Altitude of max. friction angle
            & 0
            & m
            & -- \\

    $b_1$   & Altitude of min. friction angle
            & 200
            & m
            & \citet{Clague.1981} \\

    $\phi_0$& Minimum friction angle
            & 15
            & \degree
            & -- \\

    $\phi_1$& Maximum friction angle
            & 45
            & \degree
            & -- \\

    \cline{1-5}
    \multicolumn{2}{l}{{Bedrock and lithosphere}} \\
    \cline{1-5}

    $q_{\mathrm{G}}$   & Geothermal heat flux
            & 70.0
            & \unit{mW\,m^{-2}}
            & -- \\

    $\rho_{\mathrm{b}}$& Bedrock density
            & 3300
            & \unit{kg\,m^{-3}}
            & -- \\

    $c_{\mathrm{b}}$   & Bedrock specific heat capacity
            & 1000
            & \unit{J\,kg^{-1}\,K^{-1}}
            & -- \\

    $k_{\mathrm{b}}$   & Bedrock thermal conductivity
            & 3.0
            & \unit{J\,m^{-1}\,K^{-1}\,s^{-1}}
            & -- \\

    $\nu_{\mathrm{m}}$ & Astenosphere viscosity
            & $1\times10^{19}$
            & \unit{Pa\,s}
            & \citet{James.etal.2009} \\

    $\rho_{\mathrm{l}}$& Lithosphere density
            & 3300
            & \unit{kg\,m^{-3}}
            & \citet{Lingle.Clark.1985} \\

    $D$     & Lithosphere flexural rigidity
            & $5.0\times10^{24}$
            & \unit{N}
            & \citet{Lingle.Clark.1985} \\

    \cline{1-5}
    \multicolumn{2}{l}{{Surface and atmosphere}} \\
    \cline{1-5}

    $T_{\mathrm{s}}$   & Temperature of snow precipitation
            & 273.15
            & \unit{K}
            & -- \\

    $T_{\mathrm{r}}$   & Temperature of rain precipitation
            & 275.15
            & \unit{K}
            & -- \\

    $F_{\mathrm{s}}$   & Degree-day factor for snow
            & $3.04\times10^{-3}$
            & \unit{m\,K^{-1}\,day^{-1}}
            & \citet{Shea.etal.2009} \\

    $F_{\mathrm{i}}$   & Degree-day factor for ice
            & $4.59\times10^{-3}$
            & \unit{m\,K^{-1}\,day^{-1}}
            & \citet{Shea.etal.2009} \\

    $\gamma$& Air temperature lapse rate
            & $6\times10^{-3}$
            & \unit{K\,m{-1}}
            & -- \\

    \bottomhline
\end{tabular}}
\belowtable{%
      $^1$Default value. Alternative values used in sensitivity tests are
      given in Table~\ref{tab:sens_params}.}
\end{table*}


\begin{table*}
\caption{%
      Palaeo-temperature proxy records and scaling parameters yielding
      temperature offset time-series used to force the ice sheet model
      through the last glacial cycle (Fig.~\ref{fig:lr_ts}). $f$ corresponds
      to the scaling factor adopted to yield last glacial maximum ice limits
      in the vicinity of mapped end moraines, and
      $[{\Delta}T_{\text{TS}}]_{32}^{22}$ refers to the resulting mean
      temperature anomaly during the period 32 to~22\,\unit{ka} after
      scaling.}
\label{tab:records}
%\scalebox{.850}[.850]
{\begin{tabular}{l|ccc|ccc|l}
    \tophline
    Record & Latitude & Longitude & Elev.
           & Proxy & $f$ & $[{\Delta}T_{\text{TS}}]_{32}^{22}$
           & Reference\\
    & & & (\unit{m\,a.s.l.}) & & & (K) & \\
    \middlehline
    GRIP     &  72{\degree}35$^{\prime}$\,N   % 72.58 (decimal)
             &  37{\degree}38$^{\prime}$\,W   % 37.64 (decimal)
             & 3238
             & \chem{\delta^{18}O}
             & 0.38 & $-$6.2  % -16.4126 (before scaling)
             & \citet{Dansgaard.etal.1993} \\

    NGRIP    &  75{\degree}06$^{\prime}$\,N   % 75.10
             &  42{\degree}19$^{\prime}$\,W   % 42.32
             & 2917
             & \chem{\delta^{18}O}
             & 0.25 & $-$6.6  % -26.7098
             & \citet{Andersen.etal.2004} \\

    EPICA    &  75{\degree}06$^{\prime}$\,S   % 75.1
             & 123{\degree}21$^{\prime}$\,E   % 123.35
             & 3233
             & \chem{\delta^{18}O}
             & 0.64 & $-$5.9  % -9.2055
             & \citet{Jouzel.etal.2007} \\

    Vostok   &  78{\degree}28$^{\prime}$\,S   % 78.8
             & 106{\degree}50$^{\prime}$\,E   % 106.8
             & 3488
             & \chem{\delta^{18}O}
             & 0.75 & $-$6.0  % -7.9550
             & \citet{Petit.etal.1999} \\

    ODP~1012 &  32{\degree}17$^{\prime}$\,N
             & 118{\degree}23$^{\prime}$\,W
             & $-$1772
             & \chem{U^{K'}_{37}}
             & 1.61 & $-$6.1  % -3.7889
             & \citet{Herbert.etal.2001} \\

    ODP~1020 &  41{\degree}00$^{\prime}$\,N
             & 126{\degree}26$^{\prime}$\,W
             & $-$3038
             & \chem{U^{K'}_{37}}
             & 1.18 & $-$5.9  % -5.0000
             & \citet{Herbert.etal.2001} \\
    \bottomhline
\end{tabular}}
\end{table*}


\begin{table*}
\caption{%
      Parameter values used in the sensitivity test.}
\label{tab:sens_params}
%\centering\makebox[\textwidth]
{\begin{tabular}{l|ccccc|cc|cc}
    \tophline
            & \multicolumn{5}{c|}{Rheology}
            & \multicolumn{2}{c|}{Sliding}
            & \multicolumn{2}{c}{GRIP scaling} \\
    Config. & $A_{\mathrm{c}}$ & $A_{\mathrm{w}}$
            & $Q_{\mathrm{c}}$ & $Q_{\mathrm{w}}$
            & $E_{\text{SIA}}$ & $\delta$ & $W_{\text{max}}$
            & $f$ & $T_{[32, 22]}$ \\
            & \multicolumn{2}{c}{(\unit{Pa^{-3}\,s^{-1}})}
            & \multicolumn{2}{c}{(\unit{J\,mol^{-1}})}
            & & & (\unit{m}) \\
    \middlehline
    Default$^1$  & $ 3.61\times 10^{-13}$
                 & $ 1.73\times 10^3$
                 & $   60\times 10^3$
                 & $  139\times 10^3$
                 & 1 & 0.02 & 2 & 0.38 & 6.2 \\
    \cline{1-10}
    Soft ice$^2$ & $2.847\times 10^{-13}$
                 & $2.356\times 10^{-2}$
                 & $   60\times 10^3$
                 & $  115\times 10^3$
                 & 5 & 0.02 & 2 & 0.40 & 6.6 \\
    Hard ice$^2$ & $2.847\times 10^{-13}$
                 & $2.356\times 10^{-2}$
                 & $   60\times 10^3$
                 & $  115\times 10^3$
                 & 1 & 0.02 & 2 & 0.37 & 6.0 \\
    \cline{1-10}
    Soft bed     & $ 3.61\times 10^{-13}$
                 & $ 1.73\times 10^3$
                 & $   60\times 10^3$
                 & $  139\times 10^3$
                 & 1 & 0.01 & 1 & 0.40 & 6.5 \\
    Hard bed     & $ 3.61\times 10^{-13}$
                 & $ 1.73\times 10^3$
                 & $   60\times 10^3$
                 & $  139\times 10^3$
                 & 1 & 0.05 & 5 & 0.36 & 5.9 \\
    \bottomhline
\end{tabular}}
\belowtable{%
      After $^1$\citet{Paterson.Budd.1982,Bueler.Pelt.2015};
      and $^2$\citet{Cuffey.Paterson.2010}.}
\end{table*}


\clearpage{}  % avoid three tables on one page


\begin{table*}
\caption{%
      Extremes in Cordilleran ice sheet volume and extent corresponding to
      MIS~4, 3 and 2 for each of the six low-resolution simulations
      (Fig.~\ref{fig:lr_ts}).}
\label{tab:extrema}
%\scalebox{.50}[.50]
{\begin{tabular}{l|ccc|ccc|ccc}
    \tophline
             & \multicolumn{3}{c|}{Age (ka)}
             & \multicolumn{3}{c|}{Ice extent (\unit{10^6\,km^2})}
             & \multicolumn{3}{c}{Ice volume (m\,s.l.e.)} \\
    Record   &  MIS~4 &  MIS~3 &  MIS~2
             &  MIS~4 &  MIS~3 &  MIS~2
             &  MIS~4 &  MIS~3 &  MIS~2 \\
    \middlehline
    GRIP     &  57.59 &  42.91 &  19.14
             &   1.93 &   0.67 &   2.09
             &   7.43 &   1.54 &   8.62 \\
    NGRIP    &  60.27 &  45.87 &  22.85
             &   2.13 &   0.73 &   2.11
             &   8.71 &   1.70 &   8.60 \\
    EPICA    &  61.90 &  52.40 &  17.36
             &   1.48 &   0.98 &   2.08
             &   4.84 &   2.55 &   8.56 \\
    Vostok   &  62.17 &  55.95 &  16.83
             &   1.52 &   1.04 &   2.17
             &   5.04 &   2.94 &   9.07 \\
    ODP 1012 &  56.91 &  47.69 &  23.10
             &   1.34 &   0.87 &   2.06
             &   4.10 &   2.20 &   8.40 \\
    ODP 1020 &  60.72 &  52.88 &  20.99
             &   1.21 &   0.69 &   2.06
             &   3.54 &   1.54 &   8.39 \\
    \cline{1-10}
    Minimum  &  56.91 &  42.91 &  16.83
             &   1.21 &   0.67 &   2.06
             &   3.54 &   1.54 &   8.39 \\
    Maximum  &  62.17 &  55.95 &  23.10
             &   2.13 &   1.04 &   2.17
             &   8.71 &   2.94 &   9.07 \\
    \bottomhline
\end{tabular}}
\end{table*}


\begin{table*}[p]
\caption{%
      Extremes in Cordilleran ice sheet volume and extent corresponding to
      MIS~4, 3 and 2 using the GRIP paleo-climate forcing with each
      parameter configuration (Fig.~3). Relative differences (R. diff.) give
      rough error estimates related to varying selected ice rheology and
      basal sliding parameters (Table~\ref{tab:sens_params}).}
\label{tab:sens_extrema}
%\centering\makebox[\textwidth]
{\begin{tabular}{l*{3}{|ccc}}
    \tophline
             & \multicolumn{3}{c|}{Age (ka)}
             & \multicolumn{3}{c|}{Ice extent (\unit{10^6\,km^2})}
             & \multicolumn{3}{c}{Ice volume (m\,s.l.e.)} \\
    Config.  &  MIS~4 &  MIS~3 &  MIS~2
             &  MIS~4 &  MIS~3 &  MIS~2
             &  MIS~4 &  MIS~3 &  MIS~2 \\
    \middlehline
    Default  &  57.59 &  42.91 &  19.14
             &   1.93 &   0.67 &   2.09
             &   7.43 &   1.54 &   8.62 \\
    \cline{1-10}
    Soft ice &  58.89 &  49.97 &  21.57
             &   1.96 &   0.54 &   2.08
             &   6.58 &   1.03 &   6.88 \\
    Hard ice &  57.32 &  42.90 &  19.14
             &   1.90 &   0.75 &   2.12
             &   7.83 &   1.91 &   9.46 \\
    R. diff. &    3\,\unit{\%} &   16\,\unit{\%} &   13\,\unit{\%}
             &    3\,\unit{\%} &   31\,\unit{\%} &    2\,\unit{\%}
             &   17\,\unit{\%} &   57\,\unit{\%} &   30\,\unit{\%} \\
    \cline{1-10}
    Soft bed &  58.90 &  49.21 &  19.53
             &   1.88 &   0.55 &   2.05
             &   6.46 &   1.03 &   7.52 \\
    Hard bed &  57.31 &  42.91 &  19.14
             &   1.93 &   0.96 &   2.13
             &   7.99 &   2.89 &   9.31 \\
    R. diff. &    3\,\unit{\%} &   15\,\unit{\%} &    2\,\unit{\%}
             &    3\,\unit{\%} &   62\,\unit{\%} &    4\,\unit{\%}
             &   21\,\unit{\%} &  120\,\unit{\%} &   21\,\unit{\%} \\
    \bottomhline
\end{tabular}}
\end{table*}


\clearpage{}  % new page for figures and avoid too many unprocessed floats


\begin{figure*}%
\includegraphics{locmap}
\caption{%
      Relief map of the northern American Cordillera showing cumulative last
      glacial maximum ice cover between 21.4 and
      16.8\,\unit{\chem{^{14}C}\,cal\,ka} \citep[red line]{Dyke.2004}, and
      the modelling domain used in this study (black rectangle). The
      background map consists of ETOPO1 \citep{Amante.Eakins.2009} and
      Natural Earth Data \citep{Patterson.Kelso.2015}.}
\label{fig:locmap}%
\end{figure*}%


\begin{figure}%
\includegraphics{sens_plot_rheo}
\caption{%
      Ice softness parameter, $A$, as a function of pressure-adjusted
      temperature, $T_{\text{pa}}$, for the default
      \citep{Paterson.Budd.1982}, hard ice \citep[with
      $E_{\text{SIA}}=1$]{Cuffey.Paterson.2010}, and soft ice \citep[with
      $E_{\text{SIA}}=5$]{Cuffey.Paterson.2010} rheologies, using a linear
      scale (top panel) and logarithmic scale (bottom panel). Figure made
      using Eqn.~\ref{eqn:softness} with parameters from
      Table~\ref{tab:sens_params}.}
\label{fig:sens_plot_rheo}%
\end{figure}%


\begin{figure}%
\includegraphics{sens_plot_ntil}
\caption{%
      Effective pressure, $N$, as a function of water content in the till,
      $W$, for the default ($\delta=0.02$, $W_{\text{max}}=2$\,m), hard bed
      ($\delta=0.05$, $W_{\text{max}}=5$\,m), and soft bed ($\delta=0.01$,
      $W_{\text{max}}=1$\,m) sliding parametrisations, using a linear scale
      (top panel) and a logarithmic scale (bottom panel). Calculations are
      made for an ice thickness, $h$, of 1000\,m. Figure made using
      Eqn.~\ref{eqn:ntil} with parameters from Table~\ref{tab:sens_params}
      after \citet[Fig.~1]{Bueler.Pelt.2015}.}
\label{fig:sens_plot_ntil}%
\end{figure}%


\begin{figure}%
\includegraphics{atm}
\caption{%
      Monthly mean near-surface air temperature, precipitation, and standard
      deviation of daily mean temperature (PDD SD) for January and July from
      the North American Regional Reanalysis
      \citep[NARR;][]{Mesinger.etal.2006}, used to force the surface mass
      balance (PDD) component of the ice sheet model. Note the strong
      contrasts in seasonality, timing of the precipitation peak, and
      temperature variability over the model domain, notably between coastal
      and inland regions.}
\label{fig:atm}%
\end{figure}%


\begin{figure*}%
\includegraphics{lr_ts}
\caption{%
      Temperature offset time-series from ice core and ocean records
      (Table~\ref{tab:records}) used as palaeo-climate forcing for the ice
      sheet model (top panel), and modelled ice volume (bottom panel)
      through the last 120\,\unit{ka}. Ice volumes are expressed in meters
      of sea level equivalent (\unit{m\,s.l.e.}). Gray fields indicate Marine
      Oxygen Isotope Stage (MIS) boundaries for MIS~2 and MIS~4 according to
      a~global compilation of benthic \chem{\delta^{18}O} records
      \citep{Lisiecki.Raymo.2005}. Hatched rectangles highlight the
      time-volume span for ice volume extremes corresponding to MIS~4
      (61.9--56.5\,\unit{ka}), MIS~3 (53.0--41.3\,\unit{ka}), and MIS~2
      (LGM, 23.2--16.8\,\unit{ka}). Dotted lines correspond to GRIP- and
      EPICA-driven 5\,\unit{km}-resolution runs.}
\label{fig:lr_ts}%
\end{figure*}%


\begin{figure*}%
\includegraphics{lr_maps}
\caption{%
      Snapshots of modelled surface topography (500\,\unit{m} contours)
      corresponding to the ice volume extremes indicated on
      Fig.~\ref{fig:lr_ts}. An ice cap persists over the Skeena Mountains
      (SM) during MIS~3. Note the occurence of spatial similarities despite
      large differences in timing.}
\label{fig:lr_maps}%
\end{figure*}%


\begin{figure*}
\includegraphics{sens_ts}
\caption{%
      Modelled sea-level relevant ice volume through the last 120\,ka in the
      simulation forced by the GRIP paleo-climate record, using default
      parameters (black curves), different ice rheology parameters (top
      panel), and different basal sliding parameters (bottom panel). Gray
      fields indicate Marine Oxygen Isotope Stage (MIS) boundaries for MIS~2
      and MIS~4 according to a global compilation of benthic
      \chem{\delta^{18}O} records \citep{Lisiecki.Raymo.2005}.}
\label{fig:sens_ts}
\end{figure*}


\begin{figure*}%
\includegraphics{hr_maps_mis2}
\caption{%
      Modelled surface topography (200\,\unit{m} contours) and surface
      velocity (colour mapping) corresponding to the maximum ice volume
      during MIS~2 in the GRIP and EPICA high-resolution simulations.}
\label{fig:hr_maps_mis2}%
\end{figure*}%


\begin{figure*}%
\includegraphics{hr_maps_mis4}
\caption{%
      Modelled surface topography (200\,\unit{m} contours) and surface
      velocity (colour mapping) corresponding to the maximum ice volume
      during MIS~4 in the GRIP and EPICA high-resolution simulations.}
\label{fig:hr_maps_mis4}%
\end{figure*}%


\begin{figure*}%
\includegraphics{hr_geom_duration}
\caption{%
      Modelled duration of ice cover during the last 120\,\unit{ka} using
      GRIP and EPICA climate forcing. Note the irregular colour scale.
      A~continuous ice cover spanning from the Alaska Range (AR) to the
      Coast Mountains (CM) and the Columbia and Rocky mountains (CRM) exists
      for about 32\,\unit{ka} in the GRIP simulation and 26\,\unit{ka} in
      the EPICA simulation. The maximum extent of the ice sheet generally
      corresponds to relatively short durations of ice cover, but ice cover
      persists over the Skeena Mountains (SM) during most of the simulation.
      See Fig.~\ref{fig:locmap} for a~list of abbreviations.}
\label{fig:hr_geom_duration}%
\end{figure*}%


\begin{figure*}%
\includegraphics{hr_geom_warmbase}
\caption{%
      Modelled duration of warm-based ice cover during the last
      120\,\unit{ka}. Long ice cover durations combined with basal
      temperatures at the pressure melting point may explain the strong
      glacial erosional imprint of the Skeena Mountains (SM) landscape.
      Hatches indicate areas that were covered by cold ice only.}
\label{fig:hr_geom_warmbase}%
\end{figure*}%


\begin{figure*}%
\includegraphics{hr_geom_warmfrac}
\caption{%
      Modelled fraction of warm-based ice cover during the ice-covered
      period. Note the dominance of warm-based conditions on the continental
      shelf and major glacial troughs of the coastal ranges. Hatches
      indicate areas that were covered by cold ice only.}
\label{fig:hr_geom_warmfrac}%
\end{figure*}%


\begin{figure}%
\includegraphics{hr_ts_deglac}
\caption{%
      Temperature offset time-series from the GRIP and EPICA ice core
      records (Table~\ref{tab:records}) (top panel), and modelled ice volume
      during the deglaciation, expressed in meters of sea-level equivalent
      (bottom panel).}
\label{fig:hr_ts_deglac}%
\end{figure}%


\begin{figure*}%
\includegraphics{hr_maps_deglac}
\caption{%
      Snapshots of modelled surface topography (200\,\unit{m} contours) and
      surface velocity (colour mapping) during the last deglaciation from
      the GRIP (top panels) and EPICA (bottom panels) 5\,\unit{km}
      simulations. Dashed segments \textbf{(A--D)} indicate the location of
      profiles used in Figs.~\ref{fig:hr_pf_grip}
      and~\ref{fig:hr_pf_epica}.}
\label{fig:hr_maps_deglac}%
\end{figure*}%


\begin{figure*}%
\includegraphics{hr_geom_deglacage}
\caption{%
      Modelled age of the last deglaciation. Areas that have been covered
      only before the last glacial maximum are marked in green. Hatches
      denote re-advance of mountain-centred ice caps and the decaying ice
      sheet between 14 and 10\,\unit{ka}. Dashed segments \textbf{(A--D)}
      indicate the location of profiles used in Figs.~\ref{fig:hr_pf_grip}
      and~\ref{fig:hr_pf_epica}.}
\label{fig:hr_geom_deglacage}%
\end{figure*}%


\begin{figure*}%
\includegraphics{hr_geom_lastflow}
\caption{%
      Modelled deglacial basal ice velocities. Hatches indicate areas that
      remain non-sliding throughout deglaciation (22.0--8.0\,\unit{ka}),
      notably including parts of the Interior Plateau (IP). Note the
      concentric patterns of deglacial flow in the Liard Lowland (LL).
      Sliding grid cells were distinguished from non-sliding grid cells
      using a~basal velocity threshold of 1\,\unit{m\,yr^{-1}}.}
\label{fig:hr_geom_lastflow}%
\end{figure*}%


\begin{figure}%
\includegraphics{hr_pf_grip}
\caption{%
      Modelled bedrock (black) and ice surface (blue) topography profiles
      during deglaciation (22.0--8.0\,\unit{ka}) in the GRIP 5\,\unit{km}
      simulation, corresponding to the four transects indicated in
      Figs.~\ref{fig:hr_maps_deglac}
      and~\ref{fig:hr_geom_deglacage}.}
\label{fig:hr_pf_grip}%
\end{figure}%


\begin{figure}%
\includegraphics{hr_pf_epica}
\caption{%
      Modelled bedrock (black) and ice surface (red) topography profiles
      during deglaciation (22.0--8.0\,\unit{ka}) in the EPICA 5\,\unit{km}
      simulation, corresponding to the four transects indicated in
      Figs.~\ref{fig:hr_maps_deglac}
      and~\ref{fig:hr_geom_deglacage}.}
\label{fig:hr_pf_epica}%
\end{figure}


\end{document}
\endinput%%%%%%
