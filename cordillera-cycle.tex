% cordillera-climate.tex
% ----------------------------------------------------------------------

% Copernicus manuscript
\documentclass[tc, ms]{copernicus}

% Copernicus final print
%\documentclass[tc]{copernicus}

% Copernicus discussion paper
%\documentclass[tcd, hvmath]{copernicus_discussions}

% Copernicus-like latex2rtf compatible
%% copernicus_rtf.tex
% ------------------

% Base class and packages
\documentclass{article}
\usepackage{color}
\usepackage{geometry}
\usepackage{graphicx}
\usepackage{setspace}
\onehalfspacing

% Replacements for bibtex commands
\newcommand{\citep}[1]{(\textcolor{blue}{#1})}
\newcommand{\citet}[1]{\textcolor{blue}{#1}}

% Replacements for Copernicus commands
\newcommand{\introduction}[0]{\section{Introduction}}
\newcommand{\conclusions}[0]{\section{Conclusions}}
\newcommand{\tophline}[0]{\hline}
\newcommand{\middlehline}[0]{\hline}
\newcommand{\bottomhline}[0]{\hline}
\newcommand{\unit}[1]{\ensuremath{\mathrm{#1}}}
\newcommand{\degree}[0]{\ensuremath{^{\circ}}}

% Ignore other Copernicus commands
\newcommand{\runningtitle}[1]{}
\newcommand{\runningauthor}[1]{}
\newcommand{\received}[1]{}
\newcommand{\correspondence}[1]{}
\newcommand{\pubdiscuss}[1]{}
\newcommand{\revised}[1]{}
\newcommand{\accepted}[1]{}
\newcommand{\published}[1]{}



% Coloured hyperlinks
\usepackage[colorlinks]{hyperref}

% Encoding
\usepackage[T1]{fontenc}

% Figure directory
\graphicspath{{figures/}}

% My commands
\newcommand{\idea}[1]{\textbf{[IDEA: #1]}}
\newcommand{\note}[1]{\textbf{[NOTE: #1]}}
\newcommand{\todo}[1]{\textbf{[TODO: #1]}}
\newcommand{\aref}[0]{\textbf{[ref.]}}
\renewcommand{\citep}[1]{\aref}
\renewcommand{\citet}[1]{\aref}

% ----------------------------------------------------------------------
\begin{document}\hack{\sloppy}
% ----------------------------------------------------------------------

% Title
\title{Numerical simulation of the Cordilleran ice sheet
       through the last glacial cycle}

% Authors
\author[1,2]{J.~Seguinot}
\author[3]{M.~Margold}
\author[2]{I.~Rogozhina}
\author[1]{A.~Stroeven}
\runningauthor{J.~Seguinot et~al.}
\correspondence{J.~Seguinot (julien.seguinot@natgeo.su.se)}

% Running title
\runningtitle{Climate forcing for Cordilleran ice sheet simulations}

% Affiliations
\affil[1]{Department of Physical Geography and Quaternary Geology and the
          Bolin Centre for Climate Research, Stockholm University,
          Stockholm, Sweden}
\affil[2]{Helmholtz Centre Potsdam, GFZ German Research Centre for Geosciences,
          Potsdam, Germany}
\affil[3]{Department of Geography, Durham University, UK}

% For Copernicus
\received{}
\accepted{}
\published{}

% Title
\firstpage{1}
\maketitle

% Abstract
\begin{abstract}

  Despite more than a century of geological observations, the Cordilleran ice
  sheet of North America remains poorly understood in terms of its former
  extent, volume and dynamics. Although geomorphological evidence is abundant,
  its complexity is such that whole ice-sheet reconstructions of advance and
  retreat patterns are lacking. Here we use a numerical ice sheet model
  calibrated against field-based evidence to attempt a quantitative
  reconstruction of the Cordilleran ice sheet history through the last glacial
  cycle. A series of simulations is driven by time-dependent temperature
  offsets from six proxy records located around the globe. Although this
  approach reveals large variations in model response to evolving atmospheric
  forcing, all simulations produce two major glaciation events during MIS~4
  (61.9--55.4\,\unit{kyr}) and MIS~2 (29.5--16.9\,\unit{kyr}). The timing of
  glaciation is
  better reproduced using temperature reconstructions from Greenland and
  Antarctic ice cores than from regional ocean sediment cores. During most of
  the last glacial cycle, the modelled ice cover is discontinuous and
  restricted to high mountain areas. However, widespread precipitation over the
  Skeena mountains favours the persistence of a central ice dome throughout the
  glacial cycle. It acts as a nucleation centre before the last glacial maximum
  and hosts the last remains of Cordilleran ice during the
  early Holocene (10.9--9.5\,\unit{kyr}).

\end{abstract}

\begin{center}
\begin{tabular}{>{\bfseries}ll}
    \tophline
    Introduction & Happy with it except for missing refs.\\
    Model setup  & Needs minor fixes to avoid repeating the previous paper.\\
    Results      & Almost happy with it. \\
    Discussion   & Large gaps still have to be filled.\\
    Conclusions  & Main contents are here but the text needs to be improved.\\
    Figures      & I have plans for a couple extra ones.\\
    \bottomhline
\end{tabular}
\end{center}

% ----------------------------------------------------------------------
\introduction
\label{sec:intro}
% ----------------------------------------------------------------------

During the last glacial cycle, glaciers and ice caps of the North American
Cordillera were more extensive than today. At the last glacial maximum, a
contiguous blanket of ice extended from the Alaska Range in the north to the
North Cascades in the south \aref. This ice mass is known as the former
Cordilleran ice sheet (Fig.~\ref{fig:locmap}).

For more than a century, exploration and geological investigation of the
Cordillera have led to a large collection of evidence of the former ice cover
\aref. This evidence consists of mapped boundaries of the former ice extent
\aref, directions of past sliding velocities {\aref} and locations of former
melt-water streams \aref, as well as the timing of glaciation identified from
radiocarbon dating \aref, cosmogenic dating \aref, and the offshore sedimental
record \aref.

Field-based evidence allowed to reconstruct, with a high degree of confidence,
the maximal extent attained by the Cordilleran ice sheet during the last glacial
cycle \aref. However, former ice thickness and the ice sheet's contribution to the
last glacial maximum sea-level low-stand remain poorly constrained. Moreover,
our understanding of the glaciation history is mainly restricted to the
deglaciation phase and a regional scale, while little is known about the ice
sheet evolution prior to the last glacial maximum extent \aref. Although
time-evolving, whole ice-sheet reconstructions of glacial advance and retreat
patterns are available for the Laurentide and Eurasian ice sheets \aref, such is not
the case for the Cordilleran ice sheet, where complex arrangements of ice flow
directions emanating from multiple glaciation centres have been identified, but
remain poorly understood \aref.

The present study uses a numerical ice sheet model (PISM, \aref) calibrated against
field-based evidence to perform a quantitative reconstruction of the
Cordilleran ice sheet history through the last glacial cycle. Although
numerical modelling has been established as a useful tool to improve our
understanding
of the Cordilleran ice sheet more than twenty years ago \aref, the ubiquitously
mountainous topography of the region has presented a major challenge to its
application, which has only been overcome by recent development of numerical ice
sheet models and underlying scientific computing tools \aref. In addition to the topographic
complexity, strong climatic contrasts characteristic of the North American
Cordillera require the use of a high-resolution temperature and precipitation
forcing as input data to the ice sheet model \aref.

Because past climate conditions are subject to considerable uncertainty, our
palaeoclimate forcing is a simplistic approximation of time-evolved temperature
and precipitation fields derived from a combination of a present-day
atmospheric reanalysis (NARR, \aref), lapse-rate corrections,
and temperature offset time series. The latter are
obtained by scaling six different palaeo-temperature reconstructions from
proxy records around the globe, including two \chem{\delta^{18}O} records from
Greenland ice cores \aref, two \chem{\delta^{18}O} records from Antarctic ice cores \aref,
and two alkenone unsaturation index records from Northwest Pacific oceanic
sediment cores \aref. Model output is compared to geomorphological evidence in terms
of timing and extent of glaciation and patterns of deglaciation.


% ----------------------------------------------------------------------
\section{Model setup}
\label{sec:model}
% ----------------------------------------------------------------------

\todo{Small adjustments to avoid repetitions with the previous paper.}

The simulations presented here were run using Parallel Ice Sheet Model (PISM,
development version~11b0a7f), an open-source, finite-difference, shallow ice
sheet model \aref. The model inputs basal topography, sea level, geothermal
heat flux and climate forcing, and computes the evolution of ice extent
and thickness in time, the thermal and dynamic state of the ice sheet, and
the associated lithospheric response. Our modelling domain encompasses the
entire area covered by the Cordilleran ice sheet at the last glacial maximum
(Fig.~\ref{fig:locmap}).

To reconstruct the successive phases of growth and decay of the last Cordilleran
ice sheet, palaeo-climatic conditions of the last glacial cycle are mimicked
by applying time-dependent temperature offsets derived from multiple
palaeo-temperature proxy records. Each simulation starts from assumed ice-free
conditions at -120\,kyr, and runs to present. These computations were
performed on 16 to 128 cores at the Swedish National Supercomputing
Center.

\subsection{Ice thermodynamics}

PISM is a~shallow model, which implies that the balance of stresses is
approximated based on their predominant components.
The Shallow Shelf Approximation (SSA) is used as a ``sliding law'' for the
Shallow Ice Approximation (SIA) \citep{bueler-brown-2009,winkelmann-etal-2011}.
SIA and SSA velocities are computed by finite difference methods on a~10\,km
resolution horizontal grid of 300 by 150 points (the modelling domain). Ice
softness depends on temperature and water content through an enthalpy
formulation \citep{aschwanden-blatter-2009,aschwanden-etal-2012}. Enthalpy is
computed in three dimensions in up to 51 layers within the
ice, and temperature is additionally computed in 31 layers in bedrock to
a~depth of 3\,km. Surface air temperature from the atmospheric forcing
provides the upper boundary condition to the ice enthalpy model, and
a~geothermal heat flux of 70\,\unit{mW\,m^{-2}} provides the lower boundary
condition to the bedrock thermal model. Although this uniform value does not
account for the high spatial geothermal variability in the region, it is on
average representative of available heat flow measurements
\citep{artemieva-mooney-2001,blackwell-richards-2004}.

A~pseudo-plastic sliding law \citep{aschwanden-etal-2013} relates the
bed-parallel shear stress and the sliding velocity. The yield stress is
modelled using the Mohr--Coulomb criterion. The till friction angle $\phi$
varies from 15 to 45{\degree}. It is taken as a~function of modern bed
elevation, with lowest values occurring at low elevation, thus accounting
for a~weakening of till associated with the presence of marine sediments
\citep{martin-etal-2011,aschwanden-etal-2013}. The till effective pressure is
related to the modelled amount of water in the till, using a formula derived
from laboratory experiments with till extracted from an Antarctic ice stream.
Basal topography is derived from the ETOPO1 combined topography
and bathymetry dataset with a~resolution of 1\,arc-min \citep{data:etopo1}.

Sea level is lowered as a function of time according to the SPECMAP
reconstruction, and basal topography responds to ice load
following a bedrock deformation model that includes point-wise isostasy,
elastic lithosphere flexure and viscous mantle deformation in a~semi-infinite
half-space \citep{lingle-clark-1985,bueler-etal-2007}.

\subsection{Surface mass-balance}

Ice surface accumulation and ablation are computed from monthly mean
near-surface air temperature, monthly precipitation, and monthly standard
deviation of near-surface temperature by a~temperature-index (positive
degree-day) model \citep{hock-2003}. Ice accumulation is equal to precipitation
when temperature is below 0\,\unit{{\degree}C}, and decreases to zero linearly
with temperature between 0 and 2\,\unit{{\degree}C}. Ice ablation is computed
from the number of positive degree-days, defined as the integral of
temperatures above 0\,\unit{{\degree}C} in one year.

The positive degree-day integral \citep{calov-greve-2005} is numerically
approximated using week-long sub-intervals. It accounts for temperature
variability assuming a~normal distribution along a~central (input) value. The
temperature standard deviation is part of the forcing climatology. It was
computed from daily temperature values from the North American Regional
Reanalysis \citep{data:narr}, excluding variability associated with the
seasonal cycle itself \citet{seguinot-rogozhina-2014}. The
ablation model incorporates degree-day factors of
3.04\,\unit{mm\,{\degree}C^{-1}\,day^{-1}} for snow and
4.59\,\unit{mm\,{\degree}C^{-1}\,day^{-1}} for ice, as derived from
mass-balance measurements on contemporary glaciers from the Coast Mountains and
Rocky Mountains in British Columbia \citep{shea-etal-2009}.

\subsection{Atmospheric forcing}

Atmospheric forcing of the model consists of a present-day monthly climatology
$\{T_m0, P_m0\}$, modified by temperature-offset time series ${\Delta}T_{TS}$
and a lapse-rate correction ${\Delta}T_{LR}$.
\begin{align}
    T_m(t, x, y) &= T_{m0}(x, y) + {\Delta}T_{TS}(t) + {\Delta}T_{LR}(t, x, y) \\
    P_m(t, x, y) &= P_{m0}(x, y)
\end{align}

The present-day climatology $\{T_m0, P_m0\}$ was computed from the near-surface air temperature and
precipitation rate fields of the North American Regional Reanalysis (NARR) over
the period 1979--2000 (Fig.~\ref{fig:atm}). Modern climate of the North
American Cordillera is characterised by strong geographic variations of
seasonality, timing of the precipitation peak and daily temperature variability.
Our choice of data from the NARR is motivated by the need for an accurate,
high-resolution precipitation forcing, as outlined in a previous sensitivity
study.

Temperature offset time-series ${\Delta}T_{TS}$ are derived from proxy records from
the Greenland Ice Core Project (GRIP), the North Greenland Ice Core Project
(NGRIP), the European Project for Ice Coring in Antarctica (EPICA), the Lake
Vostok ice core, and Ocean Drilling Program (ODP) sites 1012 and 1020, both
located off the Californian shore. Palaeo-temperatures from the GRIP and NGRIP
records were calculated using a quadratic equation \aref:
\begin{equation}
    {\Delta}T_{TS}(t) = -11.88[\chem{\delta^{18}O(t)}-\chem{\delta^{18}O}(0)]
                        -0.1925[\chem{\delta^{18}O(t)}^2-\chem{\delta^{18}O}(0)^2]
\end{equation}
while the temperature reconstructions from the Antarctic and oceanic cores were
provided by their authors \aref. All records were scaled linearly in
order to simulate realistic and comparable ice extents at the last
glacial maximum (Table~\ref{tab:records}, Fig.~\ref{fig:timeseries}).

Prior to surface mass balance computation, the model dynamically applies
a~lapse-rate correction to surface air temperature,
\begin{align}
    {\Delta}T_{LR}(t, x, y) &= -\gamma [z_{s}(t, x, y)-z_{ref}] \\
                            &= -\gamma [h(t, x, y)+z_{b}(t, x, y)-z_{ref}]
\end{align}

This correction accounts for the evolution of ice thickness $h$ on the one
hand, and for differences between the climate forcing reference topography and
the ice flow model basal topography $z_{b}-z_{ref}$ on the other hand. An
annual lapse rate of $\gamma = 6\,\unit{{\degree}C\,km^{-1}}$ is used in all
simulations.


% ----------------------------------------------------------------------
\section{Results}
\label{sec:results}
% ----------------------------------------------------------------------

Despite large differences in the input temperature offset time-series, the
model output presents consistent features that can be observed across the range
of forcing data used. In all simulations, total ice volume is relatively low
during most of the glacial cycle, while two major glaciation events occur
during MIS~4 (61.9--55.4\,\unit{kyr}) and MIS~2 (29.5--16.9\,\unit{kyr},
Fig.~\ref{fig:timeseries}). A local ice minimum is consistently attained in
between these two events during MIS~3 (52.2--45.6\,\unit{kyr}).
However, the magnitude and timing of these events vary significantly
between our simulations.

Simulations forced by the Greenland ice core records (GRIP, NGRIP) produce
highest variability in modelled ice volume throughout the glaciation history.
Although simulations forced by the Antarctic (EPICA, Vostok) and GRIP records
reach the last glacial maximum in ice volume at 19.5--16.9\,\unit{kyr}, those
driven by oceanic (ODP~1012, ODP~1020) and the NGRIP record attain it several
thousands of years earlier. The NGRIP record is the only temperature forcing
that predicts a more extensive glaciation during MIS~4 than at the LGM.

Simulations forced by oceanic and Antarctic records generally result in lower
modelled ice volume during MIS~4 and larger ice volume during MIS~3, thus lower
ice volume variability throughout the simulation length. The ODP~1012
record-driven simulation predicts rapid deglaciation prior to 20\,\unit{kyr}
before present due to a local temperature maximum inconsistent with all other
temperature forcing time series used. The ODP~1012 record-driven simulation
predicts an early last glacial volume maximum, followed by slower deglaciation
than modelled using other records. Finally, simulations forced by Antarctic ice
core records produce a continuous, rapid deglaciation after the last glacial
maximum, whereas the simulations driven by the GRIP records produce a rapid
deglaciation in three steps, including a glacial re-advance around the Younger
Dryas event.

Despite the different timing, snapshots of model output show relatively
consistent patterns of glaciation (Fig.~\ref{fig:snapshots}). As a result of
the different scaling factors applied to palaeo-temperature proxy records, the
ice sheet geometry at the last glacial maximum appears very similar from one
simulation to the next, including a central divide located at about
3\,500\,\unit{m} elevation along the spine of the Rocky Mountains. In all
simulations, a central ice cap persists over the Skeena mountains during MIS~3
between the two glaciation events. However, the size of this ice cap, as well
as the magnitude of the MIS~4 glaciation, depends sensitively on the choice of
palaeo-temperature proxy record used to drive the model.

% ----------------------------------------------------------------------
\section{Discussion}
\label{sec:discussion}
% ----------------------------------------------------------------------

\todo{Develop the three subsections in order to reach the conclusions.}

\subsection{Timing of the glacial maxima}

During the last glacial cycle, geomorphological evidence has allowed to
identify two major advances of the Cordilleran ice sheet. These correspond to
the last glacial maximum \aref, and the older xxx glaciation, found in the
northernmost parts of the ice sheet \aref. It should be noted that all
simulations presented here reproduce these two major events, which constitutes
a first-order agreement between model results and geomorphological evidence.

However, the timing of these events depends highly on the choice of temperature
time-series used to force the model. For instance, the timing of the last
glacial maximum is best reproduced by the Antarctic records, while simulations
driven by the NGRIP and oceanic records pre-date the last glacial maximum by
several thousands of years. Regarding the oceanic records, the early volume
maxima can be explain by an early warming in the palaeo-temperature record.
This early warming has been previously documented, and explained as a
weakening of the California current, precisely associated with the presence of
the Cordilleran ice sheet \aref. This precocious glacial maximum illustrates
the complexity of climate-ice-sheet feedbacks on a regional scale, and shows
that, although closely located to the modelling domain, the ODP~1012 and
ODP~1020 proxy records records can't be reliably used as a temperature forcing
to model the last Cordilleran ice sheet.

\subsection{Glaciated areas through the last glacial cycle}

During most of the glacial cycle, ice cover is restricted to high mountain
areas. This is particularly true when using the Greenland-based forcing
(Fig~\ref{fig:duration}). Most of the glacial maxima margin has a transient
character, showing that the ice sheet is far out of balance with climate. There
persists a central ice dome over the Skeena mountains over the entire glacial
period. This ice dome serves as a nucleation centre at the initiation of the
advance phase preceding the last glacial maximum. It is here because of more
widespread precipitation over the Skeena mountains than across other East-West
transects towards south or north of this region. We interpret this as an
orographic precipitation effect resulting from a "window" through the sharp
range of the Coast Mountains at this latitude (Fig?).

\subsection{History of the last deglaciation}

Deglaciation happens first in the southern half and then in the northern half
of the ice-covered area (Fig.~\ref{fig:deglac}). Retreat patterns are different
than reconstructed over central BC, however they match Martin's reconstruction
of Liard lowland "unzipping" pretty well. In the GRIP-driven simulation, the
patterns of de-glaciation and Younger Dryas re-advance match with recent
cosmogenic exposure dates.

% ----------------------------------------------------------------------
\conclusions
\label{sec:concl}
% ----------------------------------------------------------------------

\todo{Improve the style.}

Numerical simulations of the Cordilleran ice sheet through the last glacial
cycle presented in this study consistently produce two glaciation maxima during
MIS~4 (61.9--55.4\,\unit{kyr}) and MIS~2 (29.5--16.9\,\unit{kyr}), two periods
corresponding to documented palaeo-glaciation. This result is independent
of the choice of palaeo-temperature record used for climate forcing, and
can be seen as a first-order agreement between the model and geomorphological
evidence. However, the timing of the two glaciation peaks highly depends on
which record
is used to drive the model. The timing of the last glacial maximum is best
reproduced by the Antarctic record, and occurs too early in all simulations
that were driven by other records. The mismatch is greatest when forcing the
model with oceanic records from the Pacific Northwest, which are affected by a
weakening of the California current during the last glacial maximum.

In all simulations presented here, ice cover is limited to disjoint mountain ice
caps during most of the glacial cycle. This agrees with earlier interpretations
of the geological evidence preceding the last glacial maximum. However, our
simulations produce persistent ice cover over the Skeena mountains during
the entire glacial cycle. At time when the Cordilleran ice sheet is absent
the Skeena ice cap appears to be fed by an eastwards precipitation
intrusion through a topographic window in the Coast Mountains. The ice cap acts
as a nucleation centre at the onset of the last glacial maximum re-advance, and
may have contributed to the spectacular glacial imprint of the Skeena Mountains
erosional landscape.

Concerning the deglaciation phase, when most geological evidence is available,
none of the palaeo-temperature records used produce an ideal agreement between
model results and geological evidence. The
timing of the last glacial maximum and early deglaciation is generally best
reproduced by forcing the model with an Antarctic
record. However, the Younger Dryas re-advance is best reproduced at places
where it has been documented using temperature forcing from the GRIP record.
Nonetheless, the patterns of deglaciation are consistent between different
simulations. They show a rapid deglaciation of the southern half of the ice
sheet, followed by unzipping of the Liard Lowland, and concentric retreat of
the ice margin towards the last palaeo-ice cover in the Skeena Mountains
during the early Holocene (10.9--9.5\,\unit{kyr}).

One must keep in mind, however, that these results are only accurate for our
choices of ice-sheet model (PISM), surface mass balance model (PDD), study
object (the Cordilleran ice sheet). Most importantly, our simplistic
palaeo-climate forcing does not include precipitation corrections in response
to the presence of the ice sheet, likely leading to overestimates of glacial
extent and volume. Nevertheless, our results identify the largely understudied
Skeena Mountains as a key area to understanding glacial dynamics of the
Cordilleran ice sheet, highlighting the need for further geological
investigation in this region.

% Acknowledgements
%\begin{acknowledgements}
  % Author contributions
  %\hack{\noindent}\textit{Author contributions.}
%\end{acknowledgements}

% References
%\input{references.bbl}
\newpage

% ----------------------------------------------------------------------
% Floats
% ----------------------------------------------------------------------

% tab:records
\begin{table*}[t]
  \caption{Palaeo-temperature proxy records and scaling parameters used to
           prepare temperature offset time-series used to force the ice sheet
           model through the last 120\,\unit{kyr}. $T_{[32;22]}$ refers to the
           mean temperature anomaly during the period -32 to~-22~\unit{kyr} after
           scaling.}
  \label{tab:records}
  {\begin{tabular}{lcccc}
    \tophline
    Record & Proxy & Scaling factor & $T_{[32;22]}$ & Source\\
    \middlehline
    GRIP     & \chem{\delta^{18}O} & ?\% & -5.8{\degree}C & \aref \\
    NGRIP    & \chem{\delta^{18}O} & ?\% & -6.1{\degree}C & \aref \\
    EPICA    & \chem{\delta^{18}O} & ?\% & -5.6{\degree}C & \aref \\
    Vostok   & \chem{\delta^{18}O} & ?\% & -5.6{\degree}C & \aref \\
    ODP~1012 & \chem{U^{K'}_{37}}  & ?\% & -5.8{\degree}C & \aref \\
    ODP~1020 & \chem{U^{K'}_{37}}  & ?\% & -5.8{\degree}C & \aref \\
    \bottomhline
  \end{tabular}}
  \belowtable{}
\end{table*}

% fig:locmap
\begin{figure}
  \includegraphics{locmap}
  \caption{Relief map of northern North America showing a reconstruction of the
           areas once covered by the Cordilleran (CIS), Laurentide (LIS),
           Innuitian (IIS) and Greenland (GIS) ice sheets during the last
           18\,\unit{\chem{^{14}C}\,kyr\,BP} (21.4\,cal\,kyr\,BP)
           \citep{dyke-2004}. The rectangular box denotes the location of the
           modelling domain used in this study. Major mountain ranges covered
           by the ice sheet include the Alaska Range (AR), the Wrangell and
           St.-Elias mountains (WSE), the Selwyn and MacKenzie mountains (SMK),
           the Skeena Mountains, the Coast Mountains (CM), the Rocky Mountains
           (RM) and the North Cascades (NC). The background
           map consists of ETOPO1 \citep{data:etopo1} and Natural Earth Data
           \citep{data:naturalearth}.
           \todo{Mark palaeo-ice sheets and mountain ranges on the map.}}
  \label{fig:locmap}
\end{figure}

% fig:atm
\begin{figure}
  \includegraphics{atm}
  \caption{Monthly mean near-surface air temperature, precipitation and
           standard deviation of daily mean temperature for January and July
           months from the North American Regional Reanalysis (NARR)
           climatology, used to force the ice sheet model. Note the
           strong contrasts in seasonality, timing of the precipitation peak,
           and temperature variability over the model domain, notably between
           the maritime and continental regions.}
  \label{fig:atm}
\end{figure}

% fig:timeseries
\begin{figure}
  \includegraphics{timeseries}
  \caption{Temperature offset time-series from ice core and sediment core
           records (Table~\ref{tab:records}) used as palaeo-climate forcing for
           the ice sheet model \textbf{(top)}, and modelled ice volume
           through the last 120\,\unit{kyr}, expressed in meters of sea-level
           equivalent \textbf{(bottom)}. Gray spans indicate Marine Isotope
           Stages (MIS) according to a global compilation of benthic
           \chem{\delta^{18}O} records \citep{lisiecki-raymo-2005}. Hatched
           rectangles indicate modelled ice volume extrema corresponding to
           MIS~4 (61.9--55.4\,\unit{kyr}), MIS~3 (52.2--45.6\,\unit{kyr}), and
           MIS~2 (last glacial maximum, 29.5--16.9\,\unit{kyr}).}
  \label{fig:timeseries}
\end{figure}

% fig:snapshots
\begin{figure}
  \includegraphics{snapshots}
  \caption{Snapshots of modelled surface topography (500\,\unit{m} contours)
           corresponding to the ice volume extrema indicated on
           Fig.~\ref{fig:timeseries}. Note the occurence of spatial similarities
           despite large differences in timing.}
  \label{fig:snapshots}
\end{figure}

% fig:duration
\begin{figure}
  \includegraphics{duration}
  \caption{Modelled duration of ice cover during the last 120\,\unit{kyr}.
           A contiguous ice cover spanning from the Alaska Range (AR) to the
           southern Coast Mountians (CM) and Rocky Mountains (RM) exists for
           about 28\,\unit{kyr} in the GRIP simulation and about 29\,\unit{kyr}
           in the EPICA simulation. Note the irregular colour scale. A central
           ice cover persists over the Skeena Mountains (SM) during most of the
           simulation. On the other hand, the maximal extent of the ice sheet
           generally corresponds to relatively short durations of ice cover.}
  \label{fig:duration}
\end{figure}

% fig:warmbase
\begin{figure}
  \includegraphics{warmbase}
  \caption{Modelled duration of warm-based ice cover during the last
           120\,\unit{kyr}. Long ice cover durations combined with basal
           temperatures at the melting point may explain the strong glacial
           erosional imprint of the Skeena Mountains (SM) landscape. Hatches
           indicate areas that were covered by cold ice only.
           \note{indicate location of Skeena Mountains.}}
  \label{fig:warmbase}
\end{figure}

% fig:erosion
\begin{figure}
  \includegraphics{erosion}
  \caption{Modelled cumulative basal displacement (integrand of basal velocity)
           over the last 120\,\unit{kyr}.
           \note{Even after carefully adjusting the colours, I still find this
                 figure quite ugly and feel that it does not show much. I may
                 remove it from this paper and keep the idea for later use.}}
  \label{fig:erosion}
\end{figure}

% fig:deglac
\begin{figure}
  \includegraphics{deglac}
  \caption{Modelled age of the last deglaciation. Areas where the MIS~4 glacial
           advance exceeded the last glacial maximum advanced are marked in
           green. Hatches denote the Younger Dryas re-advance, which is most
           significant in the GRIP-driven simulation.}
  \label{fig:deglac}
\end{figure}

% fig:deglacshots
\begin{figure}
  \includegraphics{deglacshots}
  \caption{Snapshots of modelled surface topography (200\,\unit{m} contours)
           and surface velocity (colour mapping) from the GRIP simulation,
           corresponding to the last glacial maximum (-19.5 kyr) and the last
           deglaciation.}
  \label{fig:deglacshots}
\end{figure}

% fig:lastflow
\begin{figure}
  \includegraphics{lastflow}
  \caption{Modelled directions of the last basal ice velocities. Hatches
           indicate areas that remain non-sliding throughout the simulation.
           Sliding grid cells were distinguished from non-sliding grid cells
           using a velocity threshold of 1\,\unit{m\,yr^{-1}}.}
  \label{fig:lastflow}
\end{figure}

% ----------------------------------------------------------------------
\end{document}
\endinput
% ----------------------------------------------------------------------
