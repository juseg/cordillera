% cordillera-climate.tex
% ----------------------------------------------------------------------

% Copernicus manuscript
\documentclass[tc, manuscript]{copernicus}

% Copernicus final print
%\documentclass[tc]{copernicus}

% Copernicus discussion paper
%\documentclass[tcd, hvmath, online]{copernicus}

% Copernicus-like latex2rtf compatible
%% copernicus_rtf.tex
% ------------------

% Base class and packages
\documentclass{article}
\usepackage{color}
\usepackage{geometry}
\usepackage{graphicx}
\usepackage{setspace}
\onehalfspacing

% Replacements for bibtex commands
\newcommand{\citep}[1]{(\textcolor{blue}{#1})}
\newcommand{\citet}[1]{\textcolor{blue}{#1}}

% Replacements for Copernicus commands
\newcommand{\introduction}[0]{\section{Introduction}}
\newcommand{\conclusions}[0]{\section{Conclusions}}
\newcommand{\tophline}[0]{\hline}
\newcommand{\middlehline}[0]{\hline}
\newcommand{\bottomhline}[0]{\hline}
\newcommand{\unit}[1]{\ensuremath{\mathrm{#1}}}
\newcommand{\degree}[0]{\ensuremath{^{\circ}}}

% Ignore other Copernicus commands
\newcommand{\runningtitle}[1]{}
\newcommand{\runningauthor}[1]{}
\newcommand{\received}[1]{}
\newcommand{\correspondence}[1]{}
\newcommand{\pubdiscuss}[1]{}
\newcommand{\revised}[1]{}
\newcommand{\accepted}[1]{}
\newcommand{\published}[1]{}



% Coloured hyperlinks
\hypersetup{colorlinks, citecolor=blue}

% Figure directory
\graphicspath{{figures/}}

% My commands
\newcommand{\note}[1]{\textbf{[NOTE: #1]}}
\newcommand{\todo}[1]{\emph{[\textbf{Todo:} #1]}}
\newcommand{\aref}[0]{\textbf{[ref.]}}

% ----------------------------------------------------------------------
\begin{document}
\linenumbers
% ----------------------------------------------------------------------

% Title
\title{Numerical simulation of the Cordilleran ice sheet
       through the last glacial cycle}

% Authors
\Author[1,2]{J.}{Seguinot}
\Author[3]{M.}{Margold}
\Author[2]{I.}{Rogozhina}
\Author[1]{A.~P.}{Stroeven}
\runningauthor{J.~Seguinot et~al.}
\correspondence{J.~Seguinot (julien.seguinot@natgeo.su.se)}

% Running title
\runningtitle{Climate forcing for Cordilleran ice sheet simulations}

% Affiliations
\affil[1]{Department of Physical Geography and Quaternary Geology and the
          Bolin Centre for Climate Research, Stockholm University,
          Stockholm, Sweden}
\affil[2]{Helmholtz Centre Potsdam, GFZ German Research Centre for Geosciences,
          Potsdam, Germany}
\affil[3]{Department of Geography, Durham University, UK}

% For Copernicus
\received{}
\pubdiscuss{}
\revised{}
\accepted{}
\published{}

% Title
\firstpage{1}
\maketitle

% Abstract
\begin{abstract}

  Despite more than a century of geological observations, the Cordilleran ice
  sheet of North America remains poorly understood in terms of its former
  extent, volume and dynamics. Although geomorphological evidence is abundant,
  its complexity is such that whole ice-sheet reconstructions of advance and
  retreat patterns are lacking. Here we use a numerical ice sheet model
  calibrated against field-based evidence to attempt a quantitative
  reconstruction of the Cordilleran ice sheet history through the last glacial
  cycle. A series of simulations is driven by time-dependent temperature
  offsets from six proxy records located around the globe. Although this
  approach reveals large variations in model response to evolving atmospheric
  forcing, all simulations produce two major glaciation events during
  marine isotope stages~4 (61.9--55.4\,kyr) and~2
  (29.5--16.9\,kyr). The timing of glaciation is
  better reproduced using temperature reconstructions from Greenland and
  Antarctic ice cores than from regional oceanic sediment cores. During most of
  the last glacial cycle, the modelled ice cover is discontinuous and
  restricted to high mountain areas. However, widespread precipitation over the
  Skeena mountains favours the persistence of a central ice dome throughout the
  glacial cycle. It acts as a nucleation centre before the last glacial maximum
  and hosts the last remains of Cordilleran ice during the
  early Holocene (10.9--9.5\,kyr).

\end{abstract}

% ----------------------------------------------------------------------
\introduction
\label{sec:intro}
% ----------------------------------------------------------------------

During the last glacial cycle, glaciers and ice caps of the North American
Cordillera were more extensive than today. At the last glacial maximum, a
contiguous blanket of ice extended from the Alaska Range in the north to the
North Cascades in the south (Fig.~\ref{fig:locmap}). This ice mass is known as
the former Cordilleran ice sheet \citep{Dawson.1888}.

For more than a century, exploration and geological investigation of the
Cordillera have led to a large collection of evidence of the former ice cover
\citep{Jackson.Clague.1991}. This evidence consists of mapped boundaries of the
former ice extent (\citealp[Fig.~1.12]{Clague.1989}; \citealp{Duk-Rodkin.1999};
\citealp{Dyke.2004}), directions of past sliding velocities
(\citealp{Prest.etal.1968}; \citealp[Fig.~2]{Kleman.etal.2010}), and locations
of former
melt-water streams \citep{Margold.etal.2011, Margold.etal.2013a}, as well as the
timing of glaciation identified from radiocarbon dating \citep[e.g.,][]
{Clague.1981, Porter.Swanson.1998}, cosmogenic dating \citep[e.g.,][]
{Ward.etal.2007, Menounos.etal.2008, Stroeven.etal.2010, Stroeven.etal.2013},
and the offshore sedimentary record \citep{Cosma.etal.2008, Davies.etal.2011}%
\footnote{\textbf{Arjen, Martin,}, please comment on my choice of references
    in this paragraph. Is it a fair sample of the current state of the field?
    Am I missing some other important works?}.

Field-based evidence allowed to reconstruct, with a high degree of confidence,
the maximal extent attained by the Cordilleran ice sheet during the last
glacial cycle (\citealp[Fig. 1.12]{Clague.1989}; \citealp{Dyke.2004}). However,
former ice thickness
and the ice sheet's contribution to the last glacial maximum sea-level
low-stand remain poorly constrained. Moreover, our understanding of the
glaciation history is mainly restricted to the deglaciation phase and a
regional scale, while little is known about the ice sheet evolution prior to
the last glacial maximum extent \citep[Fig.~6]{Kleman.etal.2010}. Although
time-evolving, whole ice-sheet reconstructions of glacial advance and retreat
patterns are available for the Eurasian and Laurentide and ice sheets
\citep{Kleman.etal.1997, Kleman.etal.2010}, such is not the case for the
Cordilleran ice sheet, where complex arrangements of ice flow directions
emanating from multiple glaciation centres have been identified
\citep[Fig.~1.12]{Prest.etal.1968, Clague.1989}, but remain poorly understood
\citep[p.~2049]{Kleman.etal.2010}.

The present study uses a numerical ice sheet model \citep{PISM-authors.2014},
calibrated against field-based evidence to perform a quantitative
reconstruction of the Cordilleran ice sheet history through the last glacial
cycle. Although numerical modelling has been established as a useful tool to
improve our understanding of the Cordilleran ice sheet more than twenty years
ago (\citealp[p.~227]{Jackson.Clague.1991}; \citealp{Robert.1991}), the
ubiquitously mountainous
topography of the region has presented a major challenge to its application.
In fact, only recent development of numerical ice sheet models and underlying
scientific computing tools \citep{Bueler.Brown.2009, Balay.etal.2014} allows
high-resolution numerical modelling of palaeo-glaciers on mountainous terrain
over millenial time scales \citep{Golledge.etal.2012}. In the North
American Cordillera, this complex topography induces strong gradients in
seasonality and distinct patterns of precipition, thus requiring the use of
high-resolution atmospheric forcing fields as input to the ice sheet model
\citep{Seguinot.etal.2013}.

Because past climate conditions are subject to considerable uncertainty, our
palaeoclimate forcing is a simplistic approximation of time-evolved temperature
and precipitation fields derived from a combination of a present-day
atmospheric reanalysis \citep{Mesinger.etal.2006}, lapse-rate corrections,
and temperature offset time series. The latter are obtained by scaling six
different palaeo-temperature reconstructions from proxy records around the
globe, including two \chem{\delta^{18}O} records from Greenland ice cores
\citep{Dansgaard.etal.1993, Andersen.etal.2004}, two \chem{\delta^{18}O}
records from Antarctic ice cores \citep{Petit.etal.1999,Jouzel.etal.2007},
and two alkenone unsaturation index records from Northwest Pacific oceanic
sediment cores \citep{Herbert.etal.2001}. Model output is compared to
geomorphological evidence in terms of timing and extent of glaciation and
patterns of deglaciation.


% ----------------------------------------------------------------------
\section{Model setup}
\label{sec:model}
% ----------------------------------------------------------------------

The simulations presented here were run using Parallel Ice Sheet Model (PISM,
development version~11b0a7f and stable version~0.6.1), an open-source,
finite-difference, shallow ice sheet model \citep{PISM-authors.2014}.
The model inputs basal topography, sea level, geothermal
heat flux and climate forcing, and computes the evolution of ice extent
and thickness in time, the thermal and dynamic state of the ice sheet, and
the associated lithospheric response. Our modelling domain encompasses the
entire area covered by the Cordilleran ice sheet at the last glacial maximum
(Fig.~\ref{fig:locmap}).

To reconstruct the successive phases of growth and decay of the last Cordilleran
ice sheet, palaeo-climatic conditions of the last glacial cycle are mimicked
by applying time-dependent temperature offsets derived from multiple
palaeo-temperature proxy records. Each simulation starts from assumed ice-free
conditions at -120\,kyr, and runs to present. These computations were
performed on 16 to 128 cores at the Swedish National Supercomputing Centre.

\subsection{Ice thermodynamics}

PISM is a~shallow model, which implies that the balance of stresses is
approximated based on their predominant components.
The Shallow Shelf Approximation (SSA) is used as a ``sliding law'' for the
Shallow Ice Approximation (SIA) by adding both velocity solutions
\citep[Eqn.~15]{Winkelmann.etal.2011}.
Ice softness depends on temperature and water content through an enthalpy
formulation \citep{Aschwanden.etal.2012}. Surface air
temperature from the atmospheric forcing provides the upper boundary condition
to the ice enthalpy model. Temperature is computed subglacially to a~depth of
3\,km, where it is conditioned by a~geothermal heat flux of
70\,\unit{mW\,m^{-2}}. Although this uniform value does not
account for the high spatial geothermal variability in the region, it is on
average representative of available heat flow measurements
\citep{Blackwell.Richards.2004}.

A~pseudo-plastic sliding law \citep{PISM-authors.2014}\footnote{I need to ask
    Ed Bueler if the sliding will be published somewhere soon.} relates the
bed-parallel shear stress and the sliding velocity. The yield stress is
modelled using the Mohr--Coulomb criterion. The till friction angle $\phi$
varies from 15 to 45{\degree}. It is taken as a~function of modern bed
elevation, with lowest values occurring at low elevation, thus accounting
for a~weakening of till associated with the presence of marine sediments.
The till effective pressure is related to the modelled amount of water in the
till, using a formula derived from laboratory experiments with till extracted
from an Antarctic ice stream \citep[Eqn.~2]{Tulaczyk.etal.2000}. Basal
topography is
derived from the ETOPO1 combined topography and bathymetry dataset with
a~resolution of 1\,arc-min \citep{Amante.Eakins.2009}. Sea level is lowered as
a function of time based on the Spectral Mapping Project (SPECMAP)
\citep[SPECMAP,][]{Imbrie.etal.1989} time scale.
Basal topography responds to ice load
following a bedrock deformation model that includes point-wise isostasy,
elastic lithosphere flexure and viscous mantle deformation in a~semi-infinite
half-space \citep{Lingle.Clark.1985,Bueler.etal.2007}.

\subsection{Surface mass-balance}

Ice surface accumulation and ablation are computed from monthly mean
near-surface air temperature, monthly precipitation, and monthly standard
deviation of near-surface temperature by a~temperature-index model
\citep[e.g.,][]{Hock.2003}. Ice accumulation is equal to precipitation
when temperature is below 0\,\unit{{\degree}C}, and decreases to zero linearly
with temperature between 0 and 2\,\unit{{\degree}C}. Ice ablation is computed
from the number of positive degree-days, defined as the integral of
temperatures above 0\,\unit{{\degree}C} in one year.

The positive degree-day integral \citep[Eqn.~6]{Calov.Greve.2005} is
numerically
approximated using week-long sub-intervals. It accounts for temperature
variability assuming a~normal distribution along a~central (input) value. The
temperature standard deviation is part of the forcing climatology. It was
computed from daily temperature values from the North American Regional
Reanalysis \citep[NARR,][]{Mesinger.etal.2006}, in order to account for the
effects of spatial and seasonal variations of temperature variability
\citep[Fig.~\ref{fig:atm};][]{Seguinot.2013}, after excluding variability
associated with the seasonal cycle itself \citep[cf.][]
{Seguinot.Rogozhina.2014}. The ablation model incorporates degree-day factors
of 3.04\,\unit{mm\,{\degree}C^{-1}\,day^{-1}} for snow and
4.59\,\unit{mm\,{\degree}C^{-1}\,day^{-1}} for ice, as derived from
mass-balance measurements on contemporary glaciers from the Coast Mountains and
Rocky Mountains in British Columbia \citep{Shea.etal.2009}.

\subsection{Atmospheric forcing}

Atmospheric forcing of the model consists of a present-day monthly climatology
$\{T_{m0}, P_{m0}\}$, modified by a lapse-rate correction ${\Delta}T_{LR}$ and
temperature-offset time series ${\Delta}T_{TS}$.
\begin{align}
    T_m(t, x, y) &= T_{m0}(x, y) + {\Delta}T_{LR}(t) + {\Delta}T_{TS}(t, x, y) \\
    P_m(t, x, y) &= P_{m0}(x, y)
\end{align}

The present-day climatology $\{T_{m0}, P_{m0}\}$ was computed from
near-surface air temperature and precipitation rate fields of the NARR
\citep{Mesinger.etal.2006} over the period 1979--2000. Modern climate of the
North American Cordillera is characterised by strong geographic variations of
seasonality, timing of the precipitation peak and daily temperature variability
(Fig.~\ref{fig:atm}).
Our choice of data from the NARR is motivated by the need for an accurate,
high-resolution precipitation forcing, as identified in a previous sensitivity
study \citep{Seguinot.etal.2013}.

The lapse-rate correction ${\Delta}T_{LR}$ is computed using the NARR surface
geopotential height invariant field as a reference topography $z_{ref}$,
\begin{align}
    {\Delta}T_{LR}(t, x, y) &= -\gamma [z_{s}(t, x, y)-z_{ref}] \\
                            &= -\gamma [h(t, x, y)+z_{b}(t, x, y)-z_{ref}],
\end{align}

thus accounting for the evolution of ice thickness $h$ on the one hand, and
for differences between the the ice flow model basal topography $z_{b}$ and the
NARR topography $z_{ref}$ on the other hand. All simulations use an annual temperature lapse rate of $\gamma = 6\,\unit{{\degree}C\,km^{-1}}$.

Temperature offset time-series ${\Delta}T_{TS}$ are derived from proxy records from
the Greenland Ice Core Project \citep[GRIP,][]{Dansgaard.etal.1993}, the
North Greenland Ice Core Project \citep[NGRIP,][]{Andersen.etal.2004},
the European Project for Ice Coring in Antarctica \citep[EPICA,][]
{Jouzel.etal.2007}, the Lake Vostok ice core \citep{Petit.etal.1999}, and Ocean
Drilling Program (ODP) sites 1012 and 1020, both located off the Californian
shore \citep{Herbert.etal.2001}. Palaeo-temperatures from the GRIP and NGRIP
records were calculated using a quadratic equation \citep{Johnsen.etal.1995},
\begin{equation}
    {\Delta}T_{TS}(t) = -11.88[\chem{\delta^{18}O(t)}-\chem{\delta^{18}O}(0)]
                        -0.1925[\chem{\delta^{18}O(t)}^2-\chem{\delta^{18}O}(0)^2],
\end{equation}
while the temperature reconstructions from the Antarctic and oceanic cores were
provided as such. All records were scaled linearly in
order to simulate realistic and comparable ice extents at the last
glacial maximum (Table~\ref{tab:records}, Fig.~\ref{fig:timeseries}).

For all proxy records, we firstly run a low-resolution simulation, using
a~10\,km, 300 by 150~points horizontal grid for SSA and SIA velocities,
up to 51~enthalpy layers in the ice, and 31~temperature layers in the bedrock.
For the GRIP and EPICA records, we then run a second simulation using a higher
horizontal resolution of 6\,km, up to 101~ice layers, and 61~bedrock
layers. In the rest of this paper, we occasionally refer to different model
through by the proxy record used for palaeo-temperature forcing.


% ----------------------------------------------------------------------
\section{Results}
\label{sec:results}
% ----------------------------------------------------------------------

Despite large differences in input temperature offset time-series, the model
output presents consistent features that can be observed across the range of
forcing data used. In all simulations, total ice volume is relatively low
during most of the glacial cycle, while two major glaciation events occur
during Marine Isotope Stage (MIS)~4 (61.9--55.4\,kyr) and MIS~2
(29.5--16.9\,kyr, Fig.~\ref{fig:timeseries}). A local ice minimum is
consistently attained during MIS~3 (52.2--45.6\,kyr). However, the
magnitude and timing of these three events depend significantly on the choice
of proxy record used for palaeo-temperature forcing.

Simulations forced by the Greenland ice core records (GRIP, NGRIP) produce
highest variability in modelled ice volume throughout the glaciation history.
Model runs using oceanic (ODP~1012, ODP~1020) and Antarctic (EPICA, Vostok)
records generally result in lower
modelled ice volume during MIS~4 and larger ice volume during MIS~3, thus
suggesting lower ice volume variability throughout the simulation length. The
NGRIP palaeo-temperature forcing is the only one that results in larger
glaciation during MIS~4 than during the last glacial maximum (MIS~2).

While simulations driven by the Antarctic and GRIP records
reach a last glacial maximum in ice volume between 19.5 and 16.9\,kyr,
those steered by oceanic and the NGRIP records attain it
several thousands of years earlier. Moreover, the ODP~1012 run yields rapid
deglaciation prior to 20\,kyr due to a local temperature maximum that is
inconsistent with all other records used. The ODP~1020 simulation predicts an
early last glacial maximum in ice volume at 29.5\,kyr, followed by
slower deglaciation than modelled using other records. Finally, model runs
informed by Antarctic ice core records produce a continuous, rapid deglaciation
after the last glacial maximum, whereas the simulations driven by the GRIP
record produce a rapid deglaciation in three steps, including stand-stills and
readvances of the ice margin.

Despite the different timing, snapshots of model output show relatively
consistent patterns of glaciation (Fig.~\ref{fig:snapshots}). As a result of
different scaling factors applied to palaeo-temperature proxy records,
modelled ice sheet geometries at the last glacial maximum appear very similar
from one simulation to the next. They include a central divide located at about
3\,500\,m elevation along the spine of the Rocky Mountains. In all
simulations, a central ice cap persists over the Skeena mountains during MIS~3
between the two glaciation events. However, the size of this ice cap, as well
as the magnitude of the MIS~4 glaciation, depend sensitively on the choice of
palaeo-temperature proxy record used to drive the model.


% ----------------------------------------------------------------------
\section{Discussion}
\label{sec:discussion}
% ----------------------------------------------------------------------

\subsection{Timing of the glaciation peaks}

Independently of the choice of palaeo-temperature record used as forcing
time-series for the ice sheet model, our simulations consistently produce two
glaciation maxima during MIS~4 (61.9--55.4\,kyr) and MIS~2
(29.5--16.9\,kyr). These events broadly correspond in timing to the
Gladstone (MIS~4) and McConell (MIS~2) glaciations documented by geological
evidence in the northern part of the model domain \aref, and to the Fraser
(MIS~2) glaciation in the southern part \aref
\footnote{\textbf{Arjen, Martin, Johan,} can you check that I use these names
    correctly, and help me to fill the missing refs in this sentence? Is is
    possible to trace back when these names where first applied?}.

However, the timing of the modelled glacial maxima strongly depends on the
choice of palaeo-temperature record used, and a more precise agreement with
timing reconstructed from geological evidence is only attained in some of our
simulations. In the Puget lowland, the last glacial maximum advance of the
southern Cordilleran ice sheet margin has been constrained by radiocarbon
dating between 17.4 and 16.4\,\unit{\chem{^{14}C}\,cal\,kyr\,BP}
\citep[Fig.~4]{Porter.Swanson.1998}.
These dates have been confirmed by the offshore sediment record, which shows an
increase of glaciomarine sedimentation between 19.5 and
16.2\,\unit{\chem{^{14}C}\,cal\,kyr\,BP} \citep{Cosma.etal.2008}.

Among simulations presented here, only those forced by the GRIP, EPICA and
Vostok records place the last glacial maximum within this range
(Fig.~\ref{fig:timeseries}, lower panel). Simulations driven by the NGRIP,
ODP~1012 and ODP~1020 records, on the contrary, produce a last glacial maximum
that pre-dates reconstructed ages by several thousand years. Concerning the
simulations driven by oceanic records, this early deglaciation is caused by an
early warming sealed in the palaeo-temperature reconstructions
(Fig.~\ref{fig:timeseries}, upper panel). However, it has been shown that this
early warming is a local effect, corresponding to a weakening of the California
current \citep[Fig.~3]{Herbert.etal.2001}. The California current, driving cold
waters southwards
along the western coast of North America, is thought to have weakened in the
presence of the Cordilleran ice sheet, resulting in paradoxically warmer
sea-surface temperatures at the locations of the ODP~1012 and ODP~1020 sites
during the last glacial maximum \citep{Herbert.etal.2001}.

Because most of the Cordilleran ice sheet marine front lies in a different
oceanic circulation gyre than that holding the California current, it is
unlikely that it was affected by this early warming. However, the above paradox
illustrates the complexity of ice-sheet feedbacks on regional climate, and
demonstrates that, although located in the neighbourhood of the modelling
domain, the ODP~1012 and ODP~1020 palaeo-temperature proxy records can not be
used as a realistic forcing time-series to model the Cordilleran ice sheet
through the last glacial cycle.

These results make us believe that a palaeo-temperature proxy record located
in the direct vicinity of the former Cordilleran ice sheet would be of great
help to further understand its glacial history. Because, to our knowledge, no
such record yet exists, we focus the rest of our analysis on simulations forced
by temperature records from Greenland (GRIP) and Antarctic (EPICA) ice cores
(Fig.~\ref{fig:timeseries}, lower panel, dotted lines).

\subsection{Spatial imprint of ice cover}

Although snapshots from the model output (Fig.~\ref{fig:snapshots}) give a
crisp picture of modelled ice dynamics at specific time frames, they often
can not be directly compared to geomorphological evidence, which is by nature
time-transgressive. In this section, we use the 6-km horizontal-resolution
model output from simulations driven by the GRIP and EPICA records to produce
a series of time-integrated plots containing information aggregated over the
whole simulation period.

During most of the glacial cycle, modelled ice cover is restricted to disjoint
ice caps centred on major mountain ranges of the North American Cordillera
(Fig.~\ref{fig:duration}). In both simulations driven by the GRIP and EPICA
records, a~contiguous ice cover spanning from the Alaska Range to the Rocky
Mountains exists for about 29\,kyr in total over MIS~2 and~4.
With the exception of the oceanic front and the northern foothills of the Alaska
Range, most of the maximal extent of the ice sheet is associated with short
durations of ice cover, generally a few thousand years, distributed over one
(GRIP) or two (EPICA) glaciations (Fig.~\ref{fig:duration}). This illustrates
that the most extensive stages of the modelled ice sheet are both short-lived
and out of balance with contemporary climate states, which confirms previous
inferences of a short-lived Cordilleran ice sheet based on the geological
record \citep{Clague.etal.1980, Stroeven.etal.2010}.

However, our simulations depict, independently of the choice of record used for
palaeo-temperature forcing, a central Skeena mountain ice cap persisting
throughout the entire last glacial cycle (Figs.~\ref{fig:snapshots}
and~\ref{fig:duration}). In particular, this ice cap persists during MIS~3,
providing a nucleation centre for the onset of glacial
readvance towards the last glacial maximum (MIS~2).

During time periods when a fully-grown Cordilleran ice sheet is absent from our
simulation, the presence of the Skeena ice cap can be explained by a more
widespread winter precipitation there than in other parts of the modelling
domain (Fig.~\ref{fig:atm}). Along most of the north-western North American
shore, coastal mountain ranges provide a sharp orographic barrier to westerly
winds, capturing atmospheric moisture in the form of precipitation, and leaving
interior lowlands arid. However, near the centre of our modelling domain, this
barrier is less pronounced than
elsewhere, allowing westerly winds to carry moisture further inland, until it
is progressively captured by the broad and mildly elevated group of the Skeena
Mountains.

In addition, the erosional landscape of the Skeena mountain bears one of the
strongest glacial imprint found within the area formerly covered by the
Cordilleran ice sheet \aref%
\footnote{\textbf{Arjen, Martin, Johan,} can we use a reference here to support
    this point (observations, mapping, shaded relief)? Alternatively, we could
    maybe use a photograph, if any of you has one. It would add a bit of
    variation into the figure list. \textbf{Johan}, you seemed to support this
    this point during our discussion, any idea?}.
Here we suggest that persistent ice cover (Fig.~\ref{fig:duration})
associated with basal ice temperatures at the pressure-melting point
(Fig.~\ref{fig:warmfrac} and ~\ref{fig:warmbase}%
\footnote{Fig.~\ref{fig:erosion} is tricky because it may be significantly
    affected by the constant-geothermal heat flux assumption, which is not
    valid for this region. Arjen, Johan and I decided to leave it for my kappa
    for now. Regarding Figs.~\ref{fig:warmfrac} and ~\ref{fig:warmbase}, I
    still wonder if it is useful to show both. Basically we have
    Fig.~\ref{fig:warmbase}\,=\,Fig.~\ref{fig:duration}\,*\,Fig.~\ref{fig:warmfrac}.
    Maybe the reader can perform this multiplication visually without need of
    Fig.~\ref{fig:warmbase}? Tell me what you think of it.})
may partly explain this spectacular landscape.

\subsection{Deglaciation history}

In the North American Cordillera alike other glaciated regions, the large
majority of the glacial geologic record relates to the last few millennia of
glacier cover, most of the older evidence having been overridden by ice retreat
retreat during the deglaciation \footnote{\textbf{Johan,} can we refer to one
of your papers here?}.

In our simulations, the timing of the last glacial maximum, and in turn that of
the deglaciation, depends on the palaeo-temperature time-series used to force
the model (Fig.~\ref{fig:timeseries}). The
6\,km-resolution simulation forced by the EPICA record reaches a maximal
ice volume at 17.3\,kyr, followed by a continuous deglaciation between
about 17 and 11\,kyr (Fig.~\ref{fig:deglacseries}, red dotted curve).
That driven by the GRIP record reaches maximal
volume at 19.1\,kyr, and produces a deglaciation in three steps,
including the first stand-still between 16.6 and 14.5\,kyr, and a late
glacial re-advance between 12.6 and 11.6\,kyr (Fig.~\ref{fig:deglacseries},
blue dotted curve).

The possibly of a widespread late-glacial re-advance of decaying glaciers of
the North American Cordilleran has been debated for long. On the southern tip
of the Coast Mountains, radiocarbon-dated frontal moraines indicate multiple
stages of glacial re-advance in the Fraser and Squamish valleys, of which
one corresponds to the Younger Dryas chronozone \citep{Clague.etal.1997,
Friele.Clague.2002, Friele.Clague.2002a, Kovanen.2002,
Kovanen.Easterbrook.2002}. Additional evidence for a late-glacial re-advance
was found in the Finlay River area, northern Rocky Mountains, where
sharp-crested moraines show an abundant remobilization of late-glacial
sediments by alpine glaciers in interaction with the main body of the decaying
ice sheet \citep{Lakeman.etal.2008}. Although further work is needed to
constrain the timing of the North American Cordillera late-glacial re-advance,
to seize its extent and identify potential climatic triggers
\citep{Menounos.etal.2008}, it is interesting to note that the simulation
driven by the GRIP record produces a late-glacial re-advance in the Coast
Mountains, Rocky Mountains and Finlay River area, corresponding to where it
has been identified in the geologic record (Fig.~\ref{fig:deglac}, left panel),
whereas the EPICA-driven simulation produces a nearly-continuous deglaciation
with very restrained glacial re-advance (Fig.~\ref{fig:deglac}, right panel).

Although the predicted timing of glacier retreat differs between the two
simulations, its patterns are relatively consistent (Fig.~\ref{fig:deglac}). In
both runs, the southern half of the modelling domain, including the Puget
Lowland, the Coast and Rocky Mountains and the Interior Plateau of British
Columbia, are completely deglaciated by 12\,kyr, whereas significant ice cover
remains on the Skeena, Selwyn, MacKenzie, Wrangell and Saint-Elias Mountains
in the northern half of the domain. After 12\,kyr, deglaciation proceeds
through an opening of the Liard lowland with radial margin retreat towards the
surrounding mountain ranges and inward flow towards the depression, in full
compatibility with the regional melt water record \citep{Margold.etal.2013}.
Remaining ice continues its decay by retreating towards the Selwyn Mountains
and, then, the Skeena Mountains. The last remains of the palaeo-ice sheet
vanish in the Skeena Mountains at around 10.6\,kyr in the EPICA simulation
versus 9.4\,kyr in the GRIP simulation.

Once more, snapshots of modelled deglacial ice sheet configurations
(Figs.~\ref{fig:deglacshots-grip} and~\ref{fig:deglacshots-epica}) provide a
poor medium for direct comparison with geomorphological evidence of the ice
retreat. Here, we compute a map of de-glacial, basal flow direction immediately
preceding exposure of ice-free land or cessation of sliding upon the approach
of the retreating margin (Fig.~\ref{fig:lastflow}). On the northern and
southern front of the ice sheet and in the Liard Lowland, the resulting map
show similarities with patterns of glacial lineations
(\citealp{Prest.etal.1968}; \citealp[Fig.~1.12]{Clague.1989};
\citealp[Fig.~2]{Kleman.etal.2010}; \citep[Fig.~2]{Margold.etal.2013}).
However, such is not the case on the Interior Plateau of British Columbia,
where the model produce no deglacial flow directions (Fig.~\ref{fig:lastflow}),
whereas glacial lineation maps indicate eastward flow.

On the one hand, this could indicate that the model performs poorly in this
region, perhaps because the modelled last glacial maximum ice sheet is too
thick, or too extensive to the west. Because feedback mechanisms between ice
sheet topography and regional climate such as wind redistribution, orographic
precipitation and latent warming of moisture-depleted air are absent in our
model, we regard this possibility as fully probable \citep{Seguinot.etal.2013}.

On the other hand, in both simulations driven by the GRIP and EPICA records,
the deglaciation of central British Columbia consists of a rapid northwards
retreat associated with southwards-flowing, non-sliding ice lobes on the
Interior Plateau, while the surrounding mountain ranges are already ice-free
(Figs.~\ref{fig:profiles-grip} and~\ref{fig:profiles-epica}). Thus the model
suggests that glacial lineations on the Interior Plateau of British Columbia
may be of older age than the last glacial maximum, and may have remained intact
throughout the deglaciation.

% ----------------------------------------------------------------------
\conclusions
\label{sec:concl}
% ----------------------------------------------------------------------

Numerical simulations of the Cordilleran ice sheet through the last glacial
cycle presented in this study consistently produce two glaciation maxima during
MIS~4 (61.9--55.4\,kyr) and MIS~2 (29.5--16.9\,kyr), two periods
corresponding to documented extensive palaeo-glaciations. This result is
independent of the choice of palaeo-temperature record used to approximate the
past evolution of climate forcing, and
can be seen as a first-order agreement between the model and geomorphological
evidence. However, the timing of the two glaciation peaks depends sensitively
on which record
is used to drive the model. The timing of the last glacial maximum is best
reproduced by the Antarctic record, and occurs too early in all simulations
that are driven by other records. The mismatch is greatest when using
oceanic records from the Pacific Northwest, which are affected by the
weakening of the California current during the last glacial maximum.

In all simulations presented here, ice cover is limited to disjoint mountain ice
caps during most of the glacial cycle. This agrees with earlier interpretations
of the geological evidence preceding the last glacial maximum. However, our
simulations produce persistent ice cover over the Skeena mountains during
the entire glacial cycle. At the time when a full-size Cordilleran ice sheet is
absent, the Skeena ice cap appears to be fed by the eastwards precipitation
intrusion through a topographic window in the Coast Mountains. The ice cap acts
as a nucleation centre at the onset of the last glacial maximum re-advance, and
may have contributed to the spectacular glacial imprint of the Skeena Mountains
erosional landscape.

Concerning the deglaciation phase, when most geological evidence is available,
none of the palaeo-temperature records used produces an ideal agreement between
model results and geological evidence. The timing of the last glacial maximum
and early deglaciation is generally best reproduced by forcing the model with
Antarctic records. However, the Younger Dryas re-advance is only reproduced
using temperature forcing from the GRIP record at places where it has been
documented. Nonetheless, the patterns of deglaciation are consistent between
both simulations driven by the GRIP and EPICA record. They show a rapid
deglaciation of the southern half of the ice
sheet, followed by unzipping of the Liard Lowland, and concentric retreat of
the ice margin towards the last palaeo-ice cover in the Skeena Mountains
during the early Holocene (10.9--9.5\,kyr).

One must keep in mind, however, that these results are only accurate for our
choices of ice-sheet model (PISM), surface mass balance model (PDD) and climate
forcing. Most importantly, our simplistic
palaeo-climate forcing does not include precipitation corrections in response
to the presence of an ice cover, likely leading to overestimated glacial
extent and volume. Nevertheless, our results identify that the largely
understudied Skeena Mountains are a key area to understanding glacial dynamics
of the Cordilleran ice sheet, highlighting the need for further geological
investigation of this region.

% Acknowledgements
%\begin{acknowledgements}
  % Author contributions
  %\hack{\noindent}\textit{Author contributions.}
%\end{acknowledgements}

% References
\bibliographystyle{copernicus}
\bibliography{refs/references.bib}
\newpage

% ----------------------------------------------------------------------
% Floats
% ----------------------------------------------------------------------

% tab:records
\begin{table*}[t]
  \caption{Palaeo-temperature proxy records and scaling parameters used to
           prepare temperature offset time-series used to force the ice sheet
           model through the last 120\,kyr. $T_{[32;22]}$ refers to the
           mean temperature anomaly during the period -32 to~-22~kyr after
           scaling.}
  \label{tab:records}
  {\begin{tabular}{lcccc}
    \tophline
    Record & Proxy & Scaling factor & $T_{[32;22]}$ & Original reference\\
    \middlehline
    GRIP     & \chem{\delta^{18}O} & ?\% & -5.8{\degree}C
        & \citet{Dansgaard.etal.1993} \\
    NGRIP    & \chem{\delta^{18}O} & ?\% & -6.1{\degree}C
        & \citet{Andersen.etal.2004} \\
    EPICA    & \chem{\delta^{18}O} & ?\% & -5.6{\degree}C
        & \citet{Jouzel.etal.2007} \\
    Vostok   & \chem{\delta^{18}O} & ?\% & -5.6{\degree}C
        & \citet{Petit.etal.1999} \\
    ODP~1012 & \chem{U^{K'}_{37}}  & ?\% & -5.8{\degree}C
        & \citet{Herbert.etal.2001} \\
    ODP~1020 & \chem{U^{K'}_{37}}  & ?\% & -5.8{\degree}C
        & \citet{Herbert.etal.2001} \\
    \bottomhline
  \end{tabular}}
  \belowtable{}
\end{table*}

% fig:locmap
\begin{figure}
  \includegraphics{locmap}
  \caption{Relief map of northern North America showing a reconstruction of the
           areas once covered by the Cordilleran (CIS), Laurentide (LIS),
           Innuitian (IIS) and Greenland (GIS) ice sheets during the last
           18\,\unit{\chem{^{14}C}\,kyr\,BP} (21.4\,cal\,kyr\,BP)
           \citep{Dyke.2004}. The rectangular box denotes the location of the
           modelling domain used in this study. Major mountain ranges covered
           by the ice sheet include the Alaska Range (AR), the Wrangell and
           St.-Elias mountains (WSE), the Selwyn and MacKenzie mountains (SMK),
           the Skeena Mountains, the Coast Mountains (CM), the Rocky Mountains
           (RM) and the North Cascades (NC). The background
           map consists of ETOPO1 \citep{Amante.Eakins.2009} and Natural Earth
           Data \citep{Patterson.Kelso.2014}.
           \todo{Mark palaeo-ice sheets and mountain ranges on the map.}}
  \label{fig:locmap}
\end{figure}

% fig:atm
\begin{figure}
  \includegraphics{atm}
  \caption{Monthly mean near-surface air temperature, precipitation and
           standard deviation of daily mean temperature for January and July
           months from the North American Regional Reanalysis (NARR)
           climatology, used to force the ice sheet model. Note the
           strong contrasts in seasonality, timing of the precipitation peak,
           and temperature variability over the model domain, notably between
           the maritime and continental regions.}
  \label{fig:atm}
\end{figure}

% fig:timeseries
\begin{figure*}
  \includegraphics{timeseries}
  \caption{Temperature offset time-series from ice core and sediment core
           records (Table~\ref{tab:records}) used as palaeo-climate forcing for
           the ice sheet model \textbf{(top)}, and modelled ice volume
           through the last 120\,kyr, expressed in meters of sea-level
           equivalent \textbf{(bottom)}. Gray spans indicate Marine Isotope
           Stages (MIS) according to a global compilation of benthic
           \chem{\delta^{18}O} records \citep{Lisiecki.Raymo.2005}. Hatched
           rectangles highlight modelled ice volume extrema corresponding to
           MIS~4 (61.9--55.4\,kyr), MIS~3 (52.2--45.6\,kyr), and
           MIS~2 (last glacial maximum, 29.5--16.9\,kyr). Dotted lines
           correspond to the GRIP and EPICA 6\,km-resolution runs.
           \todo{add grid lines to improve readability}}
  \label{fig:timeseries}
\end{figure*}

% fig:snapshots
\begin{figure*}
  \includegraphics{snapshots}
  \caption{Snapshots of modelled surface topography (500\,m contours)
           corresponding to the ice volume extrema indicated on
           Fig.~\ref{fig:timeseries}. Note the occurence of spatial similarities
           despite large differences in timing.}
  \label{fig:snapshots}
\end{figure*}

% fig:duration
\begin{figure*}
  \includegraphics{duration}
  \caption{Modelled duration of ice cover during the last 120\,kyr.
           Note the irregular colour scale. A contiguous ice cover spanning
           from the Alaska Range (AR) to the southern Coast Mountains (CM) and
           Rocky Mountains (RM) exists for about 29\,kyr in both
           simulations. A central
           ice cover persists over the Skeena Mountains (SM) during most of the
           simulation. On the other hand, the maximal extent of the ice sheet
           generally corresponds to relatively short durations of ice cover.}
  \label{fig:duration}
\end{figure*}

% fig:warmfrac
\begin{figure*}
  \includegraphics{warmfrac}
  \caption{Modelled fraction of warm-based ice cover during the ice-covered
           period. Note the dominance of warm-based conditions on the
           continental shelf and major glacial troughs of the coastal ranges.
           Hatches indicate areas that were covered by cold ice only.
           \todo{indicate location of the Skeena Mountains.}}
  \label{fig:warmfrac}
\end{figure*}

% fig:warmbase
\begin{figure*}
  \includegraphics{warmbase}
  \caption{Modelled duration of warm-based ice cover during the last
           120\,kyr. Long ice cover durations combined with basal
           temperatures at the pressure-melting point may explain the strong
           glacial erosional imprint of the Skeena Mountains (SM) landscape.
           Hatches indicate areas that were covered by cold ice only.
           \todo{indicate location of the Skeena Mountains.}}
  \label{fig:warmbase}
\end{figure*}

% fig:erosion
\begin{figure*}
  \includegraphics{erosion}
  \caption{Modelled cumulative basal displacement (integrand of basal velocity)
           over the last 120\,kyr.
           \todo{remove this figure (see footnote in text).}}
  \label{fig:erosion}
\end{figure*}

% fig:deglacseries
\begin{figure}
  \includegraphics{deglacseries}
  \caption{Temperature offset time-series from the GRIP and EPICA ice core
           records (Table~\ref{tab:records}) \textbf{(top)}, and modelled ice
           volume during the deglaciation, expressed in meters of sea-level
           equivalent \textbf{(bottom)}.
           \todo{add grid lines to improve readability}}
  \label{fig:deglacseries}
\end{figure}

% fig:deglac
\begin{figure*}
  \includegraphics{deglac}
  \caption{Modelled age of the last deglaciation. Areas where the MIS~4 glacial
           advance exceeded the last glacial maximum advanced are marked in
           green. Hatches denote re-advance of mountain-centred ice caps and
           and the decaying ice sheet between 14 and 10\,kyr., which is more
           pronounced in the GRIP-driven simulation.}
  \label{fig:deglac}
\end{figure*}

% fig:deglacshots-grip
\begin{figure*}
  \includegraphics{deglacshots-grip}
  \caption{Snapshots of modelled surface topography (200\,m contours)
           and surface velocity (colour mapping) from the GRIP simulation,
           corresponding to the last glacial ice volume maximum (-19.1 kyr) and
           the last deglaciation.}
  \label{fig:deglacshots-grip}
\end{figure*}

% fig:deglacshots-epica
\begin{figure*}
  \includegraphics{deglacshots-epica}
  \caption{Snapshots of modelled surface topography (200\,m contours)
           and surface velocity (colour mapping) from the EPICA simulation,
           corresponding to the last glacial ice volume maximum (-17.3 kyr) and
           the last deglaciation.}
  \label{fig:deglacshots-epica}
\end{figure*}


% fig:profiles-grip
\begin{figure}
  \includegraphics{profiles-grip}
  \caption{Modelled bedrock (black) and ice surface (blue) topography profiles
           during deglaciation (22.0--8.0\,kyr) in the GRIP 6\,km
           simulation, corresponding to the four transects indicated in
           Fig.~\ref{fig:deglac}.}
  \label{fig:profiles-grip}
\end{figure}

% fig:profiles-epica
\begin{figure}
  \includegraphics{profiles-epica}
  \caption{Modelled bedrock (black) and ice surface (blue) topography profiles
           during deglaciation (22.0--8.0\,kyr) in the EPICA 6\,km
           simulation, corresponding to the four transects indicated in
           Fig.~\ref{fig:deglac}.}
  \label{fig:profiles-epica}
\end{figure}

% fig:lastflow
\begin{figure*}
  \includegraphics{lastflow}
  \caption{Modelled directions of the last basal ice velocities. Hatches
           indicate areas that remain non-sliding throughout the simulation.
           Sliding grid cells were distinguished from non-sliding grid cells
           using a velocity threshold of 1\,\unit{m\,yr^{-1}}.}
  \label{fig:lastflow}
\end{figure*}

% ----------------------------------------------------------------------
\end{document}
\endinput
% ----------------------------------------------------------------------
