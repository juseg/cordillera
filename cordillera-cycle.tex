% cordillera-climate.tex
% =============================================================================

% Copernicus-like latex2rtf compatible
\newif\iflatextortf
\iflatextortf
% copernicus_rtf.tex
% ------------------

% Base class and packages
\documentclass{article}
\usepackage{color}
\usepackage{geometry}
\usepackage{graphicx}
\usepackage{setspace}
\onehalfspacing

% Replacements for bibtex commands
\newcommand{\citep}[1]{(\textcolor{blue}{#1})}
\newcommand{\citet}[1]{\textcolor{blue}{#1}}

% Replacements for Copernicus commands
\newcommand{\introduction}[0]{\section{Introduction}}
\newcommand{\conclusions}[0]{\section{Conclusions}}
\newcommand{\tophline}[0]{\hline}
\newcommand{\middlehline}[0]{\hline}
\newcommand{\bottomhline}[0]{\hline}
\newcommand{\unit}[1]{\ensuremath{\mathrm{#1}}}
\newcommand{\degree}[0]{\ensuremath{^{\circ}}}

% Ignore other Copernicus commands
\newcommand{\runningtitle}[1]{}
\newcommand{\runningauthor}[1]{}
\newcommand{\received}[1]{}
\newcommand{\correspondence}[1]{}
\newcommand{\pubdiscuss}[1]{}
\newcommand{\revised}[1]{}
\newcommand{\accepted}[1]{}
\newcommand{\published}[1]{}


\else

% Copernicus manuscript
\documentclass[tc, manuscript]{copernicus}

% Copernicus final print
%\documentclass[tc]{copernicus}

% Copernicus discussion paper
%\documentclass[tcd, hvmath, online]{copernicus}
\fi

% Coloured hyperlinks
\hypersetup{colorlinks, citecolor=blue}

% Figure directory
\graphicspath{{figures/}}

% My commands
%\newcommand{\renote}[1]{\footnote{\textbf{Comment reply}: #1}}
%\newcommand{\todo}[1]{\emph{[\textbf{Todo:} #1]}}
%\newcommand{\aref}[0]{\textbf{[ref.]}}

% Todo
% change Fraser Plateau to Interior Plateau

% =============================================================================
\begin{document}
\linenumbers
% =============================================================================

% Title
\title{Numerical simulations of the Cordilleran ice sheet
       through the last glacial cycle}

% Authors
\Author[1,2]{J.}{Seguinot}
\Author[2]{I.}{Rogozhina}
\Author[1]{A.~P.}{Stroeven}
\Author[3]{M.}{Margold}
\Author[1]{J.}{Kleman}
\runningauthor{J.~Seguinot et~al.}
\correspondence{J.~Seguinot (julien.seguinot@natgeo.su.se)}

% Running title
\runningtitle{Climate forcing for Cordilleran ice sheet simulations}

% Affiliations
\affil[1]{Department of Physical Geography and Quaternary Geology and the
          Bolin Centre for Climate Research, Stockholm University,
          Stockholm, Sweden}
\affil[2]{Helmholtz Centre Potsdam, GFZ German Research Centre for Geosciences,
          Potsdam, Germany}
\affil[3]{Department of Geography, Durham University, UK}

% For Copernicus
\received{}
\pubdiscuss{}
\revised{}
\accepted{}
\published{}

% Title
\firstpage{1}
\maketitle

% Abstract
\begin{abstract}

  Despite more than a century of geological observations, the Cordilleran ice
  sheet of North America remains poorly understood in terms of its former
  extent, volume and dynamics. Although geomorphological evidence is abundant,
  its complexity is such that whole ice-sheet reconstructions of advance and
  retreat patterns are lacking. Here we use a numerical ice sheet model
  calibrated against field-based evidence to attempt a quantitative
  reconstruction of the Cordilleran ice sheet history through the last glacial
  cycle. A series of simulations is driven by time-dependent temperature
  offsets from six proxy records located around the globe. Although this
  approach reveals large variations in model response to evolving climate
  forcing, all simulations produce two major glaciation events during
  marine oxygen isotope stages~4 (61.9--55.4\,ka) and~2
  (29.5--16.9\,ka). The timing of glaciation is
  better reproduced using temperature reconstructions from Greenland and
  Antarctic ice cores than from regional oceanic sediment cores. During most of
  the last glacial cycle, the modelled ice cover is discontinuous and
  restricted to high mountain areas. However, widespread precipitation over the
  Skeena Mountains favours the persistence of a central ice dome throughout the
  glacial cycle. It acts as a nucleation centre before the last glacial maximum
  and hosts the last remains of Cordilleran ice during the
  early Holocene (10.9--9.5\,ka).

\end{abstract}

% =============================================================================
\introduction
\label{sec:intro}
% =============================================================================

During the last glacial cycle, glaciers and ice caps of the North American
Cordillera have been more extensive than today. At the Last Glacial
Maximum (LGM),
a continuous blanket of ice, the Cordilleran ice sheet
\citep{Dawson.1888}, extended from the Alaska Range in the north to the
North Cascades in the south (Fig.~\ref{fig:locmap}).
In addition, it extended offshore, where it calved
into the Pacific Ocean, and merged with the western margin of its much larger
neighbour, the Laurentide ice sheet, east of the Rocky Mountains.

More than a century of exploration and geological investigations of the
Cordilleran mountains have led to many observations in support of the former
ice sheet
    \citep{Jackson.Clague.1991}.
Despite the lack of documented end moraines offshore, in the zone of confluence
with the Laurentide ice sheet, and in areas swept by the Missoula floods
    \citep{Carrara.etal.1996},
moraines that demarcate the northern and south-western margins provide key
constraints that allow reasonable reconstructions of maximum ice sheet extents
    (\citealp{Prest.etal.1968}; \citealp[Fig. 1.12]{Clague.1989};
     \citealp{Duk-Rodkin.1999};
     \citealp{Booth.etal.2003}; \citealp{Dyke.2004}).
The LGM Cordilleran ice sheet maximum extent was short-lived, as indicated by
field evidence from radiocarbon dating
    \citep{Clague.etal.1980, Clague.1985, Clague.1986, Porter.Swanson.1998,
           Menounos.etal.2008},
cosmogenic exposure dating
    \citep{Stroeven.etal.2010, Stroeven.etal.2014, Margold.etal.2014},
bedrock deformation in response to former ice loads
    \citep{Clague.James.2002, Clague.etal.2005},
and offshore sedimentary records
    \citep{Cosma.etal.2008, Davies.etal.2011}.
However, former ice thicknesses and, therefore, the ice sheet's contribution to
the LGM sea level low stand
    \citep{Carlson.Clark.2012, Clark.Mix.2002}
remain uncertain.

Our understanding of the Cordilleran glaciation history prior to the LGM is
even more fragmentary
    \citep{Barendregt.Irving.1998, Kleman.etal.2010, Rutter.etal.2012},
although it is clear that maximum glaciation of the Cordilleran ice sheet
predates the last glacial cycle
    \citep{Hidy.etal.2013}.
In parts of the Yukon Territory and Alaska, the distribution of tills
    \citep{Turner.etal.2013}
and dated glacial erratics
    \citep{Ward.etal.2007, Ward.etal.2008, Briner.Kaufman.2008,
           Stroeven.etal.2010, Stroeven.etal.2014}
indicate an extensive Marine Oxygen Isotope Stage (MIS)~4 glaciation,
yet it is not known whether other regions in the study area were affected.
Landforms in the interior regions include flow sets that are likely
older than the LGM
    \citep[Fig.~2]{Kleman.etal.2010},
but their absolute age remains uncertain.

In contrast, evidence for the deglaciation history of the Cordilleran
ice sheet since the LGM is considerable, albeit mostly at a regional scale.
Geomorphological evidence from south-central British Columbia indicates a rapid
deglaciation, including an early emergence of the elevated areas while thin,
stagnant ice still covered the surrounding lowlands
    \citep{Fulton.1967, Fulton.1991, Margold.etal.2011, Margold.etal.2013a}.
This model, although credible, may not apply in all areas of the Cordilleran
ice sheet
    \citep{Margold.etal.2013}.
Although solid evidence for late-glacial glacier re-advances have been found in
the Coast, Columbia and Rocky mountains
    \citep{Clague.etal.1997, Friele.Clague.2002, Friele.Clague.2002a,
           Kovanen.2002, Kovanen.Easterbrook.2002, Lakeman.etal.2008,
           Menounos.etal.2008},
it appears to be more sparse than for formerly glaciated regions surrounding
the North Atlantic
    \citep[e.g.,][]{Sissons.1979, Lundqvist.1987,
                    Ivy-Ochs.etal.1999, Stea.etal.2011},
thus allowing for considerable uncertainty concerning
the possibility of a regional late glacial cold reversal.

In general, the topographic complexity of the North American Cordillera and its
effect on glacial history have inhibited the reconstruction of ice sheet-wide
glacial advance and retreat patterns such as
those available for the Fennoscandian and Laurentide ice sheets
     \citep{Boulton.etal.2001, Dyke.Prest.1987, Dyke.etal.2003,
            Kleman.etal.1997, Kleman.etal.2010}.
Here, we use a numerical ice sheet model \citep{PISM-authors.2014},
calibrated against field-based evidence, to perform a quantitative
reconstruction of the Cordilleran ice sheet history through the last glacial
cycle, and
analyse some of the long-standing questions related to its evolution:

\begin{itemize}
  \item How much ice was locked in the Cordilleran ice sheet during the
    last glacial maximum?
  \item What was the scale of glaciation prior to the last glacial maximum?
  \item Which were the primary dispersal centres? Do they reflect stable or
    ephemeral configurations?
  \item How rapid was the last deglaciation? Did it include late glacial
    standstills or readvances?
\end{itemize}

Although numerical ice sheet modelling has been established as a useful tool to
improve our understanding of the Cordilleran ice sheet
    (\citealp[p.~227]{Jackson.Clague.1991}; \citealp{Robert.1991},
     \citealp{Marshall.etal.2000}),
the ubiquitously mountainous
topography of the region has presented two major challenges to its application.
First, only recent development of numerical ice sheet models and underlying
scientific computing tools \citep{Bueler.Brown.2009, Balay.etal.2014} has
allowed for high-resolution numerical modelling of glaciers and ice sheets on
mountainous terrain
over millenial time scales \citep[e.g.,][]{Golledge.etal.2012}. Second, the
complex
topography of the North American Cordillera also induces strong gradients in
seasonality and distinct patterns of precipition, thus requiring the use of
high-resolution climate forcing fields as an input to the ice sheet model
\citep{Seguinot.etal.2014}.

Because climate conditions over the last glacial cycle are subject to
considerable uncertainty, our palaeo-climate forcing is a simplistic approach
including temperature and precipitation fields derived from a
present-day atmospheric reanalysis \citep{Mesinger.etal.2006,
Seguinot.etal.2014} supplemented by lapse-rate corrections
and temperature offset time series. The latter are obtained by scaling six
different palaeo-temperature reconstructions from proxy records around the
globe, including two oxygen isotope records from Greenland ice cores
\citep{Dansgaard.etal.1993, Andersen.etal.2004}, two oxygen isotope
records from Antarctic ice cores \citep{Petit.etal.1999,Jouzel.etal.2007},
and two alkenone unsaturation index records from Northwest Pacific ocean
sediment cores \citep{Herbert.etal.2001}. We then proceed to compare the model
output to geological evidence and discuss the timing and extent of glaciation
and the patterns of deglaciation.


% =============================================================================
\section{Model setup}
\label{sec:model}
% =============================================================================

% -----------------------------------------------------------------------------
\subsection{Overview}
\label{sec:overview}
% -----------------------------------------------------------------------------

The simulations presented here were run using the Parallel Ice Sheet Model
(PISM, development version~11b0a7f and stable version~0.6.1), an open source,
finite difference, shallow ice sheet model \citep{PISM-authors.2014}. The model
requires input on basal topography, sea level, geothermal heat flux and
climate forcing. It computes the evolution of ice extent
and thickness over time, the thermal and dynamic
states of the ice sheet, and the associated lithospheric response.

Basal topography is derived from the ETOPO1 combined topography and bathymetry
dataset with a~resolution of 1\,arc-min \citep{Amante.Eakins.2009}. Sea level
is lowered as a function of time based on the Spectral Mapping Project
\citep[SPECMAP;][]{Imbrie.etal.1989} time scale. Geothermal heat flux
is applied as a constant value of 70\,\unit{mW\,m^{-2}} at 3\,km depth
(Sect.~\ref{sec:icedyn}). Surface mass balance is computed using a positive
degree-day (PDD) model (Sect.~\ref{sec:surface}). Climate forcing is
provided by a monthly climatology averaged from 1979 to 2000 from the North
American Regional Reanalysis \citep[NARR;][]{Mesinger.etal.2006},
perturbated by time-dependent and
lapse-rate temperature corrections (Sect.~\ref{sec:atm}).

Each simulation starts from assumed ice-free conditions at 120000 years ago
(120\,ka), and runs to the present. Our modelling domain of 1500 by 3000\,km
encompasses the entire area covered by the Cordilleran ice sheet at the LGM
(Fig.~\ref{fig:locmap}). The simulations were run on two distinct grids, using
a lower horizontal resolution of 10\,km, and a higher horizontal resolution of
6\,km. These computations were performed on 16 to 128 computing cores at the
Swedish National Supercomputing Centre.

% -----------------------------------------------------------------------------
\subsection{Ice thermodynamics}
\label{sec:icedyn}
% -----------------------------------------------------------------------------

Ice sheet dynamics are typically modelled using a combination of internal
deformation and basal sliding. PISM is a~shallow ice sheet model, which implies
that the balance of stresses is approximated based on their predominant
components. The Shallow Shelf Approximation (SSA) is used as a ``sliding law''
for the Shallow Ice Approximation (SIA) by adding velocity solutions of the
two approximation
\citep[Eqns.~7--9 and 15]{Bueler.Brown.2009, Winkelmann.etal.2011}.
Ice rheology depends on temperature and water content through an enthalpy
formulation \citep{Aschwanden.etal.2012}. Surface air temperature derived from
the climate forcing (Sect.~\ref{sec:atm}) provides the upper boundary
condition to the ice enthalpy model. Temperature is computed subglacially to
a~depth of 3\,km,
where it is conditioned by a lower boundary geothermal heat flux of
70\,\unit{mW\,m^{-2}}. Although this uniform value does not account for the
high spatial geothermal variability in the region
\citep{Blackwell.Richards.2004}, it is, on average, representative of available
heat flow measurements. In the low-resolution simulations, the vertical grid
consists of up to 51~enthalpy layers in the ice sheet and 31~temperature layers
in the bedrock. In the high-resolution simulations, up to 101~ice layers, and
61~bedrock layers are used.

A~pseudo-plastic sliding law,
\begin{equation}
    \label{eqn:pseudoplastic}
    \vec{\tau}_b = -\tau_c \frac{\vec{v}_b}{{v_{th}}^q\,|\vec{v}_b|^{1-q}} \,,
\end{equation}
relates the bed-parallel shear stresses, $\vec{\tau}_b$, to the sliding
velocity, $\vec{v}_b$ (Table~\ref{tab:params}).
The yield stress, $\tau_c$,
is modelled using the Mohr--Coulomb criterion,
\begin{equation}
   \tau_c = c_0 + N\,\tan{\phi} \,,
\end{equation}
where cohesion, $c_0$, is assumed to be zero. The friction angle, $\phi$,
varies from 15 to 45\degree as a piecewise-linear function of modern bed
elevation, with the lowest value occuring below modern sea level (0\,m above
sea level, m~a.s.l) and the highest value occuring above the generalised
elevation of the highest shorelines (200\,m~a.s.l), thus accounting for
a~weakening of till associated with the presence of marine sediments
(cf. \citealp{Martin.etal.2011}, \citealp[supplement]{Aschwanden.etal.2013},
\citealp{PISM-authors.2014}). Effective pressure, $N$,
is related to the ice overburden stress, $\rho gh$, and the modelled amount of
subglacial water, using a formula derived from laboratory experiments with till
extracted from an Antarctic ice stream \citep{Tulaczyk.etal.2000,
Bueler.Pelt.2014},
\begin{equation}
    N = \delta \rho gh \, 10^{(e_0/C_c) (1 - (W/W_{max}))} \,,
\end{equation}
where $\delta$ is chosen as 0.02, $e_0$ is a measured reference void ratio and
$C_c$ a measured compressibility coefficient (Table~\ref{tab:params}). The
amount of water at the base, $W$, varies from zero to $W_{max}=2$\,m, a
threshold above which instantaneous drainage is assumed.
Finally, the bedrock topography responds to ice load
following a bedrock deformation model that includes local isostasy,
elastic lithosphere flexure and viscous mantle deformation in the lower
half-space
\citep[Table~\ref{tab:params};][]{Lingle.Clark.1985,Bueler.etal.2007}.

% -----------------------------------------------------------------------------
\subsection{Surface mass balance}
\label{sec:surface}
% -----------------------------------------------------------------------------

Ice surface accumulation and ablation are computed from monthly mean
near-surface air temperature, $T_m$, monthly standard deviation of near-surface
air temperature, $\sigma$, and monthly precipitation, $P_m$, using
a~temperature-index model \citep[e.g.,][]{Hock.2003}.
Accumulation is equal to
precipitation when air temperatures are below 0\,\unit{{\degree}C}, and
decreases to zero linearly with temperatures between 0 and
2\,\unit{{\degree}C}. Ablation is computed from the number of positive
degree-days (PDD), defined as the integral of temperatures above
0\,\unit{{\degree}C} in one year.

The PDD computation accounts for stochastic temperature variations by assuming
a normal temperature distribution of standard deviation $\sigma$ aroung the
expected value $T_m$. It is expressed by an error-function formulation
\citep{Calov.Greve.2005},
\begin{equation}
    \label{eqn:calovgreve}
    \mathrm{PDD} = \int_{t_1}^{t_2} \mathrm{d}t
        \left[\frac{\sigma}{\sqrt{2\pi}}
                \exp\left({-\frac{T_{m}^2}{2\sigma^2}}\right)
              + \frac{T_{m}}{2} \, \mathrm{erfc}
                \left(-\frac{T_{m}}{\sqrt{2}\sigma}\right)\right] \,,
\end{equation}
which is numerically approximated using week-long sub-intervals. In order to
account for the effects of spatial and seasonal variations of temperature
variability \citep{Seguinot.2013}, $\sigma$ is computed from daily temperature
values from the NARR, after excluding variability associated
with the seasonal cycle itself \citep[cf.][]{Seguinot.Rogozhina.2014}.
Degree-day factors for snow and ice melt are derived from
mass-balance measurements on contemporary glaciers from the Coast Mountains and
Rocky Mountains in British Columbia
\citep[Table~\ref{tab:params};][]{Shea.etal.2009}.

% -----------------------------------------------------------------------------
\subsection{Climate forcing}
\label{sec:atm}
% -----------------------------------------------------------------------------

Climate forcing of the model consists of a present-day monthly climatology,
$\{T_{m0}, P_{m0}\}$, where temperatures are modified by offset time series,
${\Delta}T_{TS}$, and lapse-rate corrections, ${\Delta}T_{LR}$:
\begin{align}
    T_m(t, x, y) &= T_{m0}(x, y) + {\Delta}T_{TS}(t)
                    + {\Delta}T_{LR}(t, x, y) \,, \\
    P_m(t, x, y) &= P_{m0}(x, y) \,.
\end{align}

The present-day monthly climatology was computed from
near-surface air temperature and precipitation rate fields from the NARR,
averaged from 1979 to 2000. Modern climate of the
North American Cordillera is characterised by strong geographic variations in
temperature seasonality, timing of the maximum annual precipitation, and
daily temperature variability (Fig.~\ref{fig:atm}).
The use of NARR is motivated by the need for an accurate,
high-resolution precipitation forcing, as identified in a previous sensitivity
study \citep{Seguinot.etal.2014}.

Temperature offset time-series, ${\Delta}T_{TS}$, are derived from
palaeo-temperature proxy records from
the Greenland Ice Core Project \citep[GRIP,][]{Dansgaard.etal.1993}, the
North Greenland Ice Core Project \citep[NGRIP,][]{Andersen.etal.2004},
the European Project for Ice Coring in Antarctica \citep[EPICA,][]
{Jouzel.etal.2007}, the Vostok ice core \citep{Petit.etal.1999}, and Ocean
Drilling Program (ODP) sites 1012 and 1020, both located off the coast of
California \citep{Herbert.etal.2001}. Palaeo-temperatures from the GRIP and
NGRIP
records were calculated from oxygen isotope (\chem{\delta^{18}O}) measurements
using a quadratic equation \citep{Johnsen.etal.1995},
\begin{equation}
    {\Delta}T_{TS}(t) = -11.88 [\chem{\delta^{18}O}(t)
                               -\chem{\delta^{18}O}(0)]
                        -0.1925[\chem{\delta^{18}O}(t)^2
                                -\chem{\delta^{18}O}(0)^2] \,,
\end{equation}
while temperature reconstructions from Antarctic and oceanic cores were
provided as such. All records were scaled linearly (Table~\ref{tab:records}) in
order to simulate comparable ice extents at the LGM  (Table~\ref{tab:extrema})
and realistic outlines (Fig.~\ref{fig:snapshots}).

Finally, lapse-rate corrections, ${\Delta}T_{LR}$, are computed as a function
of ice surface elevation, $s$, using the NARR surface geopotential height
invariant field as a reference topography, $b_{ref}$:
\begin{align}
    {\Delta}T_{LR}(t, x, y) &= -\gamma [s(t, x, y)-b_{ref}] \\
                            &= -\gamma [h(t, x, y)+b(t, x, y)-b_{ref}],
\end{align}

thus accounting for the evolution of ice thickness, ${h=s-b}$,
on the one hand, and for differences between the basal topography of the ice
flow model, $b$, and the
NARR reference topography, $b_{ref}$, on the other hand. All simulations use an
annual temperature lapse rate of $\gamma = 6\,\unit{K\,km^{-1}}$.
In the rest of this paper, we refer to different model runs by the name of the
proxy record used for the palaeo-temperature forcing.

% =============================================================================
\section{Sensitivity to climate forcing time-series}
\label{sec:results}
% =============================================================================

% -----------------------------------------------------------------------------
\subsection{Evolution of ice volume}
% -----------------------------------------------------------------------------

Despite large differences in the input climate forcing
(Fig.~\ref{fig:timeseries}, upper panel), model output presents consistent
features that can be observed across the range of forcing data used. In all
simulations, modelled ice volumes remain relatively low during most of the
glacial cycle, except during two major glacial events which occur between 61.9
and 55.4\,ka during MIS~4, and between 29.5 and 16.9\,ka during MIS~2
(Fig.~\ref{fig:timeseries}, lower panel). An ice volume minimum is
consistently reached between 52.2 and 45.6\,ka during MIS~3. However, the
magnitude and precise timing of these three events depend significantly on the
choice of proxy record used for time-dependent climate forcing
(Table~\ref{tab:extrema}).

Simulations forced by the Greenland ice core palaeo-temperature
records (GRIP, NGRIP) produce the highest variability in modelled ice volume
throughout the last glacial cycle. In contrast, simulations driven by oceanic
(ODP~1012, ODP~1020) and Antarctic (EPICA, Vostok) palaeo-temperature records
generally result in smaller
modelled ice volumes during MIS~4 and larger ice volumes during MIS~3.
Hence, they produce lower ice volume variability
throughout the simulation length. The NGRIP climate forcing is the only one
that results in a larger ice volume during MIS~4 than during MIS~2.

While simulations driven by the GRIP and the two Antarctic palaeo-temperature
records attain a last ice volume maximum between 19.5 and 16.9\,ka, those
informed by the NGRIP and the two oceanic palaeo-temperature records attain their
maximum ice volumes thousands of years earlier. Moreover, the ODP~1012 run
yields a rapid deglaciation of the modelled area prior to 20\,ka. The ODP~1020
simulation predicts an early maximum in ice volume at 29.5\,ka, followed by
slower deglaciation than modelled using the other palaeo-temperature records.
Finally, model runs forced by Antarctic palaeo-temperature records result in a
rapid and uninterrupted deglaciation after the last glacial maximum, whereas the
simulation driven by the GRIP palaeo-temperature record also results in a rapid
deglaciation but in three steps, separated by two ice volume standstills
(Fig.~\ref{fig:timeseries}).

% -----------------------------------------------------------------------------
\subsection{Extreme configurations}
% -----------------------------------------------------------------------------

Despite such differences in the timing of attained volume extrema
(Table~\ref{tab:extrema}), all the model runs show relatively consistent
patterns of glaciation. During MIS~4, all simulations produce an extensive ice
sheet, covering an area of at least half of that attained during MIS~2
(Table~\ref{tab:extrema}; Fig.~\ref{fig:snapshots}, upper panels).
Corresponding maximum ice volumes also differ significantly between model runs,
and vary between 3.88 and~8.69\,m sea level equivalents (m.~s.l.e;
Table~\ref{tab:extrema}).

During MIS~3 ice volume minima, a central ice cap persists over the Skeena
Mountains (Fig.~\ref{fig:snapshots}, middle panels). Although this ice cap is
present in all simulations, its dimensions depend sensitively on the choice of
the applied palaeo-temperature record. Modelled ice volume minima spread over a
wide range between 0.89 and 2.44\,m~s.l.e. (Table~\ref{tab:extrema}).

Modelled ice sheet geometries during the LGM (MIS~2; Fig.~\ref{fig:snapshots},
lower panels) invariably include a ca. 1500\,km-long central divide located
about 3500\,m~a.s.l. along the spine of the Rocky Mountains. Although the
similatity of modelled ice extents is a direct result from the choice of scaling
factors applied to different palaeo-temperature proxy records
(Table~\ref{tab:records}), it is interesting to note that modelled maximum ice
volumes also fall within a tight range of 8.24 to 8.66\,m~s.l.e.
(Table~\ref{tab:extrema}).


% =============================================================================
\section{Comparison to the geologic record}
\label{sec:discussion}
% =============================================================================

Large variations in the model response reveal its sensitivity to evolving
climate forcing. Nevertheless, geological evidence of former glaciations
provide a basis for validation of our runs, while the results from numerical
modelling can perhaps help to analyse some of the complexity of this evidence.
In this section, we compare model output to the geologic record, in terms of
timing and configuration of the maximum stages, location and durability of the
major nucleation centres, and patterns of ice retreat during the last
deglaciation.

% -----------------------------------------------------------------------------
\subsection{Glacial maxima}
% -----------------------------------------------------------------------------

\subsubsection{Timing of glaciation}
\label{sec:timing}

Independently of the palaeo-temperature records
used to force the ice sheet model, our simulations consistently produce two
glacial maxima during the last glacial cycle. The first maximum configuration
is obtained during MIS~4 (61.9--55.4\,ka) and the second during MIS~2
(29.5--16.9\,ka; Figs.~\ref{fig:timeseries}, \ref{fig:snapshots};
Table~\ref{tab:extrema}). These events broadly correspond in timing to the
Gladstone
(MIS~4) and McConnell (MIS~2) glaciations documented by geological evidence for
the northern sector of the Cordilleran ice sheet
    \citep{Duk-Rodkin.etal.1996, Ward.etal.2007,
           Stroeven.etal.2010, Stroeven.etal.2014},
and to the Fraser Glaciation (MIS~2) documented for its southern sector
    \citep{Porter.Swanson.1998, Margold.etal.2014}.
There is patchy stratigraphical evidence for glaciations older than the Fraser
Glaciation \citep{Clague.Ward.2011} in British Columbia, but their extent and
timing are still highly conjectural
    \citep[perhaps MIS~4 or early MIS~3; e.g.,][]{Cosma.etal.2008}.

The exact timing of modelled MIS~2 maximum ice volume depends strongly on the
choice of applied palaeo-temperature record, which allows for a more in-depth
comparison with geological evidence for the timing of maximum Cordilleran ice
sheet extent. In the Puget lowland (Fig.~\ref{fig:locmap}), the LGM advance of
the southern Cordilleran ice sheet margin has been constrained by radiocarbon
dating on wood between 17.4 and 16.4\,\unit{\chem{^{14}C}\,cal\,ka}
\citep{Porter.Swanson.1998}. These dates are consistent with radiocarbon dates
from the offshore sedimentary record, which reveals
an increase of glaciomarine sedimentation between 19.5 and
16.2\,\unit{\chem{^{14}C}\,cal\,ka} \citep{Cosma.etal.2008}. Radiocarbon
dating of the northern Cordilleran ice sheet margin is much less constrained
but straddles presented constraints from the southern margin. However,
cosmogenic exposure dating
places the timing of maximum CIS extent during the McConnell glaciation close
to 17\,ka \citep{Stroeven.etal.2010, Stroeven.etal.2014}.

Among the simulations presented here, only those forced with the GRIP, EPICA
and Vostok palaeo-temperature records yield Cordilleran ice sheet maximum
extents that may be compatible with these field constraints
(Fig.~\ref{fig:timeseries}, lower panel; Table~\ref{tab:extrema}).
Simulations driven by the NGRIP, ODP~1012 and ODP~1020
palaeo-temperature records, on the contrary, yield MIS~2 maximum Cordilleran
ice sheet extents that pre-date field-based constraints by several thousands of
years. Concerning the simulations driven by oceanic records, this early
deglaciation is caused by an early warming present in the alkenone
palaeo-temperature reconstructions (Fig.~\ref{fig:timeseries}, upper panel;
\citealp[Fig.~3]{Herbert.etal.2001}). However, this
early warming is a local effect, corresponding to a weakening of the California
current \citep{Herbert.etal.2001}. The California current, driving cold
waters southwards along the south-western coast of North America,
has been shown to have weakened during each peak global glaciation (in SPECMAP)
during the past 550\,ka, including the LGM, resulting in paradoxically warmer
sea-surface temperatures at the locations of the ODP~1012 and ODP~1020 sites
\citep{Herbert.etal.2001}.

Because most of the marine margin of the Cordilleran ice sheet terminated in a
sector of the Pacific Ocean unaffected by variations in the California current,
it probably remained insensitive to this local phenomenum. However, the above
paradox
illustrates the complexity of ice-sheet feedbacks on regional climate, and
demonstrates that, although located in the neighbourhood of the modelling
domain, the ODP~1012 and ODP~1020 palaeo-temperature records cannot be
used as a realistic forcing to model the Cordilleran ice sheet
through the last glacial cycle.

Similarly, the simulation using the NGRIP palaeo-temperature record depicts an
early onset of deglaciation (Fig.~\ref{fig:timeseries}) following its last
glacial volume maximum attained about 6\,ka earlier (-22.9\,ka,
Table~\ref{tab:extrema}) than dated evidence of the last glacial maximum
advance. Hence, we focus the rest of our analysis on simulations forced by
palaeo-temperature records from the GRIP and EPICA ice cores that appear to
produce the most realistic reconstructions of regional glaciation history among
all records tested. To allow for a more
detailed comparison against the geologic record, these two simulations were
re-run using the high-resolution grid (Fig.~\ref{fig:timeseries}, lower panel,
dotted lines).


\subsubsection{Ice configuration during MIS~2}
\label{sec:mis2}

During maximum glaciation, both simulations position the main meridional ice
divide over
the western flank of the Rocky Mountains (Fig.~\ref{fig:snapshots}, lower
panels; Fig.~\ref{fig:mis2}). This result appears to contrast with
palaeoglaciological reconstructions for central and southern British Columbia
that inferred the ice divide in a more westerly position,
over the western margin of the Fraser Plateau
\citep{Ryder.etal.1991,
Stumpf.etal.2000, Kleman.etal.2010, Clague.Ward.2011, Margold.etal.2013a}.
These idincate that a latitudinal saddle connected ice dispersal centres in
the Columbia Mountains with the main ice divide \citep{Ryder.etal.1991,
Kleman.etal.2010, Clague.Ward.2011, Margold.etal.2013a}. A latitudinal saddle
does indeed feature in our modelling results, however, in an inverse
configuration between the main ice divide over the Columbia Mountains and a
secondary divide over the southern Coast Mountains (Fig.~\ref{fig:mis2}).

This could reflect the fact that our model does not include
feedback mechanisms between ice sheet topography and the regional climate.
Firstly, during the build-up phase preceding the LGM, rapid accumulation over
the Coast Mountains enhanced the topographic barrier formed by these mountain
ranges, which likely resulted in a decrease of precipitation and, therefore,
accumulation in the interior. Secondly, latent warming of the moisture-depleted
air parcels flowing over this enhanced topography could have resulted in an
inflow of warmer air over the eastern flank of the ice sheet, increasing melt
along the advancing margin \citep[cf.][]{Langen.etal.2012}. Because these two
processes, both with a tendency to limit of ice-sheet growth, are absent from
our model, the
eastern margin of the ice sheet and the position of the main meridional ice
divide are certainly biased towards the east in our simulations
\citep{Seguinot.etal.2014}.

However, field-based palaeoglaciological reconstructions have
struggled to reconcile the more westerly-centred ice divide in south-central
British Columbia with evidence in the Rocky Mountains and beyond, that the
Cordilleran ice sheet invaded the western Interior Plains, where it merged with
the southwestern margin of the Laurentide ice sheet and was deflected to the
south \citep{Jackson.etal.1997, Bednarski.Smith.2007, Kleman.etal.2010,
Margold.etal.2013, Margold.etal.2013a}. Ice geometries from our model runs do
not have this problem, because the position and elevation of the ice divide
results in significant ice drainage across the Rocky Mountains at the LGM
(Fig.~\ref{fig:mis2}).


\subsubsection{Ice configuration during MIS~4}
\label{sec:mis4}

The modelled ice sheet configurations corresponding to ice volume maxima during
MIS~4 are more sensitive to the choice of atmospheric forcing than those
corresponding to ice volume maxima during MIS~2 (Fig.~\ref{fig:snapshots},
upper panels;
Fig.~\ref{fig:mis4}). The GRIP simulation (Fig.~\ref{fig:mis4}, left panel)
results in a modelled maximum ice sheet extent that closely resembles that
obtained
during MIS~2, with the only major difference being slightly less extensive
northern and eastern sectors. In contrast, the EPICA simulation produces a lower
ice volume maximum (Fig.~\ref{fig:timeseries}), which translates in the
modelled ice sheet geometry into a significantly reduced southern sector, more
restricted ice cover in northern and eastern sectors, and generally lower
ice surface elevations in the interior (Fig.~\ref{fig:mis4}, left panel).

In the GRIP simulation, some sections of the MIS~4 maximum ice margin are
slightly more extensive than their MIS~2 counterparts. This is notably the case
in the Puget Lowland and along parts of the ice margin in the marine sectors,
on the north slope of the Alaska Range and in Yukon Territory. However, these
differences
are small and generally correspond to a few grid cells only. Although such
results may be corroborated by field evidence in some places
\citep[e.g.][]{Ward.etal.2007}, considering the simplicity of the applied
climate forcing, modelled differences in extent are not demonstrably
significant.


% -----------------------------------------------------------------------------
\subsection{Nucleation centres}
% -----------------------------------------------------------------------------

\subsubsection{Transient ice sheet states}

Palaeo-glaciological reconstructions are generally more robust
for maximum ice sheet extents and late ice sheet configurations than for
intermediate or minimum ice sheet extents and older ice sheet configurations
\citep{Kleman.etal.2010}. However,
these maximum stages are, by nature, extreme configurations, which do not
necessarily represent the dominant patterns of glaciation throughout the period
of ice cover \citep{Porter.1989, Kleman.Stroeven.1997}.

For the Cordilleran ice sheet, geological evidence from radiocarbon dating
    \citep{Clague.etal.1980, Clague.1985, Clague.1986, Porter.Swanson.1998,
           Menounos.etal.2008},
cosmogenic exposure dating
    \citep{Stroeven.etal.2010, Stroeven.etal.2014, Margold.etal.2014},
bedrock deformation in response to former ice loads
    \citep{Clague.James.2002, Clague.etal.2005},
and offshore sedimentary records
    \citep{Cosma.etal.2008, Davies.etal.2011}
indicate that the LGM maximum extent was short-lived. To compare this finding
to our simulations, we use numerical modelling output
to compute durations of ice cover throughout the last glacial
cycle (Fig.~\ref{fig:duration}).

The resulting maps show that, during most of the glacial cycle, modelled ice
cover is restricted to disjoint ice caps centred on major mountain ranges of
the North American Cordillera (Fig.~\ref{fig:duration}, blue areas). A
2500\,km-long continuous expanse of ice, extending from the Alaska Range in the
northeast to the Rocky Mountains in the southwest, is only in operation
for at most 29\,ka, which is less than 25\,\% of a glacial cycle
(Fig.~\ref{fig:duration}, hatched areas). However,
except for its marine margin and the northern foothills of the Alaska Range,
the maximum extent of the ice sheet is attained for an even shorter period
of time of few thousand years (Figs.~\ref{fig:duration}, red areas). This
result illustrates that the maximum extents of the modelled ice sheet during
MIS~4 and MIS~2 were both short-lived and therefore out of balance with
contemporary climate.

A notable exception to the transient character of other sectors of the
Cordilleran ice sheet
is the northern slope of the Alaska Range, where modelled glaciers are confined
to its foothills during the entire simulation period (Fig.~\ref{fig:duration},
AR). This apparent insensitivity of modelled glacial extent to temperature
fluctuations results from a combination regimes of low precipitation, high summer
temperature and large temperature standard deviation for this sector of the
modelling domain (Fig.~\ref{fig:atm}). This result could potentially explain
the local distribution of glacial deposits, which indicates that these glaciers
have remained small throughout the Pleistocene \citep{Kaufman.Manley.2004}.

\subsubsection{Major ice-dispersal centres}

It is generally believed that the Cordilleran ice sheet formed by the
coalescence of several mountain-centred ice caps \citep{Davis.Mathews.1944}.
In our simulations, major ice-dispersal centres, visible on the modelled ice
cover duration maps (Fig.~\ref{fig:duration}), are located over the Coast
Mountains (CM), the Columbia and Rocky mountains (CRM), the Skeena Mountains
(SM), and the Selwyn and Mackenzie Mountains (SMKM). Although the Coast,
Columbia, Rocky and Skeena mountains (CM, CRM, SM) are covered by mountain
glaciers for most of the last glacial cycle, providing durable nucleation
centres for an ice sheet, this is not the case for the Selwyn and
Mackenzie mountains (SMKM), where ice cover on the highest peaks is limited to a
fraction of the last glacial cycle. In other words, the Selwyn and Mackenzie
mountains appear as a secondary ice-dispersal centre which ``activates'' only
during the coldest periods of the last glacial cycle. The Northern Rocky
Mountains (Fig.~\ref{fig:duration}, NRM), do not act as a nucleation centre,
but rather as a pinning point for the Cordilleran ice sheet margin coming from
the west.

Perhaps the most striking feature displayed by the distributions of modelled
ice cover is the persistence of the Skeena Mountains ice cap throughout the
entire last glacial cycle and its predominance over the other ice-dispersal
centres
(Figs.~\ref{fig:snapshots} and~\ref{fig:duration}). Regardless of the applied
forcing, this ice cap appears to survive MIS~3 (Fig.~\ref{fig:snapshots},
middle panels), and serves as a nucleation centre at the onset of the glacial
readvance towards the LGM (MIS~2). The importance of residual ice for North
American glacial history leading up to the LGM has been illustrated by the
MIS~3 residual ice bodies in northern and eastern Canada as nucleation centres
for a much more extensive MIS~2 configuration of the Laurentide ice sheet
\citep{Kleman.etal.2010}.

The presence of a Skeena Mountains ice cap during most of the last glacial
cycle can be explained by a currently more eastwards-spread distribution of
winter precipitation there than along other transects of the
modelling domain (Fig.~\ref{fig:atm}). Along most of the north-western coast of
North America, coastal mountain ranges form a pronounced topographic barrier
for westerly winds, capturing atmospheric moisture in the form of orographic
precipitation, and resulting in arid interior lowlands. However, near the
centre of our modelling domain, this barrier is less pronounced than elsewhere,
allowing westerly winds to carry moisture
further inland, until it is captured by the extensive Skeena Mountains in
north-central British Columbia, thus resulting in a more widespread
distribution of winter precipitation.

\subsubsection{Erosional imprint on the landscape}

A correlation is observed between the modelled duration of warm based ice cover
(Fig.~\ref{fig:warmbase}) and the degree of glacial modification of the
landscape (mainly in terms of the development of deep glacial valleys and
troughs). We find evidence for this on both the southwestern and northeastern
slopes of the southern Coast Mountains, on the western slopes of the Columbia
Mountains, on the western and eastern slopes of the northern Coast Mountains
and the Saint-Elias Mountains, and radiating off the Skeena Mountains
(Figs.~\ref{fig:duration} and \ref{fig:warmbase};
\citealp[Fig.~2]{Kleman.etal.2010}).
The Skeena Mountains, for example, indeed bear a strong glacial imprint that
indicates ice drainage in a system of distinct glacial troughs emanating in a
radial pattern from the centre of the mountain range
\citep[Fig.~2]{Kleman.etal.2010}. We suggest that
persistent ice cover (Fig.~\ref{fig:duration}) associated with basal ice
temperatures at the pressure-melting point (Figs.~\ref{fig:warmfrac}
and~\ref{fig:warmbase}) explains the large-scale glacial erosional imprint on
the landscape. A well-developed network of glacial valleys west of
the Mackenzie Mountains (\citealp[Fig.~2]{Kleman.etal.2010}; \citealp[Fig.~8]
{Stroeven.etal.2010}) is modelled to have hosted warm-based ice
(Fig.~\ref{fig:warmfrac}). However, because it has only been glaciated for a
short fraction
of the last glacial cycle, (Fig.~\ref{fig:duration}) according to our
results, this perhaps appears indicate that
the observed landscape pattern originates from multiple glacial cycles and
witnesses an increased relative importance of this ice dispersal centre prior
to the Late Pleistocene \citep[cf.][]{Ward.etal.2008, Demuro.etal.2012}.

The modelled distribution of warm-based ice cover (Figs.~\ref{fig:warmbase}
and~\ref{fig:warmfrac}) is inevitably affected by our assumption of a constant,
70\,\unit{mW\,m^{-2}} geothermal heat flux at 3\,km depth
(Sect.~\ref{sec:icedyn}). However, both the Skeena Mountains and the area
west of the Mackenzie Mountains are areas of higher-than-average geothermal
heat flux \citep{Blackwell.Richards.2004}. We can therefore expect longer
durations of warm-based ice cover for these areas if we were to include
spatially variable geothermal forcing in our Cordilleran ice sheet simulations.


% -----------------------------------------------------------------------------
\subsection{The last deglaciation}
% -----------------------------------------------------------------------------

\subsubsection{Pace and patterns of deglaciation}

Similarly to other glaciated regions, most glacial traces in the North American
Cordillera relate to the last few millennia of glaciation, because most of the
older evidence has been overprinted by warm-based ice retreat
during the deglaciation \citep{Kleman.1994, Kleman.etal.2010}. From a
numerical modelling perspective, phases of glacier retreat are more challenging
than phases of growth, because they involve more rapid fluctuations of the ice
margin, increased flow velocities and longitudinal stress gradients, and poorly
understood hydrological processes. The latter are typically included in the
models through simple parametrisations \citep[e.g.][]{Clason.etal.2012,
Clason.etal.2014, Bueler.Pelt.2014}, if included at all. However, next after a
matching of modelled ice margin positions with mapped end moraines
\citep{Napieralski.etal.2007}, indicative of maximum ice sheet extents
(Sect.~\ref{sec:mis2} and~\ref{sec:mis4}), geomorphologically-based
reconstructions of patterns of of ice sheet retreat during the last
deglaciation provide the second best source of evidence for the validation of
our simulations.

In the North American Cordillera, the presence of lateral meltwater channels
at high elevation \citep{Margold.etal.2011, Margold.etal.2013a,
Margold.etal.2014}, and abundant esker systems at low elevation
\citep{Burke.etal.2012, Burke.etal.2012a, Perkins.etal.2013,
Margold.etal.2013}
indicate that abundant meltwater was produced over the entire ice sheet surface
during deglaciation. The southern and northern margins of the
Cordilleran ice sheet reached their last glacial maximum extent around
17\,ka \citep[Sect.~\ref{sec:timing};][]{Porter.Swanson.1998, Cosma.etal.2008,
Stroeven.etal.2010, Stroeven.etal.2014}, which we take as a limiting age for
the onset of ice retreat. The timing of final deglaciation is less well
constrained, but recent cosmogenic dates from north-central British Columbia
indicate that a seizable ice cap emanating from the central Coast Mountains or
the Skeena Mountains persisted into the Younger Dryas chronozone, at least
until 12.4\,ka \citep{Margold.etal.2014}.

In our simulations, the timing of peak ice volume during the last glacial
maximum and the pacing of deglaciation depend critically on the choice of
climate forcing (Table~\ref{tab:extrema}, Figs.~\ref{fig:timeseries}
and~\ref{fig:deglacseries}). Adopting the EPICA
climate forcing yields peak ice volume at 17.3\,ka and an uninterrupted
deglaciation until 10.6\,ka (Fig.~\ref{fig:deglacseries}, lower panel, red
curves). The simulation driven by the GRIP palaeo-temperature record yields
peak ice volume at 19.1\,ka and a deglaciation interrupted by two ice volume
standstills until 9.4\,ka. The first interruption occurs between 16.6 and
14.5\,ka, and the second between 12.6 and 11.6\,ka
(Fig.~\ref{fig:deglacseries}, lower panel, blue curve). Hence, the two model
runs, while similar in overall timing compared to runs with other climate
drivers, differ in detail in that the EPICA depicts peak glaciation almost
2\,ka later than the GRIP, in closer agreement with dated maximum extents, and
shows a faster, uninterrupted deglaciation which yields ice-free conditions
more than 1\,ka earlier.

Modelled patterns of ice sheet retreat are relatively consistent between the
two simulations (Figs.~\ref{fig:deglacshots} and.~\ref{fig:deglac}). The
southern sector of the modelling domain, including the Puget Lowland, the Coast
and Rocky mountains, and the Fraser Plateau, becomes
completely deglaciated by 12\,ka, whereas a significant ice cover remains over
the Skeena, Selwyn, Mackenzie, Wrangell and Saint-Elias mountains in the
northern sector of the modelling domain. After 12\,ka, deglaciation continues
to proceed across the Liard Lowland with a radial ice margin retreat towards
the surrounding mountain ranges, consistent with the regional melt water record
of the last deglaciation \citep{Margold.etal.2013}. Remaining ice continues to
decay by retreating towards the Selwyn and Skeena mountains. The last remnants
of the Cordilleran ice sheet finally disappear from the Skeena Mountains around
10.6\,ka (EPICA) and 9.4\,ka (GRIP).


\subsubsection{Late-glacial readvance}

The possibility of late glacial readvances in the North American Cordillera has
been debated for some time \citep{Osborn.Gerloff.1997}, and locally these have
been reconstructed and dated. Radiocarbon-dated end moraines in the Fraser and
Squamish valleys, off the southern tip of the Coast Mountains, indicate
consecutive glacier maxima, or standstills while in overall retreat, one of
which corresponds to the Younger Dryas chronozone \citep{Clague.etal.1997,
Friele.Clague.2002, Friele.Clague.2002a, Kovanen.2002,
Kovanen.Easterbrook.2002}. Although most of these moraines characterise
independent valley glaciers, that were probably disconnected from the waning
Cordilleran ice sheet, the Finlay River
area of the northern Rocky Mountains presents a different kind of evidence.
There, sharp-crested moraines indicate a late-glacial readvance of local alpine
glaciers and, more importantly, their interaction with larger, lingering
remnants of the main body of the Cordilleran ice sheet in the valleys
\citep{Lakeman.etal.2008}. Additional evidence for late-glacial
alpine glacier readvances includes moraines in the eastern Coast Mountains,
Rocky Mountains and the Columbia Mountains \citep{Osborn.Gerloff.1997,
Menounos.etal.2008}.

Although further work is needed to constrain the timing of the late-glacial
readvance, to assess its extents and geographical distribution, and to identify
the potential climatic triggers \citep{Menounos.etal.2008}, it is interesting
to note that the simulation driven by the GRIP record produces a late-glacial
readvance in the Coast Mountains, Rocky Mountains and the Finlay River area,
corresponding to where it has been identified in the geological record
(Fig.~\ref{fig:deglac}, left panel). In addition to matching the location of
local readvances, the GRIP-driven simulation shows that larger ice bodies
emanating from the Skeena and Mackenzie mountains may still have existed at the
time of this late-glacial readvance.
In contrast, the EPICA-driven simulation produces a nearly-continuous
deglaciation with only a tightly restricted glacial readvance
(Fig.~\ref{fig:deglac}, right panel).


\subsubsection{Deglacial flow directions}

Because a general tenant in glacial geomorphology is that the majority of
landforms (lineations and eskers) are part of the deglacial envelope
\citep[terminology from][]{Kleman.etal.2006}, having been formed
close inside the retreating margin of ice sheets \citep{Boulton.Clark.1990,
Kleman.etal.1997, Kleman.etal.2010}, we present maps of basal flow directions
immediately preceding deglaciation or at the time of cessation of sliding
inside a cold-based retreating margin (Fig.~\ref{fig:lastflow}). The
modelled deglacial flow patterns appear consistent between the GRIP and EPICA
simulations. They depict an active ice sheet retreat in the peripheral areas,
followed by stagnant ice decay in some of the interior regions. Several parts
of the modelling domain do not experience any basal sliding throughout the
deglaciation phase (Fig.~\ref{fig:lastflow}, hatched areas). This notably
includes parts of the Fraser Plateau, major
portions of the Alaskan sector of the ice sheet, and a tortuous ribbon running
from the Northern Rocky Mountains over the Skeena and Selwyn Mountains and into
the Mackenzie Mountains.

Patterns of glacial lineations formed in the northern and southern sectors of
the Cordilleran ice sheet (\citealp{Prest.etal.1968};
\citealp[Fig.~1.12]{Clague.1989}; \citealp[Fig.~2]{Kleman.etal.2010}) show
similarities with the patterns of deglacial ice flow from numerical modelling
(Fig.~\ref{fig:lastflow}). In the northern half
of the modelling domain, modelled deglacial flow directions depict an
active downhill flow as the last remnants of the ice sheet retreat towards
mountain ranges. Converging deglacial flow patterns in the Liard
Lowland, for instance (Fig.~\ref{fig:lastflow}), closely resemble the pattern
indicated by glacial lineations \citep[Fig.~2]{Margold.etal.2013}.

However, on the Fraser Plateau, both
simulations predict negligible basal sliding during deglaciation
(Fig.~\ref{fig:lastflow}), but an impressive set of glacial lineations
indicate a substantial eastward flow component of the Cordilleran ice sheet
\citep{Prest.etal.1968, Kleman.etal.2010}.
The Fraser lineation set could thus present a smoking gun for the
reliability of the presented model results. One explanation for the incongruent
results could therefore be that missing feedback mechanisms between ice sheet
topography and regional climate position the modelled ice divide of the LGM ice
sheet too far to the east (Sect.~\ref{sec:mis2}; Fig.~\ref{fig:mis2};
\citealp{Seguinot.etal.2014}). A more westerly-located LGM ice divide would
certainly result in a different deglacial flow pattern over the Interior
Plateau. However, a more westerly-positioned LGM ice divide would certainly be
associated with a thinner ice sheet than that modelled here. Decreased ice
thickness would not promote warm-based conditions but, on the
opposite, enlarge the region of negligible basal sliding
(Fig.~\ref{fig:lastflow}). Thus, another explanation for the incongruent
results could be that the Fraser Plateau lineation system predates
deglaciation, as perhaps indicated by some eskers that appear incompatible with
these glacial lineations \citep[Fig.~9]{Margold.etal.2013a}. Alternatively,
local geothermal heat associated with volcanic activity on the Fraser Plateau
could have triggered the basal sliding
\citep[cf. Greenland ice sheet,][]{Fahnestock.etal.2001}.

The modelled deglaciation of the Fraser Plateau
consists of a rapid northwards retreat (Fig.~\ref{fig:deglac}) of
southwards-flowing non-sliding ice lobes (Fig.~\ref{fig:lastflow}) positioned
in-between deglaciated mountain ranges
(Figs.~\ref{fig:profiles-grip} and~\ref{fig:profiles-epica}). This result
appears compatible with the prevailing conceptual model of deglaciation of
central British Columbia, in which mountain ranges emerge from the ice before
the plateau \citep[Fig.~7]{Fulton.1991}.

% =============================================================================
\conclusions
\label{sec:concl}
% =============================================================================

Numerical simulations of the Cordilleran ice sheet through the last glacial
cycle presented in this study consistently produce two glacial maxima during
MIS~4 (61.9--55.4\,ka, 3.8--8.7\,m~s.l.e) and MIS~2 (29.5--16.9\,ka,
8.24--8.66\,m~s.l.e.), two periods corresponding to documented extensive
glaciations. This result is independent of the palaeo-temperature record used
among the six selected for this study, and thus can be seen as a robust model
output, which broadly matches geological evidence. However, the timing of the
two glaciation peaks depends sensitively on which climate record is used to
drive the model. The timing of the LGM is best reproduced by the EPICA and
Vostok Antarctic ice core records, and occurs too early in all other
simulations. The mismatch is largest for the two northwest Pacific ODP
palaeo-temperature records, which are affected by the weakening of the
California current during the LGM.

In all simulations presented here, ice cover is limited to disjoint mountain
ice caps during most of the glacial cycle. The most persistent nucleation
centres are located in the Coast Mountains, the Columbia and Rocky mountains,
the Selwyn and Mackenzie mountains, and most importantly, in the Skeena
Mountains. Throughout the modelled last glacial cycle, the Skeena Mountains
host an ice cap which appears to be fed by moisture intruding inland from the
west through a topographic low in the Coast Mountains. The Skeena ice cap acts
as the main nucleation centre at the onset of the LGM readvance. As indicated
by persistent, warm-based ice in the model, this ice cap is perhaps associated
with the distinct glacial erosional imprint on the landscape of the Skeena
Mountains.

During deglaciation, none of the climate records used can be selected as
producing an optimal agreement between the model results and the geological
evidence. Although the EPICA-driven simulation yields the most realistic timing
of the LGM and, therefore, start of deglaciation, only the GRIP-driven
simulation produces late glacial readvances in areas where these have been
documented. Nonetheless, the patterns of ice sheet retreat are consistent
between the two simulations, and show a rapid deglaciation of the southern
sector of the ice sheet, including a rapid northwards retreat across the Fraser
Plateau of central British Columbia. The GRIP-driven simulation then produces a
late-glacial readvance of local ice caps in the Coast, Columbia and Rocky
mountains, and of the main body of the decaying Cordilleran ice sheet in the
Skeena and Selwyn mountains. In both simulations, this is followed by an
opening of the Liard Lowland, and a final retreat of the remaining ice caps
towards the Selwyn and, finally, the Skeena mountains, which hosts the last
remnant of the ice sheet during the early Holocene (10.9--9.5\,ka). Our results
identify the Skeena Mountains as a key area to understanding glacial dynamics
of the Cordilleran ice sheet, highlighting the need for further geological
investigation of this region.

% Author contributions
\section*{Author contributions}
J.~Seguinot ran the simulations and wrote most of the manuscript; I.~Rogozhina
guided experiment design; A.~P.~Stroeven, M.~Margold and J.~Kleman took part in
the interpretation and comparison of model results against geological evidence.
All co-authors contributed to improve the text.

% Acknowledgements
\begin{acknowledgements}
We are very thankful to Constantine Khroulev, Ed Bueler and Andy Aschwanden for
providing constant help and development on PISM. This work was supported by the
Swedish Research Council~(VR) grant no. 2008-3449 to A.~P.~Stroeven, by the
German Academic Exchange Service~(DAAD) grant no.~50015537 and a Knut and Alice
Wallenberg Foundation grant to J.~Seguinot.
Computer resources were provided by the Swedish National
Infrastructure for Computing (SNIC) allocation no. 2013/1-159 to A.~P.~Stroeven
at the National Supercomputing Center (NSC).
\end{acknowledgements}

% References
\bibliographystyle{copernicus}
\bibliography{refs/references.bib}
\newpage

% =============================================================================
% Floats
% =============================================================================

% tab:params
\begin{table*}
  \centering
  \caption{Parameter values used in the ice sheet model.}
  \label{tab:params}
  {\begin{tabular}{llrl}
    \tophline
    Not.    & Name & Value & Unit \\
    \middlehline

    $\rho$  & Ice density
            & 910
            & \unit{kg\,m^{-3}} \\

    $g$     & Standard gravity
            & 9.81
            & \unit{m\,s^{-2}} \\

    \multicolumn{2}{l}{\emph{Basal sliding}} \\

    $q$     & Pseudo-plastic sliding exponent
            & 0.25
            & - \\

    $v_{th}$& Pseudo-plastic threshold velocity
            & 100.0
            & \unit{m\,yr^{-1}} \\

    $c_0$   & Till cohesion
            & 0.0
            & Pa \\

    $\delta$& Effective pressure coefficient
            & 0.02
            & - \\

    $e_0$   & Till reference void ratio
            & 0.69
            & - \\

    $C_c$   & Till compressibility coefficient
            & 0.12
            & - \\

    $W_{max}$ & Maximal till water thickness
            & 2.0
            & m \\

    \multicolumn{2}{l}{\emph{Bedrock and lithosphere}} \\

    $\rho_b$& Bedrock density
            & 3300
            & \unit{kg\,m^{-3}} \\

    $c_b$   & Bedrock specific heat capacity
            & 1000
            & \unit{J\,kg^{-1}\,K^{-1}} \\

    $k_b$   & Bedrock thermal conductivity
            & 3.0
            & \unit{J\,m^{-1}\,K^{-1}\,s^{-1}} \\

    $\nu_m$ & Mantle viscosity
            & $1\times10^21$
            & \unit{Pa\,s} \\

    $\rho_l$& Lithosphere density
            & 3300
            & \unit{kg\,m^{-3}} \\

    $D$     & Lithosphere flexural rigidity
            & $5.0\times10^24$
            & \unit{N} \\

    \multicolumn{2}{l}{\emph{Surface and atmosphere}} \\

    $T_s$   & Temperature of snow precipitation
            & 273.15
            & \unit{K} \\

    $T_r$   & Temperature of rain precipitation
            & 275.15
            & \unit{K} \\

    $F_s$   & Degree-day factor for ice
            & $3.04\times10^{-3}$
            & \unit{m\,K^{-1}\,day^{-1}} \\

    $F_i$   & Degree-day factor for ice
            & $4.59\times10^{-3}$
            & \unit{m\,K^{-1}\,day^{-1}} \\

    $\gamma$& Air temperature lapse-rate
            & $6\times10^{-3}$
            & \unit{K\,m^{-1}} \\

    \bottomhline
  \end{tabular}}
  \belowtable{}
\end{table*}

% tab:records
\begin{table*}[t]
  \caption{Palaeo-temperature proxy records and scaling parameters yielding
           temperature offset time-series used to force the ice sheet model
           through the last glacial cycle (Fig.~\ref{fig:timeseries}). $f$
           corresponds to the scaling factor adopted to yield last glacial
           maximum ice limits in the vicinity of mapped end moraines, and
           $T_{[32, 22]}$ refers to the resulting mean temperature anomaly
           during the period -32 to~-22~ka after scaling.}
  \label{tab:records}
  {\begin{tabular}{l|ccc|ccc|l}
    \tophline
    Record & Latitude & Longitude & Elev. & Proxy & $f$ & $T_{[32, 22]}$
           & Reference\\
    & & & (m~a.s.l) & & & (K) & \\
    \middlehline
    GRIP     &  72{\degree} 35' N  % 72.58 (decimal)
             &  37{\degree} 38' W  % 37.64 (decimal)
             & 3238\,m
             & \chem{\delta^{18}O}
             & 0.35 & -5.8{\degree}C  % -16.4126 (before scaling)
             & \citet{Dansgaard.etal.1993} \\

    NGRIP    &  75{\degree} 06' N  % 75.10
             &  42{\degree} 19' W  % 42.32
             & 2917\,m
             & \chem{\delta^{18}O}
             & 0.22 & -6.1{\degree}C  % -26.7098
             & \citet{Andersen.etal.2004} \\

    EPICA    &  75{\degree} 06' S  % 75.1
             & 123{\degree} 21' E  % 123.35
             & 3233\,m
             & \chem{\delta^{18}O}
             & 0.60 & -5.6{\degree}C  % -9.2055
             & \citet{Jouzel.etal.2007} \\

    Vostok   &  78{\degree} 28' S  % 78.8
             & 106{\degree} 50' E  % 106.8
             & 3488\,m
             & \chem{\delta^{18}O}
             & 0.70 & -5.6{\degree}C  % -7.9550
             & \citet{Petit.etal.1999} \\

    ODP~1012 &  32{\degree} 17' N
             & 118{\degree} 23' W
             & -1772\,m
             & \chem{U^{K'}_{37}}
             & 1.53 & -5.8{\degree}C  % -3.7889
             & \citet{Herbert.etal.2001} \\

    ODP~1020 &  41{\degree} 00' N
             & 126{\degree} 26' W
             & -3038\,m
             & \chem{U^{K'}_{37}}
             & 1.16 & -5.8{\degree}C  % -5.0000
             & \citet{Herbert.etal.2001} \\
    \bottomhline
  \end{tabular}}
  \belowtable{}
\end{table*}

% tab:extrema
\begin{table*}[t]
  \caption{Extremes in Cordilleran ice sheet volume and extent corresponding to
           MIS~4, 3 and 2 for each of the six low-resolution simulations
           (Fig.~\ref{fig:timeseries}).}
  \label{tab:extrema}
  {\begin{tabular}{l*{3}{|ccc}}
    \tophline
             & \multicolumn{3}{c}{Age (ka)}
             & \multicolumn{3}{c}{Ice extent (\unit{10^6\,km^2})}
             & \multicolumn{3}{c}{Ice volume (m~s.l.e.)} \\
    Record   &  MIS~4 &  MIS~3 &  MIS~2
             &  MIS~4 &  MIS~3 &  MIS~2
             &  MIS~4 &  MIS~3 &  MIS~2 \\
    \middlehline
    GRIP     &  57.58 &  49.24 &  19.52
             &   1.98 &   0.46 &   2.13
             &   7.52 &   0.89 &   8.52 \\
    NGRIP    &  60.26 &  50.16 &  22.85
             &   2.16 &   0.50 &   2.09
             &   8.69 &   0.93 &   8.24 \\
    EPICA    &  61.87 &  45.57 &  17.10
             &   1.57 &   0.95 &   2.08
             &   5.20 &   2.44 &   8.35 \\
    Vostok   &  60.87 &  49.68 &  16.87
             &   1.55 &   0.86 &   2.14
             &   5.10 &   2.01 &   8.66 \\
    ODP 1012 &  55.41 &  47.08 &  23.21
             &   1.44 &   0.85 &   2.13
             &   4.50 &   2.06 &   8.46 \\
    ODP 1020 &  60.16 &  52.24 &  29.46
             &   1.32 &   0.70 &   2.08
             &   3.88 &   1.52 &   8.32 \\
    \middlehline
    Minimum  &  61.87 &  52.24 &  29.46
             &   1.32 &   0.46 &   2.08
             &   3.88 &   0.89 &   8.24 \\
    Maximum  &  55.41 &  45.57 &  16.87
             &   2.16 &   0.95 &   2.14
             &   8.69 &   2.44 &   8.66 \\
    \bottomhline
  \end{tabular}}
  \belowtable{}
\end{table*}

% avoid "too many unprocessed floats" errors
\clearpage

% fig:locmap
\begin{figure}
  \includegraphics{locmap}
  \caption{Relief map of northern North America showing a reconstruction of the
           areas once covered by the Cordilleran (CIS), Laurentide (LIS),
           Innuitian (IIS) and Greenland (GIS) ice sheets during the last
           18\,\unit{\chem{^{14}C}\,ka} (21.4\,cal\,ka)
           \citep{Dyke.2004}. The rectangular box denotes the location of the
           modelling domain used in this study. Major mountain ranges covered
           by the ice sheet include the Alaska Range (AR), the Wrangell and
           St.-Elias mountains (WSEM), the Selwyn and Mackenzie mountains (SMKM),
           the Skeena Mountains (SM), the Coast and Rocky
           Mountains (CRM) and the North Cascades (NC). Major depressions
           include the Liard Lowland (LL), the Fraser Plateau (FP) and the
           Puget Lowland (PL). The background
           map consists of ETOPO1 \citep{Amante.Eakins.2009} and Natural Earth
           Data \citep{Patterson.Kelso.2014}.}
  \label{fig:locmap}
\end{figure}

% fig:atm
\begin{figure}
  \includegraphics{atm}
  \caption{Monthly mean near-surface air temperature, precipitation and
           standard deviation of daily mean temperature for January and July
           from the North American Regional Reanalysis
           \citep[NARR;][]{Mesinger.etal.2006},
           used to force the ice sheet model. Note the
           strong contrasts in seasonality, timing of the precipitation peak,
           and temperature variability over the model domain, notably between
           coastal and inland regions.}
  \label{fig:atm}
\end{figure}

% fig:timeseries
\begin{figure*}
  \includegraphics{timeseries}
  \caption{Temperature offset time-series from ice core and ocean records
           (Table~\ref{tab:records}) used as palaeo-climate forcing for the ice
           sheet model (top panel), and modelled ice volume (bottom panel)
           through the last 120\,ka. Ice volumes are expressed in meters of sea
           level equivalent (m~s.l.e.). Gray fields indicate Marine Oxygen
           Isotope Stage (MIS) boundaries for MIS~2 and MIS~4 according to a
           global compilation of benthic \chem{\delta^{18}O} records
           \citep{Lisiecki.Raymo.2005}. Hatched rectangles highlight the
           time-volume space for ice volume extremes corresponding to MIS~4
           (61.9--55.4\,ka), MIS~3 (52.2--45.6\,ka), and MIS~2 (LGM,
           29.5--16.9\,ka). Dotted lines correspond to GRIP- and EPICA-driven
           6\,km-resolution runs.}
  \label{fig:timeseries}
\end{figure*}

% fig:snapshots
\begin{figure*}
  \includegraphics{snapshots}
  \caption{Snapshots of modelled surface topography (500\,m contours)
           corresponding to the ice volume extremes indicated on
           Fig.~\ref{fig:timeseries}. An ice cap persists over the Skeena
           Mountains (SM) during MIS~3. Note the occurence of spatial
           similarities despite large differences in timing.}
  \label{fig:snapshots}
\end{figure*}

% fig:mis2
\begin{figure*}
  \includegraphics{icemaps-mis2}
  \caption{Modelled surface topography (200\,m contours) and surface velocity
           (colour mapping) corresponding to the maximum ice volume during
           MIS~2 in the GRIP and EPICA high-resolution simulations.}
  \label{fig:mis2}
\end{figure*}

% fig:mis4
\begin{figure*}
  \includegraphics{icemaps-mis4}
  \caption{Modelled surface topography (200\,m contours) and surface velocity
           (colour mapping) corresponding to the maximum ice volume during
           MIS~4 in the GRIP and EPICA high-resolution simulations.}
  \label{fig:mis4}
\end{figure*}

% fig:duration
\begin{figure*}
  \includegraphics{duration}
  \caption{Modelled duration of ice cover during the last 120\,ka.
           Note the irregular colour scale. A contiguous ice cover spanning
           from the Alaska Range (AR) to the southern Coast Mountains (CM) and
           the Columbia and Rocky mountains (RM) exists for about 29\,ka in
           both simulations. The maximal extent of the ice sheet generally
           corresponds to relatively short durations of ice cover, but ice
           cover persists over the Skeena Mountains (SM) during most of the
           simulation.}
  \label{fig:duration}
\end{figure*}

% fig:warmbase
\begin{figure*}
  \includegraphics{warmbase}
  \caption{Modelled duration of warm-based ice cover during the last
           120\,ka. Long ice cover durations combined with basal
           temperatures at the pressure-melting point may explain the strong
           glacial erosional imprint of the Skeena Mountains (SM) landscape.
           Hatches indicate areas that were covered by cold ice only.}
  \label{fig:warmbase}
\end{figure*}

% fig:warmfrac
\begin{figure*}
  \includegraphics{warmfrac}
  \caption{Modelled fraction of warm-based ice cover during the ice-covered
           period. Note the dominance of warm-based conditions on the
           continental shelf and major glacial troughs of the coastal ranges.
           Hatches indicate areas that were covered by cold ice only.}
  \label{fig:warmfrac}
\end{figure*}

% fig:deglacseries
\begin{figure}
  \includegraphics{deglacseries}
  \caption{Temperature offset time-series from the GRIP and EPICA ice core
           records (Table~\ref{tab:records}) (top panel), and modelled ice
           volume during the deglaciation, expressed in meters of sea-level
           equivalent (bottom panel).}
  \label{fig:deglacseries}
\end{figure}

% fig:deglacshots
\begin{figure*}
  \includegraphics{deglacshots}
  \caption{Snapshots of modelled surface topography (200\,m contours)
           and surface velocity (colour mapping) during the last deglaciation
           from the GRIP (top panels) and EPICA (bottom panels) simulations.
           Dashed segments indicate the location of profiles used in
           Figs.~\ref{fig:profiles-grip} and~\ref{fig:profiles-epica}.}
  \label{fig:deglacshots}
\end{figure*}

% fig:deglac
\begin{figure*}
  \includegraphics{deglac}
  \caption{Modelled age of the last deglaciation. Areas where the MIS~4 glacial
           advance exceeded the last glacial maximum advanced are marked in
           green. Hatches denote re-advance of mountain-centred ice caps and
           and the decaying ice sheet between 14 and 10\,ka., which is more
           pronounced in the GRIP-driven simulation.
           Dashed segments indicate the location of profiles used in
           Figs.~\ref{fig:profiles-grip} and~\ref{fig:profiles-epica}.}
  \label{fig:deglac}
\end{figure*}

% fig:lastflow
\begin{figure*}
  \includegraphics{lastflow}
  \caption{Modelled of the deglacial basal ice velocities. Hatches
           indicate areas that remain non-sliding throughout deglaciation
           (22.0--8.0\,ka), notably including the Fraser Plateau (FP).
           Note the concentric patterns of deglacial flow in the Liard
           Lowland (LL).
           Sliding grid cells were distinguished from non-sliding grid cells
           using a velocity threshold of 1\,\unit{m\,yr^{-1}}.
           Dashed segments indicate the location of profiles used in
           Figs.~\ref{fig:profiles-grip} and~\ref{fig:profiles-epica}.}
  \label{fig:lastflow}
\end{figure*}

% fig:profiles-grip
\begin{figure}
  \includegraphics{profiles-grip}
  \caption{Modelled bedrock (black) and ice surface (blue) topography profiles
           during deglaciation (22.0--8.0\,ka) in the GRIP 6\,km
           simulation, corresponding to the four transects indicated in
           Figs.~\ref{fig:deglacshots}--\ref{fig:lastflow}.}
  \label{fig:profiles-grip}
\end{figure}

% fig:profiles-epica
\begin{figure}
  \includegraphics{profiles-epica}
  \caption{Modelled bedrock (black) and ice surface (blue) topography profiles
           during deglaciation (22.0--8.0\,ka) in the EPICA 6\,km
           simulation, corresponding to the four transects indicated in
           Figs.~\ref{fig:deglacshots}--\ref{fig:lastflow}.}
  \label{fig:profiles-epica}
\end{figure}

% =============================================================================
\end{document}
\endinput
% =============================================================================
