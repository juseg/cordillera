% sections/discuss.tex
% ----------------------------------------------------------------------
\section{Discussion}
\label{sec:discussion}
% ----------------------------------------------------------------------

The outcome of our numerical simulations shows a very large sensitivity to the choice of climate forcing data (Fig.~\ref{fig:ivolarea}--\ref{fig:cool05}). To understand the origin of these discrepancies, we quantify surface air temperature and precipitation rate differences between input climatologies and test their effect on modelled ice sheet geometries. We then compare our findings to a reconstruction of the LGM Cordilleran Ice Sheet margin, and discuss some of the key assumptions of this study.

\subsection{Comparison of forcing climatologies}

The PDD model used in this study is highly sensitive to surface air temperature and precipitation forcing. As previously outlined, winter (DJF) precipitation and summer (JJA) temperature are the predominant controls on accumulation and melt at the modelled ice sheet surface. Significant differences in their distribution exist between datasets (Fig.~\ref{fig:topo}--\ref{fig:prec}).

In this section we use the WorldClim climatology, which is derived from observations, as a reference with which to compare other climate forcing data on land. ERA-Interim, CFSR and NARR climatologies exhibit temperature distributions fairly consistent with WorldClim data (Fig.~\ref{fig:tempheatmap}). When examining the spatial distribution of temperature differences to WorldClim data, it appears that most of disparity between reanalyses and observation data is caused by topographical detail at scales unresolved by the atmospheric reanalyses (Fig.~\ref{fig:tempdiff}). Surface air temperature data from the NCAR climatology shows a large disparity with that of WorldClim, with a significant cold anomaly over most of the modelling domain (Fig.~\ref{fig:tempheatmap}--\ref{fig:tempdiff}).

Important differences in precipitation rate exist between input climatologies. All four reanalyses used in this study exhibit higher precipitation than WorldClim data (Fig.~\ref{fig:precheatmap}). However the magnitude of this anomaly differs significantly between datasets. The spatial distributions of precipitation differences between reanalyses and WorldClim data show that negative precipitation anomalies are generally constrained to the windward slope of the major mountain ranges, while positive precipitation anomalies are found on the leeward slope of these ranges and extend downwind to interior plateaux and lowlands (Fig.~\ref{fig:precdiff}).

Most likely, this is the signature of an orographic precipitation effect. As described in section~\ref{sec:climate}, the topography of the Northern American Cordillera is such that its western ranges form a continuous orographic barrier, causing high precipitation along the Pacific coast while leaving much of the interior arid. However the ability of a GCM to reproduce these contrasts in precipitation is bounded by resolution. In a model of coarser resolution, these high mountain ranges are reduced to smoothed hills and the modelled distribution of precipitation is also smoother (Fig.~\ref{fig:oroprecip}). However this effect alone can not explain the widespread and pronounced positive precipitation anomalies observed in CFSR data (Fig.~\ref{fig:precheatmap}--\ref{fig:precdiff}) despite of the high model resolution (Table~\ref{tab:reanalyses}).

\subsection{Model sensitivity to climate forcing}

To distinguish the effects of temperature and precipitation biases in the ice sheet model, we run a series of eight additional simulations. These new simulations use an “hybrid” atmosphere forcing that consist of temperature data from WorldClim, combined with precipitation data from each of the four reanalyses, and vice-versa. A single temperature offset value of 5\,K is used. This value was chosen to allow comparison with previous results (Fig.~\ref{fig:cool05}). In each case, temperature lapse rates are computed using the reference topography from the corresponding dataset.

Although only temperature or precipitation changes are applied, this experiment results once more in large differences in modelled ice sheet geometries (Fig.~\ref{fig:biatm}) as compared to the reference WorldClim 5\,K run (Fig.~\ref{fig:cool05}, top-left panel). For ERA-Interim, CFSR and NARR climatologies, precipitation anomalies clearly dominate the differences in ice sheet response, whereas temperature anomalies have relatively little effect (Fig.~\ref{fig:biatmbars}). However in the case of NCAR forcing, both the negative temperature bias (Fig.~\ref{fig:tempheatmap}--\ref{fig:tempdiff}) and the important precipitation differences (Fig.~\ref{fig:precheatmap}--\ref{fig:precdiff}) contribute to produce oversized ice sheets (Fig.~\ref{fig:biatmbars}).

\subsection{Comparison to reconstructed LGM margin}

We compare the numerically modelled ice sheet geometries to a reconstruction of the ice margin at 14\,$^{14}$C\,ka\,BP (16.8\,cal\,ka\,BP) by \citet{dyke-2004}, based on glacial geomorphology and radiocarbon dating. This corresponds to the LGM extent of most of the Cordilleran ice sheet, which occurred later than in the Laurentide ice sheet \citep{dyke-2004}.

Considering weaknesses of the CFSR and NCEP/NCAR data previously identified (Fig.~\ref{fig:tempheatmap}--\ref{fig:biatmbars}), we bound our analysis to the  WorldClim, ERA-Interim and NARR input climatologies. In order to compare modelled ice sheets of a similar size, we select for each climate forcing the simulation that leads to a final glaciated area closest to the approximate size of the LGM Cordilleran Ice Sheet of $2\,\times10^6\,\unit{km^2}$\irina{justify}. This correspond to temperature offset values of 8 (WorldClim), 6 (ERA-Interim) and 7\,K (NARR). These qualitative “best” runs are presented in Figure~\ref{fig:best} along with their associated temperature depressions and a reconstruction of the LGM ice sheet margin by \citet{dyke-2004}.

Although this results in similar-sized ice sheets, noticeable differences in shape exist. The ERA-Interim simulation produced a northerly-centred ice sheet with too much ice in the North and too little in the South. This may partly reflect unresolved orographic precipitation effects (Fig.~\ref{fig:oroprecip}) due to coarser GCM resolution (Table~\ref{tab:reanalyses}).

The LGM ice sheet margin by \citet{dyke-2004} is best reproduced by simulations driven by input climatologies from WorldClim and NARR data. However, common discrepancies between the modelled ice sheet geometry and the geomorphological reconstruction can still be observed (Fig.~\ref{fig:best}).

Firstly, the modelled eastern margin of the ice sheet extends further east than where one would expect junction between the Cordilleran and the Laurentide ice sheets (Fig.~\ref{fig:best}). However the Laurentide Ice Sheet, which is not included in our model, may have formed a buttress against the smaller Cordilleran Ice Sheet and stopped its advance onto the Canadian Prairies. In addition, potential effects of the growing ice-sheet on regional climate are not included in our model. The Cordilleran ice sheet initiated from the junction of mountain ice caps over the major reliefs \citep{clague-1989}. In our simulations, a continuous ice cover quickly forms over the western ranges, where precipitation rates are higher than in the rest of the domain. This continuous ice cover may have enhanced the topographical barrier already formed by the western ranges, resulting in less precipitation and warmer air in the interior. \citet{langen-etal-2012} demonstrated that this process, which they refer to as a ``self-inhibiting growth'', may have been limiting during the build-up of the Greenland Ice Sheet.

Secondly, our simulations produce anomalous\irina{Any evidence that there was no ice cover there? Refer to the source.} ice cover on parts of the continental shelf in the Arctic Ocean. These regions experience a marine climate, including lower summer temperature than on the adjacent land (Fig.~\ref{fig:temp}). However in our simulations, sea-level is lowered by 120~m, turning large parts of the low-sloping continental shelf into land. A similar effect occurs over Great Bear Lake in some simulations, particularly those driven by the NCEP/NCAR forcing (Fig.~\ref{fig:extent}). This anomalous ice cover is to be interpreted as an artefact arising from our simplistic temperature offset method, and has little effect on the rest of the results.

\subsection{Model sensitivity to englaciation period}

A key assumption for the present study is our choice of 10\,Ka for the length of all our simulations. Although geomorphological data shows that the period of growth of the Cordilleran Ice Sheet from nearly ice-free to full glacial conditions could not have been much longer than 10\,Ka, they provide no lower bound to this value.

To test the effect of a shorter englaciation period on model results, we use simulations run under NARR climate forcing with temperatures offsets of 7 to 11\,K, and compare modelled ice sheet extent when glaciated area reach the approximate size of the LGM Cordilleran Ice Sheet of $2\,\times10^6\,\unit{km^2}$.

This experiment shows that a shorter englaciation period leads to more restricted ice cover in the eastern part of the modelling domain, where precipitation rate control ice advance, but further glaciation along the Pacific margin to the south, where temperature is the limiting factor (Fig.~\ref{fig:durationstack}). Reducing simulation length lead to a closer match with the LGM ice sheet margin by \citet{dyke-2004} in some parts of the modelling domain, but higher discrepancy in other parts.

Therefore, our choice of 10\,Ka simulation length could certainly be discussed further. However, our representation of climate history by constant temperature offsets is voluntarily simplistic. To better understand the transient character of the LGM Cordilleran Ice Sheet, time-dependent palaeo-climate forcing would be needed, and this is left for future research.

\subsection{Other potential method improvements}

Our simplistic temperature offset method is certainly a crude simplification of past climate changes over the region. Although it is clear that these temperature changes were neither homogeneous nor constant in time, and probably associated with precipitation changes, their patterns are not trivial and may potentially display a strong inter-dependence with the evolution of the Cordilleran and Laurentide ice sheets. A more correct, yet more complex approach would be to use coupling to a GCM of intermediate complexity \citep{yoshimori-etal-2001,calov-etal-2002,abeouchi-etal-2007,charbit-etal-2013}.

Furthermore, some simplifications were made in the surface mass balance model. Our PDD model does not include refreezing. Refreezing of melted snow and ice and retention in the snow pack, however, can greatly alter mass balance at the surface of an ice sheet \citep{janssens-huybrechts-2000}. Additionally, we simulate temperature variability by using a constant, uniform value of temperature standard deviation. This approach implies large biases of surface mass-balance \citep{charbit-etal-2013,rau-rogozhina-2013,seguinot-inpress}, particularly over regions with such various climates as the Northern American Cordillera.

