% sections/discuss.tex
% ----------------------------------------------------------------------
\section{Discussion}
\label{sec:discussion}
% ----------------------------------------------------------------------

\subsection{Differences with reconstructed LGM margin}

\julien[noline]{This sentence may be better placed in the results section.}
We compare the numerically modelled ice sheet geometries to a reconstruction of the ice margin at 14\,$^{14}$C\,ka\,BP (16.8\,cal\,ka\,BP) by \citet{dyke-2004}, based on glacial geomorphology and radiocarbon dating. This corresponds to the LGM extent of most of the Cordilleran ice sheet, which occurred later than in the Laurentide ice sheet \citep{dyke-2004}.
Although the outcome of our numerical simulations is strongly dependant to which dataset is used as climate forcing, common discrepancies between the modelled ice sheet geometry and the ice margin from \citet{dyke-2004} can be observed (Figures~\ref{fig:cool06}-\ref{fig:best}).

For all climate forcing used, the modelled eastern margin of the ice sheet extends further east than the reconstructed boundary between the Cordilleran and the Laurentide ice sheets. This can be explained by the potential effects of the growing ice-sheet on regional climate not included in our temperature offset method. The Cordilleran ice sheet initiated from the junction of mountain ice caps over the major reliefs. In our simulations, a continuous ice cover quickly forms over the western ranges, where precipitation rates are higher than in the rest of the domain. This continuous ice cover may have enhanced the topographical barrier already formed by the western ranges, resulting in fewer precipitation and warmer air in the interior. \citet{langen-etal-2012} demonstrated that this process, which they refer as a ``self-inhibiting growth'', potentially prevents a regrowth of the Greenland Ice Sheet to its present size. Alternatively, the Laurentide Ice Sheet, not included in our model, may have formed a buttress against the smaller Cordilleran Ice Sheet and stopped its advance onto the Canadian Prairies.

Our simulations produces anomalous ice cover in parts on the continental shelf in the Arctic Ocean. These regions experience a marine climate, including lower summer temperature than on the adjacent land (Figures~\ref{fig:temp}). However sea-level is lowered by 120~m, turning large parts of the low-sloping continental shelf into land. A similar effect occurs over Great Bear Lake in some simulations, particularly those driven by the NCEP/NCAR forcing (Figures~\ref{fig:extent}). This anomalous ice cover is to be interpreted as an artefact from our simplistic temperature offset method, and has little effect on the rest of the results.

\subsection{Sensitivity to climate forcing}

The outcome of our numerical simulations is strongly dependant to which dataset is used for climate forcing (Figures~\ref{fig:cool06}-\ref{fig:best}). This results from temperature and precipitation differences (Figures~\ref{fig:temp}-\ref{fig:prec}) between forcing datasets, to which the glacier model is highly sensitive.

Differences between modelled ice sheet geometries are notably visible in northern Yukon Territory and Alaska, where previous modelling studies commonly produced too extensive ice cover, whereas geomorphological data shows that ice cover have been sparse. Concerning reanalysis datasets, we interpret theses differences as mainly the results of different GCM resolutions (Figures~\ref{fig:topo}, Table~\ref{tab:reanalyses}).

As described in section~\ref{sec:climate}, the topography of the Northern American Cordillera is such that its western ranges form a continuous orographic barrier, causing high precipitation along the Pacific coast while leaving much of the interior arid. However the capability of a GCM to reproduce these precipitation contrasts is bound to its resolution. In a model of coarser resolution such as used in the NCEP/NCAR reanalysis, these high mountain ranges are reduced to smoothed hills and the modelled distribution of precipitation is also smoother.

\subsection{Potential method improvements}

Our simplistic temperature offset is certainly a crude simplification of past climate changes over the region. Although it is clear that these temperature changes were neither homogeneous nor constant in time, and probably associated with precipitation changes, their patterns are not trivial and potential strongly inter-dependant with the Cordilleran and Laurentide ice sheet evolution. A more correct approach would be to use coupling to a GCM of intermediate complexity \citep{yoshimori-etal-2001,calov-etal-2002,abeouchi-etal-2007,charbit-etal-2013}.

Some simplifications were made in the surface mass-balance model. Our PDD model do not include refreezing. Refreezing of melted snow and ice, however, can form a large proportion of the surface mass-balance of an ice sheet\needref. Furthermore, we model temperature variability using a constant, homogeneous value of standard deviation. This approaches implies large biases of surface mass-balance \citep{charbit-etal-2013,rau-rogozhina-2013,seguinot-inreview}, particularly over a region with such various climates as the Northern American Cordillera.


