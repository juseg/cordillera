% sections/discuss.tex
% ----------------------------------------------------------------------
\section{Discussion}
\label{sec:discussion}
% ----------------------------------------------------------------------

\subsection{Comparison to reconstructed LGM margin}

We compare the numerically modelled ice sheet geometries to a reconstruction of the ice margin at 14\,$^{14}$C\,ka\,BP (16.8\,cal\,ka\,BP) by \citet{dyke-2004}, based on glacial geomorphology and radiocarbon dating. This corresponds to the LGM extent of most of the Cordilleran ice sheet, which occurred later than in the Laurentide ice sheet \citep{dyke-2004}.
Although the outcome of our numerical simulations strongly depends on which dataset is used as a climate forcing, common discrepancies between the modelled ice sheet geometry and the ice margin mapped by \citet{dyke-2004} can be observed (Fig.~\ref{fig:cool05}-\ref{fig:best}).

For all climate forcing used, the modelled eastern margin of the ice sheet extends further east than the reconstructed boundary between the Cordilleran and the Laurentide ice sheets (Fig.~\ref{fig:best}). This can be explained by the potential effects of the growing ice-sheet on regional climate not included in our model, or by buttressing effects against the Laurentide Ice Sheet. The Cordilleran ice sheet initiated from the junction of mountain ice caps over the major reliefs~\citep{clague-1989}. In our simulations, a continuous ice cover quickly forms over the western ranges, where precipitation rates are higher than in the rest of the domain. This continuous ice cover may have enhanced the topographical barrier already formed by the western ranges, resulting in less precipitation and warmer air in the interior. \citet{langen-etal-2012} demonstrated that this process, which they refer to as a ``self-inhibiting growth'', may have been limiting during the build-up of the Greenland Ice Sheet. Alternatively, the Laurentide Ice Sheet, which is not included in our model, may have formed a buttress against the smaller Cordilleran Ice Sheet and stopped its advance onto the Canadian Prairies.

Our simulations produce anomalous ice cover on parts of the continental shelf in the Arctic Ocean. These regions experience a marine climate, including lower summer temperature than on the adjacent land (Fig.~\ref{fig:temp}). However in our simulations, sea-level is lowered by 120~m, turning large parts of the low-sloping continental shelf into land. A similar effect occurs over Great Bear Lake in some simulations, particularly those driven by the NCEP/NCAR forcing (Fig.~\ref{fig:extent}). This anomalous ice cover is to be interpreted as an artefact arising from our simplistic temperature offset method, and has little effect on the rest of the results.

\subsection{Sensitivity to climate forcing}

The outcome of our numerical simulations strongly depends on which dataset is used for climate forcing (Fig.~\ref{fig:cool05}-\ref{fig:best}). This results from differences in temperature and precipitation (Fig.~\ref{fig:temp}-\ref{fig:prec}) between forcing datasets, to which the glacier model is highly sensitive.

Difference between modelled ice sheet geometries is notably visible in northern Yukon Territory and Alaska, where previous modelling studies commonly produced too extensive ice cover, whereas geomorphological data show that the ice cover was sparse. Concerning reanalysis datasets, we interpret these differences as mainly the result of different GCM resolutions and related model physics (Fig.~\ref{fig:topo}, Table~\ref{tab:reanalyses}).

As described in section~\ref{sec:climate}, the topography of the Northern American Cordillera is such that its western ranges form a continuous orographic barrier, causing high precipitation along the Pacific coast while leaving much of the interior arid. However the ability of a GCM to reproduce these contrasts in precipitation is bound to its resolution. In a model of coarser resolution such as used in the NCEP/NCAR reanalysis, these high mountain ranges are reduced to smoothed hills and the modelled distribution of precipitation is also smoother (Fig.~\ref{fig:oroprecip}). This interpretation does not apply to the CFSR climatology, which exhibits high inland precipitation rates despite of the high model resolution (Table~\ref{tab:reanalyses}).

\subsection{Potential method improvements}

Our simplistic temperature offset method is certainly a crude simplification of past climate changes over the region. Although it is clear that these temperature changes were neither homogeneous nor constant in time, and probably associated with precipitation changes, their patterns are not trivial and may potentially display a strong inter-dependence with the evolution of the Cordilleran and Laurentide ice sheets. A more correct, yet more complex approach would be to use coupling to a GCM of intermediate complexity \citep{yoshimori-etal-2001,calov-etal-2002,abeouchi-etal-2007,charbit-etal-2013}.

Furthermore, some simplifications were made in the surface mass balance model. Our PDD model does not include refreezing. Refreezing of melted snow and ice and retention in the snow pack, however, can greatly alter mass balance at the surface of an ice sheet \citep{janssens-huybrechts-2000}. Additionally, we simulate temperature variability by using a constant, uniform value of temperature standard deviation. This approach implies large biases of surface mass-balance \citep{charbit-etal-2013,rau-rogozhina-2013,seguinot-inreview}, particularly over regions with such various climates as the Northern American Cordillera.

