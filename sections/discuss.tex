% sections/discuss.tex
% ----------------------------------------------------------------------
\section{Discussion}
\label{sec:discussion}
% ----------------------------------------------------------------------

These simulations of the LGM Cordilleran Ice Sheet show a very large sensitivity to the choice of climate forcing data used (Figs.~\ref{fig:ivolarea}--\ref{fig:cool05}). To understand the origin of the discrepancies between model results, we quantify differences in surface air temperature and precipitation rate between the input climatologies and test their effect on the ice sheet model. We then compare our findings to a geomorphological reconstruction of the LGM Cordilleran Ice Sheet margin, and discuss some of the key assumptions of this study.

\subsection{Comparison of forcing climatologies}

As previously outlined, significant differences in the distribution of winter (DJF) precipitation and summer (JJA) temperature, which are the predominant controls on accumulation and melt at the modelled ice sheet surface, exist between datasets (Figs.~\ref{fig:topo}--\ref{fig:prec}). In this section we use the WorldClim climatology, which is derived from observations, as a reference with which to compare other climate forcing data on land. ERA-Interim, CFSR and NARR climatologies exhibit temperature distributions fairly consistent with WorldClim data (Fig.~\ref{fig:tempheatmap}). When examining the spatial distribution of temperature differences between reanalyses and WorldClim data, it appears that most of the disparity between reanalyses and observation data is caused by topographical detail at scales unresolved in the reanalysis datasets (Fig.~\ref{fig:tempdiff}). Surface air temperature data from the NCEP/NCAR climatology shows the largest disparity with that of WorldClim, with a significant cold anomaly over most of the modelling domain (Figs.~\ref{fig:tempheatmap}--\ref{fig:tempdiff}).

All four reanalyses used in this study exhibit higher precipitation rates than WorldClim data (Fig.~\ref{fig:precheatmap}). However the magnitude of this anomaly differs significantly between datasets. The spatial distributions of precipitation differences between reanalyses and WorldClim data show that negative precipitation anomalies are generally constrained to the windward slope of the major mountain ranges, while positive precipitation anomalies are found on the leeward slope of these ranges and extend downwind to interior plateaux and lowlands (Fig.~\ref{fig:precdiff}).

Most likely, this is the signature of an orographic precipitation effect. As described in section~\ref{sec:climate}, the topography of the Northern American Cordillera is such that its western ranges form a continuous orographic barrier, causing high precipitation along the Pacific coast while leaving much of the interior arid. However the ability of a GCM to reproduce these contrasts in precipitation is bounded by spatial resolution. In a model of coarser resolution, these high mountain ranges are reduced to smooth hills and the modelled distribution of precipitation is also smoother (Fig.~\ref{fig:oroprecip}). However, this effect alone cannot explain the widespread and pronounced positive precipitation anomalies observed in CFSR data (Figs.~\ref{fig:precheatmap}--\ref{fig:precdiff}) despite of its high spatial resolution (Table~\ref{tab:reanalyses}).

\subsection{Model sensitivity to climate forcing}

To distinguish the effects of temperature and precipitation biases on the ice sheet model, we run a series of eight additional simulations. These new simulations use a “hybrid” climate forcing that consists of temperature data from WorldClim, combined with precipitation data from each of the four reanalyses, and vice-versa. A single temperature offset of 5\,\unit{\degree C} is used. This value was chosen to allow for comparison with previous results (Fig.~\ref{fig:cool05}). In each case, temperature lapse rates are computed using the reference topography from the corresponding dataset.

Although only temperature or precipitation changes are applied, this experiment results once more in large differences in modelled ice sheet geometries (Fig.~\ref{fig:biatm}) as compared to the reference WorldClim 5\,\unit{\degree C} run (Fig.~\ref{fig:cool05}, top-left panel). For ERA-Interim, CFSR and NARR climatologies, precipitation anomalies clearly dominate the differences in ice sheet response, whereas temperature anomalies have relatively little effect (Fig.~\ref{fig:biatmbars}). However in the case of NCAR forcing, both the negative temperature bias (Figs.~\ref{fig:tempheatmap}--\ref{fig:tempdiff}) and the important precipitation differences (Figs.~\ref{fig:precheatmap}--\ref{fig:precdiff}) contribute approximately equally to produce oversized ice sheets (Fig.~\ref{fig:biatmbars}).

\subsection{Comparison to geomorphological LGM margin}

We compare modelled ice sheet geometries to a reconstruction of the ice margin at 14\,$^{14}$C\,ka\,BP (16.8\,cal\,ka\,BP) by \citet{dyke-2004}, based on glacial geomorphology and radiocarbon dating (Fig.~\ref{fig:locmap}). This corresponds to the LGM extent of most of the Cordilleran ice sheet, which occurred later than for the Laurentide ice sheet \citep{porter-swanson-1998,dyke-2004,stroeven-etal-2010,stroeven-etal-inpress}.

Considering previously identified weaknesses of the CFSR and NCEP/NCAR data (Figs.~\ref{fig:tempheatmap}--\ref{fig:biatmbars}), we restricted our analysis to the WorldClim, ERA-Interim and NARR input climatologies. In order to compare modelled ice sheets of a similar size, we select for each climate forcing the simulation that leads to a final glaciated area closest to the approximate size of the LGM Cordilleran Ice Sheet of $2\,\times10^6\,\unit{km^2}$. This corresponds to temperature offset values of 8\,\unit{\degree C} (WorldClim), 6\,\unit{\degree C} (ERA-Interim) and 7\,\unit{\degree C} (NARR). These qualitative “best” runs are presented in Figure~\ref{fig:best} along with their associated temperature depressions and a reconstruction of the LGM ice sheet margin by \citet{dyke-2004}.

Although this selection results in similarly sized ice sheets, noticeable differences in shape exist. The ERA-Interim simulation produces a more northerly-centred ice sheet with too much ice in the north and too little in the south. This may partly reflect unresolved orographic precipitation effects (Fig.~\ref{fig:oroprecip}) due to the coarser GCM resolution (Table~\ref{tab:reanalyses}). The LGM ice sheet margin by \citet{dyke-2004} is best reproduced by simulations driven by input climatologies from WorldClim and NARR data. However, common discrepancies between the modelled ice sheet geometry and the geomorphological reconstruction can still be observed (Fig.~\ref{fig:best}).

Firstly, the modelled eastern margin of the ice sheet extends further east than the inferred position of the junction between the Cordilleran and the Laurentide ice sheets. This can be seen for instance in the south-eastern part of our modelling domain, where the modelled ice sheets cover an ice-free corridor of the LGM reconstruction (Fig.~\ref{fig:best}). However, such comparisons have been hampered by the fact that we have not considered the buttressing effect of the Laurentide Ice Sheet, which would have inhibited the Cordilleran Ice Sheet from advancing onto the Canadian Prairies. In addition, potential effects of the growing ice-sheet on atmospheric circulation and precipitation changes are not included in our model. The Cordilleran ice sheet initiated from the formation of mountain ice caps over the high mountain ranges \citep{clague-1989}. In our simulations, a continuous ice cover quickly forms over the western ranges, where precipitation rates are higher than in the rest of the domain. In reality, this continuous ice cover enhanced the topographical barrier already formed by the western ranges, and would have resulted in less precipitation and warmer air in the interior. This process of ``self-inhibiting growth'' is not captured in our model but was demonstrated to be important in a similar setting over Greenland, where it may have been limiting the advance of the ice sheet during its build-up \citep{langen-etal-2012}.

Secondly, our simulations produce anomalous ice cover on parts of the continental shelf in the Arctic Ocean (Fig.~\ref{fig:best}). These regions experience a marine climate, including lower summer temperature than on the adjacent land (Fig.~\ref{fig:temp}). However in our simulations, sea-level is lowered by 120~m, turning large parts of the low-sloping continental shelf into land. This anomalous ice cover is to be interpreted as an artefact arising from our simplistic temperature offset method, and has little effect on the rest of the results.

Given the lack of WorldClim data offshore, we consider the NARR climate forcing as best suited for simulations of the Cordilleran Ice Sheet among those tested, and under the assumptions of the present study.

\subsection{Model sensitivity to the duration of ice sheet inception}

A key assumption for the present study is our choice of 10\,ka for the duration of all our simulations. Although geomorphological data shows that the period of Cordilleran Ice Sheet inception, from nearly ice-free to full glacial conditions, could not have been much longer than 10\,ka, they provide no lower bound to this value.

To test the effect of an inception period of shorter duration on model results, we use simulations run under NARR climate forcing with temperature offsets of 7 to 11\,\unit{\degree C}, and compare modelled ice sheet extent when glaciated area reaches the approximate size of the LGM Cordilleran Ice Sheet of $2\,\times10^6\,\unit{km^2}$.

This experiment shows that a shorter start-up period leads to more restricted ice cover in the eastern part of the modelling domain, where precipitation rate controls ice advance, but more extensive glaciation in the south-western domain along the Pacific coast, where temperature is the limiting factor (Fig.~\ref{fig:durationstack}). Hence, reducing simulation duration leads to a closer match with the LGM ice sheet margin by \citet{dyke-2004} in some parts of the modelling domain, but higher discrepancy in other parts.

\subsection{Other potential method improvements}

Our representation of climate history by constant temperature offsets is voluntarily simplistic. To better understand the transient character of the LGM Cordilleran Ice Sheet, time-dependent palaeo-climate forcing would be needed. Although it is clear that past temperature changes over the region were neither homogeneous nor constant in time, and probably associated with precipitation changes, their patterns are not trivial and may potentially display a strong inter-dependence with the evolution of the Cordilleran and Laurentide ice sheets. A more correct, yet more complex approach would be to use coupling to a GCM of intermediate complexity \citep{yoshimori-etal-2001,calov-etal-2002,abeouchi-etal-2007,charbit-etal-2013}.

Furthermore, the surface mass balance model crudely simplifies of processes taking place at the glacier surface. Refreezing of melted snow and ice and retention in the snow pack was shown to alter surface mass balance of glaciers \citep{trabant-mayo-1985}, yet it is not accounted for in our PDD model. Based on scarce observations over the Greenland Ice Sheet, different parametrizations of this effect have been developped \citep{janssens-huybrechts-2000,fausto-etal-2009b}. However, these parametrizations were shown to disagree with more complex models \citep{reijmer-etal-2012}, and their portability to a different region and time period has not been proven. In addition, we simulate temperature variability by using a constant, uniform value of daily temperature standard deviation. This approach implies large biases of surface mass-balance \citep{charbit-etal-2013,rau-rogozhina-2013,seguinot-2013}, particularly over regions with such various climates as experienced within our model domain.
