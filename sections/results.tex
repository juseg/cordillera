
% ----------------------------------------------------------------------
\section{Results}
\label{sec:results}
% ----------------------------------------------------------------------

Using climate forcing from WordClim data and the NCEP/NCAR, ERA-Iterim, CFSR and NARR reanalyses described in section~\ref{sec:climate} and the ice sheet model described in section~\ref{sec:model}, we run simulations of glacial inception and growth of the Cordilleran ice sheet using different temperature offsets.

\begin{figure*}[t]
	\vspace*{2mm}
	\begin{center}
		\includegraphics[width=13cm]{cordillera-climate-cool06}
	\end{center}
	\todo[inline]{Change panel order to match climate section}
	\caption{Ice surface topography (black contours every 1000\,m) and velocity (\unit{m\,yr^{-1}}) after 10\,kyr under a climate 6\degC colder than present for each climate forcing.}
	\label{fig:cool06}
\end{figure*}

Figure~\ref{fig:cool06} shows the outcome simulations run with a 6\degC temperature offset for each of the six climate forcings. In all simulations, ice accumulate and glaciers form as a result of the artificially cooled climate. Yet the magnitude of glaciation appears very different from one climate forcing to the next. Whereas NCAR and CFSR forcings produce a large ice sheet that covers most of the model domain land, WorldClim, ERA-Interim and NARR forcings lead to more restrictive ice cover, bounded to mountain ice caps in the case of WorldClim data. The smoothing of CFSR precipitation data has limited visual effect on the resulting ice sheet geometry.

\begin{figure*}[t]
	\vspace*{2mm}
	\begin{center}
		\includegraphics[width=13cm]{cordillera-climate-extent}
	\end{center}
	\todo[inline]{Change panel order to match climate section}
	\qiong[inline]{Move ticks on the colorbar. The 1 is confusing.}
	\todo[inline]{This figure is heavy. It should be possible to simply paths and reduce the file size. Matplotlib's mpl.rc('path', simplify=True) may be the answer.}
	\caption{Extent of ice cover after 10\,kyr as a function of applied temperature offsets for each climate forcing.}
	\label{fig:extent}
\end{figure*}

As we aim to model an ice sheet approaching LGM size, and consider temperature offset as an unknown parameter in this study, we ran simulations using offset values ranging from 2 to 9\degC for each climate forcing.

Figure~\ref{fig:extent} shows the extent of ice cover at the end of each of these (48) simulations, grouped by climate forcing. It again visible that different forcings generally lead to very different final ice cover Notably, for the smallest temperature offset value of 2\degC that was used, NCAR and CFSR simulations lead to an ice-sheet, whereas WorldClim, NARR and ERAI produced local ice caps only, restricted to the Wrangell-St.~Elias mountains in the case of WorldClim, an area presently glaciated. Several of the NCAR and CFSR simulations produced oversized ice-sheets whose extent is primirily bounded by the model domain boundary conditions rather than physical processes. Differences in regional patterns can also be noted. For instance ERAI ice seets generally appear more northernly-centered than NARR ice sheets.

\begin{figure}[t]
	\vspace*{2mm}
	\begin{center}
		\includegraphics[width=9cm]{cordillera-climate-ivolarea}
	\end{center}
	\todo[inline]{Use black-and-white-proof markers}
	\todo[inline]{Highlight the ``best'' runs.}
	\caption{Total ice volume and glaciated area after 10\,kyr as a function of temperature offset and climate forcing.}
	\label{fig:ivolarea}
\end{figure}

Final ice volume and final ice coverage at the end of all simulations is quantified in Figure~\ref{fig:ivolarea}. NCEP/NCAR and CFSR forcing lead to much larger and more extensive glaciation than ERAI, NARR and WorldClim focrings. ERAI and NARR lead to similar ice volumes and glaciated areas, yet regional patterns are different, as shown by Figure~\ref{fig:extent}.

\begin{figure*}[t]
	\vspace*{2mm}
	\begin{center}
		\includegraphics[width=13cm]{cordillera-climate-best}
	\end{center}
	\todo[inline]{Change panel order to match climate section}
	\todo[inline]{The must would be to overlay geomorphic ice extent.}
	\caption{Ice surface topography (black contours every 1000\,m) and velocity (\unit{m\,yr^{-1}}) after 10\,kyr using temperature offsets that lead to similar areas of ice cover for each climate forcing.}
	\label{fig:best}
\end{figure*}

For each climate forcing, we selected the simulation that lead to a glaciated area closest to geomorphological reconstruction. This was done using the LGM extent countour from \needref shown in Figure~\ref{fig:locmap}, which covers an area of XXX \todo{Compute geomorpho-based ice cover area}. These ``best'' runs are presented in Figure~\ref{fig:best} along with the associated emperature offset values. We observe than although similar in size, the selected ice sheets are different in shape and flow patterns. More particularly, NCEP/NCAR, CFSR and ERA-Interim ice sheets cover large territories in Yukon and Alaska that where proven to have remain ice-free for at least several glacial cycles.\needref

