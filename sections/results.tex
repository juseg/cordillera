% sections/results.tex
% ----------------------------------------------------------------------
\section{Results}
\label{sec:results}
% ----------------------------------------------------------------------

Using WorldClim, NCEP/NCAR, ERA-Interim, CFSR (smoothed and original datasets) and NARR input climatologies as a climate forcing (section~\ref{sec:climate}) for the ice sheet model PISM (section~\ref{sec:model}), we run 66~simulations of glacial inception and growth of the Cordilleran ice sheet over 10~kyr using temperature offsets ranging from 0 to 10\degC. Important differences in patterns of ice sheet growth, final glacial extent, and volume were attained (Fig. \ref{fig:cool05}-\ref{fig:best}).

To contrast the six different input climatologies, we plot simulations run with the a 5\degC temperature offset (Fig.~\ref{fig:cool05}). The magnitude of glaciation differs dramatically from one climate forcing to another. Whereas the NCEP/NCAR and CFSR climatologies produce a large ice sheet that covers most of the model domain (Fig.~\ref{fig:cool05}, lower panels), WorldClim, ERA-Interim and NARR climatologies lead to more restrictive ice cover (Fig.~\ref{fig:cool05}, upper panels). The WorldClim run only resulted in the growth of mountain ice caps bounded to the highest topography. The smoothing of CFSR precipitation yielded an insignificant effect on the resulting ice sheet geometry.

Across the range of temperature offsets used, final glaciated area and final ice volume differs widely between different input climatologies (Fig.~\ref{fig:ivolarea}). Using a 5\degC temperature offset (Fig.~\ref{fig:cool05}) final glaciated area differs between WorldClim and CFSR forcing by a factor of 6 and final volume by a factor of 12 (Fig.~\ref{fig:ivolarea}). For particular temperature offset values, given input climatologies result in similar final glaciated areas and ice volume, yet regional patterns are different (Fig.~\ref{fig:extent}).

Under present climate (0\degC offset), WorldClim, NARR and ERA-Interim forcing produce local ice caps in areas presently glaciated (Fig.~\ref{fig:extent}). This results are fairly consistent with contemporary ice coverage, although both the 10~km horizontal resolution, and the occurrence of glaciers and ice caps in the ETOPO1 input basal topography prevent a detailed comparison. CFSR forcing, on the other hand, yields continuous ice cover over the northern Coast Mountains, and NCEP/NCAR forcing results in ice-sheet growth, which is largely inconsistent with the present state of glaciation. At the opposite side of the temperature offset range, it can be noted that NCEP/NCAR and CFSR forcing produce oversized ice sheets whose extent is primarily restricted by the domain boundaries (Fig.~\ref{fig:extent}).

To compare modelled ice sheets of similar size, we selected for each climate forcing the simulation that leads to a glaciated area closest to the approximate size of the LGM Cordilleran Ice Sheet of $2\,\times10^6\,\unit{km^2}$. These qualitative ``best'' runs are presented in Figure~\ref{fig:best} along with their associated temperature depressions and a reconstruction of the ice margin at 14\,$^{14}$C\,ka\,BP (16.8\,cal\,ka\,BP) by \citet{dyke-2004}. We observe that although similar in size, the selected ice sheets are different in shape and flow patterns. More specifically, ice sheets modelled using NCEP/NCAR, CFSR and ERA-Interim input climatologies cover parts of Yukon Territory and Alaska that were shown to have remained ice-free for at least several glacial cycles \citep{dukrodkin-1999,kaufman-manley-2004}.
