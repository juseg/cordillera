% sections/results.tex
% ----------------------------------------------------------------------
\section{Results}
\label{sec:results}
% ----------------------------------------------------------------------

Using WorldClim, NCEP/NCAR, ERA-Interim, CFSR (smoothed and original datasets) and NARR input climatologies as climate forcing, we run 66~simulations of glacial inception and growth of the Cordilleran ice sheet over 10~ka using temperature offsets ranging from 0 to 10\,\unit{\degree C}. Significant differences in patterns of ice sheet growth, final glacial extent, and volume are attained (Fig. \ref{fig:ivolarea}-\ref{fig:cool05}).

Across the range of temperature offsets used, final glaciated area and final ice volume differ widely between different input climatologies (Fig.~\ref{fig:ivolarea}). For instance, using a 5\,\unit{\degree C} temperature offset, final glaciated area differs between WorldClim and CFSR forcing by a factor of~6 and final volume by a factor of~12 (Fig.~\ref{fig:ivolarea}). For particular temperature offset values, given input climatologies result in similar final glaciated areas and ice volumes, yet regional patterns are different (Fig.~\ref{fig:extent}).

Under present climate (0\,\unit{\degree C} offset), WorldClim, NARR and ERA-Interim forcing produce local ice caps in areas presently glaciated (Fig.~\ref{fig:extent}). These results are fairly consistent with contemporary ice distribution, although both the 10~km horizontal resolution and the occurrence of glaciers and ice caps surface topographies in the ETOPO1 dataset used for basal topography prevent a meaningful detailed comparison. CFSR forcing, on the other hand, yields continuous ice cover over the northern Coast Mountains, and NCEP/NCAR forcing results in ice sheet growth, both of which are largely inconsistent with the present state of glaciation. At the opposite side of the temperature offset range, it can be noted that both the NCEP/NCAR and CFSR forcing produce oversized ice sheets whose extents are primarily restricted by the domain boundaries (Fig.~\ref{fig:extent}). The smoothing of CFSR precipitation yields an insignificant effect on the resulting ice sheet geometry.

To contrast the effect of the six different input climatologies, we plot simulations using a 5\,\unit{\degree C} temperature offset (Fig.~\ref{fig:cool05}). The magnitude of glaciation differs dramatically from one climate forcing to another. Whereas the NCEP/NCAR and CFSR climatologies produce a large ice sheet that covers most of the model domain (Fig.~\ref{fig:cool05}, lower panels), WorldClim, ERA-Interim and NARR climatologies lead to more restrictive ice cover (Fig.~\ref{fig:cool05}, upper panels). The WorldClim run only resulted in the growth of mountain ice caps bounded to the highest topography.
