% sections/results.tex
% ----------------------------------------------------------------------
\section{Results}
\label{sec:results}
% ----------------------------------------------------------------------

Using WorldClim, NCEP/NCAR, ERA-Interim, CFSR (smoothed and original datasets) and NARR input climatologies (section~\ref{sec:climate}), PISM (section~\ref{sec:model}) was run using 0 to 10\degC  temperature offsets for 10~kyr to mimic glacial inception and the growth of the Cordilleran ice sheet towards its LGM configuration. Important differences in patterns of ice sheet growth, final glacial extent, and volume were attained (Fig. \ref{fig:cool06}-\ref{fig:best}).

To contrast the six different input climatologies, we plot simulations run with a 6\degC temperature offset (~\ref{fig:cool06}). Clearly, the magnitude of glaciation differs dramatically from one climate forcing to another. In fact, differences between least extent and most extent differ by a factor above 3 and least volume and most volume by a factor above 6 (Fig. \ref{fig:ivolarea}). Whereas the NCEP/NCAR and CFSR climatologies produce a large ice sheet that covers most of the model domain (lower panels, Fig.-\ref{fig:cool06}), WorldClim, ERA-Interim and NARR climatologies lead to more restrictive ice cover (upper panels, Fig.~\ref{fig:cool06}). The WorldClim run only resulted in the growth of mountain ice caps bounded to the highest topography. The smoothing of CFSR precipitation yielded an insignificant effect on the resulting ice sheet geometry.

As illustrated in Figure \ref{fig:cool06}, spatial patterns and rates of ice sheet growth differ tremendously between the input climatologies, given a certain temperature depression. Because we performed a total of 11 simulations for each input climatology, we derived an ensemble of 66 simulations (Fig. \ref{fig:extent}). Despite temperature depressions as large as 10\degC to the input reference climatologies it is again visible that each leads to a different final ice sheet extent and volume (Fig. \ref{fig:extent} and \ref{fig:ivolarea}). Notably, for the smallest temperature offset value of 0\degC, NCAR and CFSR simulations lead to an ice-sheet, whereas WorldClim, NARR and ERA-Interim produce local ice caps only. In the case of WorldClim, ice is restricted to the Wrangell-St.~Elias mountains, an area presently glaciated. Several of the NCAR and CFSR simulations produce oversized ice sheets whose extent is primarily restricted by the domain boundaries. Differences in regional patterns can also be noted. For instance ERA-Interim ice sheets generally appear more northernly-centered than NARR ice sheets.

Final ice volume and final ice coverage are shown in Figure~\ref{fig:ivolarea}. NCEP/NCAR and CFSR climatologies lead to much larger and more extensive glaciation than the ERA-Interim, NARR and WorldClim forcings. The ERA-Interim and NARR lead to similar ice volumes and glaciated areas, yet regional patterns are different, as shown by Figure~\ref{fig:extent}.

To compare modelled ice sheets of similar size, we selected for each climate forcing the simulation that leads to a glaciated area closest to the approximate size of the LGM Cordilleran Ice Sheet of $2\,\times10^6\,\unit{km^2}$. These qualitative ``best'' runs are presented in Figure~\ref{fig:best} along with their associated temperature depressions. We observe that although similar in size, the selected ice sheets are different in shape and flow patterns. More specifically, NCEP/NCAR, CFSR and ERA-Interim ice sheets cover parts of Yukon Territory and Alaska that were shown to have remained ice-free for at least several glacial cycles \citep{dukrodkin-1999,kaufman-manley-2004}.

