
% ----------------------------------------------------------------------
\section{Model setup}
\label{sec:setup}
% ----------------------------------------------------------------------

\begin{figure}\begin{framed}
	\textbf{Technical note for Constantine}

	All my simulations use the same set of options except of a different \verb|-atmosphere_given_file| (climate forcing) and \verb|-atmosphere_delta_T_file| (temperature offset). Here comes the options that I used; I think you saw them a couple of times now. I simplified paths and filenames and remove output options for readability

	\begin{verbatim}
	mpprun pismr -boot_file boot.nc -Mx 150 -My 300
	-Mz 51 -Lz 5000 -Mbz 11 -Lbz 1000 -ys 0 -ye 10000
	-topg_to_phi 10,30,0.0,200.0
	-atmosphere given,lapse_rate,delta_T -atmosphere_given_file atm.nc
	-atmosphere_given_period 1
	-temp_lapse_rate 6 -atmosphere_lapse_rate_file atm.nc
	-atmosphere_delta_T_file step-cool08.nc
	-surface pdd -pdd_annualize
	-ocean constant,delta_SL -ocean_delta_SL_file step-low120.nc
	-ssa_sliding
	-config_override config.nc
	\end{verbatim}

	And here comes the relevant contents of \verb|config.nc|:

	\begin{verbatim}
	netcdf config \{
	variables:
	  byte pism_overrides ;
	    pism_overrides:bed_deformation_model = "lc" ;
	    pism_overrides:do_thickness_calving = "true" ;
	    pism_overrides:calving_at_thickness = 200. ;
	    pism_overrides:calving_front_stress_boundary_condition = "yes" ;
	    pism_overrides:part_grid = "yes" ;
	    pism_overrides:part_redist = "yes" ;
	    pism_overrides:kill_icebergs = "yes" ;
	    pism_overrides:bootstrapping_geothermal_flux_value_no_var = 0.07 ;
	    pism_overrides:pdd_factor_snow = 0.003 ;
	    pism_overrides:pdd_factor_ice = 0.008 ;
	    pism_overrides:pdd_refreeze = 0.6 ;
	    pism_overrides:pdd_std_dev = 5 ;
	\}
 
	\end{verbatim}

\end{framed}\end{figure}

%\subsection{Surface mass-balance}
%
%Monthly temperatures and precipitation rates are converted into accumulation and ablation at the ice surface using a temperature-index (positive degree-day) model\missingref. The surface mass-balance model is certainly the most critical of the ice sheet model to the present study, and although a positive degree-day model is a very crude approximation of processes occurring at the glacier surface, we will show that a positive degree-day model captures reasonable regional patterns of accumulation and ablation when using accurate climate data.
%
%The number of positive degree-days, defined as the integrand of positive temperatures over time, is computed annually using a formula derived by \citet{calov-greve-2011} that accounts for day-to-day variability of temperature by assuming a normal distribution of variance $\sigma=5\degree C$ along a central (input) value $T$:
%
%\begin{equation}
%	\mathrm{PDD} = \int_{t_0}^{t_1} dt \left[
%		\frac{\sigma}{\sqrt{2\pi}}
%		\exp\left({-\frac{T(t)^2}{2\sigma^2}}\right)
%		+\frac{T(t)}{2}
%		\mathrm{erfc} \left(-\frac{T(t)}{\sqrt{2}\sigma}\right)
%	\right]
%\end{equation}
%
%Where $[t_0; t_1]$ is one year and $\mathrm{erfc}$ is the complementary error function.
%
%Accumulation is computed as the amount of precipitation that falls at times when temperature is below freezing \todo{Check PISM accu. model}.
%
%\subsection{Ice flow dynamics}
%To resolve ice dynamics, PISM specifically uses a heuristic combination of two classical and well-based zeroth-order shallow-ice approximations through a weighting function \citep{bueler-brown-2009}. The Shallow Shelf Approximation \citep[SSA,][]{morland-1987,weis-etal-1999} is used as a ``sliding law'' for the Shallow Ice Approximation \citep[SSA,][]{hutter-1983}. Depth-averaged velocities $\vec{v}$ are computed as a weighted sum of SIA velocities $\vec{v_{\mathrm{SIA}}}$ and SSA velocities $\vec{v_{\mathrm{SSA}}}$ by
%
%\begin{equation}
%	\vec{v} = f(|\vec{v_{\mathrm{SSA}}}|)\vec{v_{\mathrm{SIA}}}
%	+ (1-f(|\vec{v_{\mathrm{SSA}}}|))\vec{v_{\mathrm{SSA}}},
%\end{equation}
%
%with
%
%\begin{equation}
%	f(|\vec{v_{\mathrm{SSA}}}|) = 1
%	- \frac{2}{\pi}\arctan{\frac{|\vec{v_{\mathrm{SSA}}}|^2}{100^2}}.
%\end{equation}
%
%This approach, although using a heuristic, allows realistic representation of horizontal shear and longitudinal stresses where they matter \citep{bueler-brown-2009}.
%
%
%The SSA velocities are determined by a a pseudo-plastic sliding law in which the yield stress is depends on the availability of basal water and a prescibed till friction angle. We apply lower friction angles at lower elevation to model more slippery beds due to the presence of marine sediments.
%
%\begin{equation}
%	\phi = \left\{\begin{array}{llc}
%		10\,\unit{\degree C} & \mathrm{for} &               z<  0\,\unit{m} \\
%		z/10 + 10°           & \mathrm{for} &   0\,\unit{m}<z<200\,\unit{m} \\
%		30\,\unit{\degree C} & \mathrm{for} & 200\,\unit{m}<z               \\
%	\end{array}\right.
%\end{equation}
%
%\subsection{Glacier bed interface}
%
%The ice-sheet bed topography responds to the ice-loads following a regional isostasy model with flexure and relaxation.
%
%Geothermal heat flux is prescribed to 70\,\unit{W\,m^{-2}}
%
%\subsection{Simulation design and numerical implementation}
%
%The simulations were run with PISM, a parallel ice sheet model (version stable 0.5.11). Documentation on the code can be found in several publications \citep[e.g.,][]{bueler-brown-2009,aschwanden-etal-2012} and more extensively online \citep[{\url{http://www.pism-docs.org}},][]{web:pism}.

%Model equations are discretized horizontally on a 10~km-resolution, 150 by 300 points grid, and vertically within the ice on 21 irregularly-spaced layers. Time-stepping is adaptive, but a maximum allowed time step of 0.08 is used to ensure monthly capture of climate data by the surface mass-balance model.
%
%The simulations were run in parallel on 9~computation nodes of 8~cores each, for a total of 72~cores, on the supercomputer ekman at the PDC center for high performance computing of Kungliga Tekniska högskolan in Stockholm. Each simulation completed in about 10~hours.

%In this study we model glacial inception and growth of the Cordilleran ice sheet to a state nearing its last glacial maximum. For each present-day climate forcing we run a series of six simulations by applying a constant, homogeneous temperature offset of 4 to 9 \unit{\degree C} over the entire model domain. Each simulation initializes from ice-free conditions and is run for 10~000~years, a time interval we think to be representative of the rapid growth of the ice sheet from nearly ice-free conditions to last glacial maximum extent \missingref. No precipitation corrections were used.
%

