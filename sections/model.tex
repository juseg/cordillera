% sections/model.tex
% ----------------------------------------------------------------------
\section{Model setup}
\label{sec:model}
% ----------------------------------------------------------------------

We use PISM, a Parallel Ice Sheet Model (version stable 0.5.11, e.g., \citet{bueler-brown-2009,winkelmann-etal-2011,aschwanden-etal-2012,web:pism}. Given basal topography and climate forcing, the model computes the thermal and dynamic state of an ice sheet and the associated lithospheric response. Our modelling domain encompasses most of the area covered by the Cordilleran Ice Sheet at LGM with the exception of western Alaska and includes non-glaciated regions in northern Yukon Territory and interior Alaska. Our simulations are performed in parallel on 32 cores at the Swedish National Supercomputing Center.

\subsection{Ice thermodynamics}

The depth-averaged ice velocity $\vec{v}$ is computed as a weighted sum of the Shallow Ice Approximation (SIA) velocity $\vec{v_{\mathrm{SIA}}}$ and the Shallow Shelf Approximation (SSA) velocity $\vec{v_{\mathrm{SSA}}}$ \citep{bueler-brown-2009} by

\begin{equation}
	\vec{v} = f(|\vec{v_{\mathrm{SSA}}}|)\vec{v_{\mathrm{SIA}}}
  + (1-f(|\vec{v_{\mathrm{SSA}}}|))\vec{v_{\mathrm{SSA}}},
\end{equation}

where

\begin{equation}
	f(|\vec{v_{\mathrm{SSA}}}|) = 1
	- \frac{2}{\pi}\arctan{\frac{|\vec{v_{\mathrm{SSA}}}|^2}{100^2}}.
\end{equation}

SIA and SSA velocities are computed by finite difference methods on a 10km-resolution horizontal grid of 300 by 150 points (the modelling domain). Ice rheology depends on temperature and water content through an enthalpy formulation \citep{aschwanden-blatter-2009,aschwanden-etal-2012}. Enthalpy is computed in three dimensions in up to 51 irregularly spaced layers in ice and temperature is computed further in 11 regularly spaced layers in bedrock. A homogeneous geothermal heat flux of 70\,\unit{W\,m^{-2}} provides the lower boundary condition to the bedrock thermal model, and surface air temperature from the climate forcing provides the upper boundary condition to the ice enthalpy model.

SSA velocities are determined by a pseudo-plastic sliding law where the yield stress depends on the availability of basal water and a prescribed till friction angle $\phi$. We apply a range of till friction angles between 10 and 30\,\degree, where the lower friction angles occur at low elevation to mimic the more deformable beds associated with presence of marine sediments:
\julien{phi here is independent from basal water which is computed by the model (and sliding depends on both) so the 10 to 30 degree range is meant to reflect different bedrock/sediment surfaces rather than dry vs wet beds.}

\begin{equation}
	\phi = \left\{\begin{array}{llrll}
		10      & \mathrm{for} &      &z&<  0 \\
		z/10+10 & \mathrm{for} &   0 <&z&<200 \\
		30      & \mathrm{for} & 200 <&z&     \\
	\end{array}\right.
\end{equation}

Where $\phi$ is given in degrees and $z$ in meters above contemporary sea level. Basal topography is derived from the ETOPO1\citep{data:etopo1} combined topography and bathymetry dataset (Fig.~\ref{fig:topo}). It responds to ice load following a regional isostasy model that includes lithospheric flexure and mantle relaxation \citep{lingle-clark-1985}.

% ----------------------------------------------------------------------

\subsection{Surface mass-balance}

Ice surface accumulation and ablation are computed from monthly mean surface air temperature and monthly precipitation by a temperature-index (positive degree-day) model\citep{hock-2003}. Ice accumulation is equal to precipitation when temperature is below 0\,\degC, and decreases to zero linearly with temperature between 0 and 2\,\degC. Ice ablation is computed from the number of positive degree-days, defined as the integral of temperatures above 0\,\degC in one year. Positive degree-days are calculated using a expression by \citet{calov-greve-2005} that accounts for temperature variability assuming a normal distribution along a central (input) value $T$:

\begin{equation}
	\mathrm{PDD} = \int_{t_0}^{t_1} dt \left[
		\frac{\sigma}{\sqrt{2\pi}}
		\exp\left({-\frac{T(t)^2}{2\sigma^2}}\right)
		+\frac{T(t)}{2}
		\mathrm{erfc} \left(-\frac{T(t)}{\sqrt{2}\sigma}\right)
	\right],
\end{equation}

where the integration interval $[t_0; t_1]$ is one week and $\mathrm{erfc}$ is the complementary error function. The temperature standard deviation $\sigma$ is a constant model parameter and was assigned a value of 3.068094\degC which is the summer (JJA) mean, model domain-averaged monthly standard deviation of daily mean temperature from monthly mean temperature, as computed from NARR data similarly to \citet{seguinot-inreview}. The ablation model incorporates degree-day factors of 3.04~\unit{mm\,K^{-1}\,day^{-1}} for snow and 4.59~\unit{mm\,K^{-1}\,day^{-1}} for ice, as derived from mass-balance measurements on contemporary glaciers from the Coast Mountains and Rocky Mountains in British Columbia \citep{shea-etal-2009}.

% ----------------------------------------------------------------------

\subsection{Atmospheric corrections}

Prior to surface-mass balance computation, the model dynamically applies a lapse-rate correction to surface air temperature. This correction account for evolution of ice thickness on the one hand, and differences between the climate forcing reference topography and the ice flow model topography on the over hand. It uses a reference topography distinct from the basal topography and specific to each climate forcing dataset (Fig.~\ref{fig:topo}). An annual lapse-rate of 6\unit{\degree C\,km^{-1}} is used in all simulations. No lapse-rate corrections applies to precipitation.

As we aim to model glacial inception and growth of the Cordilleran Ice Sheet towards a configuration as last attained during the LGM, we mimic LGM conditions by applying constant temperature offsets homogeneously over the modelling domain. 

Each simulation starts from ice-free conditions and runs for 10\,kyr, a time interval representative of the rapid build-up of the last Cordilleran Ice Sheet from nearly ice-free to full glacial conditions \citep{clague-1989,stroeven-etal-2010}. Precipitation is corrected using the SeaRISE-Greenland formula for paleo-precipitation correction \citep{huybrechts-2002}, which implies a 7.3\% decrease of precipitation per degree Celsius of temperature lowering.

