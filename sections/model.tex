
% ----------------------------------------------------------------------
\section{Model setup}
\label{sec:model}
% ----------------------------------------------------------------------

We use PISM, a parallel ice sheet model (version stable 0.5.11). Documentation on the code can be found in several publications \citep[e.g.,][]{bueler-brown-2009,winkelmann-etal-2011,aschwanden-etal-2012} and more extensively on-line \citep[{\url{http://www.pism-docs.org}},][]{web:pism}.

\subsection{Surface mass-balance}

Monthly temperature and precipitation are converted to surface accumulation and ablation by a temperature-index (positive degree-day) model. Ice accumulation is equal to precipitation when temperature is below 0\,\degC, and decreases linearly with temperature between 0 and 2\,\degC. Ice ablation is computed from the number of positive degree-days, defined as the integral of temperatures above 0\,\degC in one year. Positive degree-days are calculated by the \citet{calov-greve-2005} formula to accounts for day-to-day temperature variability by assuming a normal distribution of variance $\sigma=5\degC$ along a central (input) value $T$:

\begin{equation}
	\mathrm{PDD} = \int_{t_0}^{t_1} dt \left[
		\frac{\sigma}{\sqrt{2\pi}}
		\exp\left({-\frac{T(t)^2}{2\sigma^2}}\right)
		+\frac{T(t)}{2}
		\mathrm{erfc} \left(-\frac{T(t)}{\sqrt{2}\sigma}\right)
	\right],
\end{equation}

where the integration interval $[t_0; t_1]$ is one week and $\mathrm{erfc}$ is the complementary error function. The ablation model uses degree-day factors of 3.04~\unit{mm\,K^{-1}\,day^{-1}} for snow and 4.59~\unit{mm\,K^{-1}\,day^{-1}} for ice, as derived from mass-balance measurements on present glaciers from the Coast and Rocky Mountains in British Columbia \citep{shea-etal-2009}.

A positive degree-day model is a crude parametrization of physical processes occurring at a glacier's surface. Yet we believe that it captures reasonably accurate patterns of accumulation and ablation for our study area considering the employed climate data.

\subsection{Ice thermodynamics}

Ice flow velocities are computed by a heuristic the Shallow Shelf Approximation (SSA) is used as a ``sliding law'' to a polythermal Shallow Ice Approximation (SIA) \citep{bueler-brown-2009}. Depth-averaged velocities $\vec{v}$ are computed as a weighted sum of SIA velocities $\vec{v_{\mathrm{SIA}}}$ and SSA velocities $\vec{v_{\mathrm{SSA}}}$ by

\begin{equation}
	\vec{v} = f(|\vec{v_{\mathrm{SSA}}}|)\vec{v_{\mathrm{SIA}}}
  + (1-f(|\vec{v_{\mathrm{SSA}}}|))\vec{v_{\mathrm{SSA}}},
\end{equation}

with

\begin{equation}
	f(|\vec{v_{\mathrm{SSA}}}|) = 1
	- \frac{2}{\pi}\arctan{\frac{|\vec{v_{\mathrm{SSA}}}|^2}{100^2}}.
\end{equation}

SIA and SSA velocities are computed by finite difference methods on a 10km-resolution horizontal grid of 300 by 150 points. SIA velocities follow temperature-dependant creep. Temperatures are derived from an enthalpy formulation \citep{aschwanden-etal-2012} and computed in three dimensions on 51 irregularly spaced layers in ice and 11 regularly spaced layers in rock. An homogeneous geothermal heat flux of 70\,\unit{W\,m^{-2}} is provided as lower boundary condition to the thermal model, and the surface air temperature from the various climate forcing is used as upper boundary condition.

SSA velocities are determined by a pseudo-plastic sliding law where the yield stress depends on the availability of basal water and a prescribed till friction angle$\phi$. We apply lower friction angles at low elevation to mimic the more slippery beds associated to the presence of marine sediments:

\begin{equation}
	\phi = \left\{\begin{array}{llrll}
		10      & \mathrm{for} &      &z&<  0 \\
		z/10+10 & \mathrm{for} &   0 <&z&<200 \\
		30      & \mathrm{for} & 200 <&z&     \\
	\end{array}\right.
\end{equation}

With $\phi$ in degrees and $z$ in meters above present sea level. The glacier bed topography responds to the ice load following a regional isostasy model with flexure and relaxation \citep{lingle-clark-1985}.

\subsection{Simulation design}

In this study we model glacial inception and growth of the Cordilleran ice sheet to a state approaching its last glacial maximum. We mimic glacial conditions by applying constant temperature offsets homogeneously over the entire domain. Each simulation starts from ice-free conditions and is run for 10,000 years, a time interval representative of the rapid build-up of the last Cordilleran Ice Sheet from nearly ice-free condition to last glacial maximum extent \citep{clague-1989}. Our simulations were run in parallel on 32 cores at the Swedish National Supercomputing Center.

