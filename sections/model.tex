% sections/model.tex
% ----------------------------------------------------------------------
\section{Model setup}
\label{sec:model}
% ----------------------------------------------------------------------

Given basal topography, sea level, geothermal heat flux and climate forcing, the model computes ice extent and thickness, its thermal and dynamic state, and the associated lithospheric response. Our modelling domain encompasses the entire area covered by the Cordilleran Ice Sheet at LGM including independent ice growing on the western Alaskan ranges and the Brooks Range and intermediate expanses of ice-free terrain in northern Yukon Territory and interior Alaska (Fig.~\ref{fig:locmap}).

As we aim to model glacial inception and growth of the Cordilleran Ice Sheet towards a configuration as last attained during the LGM, we mimic palaeo-climatic conditions by applying constant temperature offsets homogeneously over the modelling domain. Each simulation starts from ice-free conditions and runs for 10\,ka, a time interval representative of the rapid build-up of the last Cordilleran Ice Sheet from nearly ice-free to full glacial conditions \citep{clague-1989,stroeven-etal-2010}. Our simulations are performed in parallel on 32 cores at the Swedish National Supercomputing Center.

\subsection{Ice thermodynamics}

The central part of an ice sheet model consists of the computation of flow velocity which itself depends on temperature. PISM is a shallow model, which implies that the balance of stresses is approximated based on their predominant components. On the other hand, the model is polythermal: it accounts for differences in temperature and softness within the ice column.

The Shallow Shelf Approximation (SSA) is used as a ``sliding law'' for the Shallow Ice Approximation (SIA) \citep{bueler-brown-2009,winkelmann-etal-2011}. SIA and SSA velocities are computed by finite difference methods on a 10\,km-resolution horizontal grid of 300 by 150 points (the modelling domain). Ice softness depends on temperature and water content through an enthalpy formulation \citep{aschwanden-blatter-2009,aschwanden-etal-2012}. Enthalpy is computed in three dimensions in up to 51 irregularly spaced layers within the ice, and temperature is additionally computed in 11 regularly spaced layers in bedrock to a depth of 1\,km. A uniform geothermal heat flux of 70\,\unit{W\,m^{-2}} provides the lower boundary condition to the bedrock thermal model, and surface air temperature from the climate forcing provides the upper boundary condition to the ice enthalpy model.

A pseudo-plastic sliding law \citep[supplement]{aschwanden-etal-2013} relates the bed-parallel shear stress and the sliding velocity. The yield stress is modelled using the Mohr-Coulomb criterion. The till friction angle $\phi$ varies from 10 to 30\degree\ and is a function of current bed elevation, with lowest values occurring at low elevations,

\begin{equation}
	\phi = \left\{\begin{array}{llrll}
		10      & \mathrm{for} &      &z&<  0 \\
		z/10+10 & \mathrm{for} &   0 <&z&<200 \\
		30      & \mathrm{for} & 200 <&z&     \\
	\end{array}\right.
\end{equation}

where $\phi$ is given in degrees and $z$ in meters above current sea level. This models weakening of the till associated with the presence of marine sediments \citep{martin-etal-2011,aschwanden-etal-2013}. Basal topography (Fig.~\ref{fig:topo}) is derived from the ETOPO1 combined topography and bathymetry dataset with a resolution of 1\,arc-minute \citep{data:etopo1}. Sea level is lowered by 120\,m and basal topography responds to ice load following a regional isostasy model that includes lithospheric flexure and mantle relaxation \citep{lingle-clark-1985}.

% ----------------------------------------------------------------------

\subsection{Surface mass-balance}

Ice surface accumulation and ablation are computed from monthly mean surface air temperature and monthly precipitation by a temperature-index (positive degree-day) model \citep{hock-2003}. Ice accumulation is equal to precipitation when temperature is below 0\,\unit{\degree C}, and decreases to zero linearly with temperature between 0 and 2\,\unit{\degree C}. Ice ablation is computed from the number of positive degree-days, defined as the integral of temperatures above 0\,\unit{\degree C} in one year. 

The positive degree-day integral \citep{calov-greve-2005} is numerically approximated using week-long sub-intervals. It accounts for temperature variability assuming a normal distribution along a central (input) value. The temperature standard deviation is a constant model parameter and was assigned a value of 3.07\,\unit{\degree C}, which corresponds to the mean summer (JJA), model domain-averaged monthly standard deviation of daily mean temperature from monthly mean temperature, as computed from North American Regional Reanalysis data \citep{data:narr} in a manner similar to \citet{seguinot-inpress}. The ablation model incorporates degree-day factors of 3.04\,\unit{mm\,K^{-1}\,day^{-1}} for snow and 4.59\,\unit{mm\,K^{-1}\,day^{-1}} for ice, as derived from mass-balance measurements on contemporary glaciers from the Coast Mountains and Rocky Mountains in British Columbia \citep{shea-etal-2009}.

% ----------------------------------------------------------------------

\subsection{Atmospheric corrections}

Prior to surface mass balance computation, the model dynamically applies a lapse-rate correction to surface air temperature. This correction accounts for the evolution of ice thickness on the one hand, and differences between the climate forcing reference topography and the ice flow model basal topography on the other hand. It uses a reference topography distinct from the modelled basal topography and specific to each climate forcing dataset (Fig.~\ref{fig:topo}). An annual lapse rate of 6\,\unit{\degree C\,km^{-1}} is used in all simulations. No corrections apply to precipitation changes with elevation.
