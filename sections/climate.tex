% sections/climate.tex
% ----------------------------------------------------------------------
\section{Climate forcing}
\label{sec:climate}
% ----------------------------------------------------------------------

We drive the mass balance routine of PISM with temperature and precipitation data from one observational dataset and climatologies from three global and one regional reanalyses. We here present the different climate datasets used in this study.

\subsection{Observational data: WorldClim}

WorldClim is a high-resolution climate dataset built from meteorological observations \citep{data:worldclim}. These climate surfaces were generated by spatial interpolation, using 30\,arc\,s-aggregated hole-filled SRTM elevation data \citep{jarvis-etal-2008}, between a large set of measurements taken at weather stations over global land areas, using . This provides a resolution much higher than attained through GCM in areas where sufficient amount of weather stations exist.

 In the study area, the spatial distributions of summer (JJA) and winter (DJF) air surface temperatures in WorldClim generally show a decrease from south to north (JJA) and from southwest to northeast (DJF), such that regions further inland experience colder winters (Fig.~\ref{fig:temp}). It should be noted that temperature gradients are reversed (colder inland than at the coast) and much stronger in winter than in summer. Temperatures rises well above zero during the summer months over the entire modelling domain, except for the highest mountain peaks, and even regions where mean annual temperatures are well below freezing do experience warm summers. In other words, there is a strong contrast in seasonality between coastal temperate-climate and inland continental-climate regions.

The spatial distribution of JJA and DJF precipitation rates in WorldClim reveal a strong precipitation decline beyond the coastal regions (Fig.~\ref{fig:prec}), primarily as a result of the continuous orographic barrier formed by the Boundary Ranges, the Coast Mountains and the Cascades. From an ice sheet mass balance point of view, this contrast becomes even stronger by the difference in timing of the precipitation peak through the year. Whereas coastal regions experience most precipitation as snow during the accumulation season (DJF), inland regions experience dry winters and most of the annual precipitation consequently falls as rain during the ablation season (JJA).

In regions such as the northern Yukon Territory and interior Alaska, dry winters and warm summers prohibit ice to accumulate and glaciers to form, despite strongly negative mean annual temperatures. In order to account for these strong gradients in seasonality, we use monthly rather than annual averages of temperature and precipitation to drive the ice sheet model.

There are two problems inherent to the use of WorldClim data in our region of study. Firstly, the density of weather stations used by WorldClim in the Northern American Cordillera is highly inhomogeneous. Although good coverage exists for the southern part of our modelling domain, several hundred kilometres can separate neighbouring stations in the north \citep{data:worldclim}. Secondly, WorldClim lacks of data offshore, which would have been particularly useful for the on the Pacific continental shelf which was partly covered by ice during the LGM \citep{jackson-clague-1991}.

% ----------------------------------------------------------------------

\subsection{Reanalysis data: NCEP/NCAR, ERA-Interim, CFSR and NARR}

In addition to the WorldClim data, we use surface air temperatures and precipitation rates from three global climate reanalyses and one regional climate reanalysis to inform the mass balance routine of PISM: the NCEP/NCAR reanalysis, the ERA-Interim reanalysis, the Climate System Forecast Reanalysis (CFSR), and the North American Regional Reanalysis (NARR). Monthly climatologies from NCEP/NCAR and NARR reanalysis were provided by the NOAA/OAR/ESRL Physical Sciences Division, Boulder, Colorado, USA, from their website at \url{http://www.esrl.noaa.gov/psd/} whereas monthly climatologies from the ERA-Interim and CFSR reanalyses were computed from their monthly mean time series (Table \ref{tab:reanalyses}).

The spatial distributions of JJA air surface temperatures and DJF precipitation rates from the four reanalyses datasets are shown in Figures~\ref{fig:temp} and~\ref{fig:prec}. Summer temperature and winter precipitation are most relevant to the glacier model as they respectively drive summer melt and winter accumulation.

Because reanalysis are a product combinaing observations and circulation modelling, we believe that the input from reanalyses may perform better than WorldClim in sparsely monitored regions such as the northernmost American Cordillera.

% ----------------------------------------------------------------------

\subsection{Preprocessing and lapse-rate corrections}

WorldClim, NCEP/NCAR, ERA-Interim, CFSR and NARR climatologies were re-projected to Canadian Atlas Lambert conformal conic projection (EPSG code~3978) and bilinearly interpolated to a 10\,km model grid (see section~\ref{sec:model}). In addition, WorldClim data was extrapolated to cover oceans grid points using a nearest-neighbour approach. These last two steps are not presented on figures~\ref{fig:temp} and~\ref{fig:prec}.

An alternative forcing was prepared for CFSR to correct for artefacts seen in its precipitation field (Fig.~\ref{fig:prec}). A smoothing of the precipitation field was achieved by averaging data locally in a circular kernel of 7\,pixels in diameter prior to re-projection. These preprocessing steps were performed in GRASS~GIS.

Throughout the simulation, the numerical model dynamically applies temperature lapse-rate corrections that account for differences between the climate forcing reference topography and the ice flow model topography on the one hand, and the evolution of ice thickness on the over hand. For reanalysis data, the GCM surface topography (or surface geopotential height) was used as reference topography. For the WorldClim data, the aggregated hole-filled SRTM altitude field is used as reference topography. As a basal condition for the ice flow model, we use the ETOPO1\citep{data:etopo1} combined topography and bathymetry dataset. The reference topographies from the five forcing datasets, alongside the ETOPO1 data are shown in figure~\ref{fig:topo}. An annual lapse-rate of 6\unit{\degree C\,km^{-1}} is used in all simulations. No lapse-rate corrections apply to precipitation rates.

