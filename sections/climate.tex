% sections/climate.tex
% ----------------------------------------------------------------------
\section{Climate forcing}
\label{sec:climate}
% ----------------------------------------------------------------------

Our atmospheric forcing consists of monthly climatologies of surface air temperature and precipitation obtained from one observational dataset, three global reanalyses and one regional reanalysis.

\subsection{Observational data: WorldClim}

WorldClim is a high-resolution climate dataset built from meteorological observations \citep{data:worldclim}. It was built by spatial interpolation over global land areas, using 30\,arc\,s-aggregated hole-filled SRTM elevation data \citep{data:srtm} and GTOPO30 elevation data \citep{data:gtopo30}, between a large set of measurements taken at weather stations worldwide. This provides a resolution much higher than attained by GCMs in areas where sufficient amount of observations exists.

Within our modelling domain, the spatial distributions of summer (JJA) and winter (DJF) air surface temperatures in WorldClim generally show a decrease of temperature from south to north (JJA) and from south-west to north-east (DJF), such that regions further inland experience colder winters (Fig.~\ref{fig:temp}). It should be noted that temperature gradients are reversed (colder inland than at the coast) and much stronger in winter than in summer. Temperatures rises well above zero during the summer months over the entire modelling domain, except for the highest mountain peaks, and even regions where mean annual temperatures are well below freezing point do experience warm summers. In other words, there is a strong contrast in seasonality between coastal temperate-climate and inland continental-climate regions.

The spatial distribution of JJA and DJF mean precipitation rates in WorldClim reveal a strong precipitation decline beyond the coastal regions (Fig.~\ref{fig:prec})\qiong{Change unit to mm per month.}, primarily as a result of the continuous orographic barrier formed by the Boundary Ranges, the Coast Mountains and the Cascades. From an ice sheet mass balance point of view, this contrast becomes even stronger due to the difference in timing of the precipitation peak through the year. Whereas coastal regions experience most precipitation as snow during the accumulation season (DJF), inland regions experience dry winters and most of the annual precipitation consequently falls as rain during the ablation season (JJA).

In regions such as the northern Yukon Territory and interior Alaska, dry winters and warm summers prohibit ice to accumulate and glaciers to form, despite strongly negative mean annual temperatures. In order to account for these strong gradients in seasonality, we use monthly rather than annual averages of temperature and precipitation to drive the ice sheet model.

There are two problems inherent to the use of WorldClim data in our study area. Firstly, the density of weather stations used by WorldClim in the Northern American Cordillera is highly inhomogeneous. Although good coverage exists for the southern part of our modelling domain, several hundred kilometres can separate neighbouring stations in the north \citep{data:worldclim}. Secondly, WorldClim lacks data offshore, which would have been particularly useful over the Pacific continental shelf which was partly covered by ice during the LGM \citep{jackson-clague-1991}.

% ----------------------------------------------------------------------

\subsection{Reanalysis data: NCEP/NCAR, ERA-Interim, CFSR and NARR}

In addition to the WorldClim data, we use surface air temperatures and precipitation rates from three global atmospheric reanalyses and one regional atmospheric reanalysis to inform the mass balance routine of PISM: the NCEP/NCAR reanalysis \citep{data:ncar}, the ERA-Interim reanalysis \citep{data:erai}, the Climate System Forecast Reanalysis (CFSR) \citep{data:cfsr}, and the North American Regional Reanalysis (NARR) \citep{data:narr}. Monthly climatologies from NCEP/NCAR and NARR reanalyses were provided by the \citet{web:psd} whereas monthly climatologies from the ERA-Interim and CFSR reanalyses were computed from their monthly mean time series (Table \ref{tab:reanalyses}).

The spatial distributions of JJA air surface temperatures and DJF precipitation rates from the four reanalyses climatologies are shown in Figures~\ref{fig:temp} and~\ref{fig:prec}. Summer temperature and winter precipitation are most relevant to the glacier model as they respectively drive summer melt and winter accumulation. Because reanalyses include observational information through data assimilation, they are closer to observations than simple GCMs in densely monitored regions, while offering physically based output in sparsely monitored regions such as the northern part of our modelling domain \citep{bengtsson-etal-2007}.

% ----------------------------------------------------------------------

\subsection{Preprocessing and lapse-rate corrections}

WorldClim, NCEP/NCAR, ERA-Interim, CFSR and NARR climatologies were re-projected to Canadian Atlas Lambert conformal conic projection (EPSG code~3978) and bilinearly interpolated to the 10\,km-resolution model grid using the module r.proj from GRASS~GIS \citep{soft:grass}. In addition, WorldClim data was extrapolated to cover grid points in the ocean using the nearest-neighbour algorithm \citep{soft:scipy}. Note that here we choose to present original rather than interpolated data, in order to highlight differences between datasets related to spatial resolution (Fig.~\ref{fig:temp} and~\ref{fig:prec}).

The CFSR climatology presents wave-like artefacts in its precipitation field (Fig.~\ref{fig:prec}). These are a common feature in GCM simulations due to the failure of spectral models to resolve the very local nature of topography-induced precipitation events, which becomes particularly visible in the high-resolution, less spatially smoothed data. Therefore, an alternative forcing was prepared for CFSR by smoothing its precipitation field. This was achieved by averaging data locally in a circular kernel of 7\,pixels in diameter prior to re-projection using the module r.neighbors from GRASS~GIS \citep{soft:grass}. Finally, input files for PISM were prepared using \citet{web:nc4py}.
