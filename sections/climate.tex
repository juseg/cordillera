
% ----------------------------------------------------------------------
\section{Climate forcing}
\label{sec:climate}
% ----------------------------------------------------------------------

\subsection{Observation data}

WorldClim \citep{data:worldclim} is a high resolution climate dataset built from observation data. These climate surfaces were generated by spatial interpolation between a large set of weather stations using a 30\,arc\,s-aggregated flavour of hole-filled SRTM data over global land areas. This provides a resolution much higher than attained by general circulation models.

However, the density of weather stations used by WorldClim in the Northern American Cordillera is highly inhomogeneous, and if good coverage exists in the southern parts of our modelling domain, several hundred kilometers can separate nearby stations in the north \citep{data:worldclim}.

A second drawback of the dataset in an ice sheet modelling view is the lack of data on marine surfaces, particularly on the Pacific continental shelf where glaciers have advanced during the last glacial cycle\needref.

% ----------------------------------------------------------------------

\subsection{Reanalysis data}

In addition to the WorldClim data, we use surface air temperature and precipitation rate data from three global and one regional climate reanalysis to force the ice sheet model: the NCEP/NCAR reanalysis, the ERA-Interim reanalysis, the Climate System Forecast Reanalysis (CFSR) and the North American Regional Reanalysis (NARR). Monthly climatologies from NCEP/NCAR and NARR reanalysis were provided by the NOAA/OAR/ESRL PSD, Boulder, Colorado, USA, from their Web site at \url{http://www.esrl.noaa.gov/psd/} and monthly climatologies from the ERA-Interim and CFSR reanalyses were computed from their monthly mean timeseries. Further information on the data used is gatherd in Table \ref{tab:reanalyses}. As a mixed product from observations and a circulation model, we believe that reanalysis may perform best in poorly monitored regions such as the northernmost American Cordillera.

\begin{table}[t]
	\caption{Characteristic of reanalysis climatologies used to force the ice sheet model.}
	\label{tab:reanalyses}
	\vskip4mm
	\centering
	\begin{tabular}{lllll}
		\tophline
		Reanalysis& Spatial coverage& Averaging period& Resolution& Description\\
		\middlehline
		NCEP/NCAR&  global&     1981 -- 2010& 1.875\degree& \citet{data:ncar}\\
		ERA-Interim&global&     1979 -- 2011& 1.000\degree& \citet{data:erai}\\
		CFSR&       global&     1979 -- 2010& 0.325\degree& \citet{data:cfsr}\\
		NARR&       North America& 1979 -- 2000& 32\,km& \citet{data:narr}\\
		\bottomhline
	\end{tabular}
\end{table}

% ----------------------------------------------------------------------

\subsection{Cordilleran climates}

\begin{figure}[t]
	\vspace*{2mm}
	\begin{center}
		\includegraphics[width=13cm]{cordillera-climate-temp}
	\end{center}
	\caption{Summer temperature maps from the five datasets used in this study and winter temperature map from the WorldClim dataset.}
	\label{fig:temp}
\end{figure}

Figure~\ref{fig:temp} shows the spatial distribution of summer (JJA) air surface temperatures from the five climate datasets used in the study, and winter (DJF) air surface temperatures from the WorldClim interpolated observation data. Summer temperatures are most relevant to the glacier model as they drive summer melt.

Temperatures generally decrease from south to north and regions further inland experience colder winters. It should be noted that the temperature gradient is much stronger in the winter than it is in the summer. At low elevations temperature get well above zero during the summer months over the entire modelling domain. In other words, there is a strong seasonality contrast beween coastal and inland regions, and regions where mean annual temperatures are well below freezing do experience warm summers.

\begin{figure}[t]
	\vspace*{2mm}
	\begin{center}
		\includegraphics[width=13cm]{cordillera-climate-prec}
	\end{center}
	\caption{Winter precipitation rate maps from the five datasets used in this study and summer precipitation map from the WorldClim dataset.}
	\label{fig:prec}
\end{figure}

Figure~\ref{fig:prec} shows the spatial distribution of winter (DJF) precipitation rates from the five climate datasets used in the study, and summer (JJA) precipitation rates from the WorldClim interpolated observation data. Winter precipitation is most relevant to the glacier model as it drives winter accumulation.

Coastal regions generally recieve much more precipitation than inland ones as a results of the continuous topographic barrier formed by the Boundary Ranges, the Coast Mountains and the Cascadia. In a glacier mass-balance view, this contrast is made even stronger by the difference in timing of the precipitation peak through the year. Although coastal regions experience most precipitation during the accumulation season, inland regions experience dry winters and most of the precipitation falls as rain during the summer months.

In regions like Northern Yukon and Alaska, dry winters and warm summers are unfavourable conditions to accumulation and glacial inception despite of the strongly negative mean annual temperatures at present. In order to account for these strong gradients in seasonality, we use monthly means to drive the ice sheet model. Previous runs forced by annual means of temperature and precipitation not presented here systematically conducted to unrealistic ice build-up in northern Yukon and Alaska under present-day climate while leaving the Coast Mountains free of ice.

% ----------------------------------------------------------------------

\subsection{Preprocessing and lapse-rate corrections}

Air surface temperature and precipitation rates from the NCEP/NCAR, ERA-Interim, CFSR and NARR reanalyses and WorldClim data were reprojected to Canadian Atlas Lambert conformal conic projection (EPSG code~3978) and bilinearly interpolated to the model resolution of 10\,km. In addition, WorldClim data was extrapolated to marine surfaces using a nearest-neighbour approach. These last two steps are not presented on figures~\ref{fig:temp} and~\ref{fig:prec}.

For the CFSR, an alternative forcing was prepared by smoothing the precipitation field in order to correct for artifacts seen in figure~\ref{fig:prec}. This was done by averaging data locally in a circular neighbourhood of 7 pixels in diameter prior to reprojection. Yet we will show that this smoothing has very limited effect on the ice sheet model outcome. All the preprocessing steps described above were made in GRASS~GIS using scripts available on the first author's website.

\begin{figure}[t]
	\vspace*{2mm}
	\begin{center}
		\includegraphics[width=13cm]{cordillera-climate-topo}
	\end{center}
	\caption{Reference topography used for temperature lapse-rate corrections from the five climate datasets used in the study and ETOPO1 topography used as basal condition for the ice flow model.}
	\label{fig:topo}
\end{figure}

Throughout the simulation, the numerical model dynamically applies temperature lapse-rate corrections that account for differences between the climate forcing reference topography and the ice flow model topography on the one hand, and the evolution of ice thickness on the over hand. For reanalysis data, the circulation model surface topography (or surface geopotential height) was used as reference topography. For the WorldClim data, the aggregated hole-filled SRTM altitude field is used as reference topography. As a basal condition for the ice flow model, we use the ETOPO1\citep{data:etopo1} combined topography and bathymetry dataset. The reference topographies from the five forcing datasets, alongside the ETOPO1 data are shown in figure~\ref{fig:topo}. A lapse-rate of 6\unit{\degree C\,km^{-1}} is used in all simulations and no lapse-rate corrections apply to precipitation rates.

