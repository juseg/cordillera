% section/intro.tex
% ----------------------------------------------------------------------
\introduction
\label{sec:intro}
% ----------------------------------------------------------------------

At the Last Glacial Maximum (LGM), glaciers of a size comparable to the present Greenland and Antarctic ice sheets covered parts of Northern America (Laurentide, Cordilleran and Innuitian ice sheets) and northern Eurasia (Fennoscandian Ice Sheet). Numerical modelling of these former ice masses allows for a comparison between glaciological theories embedded in the models and geomorphological traces underpinning palaeo-glaciological reconstructions. Yet, a major obstacle in this exercise resides in large uncertainties concerning the climate forcing, typically atmospheric temperature and precipitation, required as input to numerical glacier models \citep{hebeler-etal-2008}. This includes uncertainty in the representation of Earth's present climate in regions of poor station coverage, and even larger uncertainty concerning accurate reconstructions of past climate change.

Arguably the most physically sound way to force an ice sheet model for simulations of glacial history is to couple it with a General Circulation Model \citep[GCM;][]{yoshimori-etal-2001,calov-etal-2002,abeouchi-etal-2007,charbit-etal-2013}. However the computational demand of GCMs is such that only models of intermediate complexity can run on the time-scales of tens of thousands of years characteristic of ice sheet growth and decay.

Climatologies obtained from GCM palaeo-climate simulations such as produced within the Paleoclimate Modelling Intercomparison Project \citep[PMIP;][]{joussaume-taylor-1995} provide perhaps a more reasonable representation of past climate. However, as such climatologies are only available for specific periods of time, using them as climate forcing for an ice sheet model requires either an assumption of steady-state ice sheet response to climatic fluctuations \citep{huybrechts-tsiobbel-1996}, or interpolation through time between climatologies from different periods, which can be linear \citep{charbit-etal-2002}, or modulated by a ``glacial index'' weighting function derived from ice core $\delta^{18}$O records \citep{marshall-clarke-1999,tarasov-peltier-2004,zweck-huybrechts-2005,gregoire-etal-2012}. An important drawback in this approach is that GCM palaeo-climate simulations themselves rely on an accurate description of the global surface topography and therefore include global ice sheet reconstructions such as ICE-4G \citep{peltier-1994} for their surface topographic boundary condition. This condition could potentially exert influence on subsequently modelled ice sheet geometries, and particularly glacier extent.

The other side of the coin is that avoiding this circular dependence unfortunately goes hand-in-hand with a need for simplifying assumptions regarding Earth's past climate. Such studies include energy balance modelling approaches \citep{tarasov-peltier-1997} and geographic parametrizations of surface mass balance \citep{robert-1991} or climate forcing \citep{johnson-fastook-2002}. Standing on a middle ground, temperature offset methods \citep{greve-etal-1999,bintanja-etal-2005} make use of the high level of detail available in present climate datasets such as gridded observation datasets, GCM output or reanalyses, while using simplifying representations of past climate deviations from the present state.

Here, we propose to address some of the uncertainties concerning climate forcing of numerical glacier models by evaluating the responses of a numerical model, in terms of glacier extent, to inputs from several climate datasets. This is not entirely a new approach as a few studies of this kind are presently available. \citet{quiquet-etal-2012}, for example, assessed the sensitivity of a Greenland ice sheet model to various atmospheric forcing, including a regional parametrization by \citet{fausto-etal-2009}, output from several GCMs and an atmospheric reanalysis. \citet{rodgers-etal-2004} and \citet{charbit-etal-2007} tested the sensitivity of a model of the Northern Hemisphere ice sheet model to climate forcing from different PMIP LGM simulations. Although using different set-ups, all three studies demonstrate the very large sensitivity of ice sheet model to the choice of climate forcing data used. In our study, to limit the degrees of freedom in our model and obtain results independent of palaeo-ice sheet reconstructions such as ICE-4G, we use a simple temperature offset similar to the approaches by \citet{greve-etal-1999} and \citet{bintanja-etal-2005} and assess ice sheet model sensitivity to the choice of present-day climate data. Hence, rather than using GCM output, we force our model with climate reanalysis data, which builds on observational information through data assimilation \citep{bengtsson-etal-2007}. Furthermore, we focus our study regionally on the former Cordilleran Ice Sheet in western Northern America.

The Cordilleran Ice Sheet (Fig.~\ref{fig:locmap}) covered an area that presently experiences strong regional variations in climate. From a numerical modelling perspective, it is one of the least studied palaeo-ice sheets of the Northern Hemisphere, despite the fact that significant stratigraphical and geomorphological data is available to constrain its extent \citep{jackson-clague-1991,dukrodkin-1999,kaufman-manley-2004,kleman-etal-2010,margold-etal-2011}. The Cordilleran Ice Sheet has previously been modelled as part of efforts to simulate ice sheets in North America \citep{marshall-clarke-1999,calov-etal-2002,tarasov-peltier-1997,tarasov-peltier-2004,gregoire-etal-2012}, the Northern Hemisphere \citep{huybrechts-tsiobbel-1996,greve-etal-1999,charbit-etal-2002,charbit-etal-2007,charbit-etal-2013,johnson-fastook-2002,rodgers-etal-2004,bintanja-etal-2005,zweck-huybrechts-2005,abeouchi-etal-2007} and world-wide \citep{yoshimori-etal-2001}. While these studies generally reproduce the magnitude of North American glaciation at LGM reasonably well, there exists a tendency in the simulations that are independent of ice sheet reconstructions such as the ICE-4G to predict excessive ice cover in parts of northern Yukon Territory and interior Alaska that have remained ice-free throughout the Pleistocene \citep{dukrodkin-1999,kaufman-manley-2004}.

Here we use PISM, a Parallel Ice Sheet Model \citep{web:pism}, to simulate the extent and thickness of the Cordilleran Ice Sheet at the LGM (Fig.~\ref{fig:locmap}). We force our model with multiple climate datasets and compare the modelled ice extent to a geomorphological reconstruction of the LGM ice sheet margin by \citet{dyke-2004}. Thereby, we aim to determine the climate dataset which is most suited for simulation of the Cordilleran Ice Sheet and to use as input for future, transient studies over a glacial cycle. Model set-up is presented in section~\ref{sec:model} and climate forcing in section~\ref{sec:climate}. Results are exposed in section~\ref{sec:results} and discussed in section~\ref{sec:discussion}. To our knowledge, this is the first modelling study that specifically focuses on the Cordilleran Ice Sheet since \citet{robert-1991}.
