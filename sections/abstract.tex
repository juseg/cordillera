% section/abstract.tex
% ----------------------------------------------------------------------

\begin{abstract}
We present an ensemble of numerical simulations of the Cordilleran Ice Sheet during the Last Glacial Maximum performed with the Parallel Ice Sheet Model (PISM), applying temperature offsets to the present-day climatologies from five different datasets. Monthly mean surface air temperature and precipitation from WorldClim, the NCEP/NCAR reanalysis, the ERA-Interim reanalysis, the Climate Forecast System Reanalysis and the North American Regional Reanalysis are used to compute surface mass balance in a positive degree-day model. Modelled ice sheet outlines and volumes appear highly sensitive to the choice of climate forcing. For three of the four reanalysis datasets used, differences in precipitation are the major source for discrepancies between model results. We assess model performance against a geomorphological reconstruction of the ice margin at the Last Glacial Maximum, and suggest that part of the mismatch is due to unresolved orographic precipitation effects caused by the coarse resolution of reanalysis data. The best match between model output and the reconstructed ice margin is obtained using the high-resolution North American Regional Reanalysis, which we retain for simulations of the Cordilleran Ice Sheet in the future.
\end{abstract}
