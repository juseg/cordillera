% section/concl.tex
% ----------------------------------------------------------------------
\conclusions
\label{sec:concl}
% ----------------------------------------------------------------------

Our study shows a strong dependency of the modelled ice sheet on the choice of climate forcing data. When using input climatologies from climate reanalyses, spatial resolution of the GCM appears critical for providing the ice sheet model with an accurate precipitation field. This confirms results obtained by \citet{quiquet-etal-2012} from various GCM forcing over the Greenland Ice Sheet.

For the Cordilleran Ice Sheet, we achieve the best fit to the mapped LGM margin by \citet{dyke-2004} using climate forcing from the high-resolution interpolated observational data WorldClim \citep{data:worldclim} and the North American Regional Reanalysis (NARR) \citep{data:narr}. The latter dataset is preferable in our case due to the lack of WorldClim data offshore. NCEP/NCAR forcing, on the other hand, produces an anomalously large ice sheet even under present-day conditions. 

One must keep in mind, however, that these results are biased by our choices of ice-sheet model (PISM), surface mass balance (PDD) model, study area (the Cordilleran Ice Sheet), and more importantly palaeo-climate representation (temperature offsets). At LGM, it is most likely that the presence of ice sheets significantly affected circulation patterns and distribution of precipitation. At the cost of greater computational expense, a more accurate representation of palaeo-climate may allow for a better fit between model results and the mapped LGM margin.
\irina{I have this feeling that the manuscript contains a number of repetitive ideas and arguments. I will try to track them throughout the text.}
