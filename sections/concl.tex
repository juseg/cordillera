% section/concl.tex
% ----------------------------------------------------------------------
\conclusions
\label{sec:concl}
% ----------------------------------------------------------------------

Our study shows a strong dependency of ice sheet model results on the choice of climate forcing data. For three of the four reanalysis datasets used, precipitation rate, over surface air temperature, causes much of the discrepancy between modelled ice sheet outlines and volumes. Furthermore, the spatial resolution of input data appears critical for providing the ice sheet model with an accurate precipitation field, confirming results obtained by \citet{quiquet-etal-2012} from various GCM forcing over the Greenland Ice Sheet.

For the Cordilleran Ice Sheet, we achieve the best fit to the mapped LGM margin by \citet{dyke-2004} using climate forcing from the high-resolution interpolated observational data WorldClim \citep{data:worldclim} and the North American Regional Reanalysis \citep[NARR;][]{data:narr}. The latter dataset is preferable in our case due to the lack of WorldClim data offshore. Climate forcing from CFSR and the NCEP/NCAR reanalysis produce largely oversized ice sheets due to too high precipitation rates, and in the second case, too low surface air temperature. The ERA-Interim data used in this study produces more reasonable results, but it may be too coarse to accurately resolve the spatial distribution of orographic precipitation associated with the rugged topography of the northern American Cordillera, resulting in a misplaced ice cover in regard to geomorphological data. Therefore, we retain NARR data for forcing simulations of the Cordilleran Ice Sheet in the future.

One must keep in mind, however, that these results are biased by our choices of ice-sheet model (PISM), surface mass balance (PDD) model, study area (the Cordilleran Ice Sheet), and more importantly palaeo-climate representation (temperature offsets). At LGM, it is most likely that the presence of ice sheets significantly affected circulation patterns and distribution of precipitation. At the cost of greater computational expense, a more accurate representation of palaeo-climate may allow for a better fit between model results and the mapped LGM margin.
