\documentclass{minimal}
\usepackage{times}
\usepackage[T1]{fontenc}
\usepackage[utf8]{inputenc}

\setlength{\textwidth}{115mm}

\begin{document}

\textbf{Backcover text}

During the last glacial cycle, continental ice sheets similar to those
currently occupying Greenland and Antarctica covered much of northern Eurasia
and North America. This study focuses on one of these former giants, the
Cordilleran ice sheet in north-western North America. Like other palaeo-ice
sheets of the Northern Hemisphere, the Cordilleran ice sheet left a
characteristic imprint on the landscape that formed its bed. For specific time
periods, this landscape record allows partial reconstructions of glacial
conditions, such as the former ice sheet extents or the directions of flow.
This thesis, however, tackles the problem of palaeo-ice sheet reconstruction
from a different angle; that of numerical ice sheet modelling based on
approximated physics of ice deformation, sliding, surface accumulation and
melt. Output from the numerical model is eventually compared with the landscape
record, with an emphasis on input climate conditions in terms of temperature,
precipitation, and temperature variability needed to simulate the former ice
sheet.

\bigskip

\textbf{Image text}

I was born and raised in the beautiful city of Lille in the flatlands of
ch'Northern France, 570\,km from the nearest glacier. I learned physics at
school and discovered programming on pocket calculators, while I gained a
fascination for icy landscapes during numerous summer hikes in the Alps. I then
studied Earth sciences at \emph{École normale supérieure} in Paris and
\emph{Université Joseph Fourier} in Grenoble, and was introduced to glacial geology at
\emph{Norges geologiske undersøkelse} in Trondheim, before moving to Sweden for
my PhD.

\end{document}
