\documentclass[12pt]{article}
\usepackage[utf8]{inputenc}
\usepackage[margin=25mm]{geometry}
\renewcommand{\familydefault}{\sfdefault}
\setlength\parindent{0pt}
\parskip=2ex
\thispagestyle{empty}
\begin{document}

% Title
\textbf{Numerical simulations of the Cordilleran ice sheet
        through the last glacial cycle}

% Authors
J.~Seguinot\textsuperscript{1,2,3},
I.~Rogozhina\textsuperscript{3},
A.~P.~Stroeven\textsuperscript{2},
M.~Margold\textsuperscript{2}
and J.~Kleman\textsuperscript{2}

% Affiliations
\textsuperscript{1}{Laboratory of Hydraulics, Hydrology and Glaciology, ETH Zürich,
          Zürich, Switzerland}\\
\textsuperscript{2}{Department of Physical Geography and the Bolin Centre for Climate
          Research, Stockholm University, Stockholm, Sweden}\\
\textsuperscript{3}{Helmholtz Centre Potsdam, GFZ German Research Centre for Geosciences,
          Potsdam, Germany}\\

\begin{abstract}

  Despite more than a century of geological observations, the Cordilleran ice
  sheet of North America remains poorly understood in terms of its former
  extent, volume and dynamics. Although geomorphological evidence is abundant,
  its complexity is such that whole ice-sheet reconstructions of advance and
  retreat patterns are lacking. Here we use a numerical ice sheet model
  calibrated against field-based evidence to attempt a quantitative
  reconstruction of the Cordilleran ice sheet history through the last glacial
  cycle. A series of simulations is driven by time-dependent temperature
  offsets from six proxy records located around the globe. Although this
  approach reveals large variations in model response to evolving climate
  forcing, all simulations produce two major glaciations during
  marine oxygen isotope stages~4 (61.9--56.5\,ka) and~2
  (23.2--16.8\,ka). The timing of glaciation is
  better reproduced using temperature reconstructions from Greenland and
  Antarctic ice cores than from regional oceanic sediment cores. During most of
  the last glacial cycle, the modelled ice cover is discontinuous and
  restricted to high mountain areas. However, widespread precipitation over the
  Skeena Mountains favours the persistence of a central ice dome throughout the
  glacial cycle. It acts as a nucleation centre before the Last Glacial Maximum
  and hosts the last remains of Cordilleran ice until the middle Holocene
  (6.6--6.2\,ka).

\end{abstract}

\end{document}
