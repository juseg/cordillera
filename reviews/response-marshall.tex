\documentclass[10pt]{article}

\usepackage[T1]{fontenc}
\usepackage[pdftex]{graphicx}
\usepackage[pdftex]{color}

\sloppy
\setlength{\parskip}{1.2ex}
\setlength{\parindent}{0em}
\renewcommand\familydefault{\sfdefault}

\graphicspath{{../figures/}}
\definecolor{journalname}{rgb}{0.341,0.596,0.824}
\definecolor{darkred}{rgb}{0.439,0.157,0.145}  % from Advances in Geoscience

\begin{document}
\textbf{Authors' response to S.~J.~Marshall}
\bigskip

% ----------------------------------------------------------------------
% Interactive comment text begins
% ----------------------------------------------------------------------

\newcommand{\sechead}[1]{\bigskip\noindent\textbf{#1}}
\newcommand{\referee}[1]{\bigskip\textcolor{journalname}{\textit{#1}}}
\newcommand{\msquote}[1]{\begin{quote}\textit{#1}\end{quote}}
\newcommand{\todo}[1]{\textcolor{darkred}{TODO: #1}}

To S.~J.~Marshall,

% ----------------------------------------------------------------------

\sechead{1 \quad Summary comments}

\referee{%
    Seguinot and colleagues provide the first detailed glaciological modelling
    of that I am aware of for the Cordilleran Ice Sheet in western North
    America, making this a novel and long overdue contribution. The authors
    have not only made new advances with this contribution, they have done so
    in an impressive leap forward. This is an excellent and carefully-presented
    study which is likely to rejuvenate interest and debate in Cordilleran Ice
    Sheet reconstructions. The balance between numerical modelling and glacial
    geological/geomorphological considerations is unusually strong, and the
    authors can be commended for this emphasis. This adds tremendous value to
    the results and increases confidence in the modelling, and I also
    appreciate that the authors point out areas where the numerical model is
    not in accord with the geological record.}

\referee{%
    The manuscript is well-written and beautifully illustrated, and I have very
    few substantive comments. The choices made by the authors are logical and
    well-explained, and they reach several well-substantiated conclusions: a
    two-phase Cordilleran glaciation, a reasonably robust estimate of CIS
    volume at LGM, the general model of CIS growth through multiple alpine
    icefields, and the importance of the Skeena Mountain inception centre. One
    can always quibble with specific aspects of the model design and climate
    scenarios, but the authors have explored a reasonable span of `solution
    space' and these aspects of the Cordilleran ice sheet history appear to be
    robust features of the simulations.}

\referee{%
    The modelling strategy and results presented here stand to be widely cited,
    and I expect that it will serve as a springboard for additional studies
    from others in the international community. I recommend this manuscript for
    publication in The Cryosphere without reservations.}

% ----------------------------------------------------------------------

\sechead{1 \quad Specific comments}

\referee{%
    The Cordilleran Ice Sheet is difficult to model due to its complex
    topography and multiple inception centres (and possibly multiple
    domes/divides), strong regional climatic gradients, which require
    relatively high-resolution climate input fields, and a dearth of
    paleoclimate proxies for western North America to inform spatial and
    temporal variations in climate conditions during the glacial period. The
    authors confront these challenges well, with a adequate ice sheet model
    resolution (5 to 10\,km) and ice physics, carefully calibrated `control'
    climatology (published in Seguinot et al., 2014), and a good exploration of
    different paleoclimate time series histories in this contribution.}

\referee{%
    Nevertheless, it is not clear that ice-core based paleoclimate proxies from
    Greenland or Antarctica are appropriate for western North America. This may
    be particularly true of Greenland proxies, where the amplitude of D-O
    (millennial) climate variations is exceptionally strong and is likely to be
    regional. Because these remote ice core records are `scaled' based on only
    one constraint, producing a CIS maximum configuration that resembles the
    geological record, it is difficult to assess the pre-LGM simulations or the
    details of the modelled ice divide structure, ice thickness, etc. The
    robustness of the conclusion that Greenland and Antarctic ice core records
    are good proxies of glacial climate variability in western North America is
    therefore not so clear, but it is admittedly hard to do better at this
    time. I do wonder if there is any hope from more regional climate proxies
    such as the Logan ice cores or the off-shore Vancouver Island sediment
    records that are cited from Cosma et al. This is worth a short discussion.}

\todo{%
    Logan ice core covers only the last 16\,Ka and is not a good proxy for
    temperature. The Vancouver SST record covers only deglaciation. Check new
    papers for updates. Discuss in the manuscript.}

\referee{%
    Similarly it is difficult to know the errors and uncertainties associated
    with the assumption of fixed modern-day spatial patterns for temperature
    and precipitation. I suspect that the sensitivity of this assumption far
    exceeds that associated with the different paleoclimate proxies. Such that,
    for example, one could readily imagine different assumptions, such as a
    maritime effect that gives reduced glacial cooling near the coast vs. in
    the interior, that is a stronger effect than the difference between
    different paleo-climate proxies with respect to the timing of LGM, ice
    divide structure, etc. But this is a very reasonable start, what the
    authors have done -- there is always going to be more parameter space to
    explore in future studies. As above, I would perhaps just suggest a small
    discussion of the authors' opinion on this question, the uncertainty or
    possible influence of this assumption of modern-day climate patterns.}

\todo{Discuss in the manuscript.}

\referee{%
    Several minor points and grammatical corrections are included in the
    attached text. Nothing that will require much thought -- this is a really
    impressive piece of research, overall, and I am hard-pressed to find any
    criticism of it. It is one of the easiest reviews I have ever done.
    Congratulations to the authors and thanks for this fine work.}

\todo{Correct mistakes.}

% ----------------------------------------------------------------------
% Interactive comment text ends
% ----------------------------------------------------------------------

\end{document}
