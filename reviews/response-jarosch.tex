% response-jarosch.tex
% ----------------------------------------------------------------------
% response-header.tex
% ----------------------------------------------------------------------

% Base class and packages
\documentclass[11pt]{article}

% Included in online comment header
\usepackage[pdftex]{graphicx}
\usepackage[pdftex]{color}
\usepackage{amssymb}
%\usepackage{times}

% Additional packages
\usepackage[T1]{fontenc}
\usepackage{geometry}
\usepackage[hidelinks]{hyperref}
\usepackage{natbib}

% Graphic path of main manuscript
\graphicspath{{../figures/}}

% Replacements for Copernicus commands
\newcommand{\unit}[1]{\ensuremath{\mathrm{#1}}}
\newcommand{\chem}[1]{\ensuremath{\mathrm{#1}}}
\newcommand{\urlprefix}[0]{}

% Default font and spacing
\renewcommand\familydefault{\sfdefault}
\setlength{\parskip}{1.2ex}
\setlength{\parindent}{0em}
\linespread{1.5}

% color defined in comment template
\definecolor{journalname}{rgb}{0.34,0.59,0.82}

% todo command which should not be used in final version
\definecolor{todored}{rgb}{0.439,0.157,0.145}  % from Advances in Geoscience
\newcommand{\todo}[1]{\textcolor{todored}{TODO: #1}}


\begin{document}
\textbf{Authors' response to A.~H.~Jarosch}
\bigskip

% ----------------------------------------------------------------------
% Interactive comment text begins
% ----------------------------------------------------------------------

\newcommand{\sechead}[1]{\bigskip\noindent\textbf{#1}}
\newcommand{\referee}[1]{\bigskip\textcolor{journalname}{\textit{#1}}}
\newcommand{\msquote}[1]{\begin{quote}\textit{#1}\end{quote}}

To A.~H.~Jarosch,

We apologize for our delayed response and thank you for this detailed review of
our manuscript. Following your comments, we decide to amend the manuscript with
a new section that explores the sensitivity of one of our runs to some of the
parameters governing ice deformation and sliding. Although the discussion of
model results against geological evidence is indeed extensive and partly
speculative, we prefer not to shorten it for reasons we will detail below.
Although you may feel we have not applied all the changes you requested, we
hope that those we do plan fill much of the voids you pointed at.

% ----------------------------------------------------------------------

\sechead{1 \quad General comments}

\referee{%
    Seguinot et al. present in this well written and structured manuscript a
    numerical modelling study of the Cordilleran ice sheet through the last
    glacial cycle. The model is driven by several temperature reconstructions
    based on proxy data and model output is subsequently compared in detail to
    the existing geological evidence in the region. The study is of significant
    relevance as it focuses on the Cordilleran ice sheet evolution in the past,
    which is still poorly understood.}

Thank you for this positive summary of our work.

\referee{%
    Nevertheless, the manuscript is quite unbalanced in its presentation as it
    focuses strongly on section~4 (Comparison with geological record) and by
    doing so neglects crucial details in section~2 (Model setup). This poses a
    fundamental challenge for understanding the science presented. If it is not
    quite clear what the model does and how it performs to start with, it
    becomes difficult to discuss the results of the modelling study and why
    there are mismatches with geological evidence.}

In our view, the fundamental challenge you refer to is not one specifically
posed by the presentation of this manuscript, but instead a more long-standing
one, related to the mutual understanding of two scientific communities that
have long remain poorly connected: that of numerical ice sheet modellers on one
side, and that of glacial geologists on the other.

Because our manuscript is aimed at both communities, it is important for us, as
a team of co-authors with different backgrounds, that a balance is kept between
the description of the physics embedded in the numerical model (documented
elsewhere) and the level of regional detail in the discussion of geological
evidence (documented elsewhere too, though perhaps in a more fragmented way).

Yet, in doing so, we may have omitted crucial details in the model set-up. We
hope that our changes described below will correct for that.

\referee{%
    An overall sensitivity study of the parameters used in the model is
    completely lacking, thereby making it almost impossible to understand
    different responses of the ice sheet model. After reading the manuscript,
    one is left with the impression that the authors assume the PISM ice sheet
    model to be a black box which just requires one initial ``correct'' setup
    with literature values. This notion is reflected in the current manuscript,
    where almost all mismatches of model output with geological evidence (as
    discussed in section~4) are attributed to climate variations lacking in the
    proxy data, or climate-ice sheet feedback mechanisms not represented in the
    model chain. Similarly in a previous study \citet{Seguinot.etal.2014} have
    focused only on the driving climate sensitivities and have omitted
    influences of the ice sheet model as well as mass balance model
    parameters even though they note in that study that these sensitivities
    require attention as well.}

As announced earlier, we have decided to add a new section to the manuscript
containing a short sensitivity study to some of the most influential rheologic
and basal sliding parameters. Thus the new outline will be as follow:
%
\begin{enumerate}
    \item{Introduction}
    \item{Model setup}
    \item{Sensitivity to climate forcing time-series}
    \item{Sensitivity to ice flow parameters}
    \item{Comparison with the geologic record}
\end{enumerate}
%
In accordance, Sect.~2 (Model setup) will be amended with an description of
default and alternative parameters for ice rheology and basal sliding with
illustrations of their role in the model. Sect.~2.2 (Ice thermodynamics) will
be divided into:
%
\begin{enumerate}
    \item[2.2]{Ice rheology}
    \item[2.3]{Basal sliding}
    \item[2.4]{Ice shelf calving}
\end{enumerate}

\referee{%
    What I advocate at this point is not a complete, strict sensitivity study
    of all parameters involved in the model setup (that would be probably a
    work package large enough to fill a science career). However several key
    parameters can be investigated with not too much effort. Contrasting the
    influence of e.g. basal sliding and ice rheology parameters with the
    influence of driving climate on the model results would help to estimate
    the overall sensitivity of the model system as well as help guiding future
    efforts performing such modelling studies. Implicitly the authors assume
    that all other model sensitivities are negligibly small in comparison to
    the driving climate. However it is obvious from an ice sheet model
    perspective that at least chosen basal sliding parameters as well as ice
    rheology parameters will strongly influence the shape and volume of the
    modelled ice sheet. Thus it would be nice to see evidence supporting the
    claim that driving climate is the only input to worry about being presented
    in the current manuscript. Or should it turn out that basal sliding and ice
    rheology play an important role too, as one would expect, then the relative
    importance of each including error estimates on the chosen parameters
    should be presented as well.}

To keep this sensitivity study as simple as possible, we choose to present only
four additional runs, two with varying rheologic parameters and two with
varying sliding parameters, using the simulation driven by the GRIP record as
a control, at the horizontal resolution of 10\,km.

In our simulations, ice deformation is governed by the constitutive law for ice
\citep{Glen.1952, Nye.1953},
%
\begin{equation}
    \label{eqn:glenslaw}
    \vec{\dot{\epsilon}} = A\,\tau_e^{n-1}\,\vec{\tau} \,.
\end{equation}
%
where $\vec{\dot{\epsilon}}$ is the the strain-rate tensor, $\vec{\tau}$ the
deviatoric stress tensor, and $\tau_e$ the equivalent stress defined by
${\tau_e}^2 = \frac{1}{2} \mathrm{tr}(\vec{\tau}^2)$.
The ice softness coefficient, $A$, depends on ice temperature, $T$, pressure, $p$, and
water content, $\omega$, through a piece-wise Arrhenius-type law
\citep[Eqs.~63--65]{Aschwanden.etal.2012},
%
\begin{equation}
    \label{eqn:softness}
    A = E\cdot
    \begin{cases}
        A_c \,e^\frac{-Q_c}{RT_{pa}}
            & \text{if}\ T_{pa} < T_c \,, \\
        A_w (1+f\omega)\,e^\frac{-Q_w}{RT_{pa}}
            & \text{if}\ T_{pa} \ge T_c \,,
    \end{cases}
\end{equation}
%
where $T_{pa}$ is the pressure-adjusted ice temperature calculated using the
Clapeyron relation, ${T_{pa} = T - \beta p}$.
$R=8.31441$\,J\,mol$^{-1}$\,K$^{-1}$ is the ideal gas constant, and $A_c$,
$A_w$, $Q_c$ and $Q_w$, are constant parameters corresponding to values
measured below and above a critical temperature threshold ${T_c=-10}^\circ$C
\citep[p.~72]{Paterson.Budd.1982,Cuffey.Paterson.2010}. The water fraction,
$\omega$, is capped at a maximum value of 0.01, above which no measurements
are available \citep[Eq.~5.7]{Lliboutry.Duval.1985, Greve.1997}. Finally,
$E$ is a non-dimensional enhancement factor which can take different values,
$E_{SIA}$, in the Shallow Ice Approximation (SIA) and $E_{SSA}$, in the Shallow
Shelf Approximation (SSA).

In our short sensitivity study, we set constant the power-law exponent, $n=3$,
according to \citet[p.~55--57]{Cuffey.Paterson.2010}, the Clapeyron constant,
$\beta=7.9\times 10^{-8}$\,K\,Pa$^{-1}$, according to \citep{Luthi.etal.2002},
the water fraction coefficient, $f=181.25$, according to
\citet{Lliboutry.Duval.1985}, and the SSA enhancement factor, $E_{SSA}=1$,
according to \citep[p.~77]{Cuffey.Paterson.2010}.

On the other hand, we test different values for the two creep parameters, $A_c$
and $A_w$, the two activation energies, $Q_c$ and $Q_w$, and the SIA
enhancement factor, $E_{SIA}$, as follow.

\begin{itemize}
    \item{Our \emph{default} configuration used in the control run and all other
          simulations in the manuscript include rheologic parameters, $A_c$,
          $A_w$, $Q_c$ and $Q_w$, derived from \citet{Paterson.Budd.1982} and
          given in \citet[Eqn.~5]{Bueler.Brown.2009}, and $E_{SIA}=1$.}
    \item{Our \emph{hard ice} configuration include rheologic parameters,
          $A_c$, $A_w$, $Q_c$ and $Q_w$, derived from
          \citet[p.~72 and 76]{Cuffey.Paterson.2010}, and $E_{SIA}=1$, which
          correspond to a stiffer rheology than that used in the control run.}
    \item{Our \emph{soft ice} configuration include rheologic parameters,
          $A_c$, $A_w$, $Q_c$ and $Q_w$, derived from
          \citet[p.~72 and 76]{Cuffey.Paterson.2010}, and $E_{SIA}=5$, the
          recommended value for ice age polar ice
          \citep[p.~77]{Cuffey.Paterson.2010}.}
\end{itemize}

An additional simulation with rheologic parameters, $A_c$, $A_w$, $Q_c$ and
$Q_w$, derived from \citet{Cuffey.Paterson.2010}, and $E_{SIA}=2$, the
recommended value for Holocene polar ice \citep[p.~77]{Cuffey.Paterson.2010}
was performed, but its results were very similar to that of our default run,
thus we decided to not present it here.

Actual parameter values for $A_c$, $A_w$, $Q_c$, $Q_w$ and $E_{SIA}$ used in
our simulations are given in Table~\ref{tab:sens_params}, while the effect of
the three different parametrisations on temperature-dependent ice softness,
$A$, is illustrated in Fig.~\ref{fig:sens_plot_rheo}.

\begin{table*}
  \caption{Parameter values used in the sensitivity test.}
  \label{tab:sens_params}
  \centering\makebox[\textwidth]
  {\begin{tabular}{l|ccccc|cc|cc}
    \tophline
            & \multicolumn{5}{c|}{Rheology}
            & \multicolumn{2}{c|}{Sliding}
            & \multicolumn{2}{c}{GRIP scaling} \\
    Config. & $A_c$ & $A_w$ & $Q_c$ & $Q_w$ & $E_{SIA}$
            & $\delta$ & $W_{max}$ & $f$ & $T_{[32, 22]}$ \\
            & \multicolumn{2}{c}{(\unit{Pa^{-3}\,s^{-1}})}
            & \multicolumn{2}{c}{(\unit{J\,mol^{-1}})}
            & & & (m) \\
    \middlehline
    Default$^1$  & $ 3.61\times 10^{-13}$
                 & $ 1.73\times 10^3$
                 & $   60\times 10^3$
                 & $  139\times 10^3$
                 & 1 & 0.02 & 2 & 0.38 & 6.2 \\
    \middlehline
    Soft ice$^2$ & $2.847\times 10^{-13}$
                 & $2.356\times 10^{-2}$
                 & $   60\times 10^3$
                 & $  115\times 10^3$
                 & 5 & 0.02 & 2 & 0.40 & 6.6 \\
    Hard ice$^2$ & $2.847\times 10^{-13}$
                 & $2.356\times 10^{-2}$
                 & $   60\times 10^3$
                 & $  115\times 10^3$
                 & 1 & 0.02 & 2 & 0.37 & 6.0 \\
    \middlehline
    Soft bed     & $ 3.61\times 10^{-13}$
                 & $ 1.73\times 10^3$
                 & $   60\times 10^3$
                 & $  139\times 10^3$
                 & 1 & 0.01 & 1 & 0.40 & 6.5 \\
    Hard bed     & $ 3.61\times 10^{-13}$
                 & $ 1.73\times 10^3$
                 & $   60\times 10^3$
                 & $  139\times 10^3$
                 & 1 & 0.05 & 5 & 0.36 & 5.9 \\
    \bottomhline
  \end{tabular}}
  \belowtable{After $^1$\citet{Paterson.Budd.1982,Bueler.Pelt.2015};
              and $^2$\citet{Cuffey.Paterson.2010}.}
\end{table*}

\begin{figure}
    \centering
    \includegraphics{sens_plot_rheo}
    \caption{Ice softness parameter, $A$, as a function of pressure-adjusted
             temperature, $T_{pa}$, for the default \citet{Paterson.Budd.1982},
             hard ice \citet{Cuffey.Paterson.2010}, and soft ice
             \citet[with $E_{SIA}=5$]{Cuffey.Paterson.2010} rheologies, using
             a linear scale (top panel) and logarithmic scale (bottom panel).
             Figure made using Eqn.~\ref{eqn:softness} with parameters from
             Table~\ref{tab:sens_params}.}
    \label{fig:sens_plot_rheo}
\end{figure}

\clearpage  % moving on to basal sliding

In our simulations, basal sliding is governed by a~pseudo-plastic sliding law,
already given in the manuscript but recalled here for the sake of completeness,
%
\begin{equation}
    \label{eqn:pseudoplastic}
    \vec{\tau}_b = -\tau_c \frac{\vec{v}_b}{{v_{th}}^q\,|\vec{v}_b|^{1-q}} \,,
\end{equation}
%
which relates the bed-parallel shear stresses, $\vec{\tau}_b$, to the sliding
velocity, $\vec{v}_b$. The yield stress, $\tau_c$, is modelled using the
Mohr--Coulomb criterion,
%
\begin{equation}
   \tau_c = c_0 + N\,\tan{\phi} \,,
\end{equation}
%
where cohesion, $c_0$, is assumed to be zero. Effective pressure, $N$, is
related to the overburden pressure, $P_0=\rho gh$, and the modelled amount of
subglacial water, using a formula derived from laboratory experiments with till
extracted from the base of Ice Stream B in West Antarctica
\citep[Eqn.~23]{Tulaczyk.etal.2000, Bueler.Pelt.2015},
\begin{equation}
    \label{eqn:ntil}
    N = \delta P_0 \, 10^{(e_0/C_c) (1 - (W/W_{max}))} \,,
\end{equation}
where $\delta$ sets the minimum ratio between the effective and
overburden pressures, $e_0$ is a measured reference void ratio
and $C_c$ is a measured compressibility coefficient. The amount of water at the
base, $W$, varies from zero to $W_{max}$, a threshold above which additional
melt water is assumed to drain off instantaneously.

In our sensitivity test, we set constant the pseudo-plastic sliding exponent,
$q=0.25$, and the threshold velocity, $v_{th}=100$\,\unit{m\,s^{-1}}, according
to values used by \citet{Aschwanden.etal.2013}, the till cohesion, $c_0=0$,
whose measured values are consistently negligible
\citep[p.~268]{Tulaczyk.etal.2000, Cuffey.Paterson.2010}, the till reference
void ratio, $e_0=0.69$, and the till compressibility coefficient, $C_c=0.12$,
according to the only measurements available to our knowledge by
\citep{Tulaczyk.etal.2000}.

We also use a constant spatial distribution of the till friction angle, $\phi$,
whose values vary from 15 to 45{\degree} as a piecewise-linear function of
modern bed elevation, with the lowest value occurring below modern sea level
(0\,m above sea level, m~a.s.l.) and the highest value occurring above the
generalised elevation of the highest shorelines
\citep[200\,m~a.s.l.,][Fig.~5]{Clague.1981}. This range of value span over the
range of measured values for glacial till of 18 to 40{\degree}
\citep[p.~268]{Cuffey.Paterson.2010}. It accounts for frictional basal
conditions associated with discontinuous till cover at high elevations, and
a~weakening of till associated with the presence of marine sediments at low
elevations (cf. \citealp{Martin.etal.2011};
\citealp[supplement]{Aschwanden.etal.2013}; \citealp{PISM-authors.2015}).

An additional simulation with a constant till friction angle, $\phi=30\degree$,
corresponding to the average value in \citet[p.~268]{Cuffey.Paterson.2010}, was
actually performed, but the induced variability was small as compared to that
which will be presented here, and therefore we decided to not include this run.

On the other hand, we test different values for the minimum ratio between the
effective and overburden pressures, $\delta$, and the maximum water
height in the till, $W_{max}$, as follow.

\begin{itemize}
    \item{Our \emph{default} configuration used in the control run and all other
          simulations in the manuscript include $\delta=0.02$ and
          $W_{max}=2$\,m as in \citet{Bueler.Pelt.2015}.}
    \item{Our \emph{soft bed} configuration use $\delta=0.01$ and
          $W_{max}=1$\,m.}
    \item{Our \emph{hard bed} configuration use $\delta=0.05$ and
          $W_{max}=5$\,m.}
\end{itemize}

The effect of the three different parametrisations on the effective pressure on
the till, $N$, in response to water content, $W$, is illustrated in
Fig.~\ref{fig:sens_plot_ntil}.
All parameter choices are recalled in Table~\ref{tab:sens_params}.

Finally, we adjusted the GRIP linear scaling factor for each run, so that they
result in a similar glaciated area at the Last Glacial Maximum (LGM) to that
modelled with the default configuration (Table~\ref{tab:sens_params}).
In other words, the sensitivity in modelled ice sheet extent at MIS~2 is
expressed through the scaling factor required to obtain a fixed target area,
since we consider this scaling factor as a free parameter in our study.

\begin{figure}
    \centering
    \includegraphics{sens_plot_ntil}
    \caption{Effective pressure, $N$, as a function of water height at the
             base, $W$, for the default ($\delta=0.02$, $W_{max}=2$\,m),
             hard bed ($\delta=0.05$, $W_{max}=5$\,m), and soft bed
             ($\delta=0.01$, $W_{max}=1$\,m) sliding parametrisations, using a
             linear scale (top panel) and a logarithmic scale (bottom panel).
             Calculations are made for an ice thickness, $h$, of 1000\,m.
             Figure made using Eqn.~\ref{eqn:ntil} with parameters from
             Table~\ref{tab:sens_params} after
             \citet[Fig.~1]{Bueler.Pelt.2015}.}
    \label{fig:sens_plot_ntil}
\end{figure}

\clearpage  % moving on to the results

By analogy to our manuscript's Table~3 (ice volume and extent extrema) and
Fig.~3 (sea-level relevant ice volume time series), the results of our
sensitivity study are presented here using a similar layout in
Table~\ref{tab:sens_extrema} and Fig.~\ref{fig:sens_ts}.

As a result of the different scaling factors applied, the resulting simulations
show very little difference in the modelled glaciated areas corresponding to
maximum ice volumes during MIS~2, but also during MIS~4
(Table~\ref{tab:sens_extrema}). However, this can not be said for the modelled
glaciated area corresponding to minimum ice volume during MIS~3. In fact, the
extent of the remnant ice cap which persists over the Skeena Mountains during
this stage shows a significant sensitivity to ice rheology of 31\%, and
an even more important sensitivity to basal sliding parameters of 62\%
(Table~\ref{tab:sens_extrema}).

The modelled sea-level relevant ice volumes show more variability than the
modelled glaciated areas (Table~\ref{tab:sens_extrema},
Fig.~\ref{fig:sens_ts}). As could be expected, softer ice and weaker till both
result in a thinner ice sheet, while harder ice and stronger till result in a
thicker ice sheet. For instance, peak ice volume during the MIS~2 (LGM) varies
by 30\% between the two parametrisations of ice rheology used, and by 21\%
between the two parametrisations of till properties used. The differences in
sea-level relevant ice volume are greatest during the MIS~3
(Table~\ref{tab:sens_extrema}, Fig.~\ref{fig:sens_ts}) where both the areal and
thickness contributions add up.

\begin{table*}
  \caption{Extremes in Cordilleran ice sheet volume and extent corresponding to
           MIS~4, 3 and 2 using the GRIP paleo-climate forcing with each
           parameter configuration (Fig.~\ref{fig:sens_ts}). Relative
           differences (r. diff.) give rough error estimates related to
           varying selected ice rheology and basal sliding parameters
           (Table.~\ref{tab:sens_params}).}
  \label{tab:sens_extrema}
  \centering\makebox[\textwidth]
  {\begin{tabular}{l*{3}{|ccc}}
    \tophline
             & \multicolumn{3}{c}{Age (ka)}
             & \multicolumn{3}{c}{Ice extent (\unit{10^6\,km^2})}
             & \multicolumn{3}{c}{Ice volume (m~s.l.e.)} \\
    Config.  &  MIS~4 &  MIS~3 &  MIS~2
             &  MIS~4 &  MIS~3 &  MIS~2
             &  MIS~4 &  MIS~3 &  MIS~2 \\
    \middlehline
    Default  &  57.59 &  42.91 &  19.14
             &   1.93 &   0.67 &   2.09
             &   7.43 &   1.54 &   8.62 \\
    \middlehline
    Soft ice &  58.89 &  49.97 &  21.57
             &   1.96 &   0.54 &   2.08
             &   6.58 &   1.03 &   6.88 \\
    Hard ice &  57.32 &  42.90 &  19.14
             &   1.90 &   0.75 &   2.12
             &   7.83 &   1.91 &   9.46 \\
    R. diff. &    3\% &   16\% &   13\%
             &    3\% &   31\% &    2\%
             &   17\% &   57\% &   30\% \\
    \middlehline
    Soft bed &  58.90 &  49.21 &  19.53
             &   1.88 &   0.55 &   2.05
             &   6.46 &   1.03 &   7.52 \\
    Hard bed &  57.31 &  42.91 &  19.14
             &   1.93 &   0.96 &   2.13
             &   7.99 &   2.89 &   9.31 \\
    R. diff. &    3\% &   15\% &    2\%
             &    3\% &   62\% &    4\%
             &   21\% &  120\% &   21\% \\
    \bottomhline
  \end{tabular}}
  \belowtable{}
\end{table*}

\begin{figure*}
  \centering
  \includegraphics{sens_ts}
  \caption{Modelled sea-level relevant ice volume through the last 120\,ka
           in the simulation forced by the GRIP paleo-climate record, using
           default parameters (black curves), different ice rheology parameters,
           (top panel) and different basal sliding parameters (bottom panel).
           Gray fields indicate Marine Oxygen Isotope Stage (MIS) boundaries
           for MIS~2 and MIS~4 according to a global compilation of benthic
           \chem{\delta^{18}O} records \citep{Lisiecki.Raymo.2005}.}
  \label{fig:sens_ts}
\end{figure*}

\clearpage  % eventually moving on to the next point

All the information detailed above, including
Eqns~\ref{eqn:glenslaw} and~\ref{eqn:softness}, default parameter values,
Tables~\ref{tab:sens_params} and~\ref{tab:sens_extrema}, and
Figs.~\ref{fig:sens_plot_rheo}--\ref{fig:sens_ts} will be included in Sect.~2
(Model setup) or in the new section (Sensitivity to ice flow parameters).
Relevant discussion points in Sect.~4(Comparison with the geologic record)
will be updated to account for these new results.

\referee{%
    Generally section~4 appears to be quite long and seems to re-summarize
    known geological evidence for the region. At times the language is quite
    speculative, for example P4162 L1 and 7, P4164 L18, P4171 L7, L9, L17, L19
    and L20 and so forth. I would recommend to shorten that section to focus
    only on the geological evidence which can or can not clearly be reproduced
    by the presented model and avoid extensive speculation on what the reasons
    for mismatch are, especially in the present form of the manuscript, where a
    sensitivity study of the model itself is completely missing. However I
    leave the choice of how much geological evidence is discussed in the
    manuscript entirely up to the authors.}

\todo{%
    I have no intention to remove half the manuscript. Argue against it here.
    Geomorphology is no exact science and has to be speculative.}

% ----------------------------------------------------------------------

\sechead{2 \quad Specific Comments}

\referee{%
    I refer to text locations in the discussion paper by page number (P) and
    line number range (L) for the specific comments.}

\referee{%
    \textbf{P4151 L11-16:} In this sentence the authors refer back to their
    previous work \citep{Seguinot.etal.2014} and highlight that the NARR
    temperature and precipitation fields are the most suitable present day
    climate datasets to be used. Especially since the NARR precipitation fields
    include steep precipitation gradients which are required as identified by
    et \citet{Seguinot.etal.2014}. NARR is delivered on a 32\,km Lambert grid,
    and thus it is questionable how ``steep'' these gradients can be, given
    the rather smooth representation of the existing topography on a 32\,km
    grid. \citet{Seguinot.etal.2014} have partly discussed that however. NARR
    precipitation and temperature fields have been evaluated in detail based on
    available station data for large parts of the study domain dealt with in
    this manuscript. This evaluation \citep{Jarosch.etal.2010} demonstrated
    that NARR has difficulties simulating orographic processes in the Coast
    Mountains which in turn results in unrealistic atmospheric conditions over
    the Rocky Mountains. \citet{Jarosch.etal.2010} further concluded that
    physics based downscaling is required to adequately drive glacier models in
    that region. The authors should argue in more detail here why they think
    that NARR precipitation fields at 32\,km are adequate to drive their model
    and reflect their arguments with the findings of \citet{Jarosch.etal.2010}.
    A solid argument here is of special importance as the authors assume the
    present day precipitation fields to be valid throughout their model time
    period (120\,ky to present) without further corrections (cf. section~2.4
    equation~6).}

\todo{%
    Decide what to do!}

\referee{%
    \textbf{P4152 L11:}
    Basal topography is ``derived'' from ETOPO1 data. What does this mean? Do
    the authors just re-sample the DEM data to their 10\,km and 5\,km model
    grids (P4152 L21-22) or is there more processing done? The ETOPO1 data
    contains the present day ice volumes within the study region.
    \citet{Clarke.etal.2013}
    have estimated the ice volume in parts of that region to be $2530
    \pm 220\,km^2$, with maximum ice thicknesses up to 200\,m. It can be argued
    that the volume is negligible in this study and the authors should do so if
    they think this is appropriate, but I wonder about the ice thicknesses.
    Assuming that the authors did not remove the present day ice cover, basal
    topography could be up to 200\,m higher that it actually is in reality.
    Given their used temperature lapse rate of $6\,K\,km^{-1}$ (P4157 L1),
    parts of the Cordilleran ice sheet growing in those regions with 200\,m too
    high topography would experience a 1.2\,K colder atmosphere than it
    actually should in reality. This favours unrealistic ice growth and thus
    the omission of present day ice cover removal should be clearly argued for
    in the manuscript.}

\todo{%
    Mention that ice cover could be overestimated in this region. The main
    problematic area are the St.-Elias Mountains. However, there is no ice
    thickness data available as far as I know. Check whether there is more info
    about that in \citet{Waechter.etal.2015}}

\referee{%
    \textbf{P4153 L2-3:}
    That the ``shallow shelf approximation'' (SSA) is used as a ``sliding law''
    for the ``shallow ice approximation'' (SIA) is a confusing statement in
    this context. Bueler has coined the term in his 2009 paper as cited in the
    manuscript. However the casual reader will be confused at this point,
    especially since the authors state the pseudoplastic sliding law the model
    actually uses in equation~1. I would recommend to leave out the statement
    on the SSA being the ``sliding law'' for the SIA.}

\todo{%
    Agree. Reformulate.}

\referee{%
    \textbf{P4153 L5-6:}
    As stated here, ice rheology within the used ice sheet model is based on
    \citet{Aschwanden.etal.2012}. This enthalpy based formulation has proofed
    itself to be very suitable for estimating ice rheology in ice sheet models,
    but it also depends on several parameters to translate enthalpy within the
    ice to ice viscosity \citep[see][equations 62-65)]{Aschwanden.etal.2012}.
    The authors do not mention any of these parameters (e.g. any of the rate
    factors or nonlinear power $n$ ) within the manuscript or in Table~1. I
    have mentioned above in the general comments section that parameters used
    in ice rheology and basal sliding formulations are important model
    parameters which will influence the ice sheet model output and that a basic
    sensitivity study on those parameters is required to understand the model
    results. Here the authors could start with listing the parameters used in
    the ice rheology formulation, than continue with estimating uncertainties
    for those from literature and afterwards perform additional model
    simulations to identify the influence of the chosen parameter sets on the
    ice volume and ice margin position history the model creates. In the end
    the authors will be able to identify the relative importance of
    uncertainties in driving climate as well as model parameters, which will
    strengthen their discussion in section~4.}

\todo{%
    Pull values from parameter table in PhD kappa. Wait for results of
    sensitivity tests.}

\referee{%
    \textbf{P4153 L8:}
    It is not clear where the geothermal heat flux boundary is located. Does
    the ``depth of 3\,km'' refer to a depth measured from the ice surface,
    which would not make much sense for a ice thickness evolving ice sheet
    model or is it measured from the ice-bedrock interface downward. In that
    case the term ``computed subglacially'' is confusing as it refers to the
    ice-bedrock interface. Please be more specific here.}

\todo{%
    From the ice-bedrock interface downward. Clarify.}

\referee{%
    \textbf{P4153 L16 - P4154 L0:}
    Here the authors describe the basal sliding setup in their model. However
    they do not explain how they came up with the parameters used in
    equations~1-3 that are listed in Table~1 (part on ``basal sliding''). What
    motivates these parameter choices (references?) and how sensitive is the
    model and its results to these choices? Both question come instantly to
    mind and need to be addressed in detail. Here a basic sensitivity study on
    how basal sliding parameters in the model control the outcome discussed in
    section~4 is in order and I strongly recommend to include one in the
    manuscript. The authors can start by estimating the uncertainties in the
    chosen basal sliding parameters and run two extra simulation runs with
    their preferred climate forcing and the end member values of the
    uncertainties. This would create the most simple sensitivity study with
    respect to basal sliding, but would be extremely helpful for the argument
    made above in my general comments.}

\todo{%
    Pull parameters and explanation from Kappa. Check for newer PISM refs on
    modelling Greenland or Antarctica. Wait for results from sensitivity tests}

\referee{%
    \textbf{P4156 L3-5:}
    In addition to what I have stated above on the NARR precipitation fields
    and their suitability, it is important to state at this location in the
    manuscript how the 32\,km NARR data is translated to the 10 and 5\,km
    computational grids of the current study. I disagree with the notion that a
    32\,km precipitation field can be called ``high- resolution'' in the
    context of 10 and 5\,km grid based ice sheet modelling. The input data is
    either 3 or 6 times coarser than the numerical grid, thus not at all
    high-resolution. \citet{Seguinot.etal.2014} state in their section~3.3 that
    the NARR data fields have been bilinearly interpolated to 10\,km resolution
    in their work. Did the authors do the same here for their 10 and 5\,km
    working grids? This is crucial information to be included in the
    manuscript. It has been demonstrated by spectral power analysis
    \citep{Jarosch.etal.2010} that the NARR precipitation fields do not contain
    any significant spacial information below approximately 39\,km resolution
    and that bilinear interpolation does not add any information whatsoever on
    smaller scales, which should come to no surprise. Physics-based downscaling
    techniques however are able to add spatial information to precipitation
    fields down to about 1\,km grid sizes \citep{Jarosch.etal.2010}. Taking
    these findings into the current context of the manuscript at hand, the NARR
    precipitation fields can hardly be called ``high resolution'' with their
    effective precipitation grid size of 39\,km. The authors should argue for
    their choice of not performing any downscaling whatsoever to their
    computational grids of 10 and 5\,km for precipitation and temperature and
    discuss their choice in the light of the findings from
    \citet{Jarosch.etal.2010}. Temperature however is better constrained in
    NARR \citep{Jarosch.etal.2010} and contains spectral information down to
    10\,km resolution, which justifies the usage of NARR temperature fields on
    the 10\,km computational grid of this study. The 5\,km grid still needs to
    be argued for.}

\todo{%
    NARR data fields have been bilinearly interpolated. Decide what to do!}

\referee{%
    \textbf{P4157 L1:}
    How is a fixed temperature lapse rate justified for simulations over 120k
    years, when there is ample published evidence that temperature lapse rates
    vary significantly within space and time? I am sure that the choice of
    $\gamma$ in this study has a significant influence on the model outcome and
    I leave it to the authors to explore this possibility.}

\todo{%
    I am not sure how to adress this at the moment. Temperature lapse rates
    vary significantly at present, but I am not sure if there is much knowledge
    on how they varied in the past.}

% ----------------------------------------------------------------------

\sechead{3 \quad Technical Corrections}

\referee{%
    \textbf{P4161 L10:}
    ``further analysis further;'' maybe change to ``further analysis'' or
    ``further analysis here''.}

\todo{%
    Correct.}

\referee{%
    \textbf{P4166 L13:}
    double ``the'' in the sentence.}

\todo{%
    Correct.}

\referee{%
    I hope the authors find my comments helpful in revising their manuscript
    and wish them success for their future endeavours.}

% ----------------------------------------------------------------------

% FIXME: no bibtex in final response
\bibliographystyle{copernicus}
\bibliography{../refs/references}

% ----------------------------------------------------------------------
% Interactive comment text ends
% ----------------------------------------------------------------------

\end{document}
