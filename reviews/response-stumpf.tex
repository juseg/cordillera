\documentclass[10pt]{article}

\usepackage[T1]{fontenc}
\usepackage[pdftex]{graphicx}
\usepackage[pdftex]{color}

\sloppy
\setlength{\parskip}{1.2ex}
\setlength{\parindent}{0em}
\renewcommand\familydefault{\sfdefault}

\graphicspath{{../figures/}}
\definecolor{journalname}{rgb}{0.34,0.59,0.82}

\begin{document}
\textbf{Authors' response to A.~Stumpf}
\bigskip

% ----------------------------------------------------------------------
% Interactive comment text begins
% ----------------------------------------------------------------------

\newcommand{\sechead}[1]{\bigskip\noindent\textbf{#1}}
\newcommand{\referee}[1]{\bigskip\textcolor{journalname}{\textit{#1}}}
\newcommand{\msquote}[1]{\begin{quote}\textit{#1}\end{quote}}
\newcommand{\todo}[1]{\textcolor{red}{TODO: #1}}

To A.~Stumpf,

\referee{%
    Seguinot et al. present a well designed numerical model for the Cordilleran
    Ice Sheet (CIS) in North America for the last two glaciations occurring
    over the past 120000 years. Although the constraints on such a model are
    not yet all fully understood, their proposed simulations attempt to take
    account all the complexity of the glacial system, and utilizes a variety of
    data sets. The modeling confirms what geologists have observed in the
    field; there was a fast decay of the CIS during MIS~2 and MIS~4 and
    non-glacial conditions existed during MIS (Olympia Nonglacial Interval;
    e.g., Plouffe and Jett\'e, 1997). I applaud them for undertaking of such a
    difficult task, and notifying the reader where input data is sparse or
    inconclusive.}

\referee{%
    I provide the following general comments and observations which may help
    the authors in revising the manuscript for final publication. Many of these
    points are both my personal suggestions and also the recommendations of
    others currently researching the CIS or who have completed studies in the
    past.}

\referee{%
    1) With the large amount of research that has been undertaken to map the
    landforms and deposits of the CIS by federal and provincial scientists,
    academic faculty, and undergraduate and graduate students and determine the
    extent, volume and dynamics ice sheet, the impression left on the reader by
    the opening sentence would be incorrect. These studies have greatly
    advanced our understanding of the CIS, and could be an important dataset to
    test against the modeling. Stumpf et al. (2014) provides a list of some of
    these studies.}

\referee{%
    2) To help the reader better understand the maps presented, I would
    recommend some spatial information be added (e.g., latitude/longitude
    grids; political boundaries; lakes and rivers, place names etc..).}

\referee{%
    3) Was the model tested against regional-scale ground-based data (e.g.,
    Ferbey et al. 2013) to constrain ice divide positions, ice flow direction,
    and ice sheet thickness?}

\referee{%
    4) For the central sector of the CIS, Stumpf et al. (2000) provides some
    insight into the chronology and effectiveness of glacial erosional during
    the MIS. In figure~5, and in the accompanying text, they describe how
    landforms on the surface formed. For example, in lake valleys east of the
    Skeena Mountains, it appeared the major glacial streamlined landforms were
    formed during a longer glacial advance phase, with ice flow paralleling the
    valleys, and later flows, some perpendicular the valley flow, only weakly
    impacted them.}

\referee{%
    5) Stumpf et al. (2000) was the first study to extensively document a
    predominant westerly directed ice flow across high elevations in the Skeena
    and Coast Mountains. This flow direction appeared to continue into the
    late-glacial period. Other subsequent studies also confirm that
    late-glacial readvance eastward out of these mountains and retreat of ice
    margins westward into these mountains was limited.}

% ----------------------------------------------------------------------
% Interactive comment text ends
% ----------------------------------------------------------------------

\end{document}
