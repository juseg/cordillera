\documentclass[10pt]{article}

\usepackage[T1]{fontenc}
\usepackage[pdftex]{graphicx}
\usepackage[pdftex]{color}
\usepackage{hyperref}
\usepackage{natbib}

\sloppy
\setlength{\parskip}{1.2ex}
\setlength{\parindent}{0em}
\renewcommand\familydefault{\sfdefault}

\graphicspath{{../figures/}}
\definecolor{journalname}{rgb}{0.34,0.59,0.82}
\definecolor{adgeored}{rgb}{0.439,0.157,0.145}  % from Advances in Geoscience

\begin{document}
\textbf{Authors' response to A.~Stumpf}
\bigskip

% ----------------------------------------------------------------------
% Interactive comment text begins
% ----------------------------------------------------------------------

\newcommand{\doi}[1]{doi:\allowbreak\href{http://dx.doi.org/#1}{#1}}
\newcommand{\sechead}[1]{\bigskip\noindent\textbf{#1}}
\newcommand{\referee}[1]{\bigskip\textcolor{journalname}{\textit{#1}}}
\newcommand{\msquote}[1]{\begin{quote}\textit{#1}\end{quote}}
\newcommand{\todo}[1]{\textcolor{adgeored}{TODO: #1}}

To A.~Stumpf,

\referee{%
    Seguinot et al. present a well designed numerical model for the Cordilleran
    Ice Sheet (CIS) in North America for the last two glaciations occurring
    over the past 120000 years. Although the constraints on such a model are
    not yet all fully understood, their proposed simulations attempt to take
    account all the complexity of the glacial system, and utilizes a variety of
    data sets. The modeling confirms what geologists have observed in the
    field; there was a fast decay of the CIS during MIS~2 and MIS~4 and
    non-glacial conditions existed during MIS (Olympia Nonglacial Interval;
    \citealp[e.g.,][]{Plouffe.Jette.1997}). I applaud them for undertaking of
    such a difficult task, and notifying the reader where input data is sparse
    or inconclusive.}

\referee{%
    I provide the following general comments and observations which may help
    the authors in revising the manuscript for final publication. Many of these
    points are both my personal suggestions and also the recommendations of
    others currently researching the CIS or who have completed studies in the
    past.}

\referee{%
    1) With the large amount of research that has been undertaken to map the
    landforms and deposits of the CIS by federal and provincial scientists,
    academic faculty, and undergraduate and graduate students and determine the
    extent, volume and dynamics ice sheet, the impression left on the reader by
    the opening sentence would be incorrect. These studies have greatly
    advanced our understanding of the CIS, and could be an important dataset to
    test against the modeling. \citet{Stumpf.etal.2014} provides a list of some
    of these studies.}

\referee{%
    2) To help the reader better understand the maps presented, I would
    recommend some spatial information be added (e.g., latitude/longitude
    grids; political boundaries; lakes and rivers, place names etc..).}

\referee{%
    3) Was the model tested against regional-scale ground-based data
    \citep[e.g.,][]{Ferbey.etal.2013} to constrain ice divide positions, ice
    flow direction, and ice sheet thickness?}

\referee{%
    4) For the central sector of the CIS, \citet{Stumpf.etal.2000} provides
    some insight into the chronology and effectiveness of glacial erosional
    during the MIS. In figure~5, and in the accompanying text, they describe
    how landforms on the surface formed. For example, in lake valleys east of
    the Skeena Mountains, it appeared the major glacial streamlined landforms
    were formed during a longer glacial advance phase, with ice flow
    paralleling the valleys, and later flows, some perpendicular the valley
    flow, only weakly impacted them.}

\referee{%
    5) \citet{Stumpf.etal.2000} was the first study to extensively document a
    predominant westerly directed ice flow across high elevations in the Skeena
    and Coast Mountains. This flow direction appeared to continue into the
    late-glacial period. Other subsequent studies also confirm that
    late-glacial readvance eastward out of these mountains and retreat of ice
    margins westward into these mountains was limited.}

% ----------------------------------------------------------------------

\begin{thebibliography}{69}

\bibitem[{Ferbey et~al.(2013)}]{Ferbey.etal.2013}
Ferbey, T., Arnold, H., and Hickin, A.: Ice-flow indicator compilation, British
  Columbia., British Columbia Geol. Surv., Victoria, BC, Open-File 2013-06,
  \url{http://www.em.gov.bc.ca/Mining/Geoscience/PublicationsCatalogue/OpenFiles/2013/Pages/2013-06.aspx},
  2013.

\bibitem[{Seguinot et~al.(2014)}]{Seguinot.etal.2014}
Seguinot, J., Khroulev, C., Rogozhina, I., Stroeven, A.~P., and Zhang, Q.: The
  effect of climate forcing on numerical simulations of the {C}ordilleran ice
  sheet at the {L}ast {G}lacial {M}aximum, The Cryosphere, 8, 1087--1103,
  \doi{10.5194/tc-8-1087-2014}, 2014.

\bibitem[{Stumpf et~al.(2000)}]{Stumpf.etal.2000}
Stumpf, A.~J., Broster, B.~E., and Levson, V.~M.: Multiphase flow of the late
  Wisconsinan Cordilleran ice sheet in western Canada, Geol. Soc. Am. Bull.,
  112, 1850--1863, \doi{10.1130/0016-7606(2000)112<1850:mfotlw>2.0.co;2}, 2000.

\bibitem[{Stumpf et~al.(2014)}]{Stumpf.etal.2014}
Stumpf, A.~J., Ferbey, T., Plouffe, A., Clague, J.~J., Ward, B.~C., Paulen,
  R.~C., and Bush, A.~B.: Discussion: ``Streamlined erosional residuals and
  drumlins in central British Columbia, Canada'' by J. Donald {McClenagan},
  (2013) Geomorphology 189, 41--54, Geomorphology, 209, 147--150,
  \doi{10.1016/j.geomorph.2013.10.019}, 2014.

\bibitem[{Plouffe and Jett{\'{e}}(1997)}]{Plouffe.Jette.1997}
Plouffe, A. and Jett{\'{e}}, H.: Middle Wisconsinan sediments and paleoecology
  of central British Columbia: sites at Necoslie and Nautley rivers, Can. J.
  Earth Sci., 34, 200--208, \doi{10.1139/e17-016}, 1997.

\end{thebibliography}

% ----------------------------------------------------------------------
% Interactive comment text ends
% ----------------------------------------------------------------------

\end{document}
