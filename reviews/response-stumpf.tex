% response-stumpf.tex
% ----------------------------------------------------------------------
% response-header.tex
% ----------------------------------------------------------------------

% Base class and packages
\documentclass[11pt]{article}

% Included in online comment header
\usepackage[pdftex]{graphicx}
\usepackage[pdftex]{color}
\usepackage{amssymb}
%\usepackage{times}

% Additional packages
\usepackage[T1]{fontenc}
\usepackage{geometry}
\usepackage[hidelinks]{hyperref}
\usepackage{natbib}

% Graphic path of main manuscript
\graphicspath{{../figures/}}

% Replacements for Copernicus commands
\newcommand{\unit}[1]{\ensuremath{\mathrm{#1}}}
\newcommand{\chem}[1]{\ensuremath{\mathrm{#1}}}
\newcommand{\urlprefix}[0]{}

% Default font and spacing
\renewcommand\familydefault{\sfdefault}
\setlength{\parskip}{1.2ex}
\setlength{\parindent}{0em}
\linespread{1.5}

% color defined in comment template
\definecolor{journalname}{rgb}{0.34,0.59,0.82}

% todo command which should not be used in final version
\definecolor{todored}{rgb}{0.439,0.157,0.145}  % from Advances in Geoscience
\newcommand{\todo}[1]{\textcolor{todored}{TODO: #1}}


\begin{document}
\textbf{Authors' response to A.~Stumpf}
\bigskip

% ----------------------------------------------------------------------
% Interactive comment text begins
% ----------------------------------------------------------------------

\newcommand{\sechead}[1]{\bigskip\noindent\textbf{#1}}
\newcommand{\referee}[1]{\bigskip\textcolor{journalname}{\textit{#1}}}
\newcommand{\msquote}[1]{\begin{quote}\textit{#1}\end{quote}}
\newcommand{\doi}[1]{doi:\allowbreak\href{http://dx.doi.org/#1}{#1}}

To A.~Stumpf,

Thank you very much for your constructive review.
We are very thankful to receive feedback from the glacial geology community,
more particularly so here on The Cryosphere Discussion which traditionally
has a stronger presence in quantitative glaciology.

\referee{%
    Seguinot et al. present a well designed numerical model for the Cordilleran
    Ice Sheet (CIS) in North America for the last two glaciations occurring
    over the past 120000 years. Although the constraints on such a model are
    not yet all fully understood, their proposed simulations attempt to take
    account all the complexity of the glacial system, and utilizes a variety of
    data sets. The modeling confirms what geologists have observed in the
    field; there was a fast decay of the CIS during MIS~2 and MIS~4 and
    non-glacial conditions existed during MIS (Olympia Nonglacial Interval;
    e.g., Plouffe and Jette, 1997). I applaud them for undertaking of
    such a difficult task, and notifying the reader where input data is sparse
    or inconclusive.}

Thank you very much for this positive summary of our work and for your
appreciative words!

\referee{%
    I provide the following general comments and observations which may help
    the authors in revising the manuscript for final publication. Many of these
    points are both my personal suggestions and also the recommendations of
    others currently researching the CIS or who have completed studies in the
    past.}

\referee{%
    1) With the large amount of research that has been undertaken to map the
    landforms and deposits of the CIS by federal and provincial scientists,
    academic faculty, and undergraduate and graduate students and determine the
    extent, volume and dynamics ice sheet, the impression left on the reader by
    the opening sentence would be incorrect. These studies have greatly
    advanced our understanding of the CIS, and could be an important dataset to
    test against the modeling. Stumpf et~al. (2014) provides a list of some
    of these studies.}

By no means our opening sentence was meant to undervalue the large amount of
geological work performed on the Cordilleran ice sheet, much of which has been
reviewed and discussed against our model results in later parts of the
manuscript. Despite all the work done, our impression is that the Cordilleran
ice sheet, due to its complexity, remains less understood than its Laurentide
and Eurasian counterparts, especially when it comes to reconstructing the
dynamics of the different phases of advance and retreat of the ice sheet
through the last glacial cycle. This is the impression that we tried to convey
in the first sentence of the abstract. To avoid any further misunderstanding,
we replaced the opening two sentences with:

\msquote{%
    After more than a century of geological research, the Cordilleran ice sheet
    of North America remains among the least understood in terms of its former
    extent, volume, and dynamics. Because of the mountainous topography on
    which the ice sheet formed, geological studies have often had only local or
    regional relevance, and shown such a complexity that ice sheet-wide spatial
    reconstructions of advance and retreat patterns are lacking.}

\referee{%
    2) To help the reader better understand the maps presented, I would
    recommend some spatial information be added (e.g., latitude/longitude
    grids; political boundaries; lakes and rivers, place names etc..).}

Graticules, rivers and lakes were added to all maps, the latter two as a
background to model result so that they do not interfere with them. Because we
feel that more geographic information would disturb the visibility of the main
figure contents, we have also reworked and enlarged Fig.~1 (location map) to
replace abbreviations by full geographic names, which should also help the
reader to follow discussions in the text.

\referee{%
    3) Was the model tested against regional-scale ground-based data
    (e.g., Ferbey et~al., 2013) to constrain ice divide positions, ice
    flow direction, and ice sheet thickness?}

The model was tested qualitatively against an extensive body of glacial
geology literature, which is the topic of Sect.~4 (Comparison to the geologic
record). The model was not tested quantitatively against geologic evidence,
because we did not know of any publicly available dataset covering a
significant part of the model domain until reading your comment.

Therefore, we thank you very much for pointing to us the recent map and dataset
by \citet{Ferbey.etal.2013}. Indeed, it would make much sense to compare this
database against modelled basal velocities in the future. However, we feel that
such a quantitative comparison is premature at this stage. In fact, modelled
basal velocities are sensitive to hydrological properties of the subglacial
till (cf. our response to A.~Jarosch in this discussion), and even more to
thermal conditions at the base, which in turn are largely influenced by
geothermal heat flow, kept constant in this study.

\referee{%
    4) For the central sector of the CIS, Stumpf et~al. (2000) provides
    some insight into the chronology and effectiveness of glacial erosional
    during the MIS. In figure~5, and in the accompanying text, they describe
    how landforms on the surface formed. For example, in lake valleys east of
    the Skeena Mountains, it appeared the major glacial streamlined landforms
    were formed during a longer glacial advance phase, with ice flow
    paralleling the valleys, and later flows, some perpendicular the valley
    flow, only weakly impacted them.}

Thank your for pointing this out. We have added a reference to
\citet{Stumpf.etal.2000} in the discussion of the erosional imprint on the
landscape (Sect.~4.2.3).

\referee{%
    5) Stumpf et~al. (2000) was the first study to extensively document a
    predominant westerly directed ice flow across high elevations in the Skeena
    and Coast Mountains. This flow direction appeared to continue into the
    late-glacial period. Other subsequent studies also confirm that
    late-glacial readvance eastward out of these mountains and retreat of ice
    margins westward into these mountains was limited.}

Thank you. We have added a reference to \citet{Stumpf.etal.2000} in the
discussion of deglacial flow directions (Sect.~4.3.3).

% ----------------------------------------------------------------------

\begin{thebibliography}{69}

\bibitem[{Ferbey et~al.(2013)Ferbey, Arnold, and Hickin}]{Ferbey.etal.2013}
Ferbey, T., Arnold, H., and Hickin, A.: Ice-flow indicator compilation, British
  Columbia., British Columbia Geol. Surv., Victoria, BC, Open-File 2013-06,
  \urlprefix\url{http://www.em.gov.bc.ca/Mining/Geoscience/PublicationsCatalogue/OpenFiles/2013/Pages/2013-06.aspx},
  2013.

\bibitem[{Plouffe and Jett{\'{e}}(1997)}]{Plouffe.Jette.1997}
Plouffe, A. and Jett{\'{e}}, H.: Middle Wisconsinan sediments and paleoecology
  of central British Columbia: sites at Necoslie and Nautley rivers, Can. J.
  Earth Sci., 34, 200--208, \doi{10.1139/e17-016}, 1997.

\bibitem[{Stumpf et~al.(2000)Stumpf, Broster, and Levson}]{Stumpf.etal.2000}
Stumpf, A.~J., Broster, B.~E., and Levson, V.~M.: Multiphase flow of the late
  Wisconsinan Cordilleran ice sheet in western Canada, Geol. Soc. Am. Bull.,
  112, 1850--1863, \doi{10.1130/0016-7606(2000)112<1850:mfotlw>2.0.co;2}, 2000.

\bibitem[{Stumpf et~al.(2014)Stumpf, Ferbey, Plouffe, Clague, Ward, Paulen, and
  Bush}]{Stumpf.etal.2014}
Stumpf, A.~J., Ferbey, T., Plouffe, A., Clague, J.~J., Ward, B.~C., Paulen,
  R.~C., and Bush, A.~B.: Discussion: ``Streamlined erosional residuals and
  drumlins in central British Columbia, Canada'' by J. Donald {McClenagan},
  (2013) Geomorphology 189, 41--54, Geomorphology, 209, 147--150,
  \doi{10.1016/j.geomorph.2013.10.019}, 2014.

\end{thebibliography}

% ----------------------------------------------------------------------
% Interactive comment text ends
% ----------------------------------------------------------------------

\end{document}
